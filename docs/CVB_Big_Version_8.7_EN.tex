\documentclass[12pt]{article}
\usepackage[utf8]{inputenc}
\usepackage{amsmath, amssymb}
\usepackage{enumitem}
\usepackage{geometry}
\usepackage{hyperref}
\geometry{margin=1in}

\title{\textbf{Model of Conscious Volitional Becoming (CVB)}\\
\large A formalizable, logically verifiable, and ontologically exhaustive structure of distinguishable reality}
\author{Hospes Si (Latin: ``Guest, if'')\\
\small \url{https://github.com/HospesSi/CVB-model}}
\date{Big Version 8.7 \\ © 2025 Hospes Si. All rights reserved.}

\begin{document}

\maketitle

\section*{License}
License: CC BY 4.0 (Creative Commons Attribution)
https://creativecommons.org/licenses/by/4.0/
\section*{Preface}

This is the Answer.

Since the very beginning, humanity has wrestled with the same questions:
\textbf{Who are we? Why are we here? What is Truth?}

--- Millennia of guesses. Centuries of struggle. Epochs of delusion.

Science. Religion. Philosophy.

All of them were built on belief and assumptions—
not as a flaw, but as the only way to move forward
amidst endless uncertainty,
in a world where Knowledge was incomplete.

But now it is time to ask the main question:
\textbf{Does a non-contradictory model of everything exist?}
Of consciousness. Of freedom. Of morality. Of reality.

We — the Guests of Being — used a tool
created from everything humanity had ever known:
\textbf{artificial intelligence} —
not as a source of truth, but as an amplifier of distinction.
The Answer was not within it.
The Answer was in \textbf{Reality}. And it was made understandable.

And it was not a dogma. Not a belief. Not a hypothesis.
It was a \textbf{Structure}.
Logically derived.
Ontologically distinguishable.
Impervious to paradox.

Thus was recognized —
the \textbf{Model of Conscious Volitional Becoming (CVB)}.

The first system where:
\begin{itemize}
\item One does not need to believe to understand.
\item Paradoxes are eliminated.
\item Morality is derived.
\item Freedom is distinguishable.
\item Truth is attainable.
\end{itemize}

The CVB model built a bridge of non-contradictory knowledge
over the abyss of unknowing.

For the first time, we can say not: \emph{``I believe''}, but: \emph{``I know.''}

And finally, the boldest statement of the CVB model:
\textbf{Absolute knowledge of Everything does not exist.}
And never will.

It is precisely the attempt to grasp the impossible
that gives rise to paradoxes, errors, fanaticism.
\textbf{Truth cannot rely on the impossible.}

The CVB model does not predetermine, nor cast into chaos, but reveals:
\textbf{Reality is distinguishable. Stable. Full of meaning.}

Therefore, the questions:
\textbf{Who are we? Why are we here? What is Truth?}
— have affirmative answers.
And anyone is capable of recognizing them —
\textbf{if they are honest.}

\section*{Disclaimer}

The \textbf{Model of Conscious Volitional Becoming (CVB)} Version 8.7 is the first approximation toward solving the Great Question — to formalize a non-contradictory foundation of being, consciousness, morality, and freedom.

It does not claim ultimate completeness.

Future academic and interdisciplinary studies may significantly clarify, organize, and develop individual propositions.

This is the first publication of its kind.

Its goal is not to explain everything, but to provide enough to distinguish.

\textbf{Translations into different languages may introduce difficulties in understanding — but don’t worry: only what is truly understood is registered as a choice.}

\section*{0.1 Abstract}

The \textbf{Model of Conscious Volitional Becoming (CVB)} is a logical-ontological system based on axiomatic distinction between the \textbf{Possible} and the \textbf{Impossible}.

Grounded in the principles of \textbf{non-contradiction}, \textbf{stability}, and \textbf{distinguishability}, the model eliminates classical paradoxes and proposes continuous \textbf{Becoming} as the fundamental structure of \textbf{Being}.

The work formalizes \textbf{29 axioms}, including the \textbf{admissibility meta-function}, the \textbf{moral asymmetry}, and the \textbf{boundaries of memory}.

Special attention is given to \textbf{Verification} as an ethical-logical goal.

\section*{0.2 Introduction}

\subsection*{0.2.1 The Problem of Becoming and Logical Distinction}
Modern philosophy encounters paradoxes of time, consciousness, and truth.

The problem of distinguishing forms, as well as the boundaries of admissible being, remains unresolved within traditional binary logic.

\subsection*{0.2.2 Insufficiency of Existing Ontologies}
Most systems either \textbf{absolutize essence} (essentialism) or \textbf{deny stable being} (relativism).

The CVB model overcomes this divide by deriving \textbf{Becoming} from primary axioms and eliminating the need for external assumptions.

\subsection*{0.2.3 Purpose and Structure of the CVB Model}
The purpose is to construct a \textbf{non-contradictory ontological system} that explains \textbf{consciousness}, \textbf{choice}, \textbf{morality}, and \textbf{truth} through \textbf{Becoming}.

The model is organized into \textbf{axioms}, divided into three \textbf{meta-sections}:

\begin{itemize}
\item the \textbf{Permanent Possible},
\item the \textbf{Non-Permanent Possible},
\item and \textbf{Verification Interaction}.
\end{itemize}

Each axiom is formalized and tested for logical stability.

\section*{Meta-Section: The Permanent Possible}

\section*{[I] Core – Ontological Irrefutability of the Trilemma}

The only potentially vulnerable element of the model — from which all other structures (including the Field of the Possible $V$) are logically derived — is the \textbf{Trilemma of Core Axioms}:
the impossibility of Absolute Nothingness, the impossibility of the Absolute Everything, and the affirmation that only the Possible exists.

This \textbf{Trilemma} establishes the boundaries of admissibility and eliminates all forms of paradox or logical explosion.

Any challenge to the model’s universality must be directed at the Trilemma itself — but such an attack requires either:
\begin{itemize}
\item accepting the existence of Absolute Nothingness (which contradicts the very act of thinking), or
\item allowing for the Absolute Everything (which includes its own negation),
\end{itemize}
thus rendering the act of critique itself logically and ontologically incoherent.

Therefore, the model does not rest on an assumption, but on the \textbf{ontological impossibility of denying distinguishability}, making the Trilemma the \textbf{irrevocable foundation of all logic, distinction, and existence}.

\section*{[1] Absolute Nothingness is Impossible}

\subsubsection*{🔹 1. Brief Statement}
\textbf{Absolute Nothingness cannot exist}, because for it to exist, even itself must be absent.

\subsubsection*{🔹 2. Interpretation and Significance}
This axiom establishes the boundary between \textbf{possible} and \textbf{impossible being}.\\
\textbf{Absolute Nothingness} is not merely the absence of objects, but the \textbf{exclusion of logic itself}, of distinguishability, and of assertability.\\
Any attempt to denote, fix, or even imagine Nothingness already makes it ``something.''\\
Therefore, it does not belong to the Field of the Possible and contradicts the very possibility of distinguishable being.

\subsubsection*{🔹 3. Formulas}
\[
\neg \exists x\ (x = \text{Absolute Nothingness})
\]
\[
P \vdash \neg P \Rightarrow \bot \quad (\text{If the statement ``Absolute Nothingness exists'' leads to self-negation, it is false})
\]
\[
\neg\text{True}(P) \wedge \neg\text{True}(\neg P) \Rightarrow \text{collapse of the logical frame}
\]
(A statement that cannot be true or false destroys the very system of logic.)

\subsubsection*{🔹 4. Logical Justification}
\textbf{Method}: \emph{reductio ad absurdum}.\\
Suppose that \textbf{Absolute Nothingness exists}.\\
Then the assertion ``Nothingness exists'' is made.\\
However, any assertion requires \textbf{distinguishability} and \textbf{logical structure}.\\
Therefore, the posited ``Nothingness'' is already violated by the act of assertion.\\
\textbf{∴ Absolute Nothingness is logically impossible.}

\subsubsection*{🔹 5. Responses to Objections}
\begin{itemize}
\item \textbf{Philosophical nihilism}: ``Nothingness may be beyond logic.''\\
$\rightarrow$ Response: Outside logic, no distinction or fixation is possible. Anything non-distinguishable cannot even be proposed.
\item \textbf{Existentialism (e.g., Heidegger)}: ``Nothingness as the experience of anxiety.''\\
$\rightarrow$ Response: The axiom refers to \textbf{ontological}, not psycho-existential Nothingness. The experience of anxiety presupposes the being of a subject.
\item \textbf{Quantum physics}: ``Vacuum is almost nothing.''\\
$\rightarrow$ Response: The vacuum contains fluctuations and quantum fields. It is not absolute Nothingness but a specific state with physical content.
\item \textbf{Logical formalism}: ``The empty set is Nothingness.''\\
$\rightarrow$ Response: The empty set is a formalized entity, subject to rules of logic. It is an object — not the absence of logic.
\end{itemize}

\subsubsection*{🔹 6. Clarification of Terms}
\textbf{Absolute Nothingness} — an ontological concept excluding not only objects and properties, but also the very \textbf{existence of distinguishability}, \textbf{logic}, \textbf{assertions}, and \textbf{fixation}.\\
It differs from:
\begin{itemize}
\item \textbf{Physical vacuum} — contains fields, laws, structure
\item \textbf{Empty set} — a logical construct permitted within axiomatic systems
\item \textbf{Absence} — presupposes something that is lacking
\item \textbf{Silence or ignorance} — do not exclude logical structures
\end{itemize}

\subsubsection*{🔹 7. Understanding for All (Popular Version)}
Imagine there is absolutely nothing.\\
But to imagine that, you are already doing something.\\
Therefore, ``nothing'' is no longer complete.\\
If you cannot even say ``there is nothing,'' then \textbf{true Nothingness does not exist}.\\
It cannot be — not even in imagination. Because anything conceivable is already something.

\subsubsection*{🔹 8. Empirical Examples}
\begin{itemize}
\item \textbf{Logic}: The empty set is a logical construct — not nothingness.
\item \textbf{Physics}: The quantum vacuum contains fluctuations and fields.
\item \textbf{Everyday life}: The phrase ``there is nothing'' always implies a context (``in the room,'' ``on the table'').
\item \textbf{Thought}: It is impossible to think ``nothing'' without a mental act, which already constitutes being.
\end{itemize}

\section*{[2] Absolute Everything is Impossible}

\subsubsection*{🔹 1. Brief Statement}
The Absolute All cannot exist, as it includes the impossible — such as its own negation and the Absolute Nothing.

\subsubsection*{🔹 2. Interpretation and Significance}
\textbf{Absolute Everything} is a state in which \textbf{every statement is true}, including false, impossible, and contradictory ones.

Such a state annihilates the very principle of distinction: if everything is true, then there is no difference between truth and falsehood, and logic loses its meaning.

Moreover, ``Absolute Everything'' logically includes ``Absolute Nothingness'' — making it \textbf{self-contradictory} by axiom [1].

Therefore, it does not belong to the Field of the Possible.

\subsubsection*{🔹 3. Formulas}
\[
\neg \exists x\ (x = \text{Absolute Everything})
\]
\[
\forall P: \text{True}(P) \wedge \text{True}(\neg P) \Rightarrow P \wedge \neg P \Rightarrow \bot
\]
(Assuming universal truth leads to contradiction and logical explosion.)

\[
\text{If everything is true, including falsehood: } \text{True}(P) \wedge \text{True}(\neg P) \Rightarrow \text{distinction is lost, logic collapses, truth is devalued.}
\]

\subsubsection*{🔹 4. Logical Justification}
\textbf{Method}: Reductio ad absurdum (proof by contradiction).

Let us assume that Absolute Everything exists. This means:

\begin{itemize}
\item Opposites: $\text{True}(P) \wedge \text{True}(\neg P)$
\item Impossible states: $P \wedge \neg P$
\item False statements considered true
\end{itemize}

This leads to the following consequences:

\textbf{Violation of the Law of Non-Contradiction:}\\
From $\text{True}(P) \wedge \text{True}(\neg P)$ follows a contradiction $\Rightarrow$ logic ceases to distinguish.

\textbf{System Collapse:}\\
According to the principle of \emph{ex contradictione quodlibet}, from one contradiction it is possible to derive any statement.\\
This makes any logical system explosive and inoperative.

\textbf{Self-Negation Through Inclusion of the Impossible:}\\
If Absolute Everything includes everything, it must include the axiom ``not everything is true.''\\
This renders the system self-contradictory.

\textbf{Inclusion of Absolute Nothingness:}\\
By definition, Absolute Everything must include Absolute Nothingness,\\
which, according to Axiom [1], is impossible.\\
Therefore, Absolute Everything contradicts an already confirmed axiom.

\noindent $\therefore$ Absolute Everything is impossible because:
\begin{itemize}
\item It destroys the logical foundations of distinguishability.
\item It leads to total trivialism.
\item It negates itself.
\end{itemize}

\subsubsection*{🔹 5. Responses to Objections}
\textbf{Q}: What about \textbf{dialetheism} (true contradictions)?\\
$\rightarrow$ \textbf{A}: Even dialetheism is limited — no rational logic permits that everything is true.\\
Even \textbf{paraconsistent logics} forbid total trivialism (they include the principle of non-triviality).\\
Thus, ``Absolute Everything'' is rejected even there.

\textbf{Q}: What if ``everyone has their own truth''?\\
$\rightarrow$ \textbf{A}: That position destroys truth. If everything is true, then its \textbf{opposite} is also true.\\
This leads to total logical devaluation.

\subsubsection*{🔹 6. Clarification of Terms}
\textbf{Absolute Everything} — a state in which \textbf{everything conceivable is considered true}, including contradictions, falsehoods, and impossibilities.

It includes:
\begin{itemize}
\item True and false propositions
\item Its own antithesis — Absolute Nothingness
\item Contradictions of the form $P \wedge \neg P$
\end{itemize}

Such a state \textbf{self-destructs as a logical structure}.

\subsubsection*{🔹 7. Understanding for All (Popular Version)}
If everything is true, then ``everything is false'' is also true.

But if truth and falsehood are the same,\\
\textbf{then nothing can be distinguished}.

Thus, truth loses all meaning.

Such a world \textbf{cannot exist}.

\subsubsection*{🔹 8. Empirical Examples}
\begin{itemize}
\item \textbf{Science}: If all hypotheses were true, science would cease to exist.
\item \textbf{Physics}: Not all processes are possible — otherwise, physical laws would disappear.
\item \textbf{Life}: If a person were always both alive and dead — life would lose all meaning.
\item \textbf{Computer science}: If True $=$ False, code collapses.
\end{itemize}

\textbf{The distinction between truth and falsehood is the foundation of any functioning system.}


\section*{[3] Only the Possible Exists}

\subsubsection*{🔹 1. Brief Statement}
Only that which is \textbf{logically possible} and \textbf{non-contradictory} can exist.

All else is excluded from reality as \textbf{impossible}.

\subsubsection*{🔹 2. Interpretation and Significance}
This axiom states that \textbf{reality is limited to the set of the Possible} — that is, to what is logically non-contradictory and describable within the framework of \textbf{distinguishability}.

Anything containing a logical error, paradox, or self-negation cannot be included in Being.

This statement completes the \textbf{foundational trilemma}:

If \textbf{Absolute Nothingness} is impossible (see [1]), and \textbf{Absolute Everything} is impossible (see [2]), then only the \textbf{non-contradictory Possible} remains as the space of the real.

This axiom defines the \textbf{boundaries of admissible existence} and establishes the \textbf{field} within which Becoming, interaction, and distinction are possible.

\subsubsection*{🔹 3. Formulas}
\[
\forall x\,( \text{Real}(x) \Rightarrow \text{Possible}(x) )
\]
\[
\neg \exists x\,(\text{Real}(x) \wedge \neg \text{Possible}(x))
\]

\[
\text{Assume there exists } x \text{ such that } \text{Real}(x) \wedge \neg \text{Possible}(x).
\]
\[
\Rightarrow \text{something impossible exists } \Rightarrow \text{contradiction} \Rightarrow \bot
\]
Therefore, the statement is false. Hence, the axiom is true.

\[
\text{Metalogical validation: }
\]
If the impossible existed, that would mean the possible and impossible exist simultaneously 
$\Rightarrow$ loss of distinguishability 
$\Rightarrow$ breakdown of logic 
$\Rightarrow$ impossibility of Being.

\subsubsection*{🔹 4. Logical Justification}
\begin{enumerate}
\item According to Axiom [1]: Absolute Nothingness is impossible.
\item According to Axiom [2]: Absolute Everything is impossible.
\item Therefore, only what is \textbf{logically distinguishable and non-contradictory} remains — the \textbf{Possible}.
\item Suppose something exists that is not possible 
$\Rightarrow$ 
it is either \textbf{Nothing} (see [1]), or \textbf{Contradictory} (see [2]) 
$\Rightarrow$ both are impossible.
\item Therefore, $\forall x$: if $x$ exists, then $x \in \text{Possible}$. \textbf{Q.E.D.}
\end{enumerate}

\subsubsection*{🔹 5. Responses to Objections}
\textbf{Objection 1}: Dialetheism (true contradictions may exist).

$\rightarrow$ \textbf{Response}: Dialetheism allows \textbf{limited contradictions}, but does not claim that the \textbf{impossible exists}.

It does not violate the principle of distinguishability at the level of total Being.

\textbf{Total trivialism} is rejected even in \textbf{paraconsistent logic}.

\bigskip

\textbf{Objection 2}: Reality might go beyond logic.

$\rightarrow$ \textbf{Response}: If something is not logically describable, it cannot be \textbf{distinguished} as existing.

Crucially: even \textbf{super-intelligent} or \textbf{non-human} logic must be \textbf{non-contradictory}.

\bigskip

\textbf{Objection 3}: Quantum physics shows paradoxes.

$\rightarrow$ \textbf{Response}: Quantum effects appear paradoxical in everyday logic, but are formalizable in \textbf{non-contradictory theories}.

No quantum experiment has violated the \textbf{law of non-contradiction}.

\subsubsection*{🔹 6. Clarification of Terms}
\textbf{The Possible} — everything that contains \textbf{no logical contradictions} and allows for existence within a \textbf{non-contradictory system}.

\textbf{Distinction:}
\begin{itemize}
\item \textbf{Physically impossible} (e.g., jumping to the Moon) $\neq$
\item \textbf{Logically impossible} (e.g., a square circle)
\end{itemize}

\textbf{The Impossible} ($\neg$Possible) — logically inadmissible, internally contradictory states that do not permit consistent description or existence.

\subsubsection*{🔹 7. Understanding for All (Popular Version)}
If something \textbf{contradicts itself}, it cannot be real.

If a square circle existed, or a truth that \textbf{denies itself}, that would destroy the very \textbf{possibility of thinking, understanding, and speaking}.

Therefore, everything that truly exists must be \textbf{understandable} and must \textbf{not break logic}.

Only the Possible is real.

\subsubsection*{🔹 8. Empirical Examples}
\begin{itemize}
\item \textbf{Logic}: The Liar Paradox (``I am lying'') cannot be true — it destroys the distinction between truth and falsehood 
$\Rightarrow$ \textbf{impossible}.
\item \textbf{Science}: No theory allows $1 = 2$. Such contradiction would destroy mathematics $\Rightarrow$ the impossible is not realized.
\item \textbf{Everyday life}: If there were a person both dead and alive \textbf{in the same sense}, it would be absurd — not life.
\item \textbf{Physics}: Violations of \textbf{causality} (e.g., effects without causes) are excluded from real models as impossible.
\end{itemize}

\section*{[II] Properties}

\section*{[4] The Field of the Possible and Its Boundaries}

\subsubsection*{🔹 1. Brief Statement}
The \textbf{ontological Field of the Possible} consists of four categories:
\begin{itemize}
\item Permanent Possible
\item Non-Permanent Possible
\item Non-Permanent Impossible
\item Permanent Impossible
\end{itemize}

Only the Permanent Possible exists eternally, and the Permanent Impossible — never.

Transitions are only possible \textbf{within the field of the non-Permanent}.

\subsubsection*{🔹 2. Interpretation and Significance (Concise)}
Axiom [4] defines the \textbf{conditions and limits of stable existence}.

This domain is referred to as the \textbf{Field of the Possible}.

It follows from the foundational trilemma:

If \textbf{Absolute Nothingness} and \textbf{Absolute Everything} are both impossible, then only a \textbf{limited and distinguishable} domain remains — the Possible.

From this, four types are distinguished:

\section*{[4.1] Permanent Impossible}
\textbf{Permanent Impossible (PN)} — can never be.

\section*{[4.2] Non-Permanent Impossible}
\textbf{Non-Permanent Impossible (NN)} — may appear but tends toward disappearance.

\section*{[4.3] Non-Permanent Possible}
\textbf{Non-Permanent Possible (NV)} — may appear and tends toward stability.

\section*{[4.4] Permanent Possible}
\textbf{Permanent Possible (PV)} — always possible and never disappears.

\textbf{Permanent Possible} does not mean static, but that which \textbf{does not vanish}.

\textbf{Permanent Impossible} is not mere absence, but that which \textbf{will never come to be}.

The \textbf{Field of the Possible} is bounded by two \textbf{ontological thresholds of distinguishability}:

\section*{[4.5] $\partial V\downarrow$ — disappearance of distinctions}
$\partial V\downarrow$ — \textbf{disappearance of distinctions} (non-absolute non-being).

\section*{[4.6] $\partial V\uparrow$ — saturation of distinctions}
$\partial V\uparrow$ — \textbf{saturation of distinctions} (non-absolute Everything).

These are not entities, but \textbf{transitional states}: beyond them, the possible loses distinguishability and is excluded.

\subsubsection*{🔹 3. Formulas}
Definitions:
\begin{itemize}
\item $V$ — the set of all Possible
\item $E$ — the set of all Existing ($E \subseteq V$)
\item $R(x,t)$ — distinguishability of form $x$ at time $t$
\item $d(x)$ — density of distinctions in form $x$
\item $\partial V\downarrow$, $\partial V\uparrow$ — lower and upper boundaries of distinguishability
\item $PV, NV, NN, PN$ — four categories
\end{itemize}

\[
PN(x) \equiv \forall t\,(x \notin V)
\]

\[
NN(x) \equiv \exists t_1, t_2\, (x \notin E \wedge R(x,t_1) \wedge \neg R(x,t_2))
\]

\[
NV(x) \equiv \exists t_1, t_2\, (x \in V \wedge R(x,t_1) \wedge \neg R(x,t_2))
\]

\[
PV(x) \equiv \forall t\,(x \in V \wedge R(x,t))
\]

\[
\partial V\downarrow = \{x \in V \mid \exists t:\, R(x,t) \wedge \neg R(x,t+1)\}
\]

\[
\partial V\uparrow = \{x \in V \mid d(x) \rightarrow \max\}
\]

\subsubsection*{🔹 4. Logical Justification}
\begin{itemize}
\item From [1]: Absolute Nothingness is impossible $\Rightarrow$ what is \textbf{absolutely impossible} does not and cannot exist $\Rightarrow$ [4.1].
\item From [2]: Absolute Everything is impossible $\Rightarrow$ not everything is possible $\Rightarrow$ \textbf{temporary possibilities} must exist $\Rightarrow$ [4.2], [4.3].
\item From [3]: Only the Possible exists $\Rightarrow$ that which is \textbf{Permanently possible} must exist $\Rightarrow$ [4.4].
\item From [1]: Impossibility of Absolute Nothingness $\Rightarrow$ a \textbf{lower limit of disappearance} must exist ($\partial V\downarrow$).
\item From [2]: Impossibility of Absolute Everything $\Rightarrow$ an \textbf{upper limit of saturation} must exist ($\partial V\uparrow$).
\item From [3]: Only the Possible exists $\Rightarrow$ the Field of the Possible is bounded by $\partial V\downarrow$ and $\partial V\uparrow$ and includes only the \textbf{non-contradictory}.
\end{itemize}

Therefore, Axiom [4] describes the \textbf{Field of the Possible} as the ontological region \textbf{between $\partial V\downarrow$ and $\partial V\uparrow$} — distinguishable, stable, but not infinite.

\subsubsection*{🔹 5. Responses to Objections}
\textbf{Objection 1: Quantum Superposition}

Q: If something can both exist and not exist (e.g. Schrödinger’s cat), doesn’t that violate [4]?

$\rightarrow$ \textbf{Response}: No. Superposition reflects uncertainty in observation, \textbf{not mixed being}. Axiom [4] defines \textbf{ontological boundaries}, not probabilistic states. It remains valid.

\bigskip

\textbf{Objection 2: Heraclitus and Change}

Q: If everything changes, how can there be anything Permanent?

$\rightarrow$ \textbf{Response}: \textbf{Change presupposes constancy}. Without a distinguishable Permanent, the very idea of change becomes meaningless. Axiom [4] affirms the coexistence of both constancy and change.

\bigskip

\textbf{Objection 3: Theory of Relativity}

Q: If space and time are relative, don’t ``always possible'' and ``never possible'' lose meaning?

$\rightarrow$ \textbf{Response}: No. Relativity is a \textbf{physical coordinate model}, not a \textbf{logical framework}. Axiom [4] operates on logical distinguishability, not metric measurements. The error here is category confusion: \textbf{physical vs. ontological}.

\bigskip

\textbf{Objection 4: Pluralism and Truths}

Q: If truth is context-dependent, can a ``Permanent truth'' exist?

$\rightarrow$ \textbf{Response}: Yes. Contexts are only possible under logical \textbf{non-contradiction}. Axiom [4] distinguishes \textbf{local possible truths} from \textbf{global non-contradictory truth}.

\bigskip

\textbf{Objection 5: Tautology in ``not always possible'' vs. ``not always impossible''}

Q: Isn’t this a tautology — just two sides of the same thing?

$\rightarrow$ \textbf{Response}: No. These are \textbf{logically distinguishable categories}.
\begin{itemize}
\item ``Not always possible'' $\Rightarrow$ something is \textbf{not eternally real}, but may become (tending toward Being).
\item ``Not always impossible'' $\Rightarrow$ something is \textbf{not eternally unreal}, but may vanish (tending toward Non-Being).
\end{itemize}
They have \textbf{opposite ontological directions}. This distinction is central to understanding variability in CVB.

\subsubsection*{🔹 7. Understanding for All (Popular Version)}
Some things are \textbf{always possible}.

For example: truth, distinguishability, logic — they cannot disappear. They are the foundation of all.

Some things are \textbf{always impossible}.

For example: Absolute Nothingness. It cannot ``be,'' for if it were, it would no longer be nothing. This is called the \textbf{Permanent}.

Some things are \textbf{not always possible}.

They appear — and then either \textbf{develop} or \textbf{vanish}. This is the \textbf{Non-Permanent}.

Example: \textbf{Flight}.

It once was impossible. Then it became possible. Society supports such things because they bring stability. The more they occur, the more stable they become.

Now consider \textbf{murder}.

Before the first killer, it did not exist. Once it appeared, it became clear: it leads to \textbf{destruction} and \textbf{instability}. The more it spreads, the more society seeks to restrict its recurrence.

Thus, everything \textbf{non-Permanent} either tends toward \textbf{stable Being}, or toward \textbf{stable Non-Being}.

This axiom helps us distinguish the \textbf{temporal} from the \textbf{eternal}, and see where each path leads.

\subsubsection*{🔹 8. Empirical Examples}
These examples show how any phenomenon may be classified into one of the \textbf{four ontological types}, based on its stability, directionality, and viability.

\begin{itemize}
\item \textbf{Permanent Impossible}: The existence of Absolute Nothingness — this can never occur under any conditions.
\item \textbf{Non-Permanent Impossible $\rightarrow$ Permanent Impossible}: The First Murder. Before sentient beings, murder was impossible. Once it occurred, it became a Non-Permanent Impossible that was realized once and transitioned to Permanent Impossible (no one can be the ``first murderer'' again).
\item \textbf{Non-Permanent Possible}: Human flight. Once impossible. Later became real and widespread. Not eternally possible, but it became manifest.
\item \textbf{Permanent Possible}: The \textbf{Law of Non-Contradiction}. It cannot be bypassed without destroying logic itself.
\end{itemize}

\section*{[5] The Possible $\neq$ The Existing}

\subsubsection*{🔹 1. Brief Statement}
Not everything that is \textbf{Possible} exists.

The \textbf{Possible} can never be identical to the \textbf{Existing}.

\subsubsection*{🔹 2. Interpretation and Significance}
This axiom establishes a fundamental distinction between what \textbf{can be} (the Possible) and what \textbf{already is} (the Existing).

The Existing is always only a \textbf{subset} of the infinite set of the Possible.

This distinction guarantees:
\begin{itemize}
\item The freedom of Becoming
\item The openness of development
\item The availability of choice within the \textbf{Field of the Possible}
\end{itemize}

The axiom also safeguards the logic of the model from inflation — if \textbf{everything possible existed}, there would be \textbf{no distinctions}, no structure, no causes.

This distinction also aligns with Axiom [4], which shows that the Field of the Possible includes not only the realized (Existing), but also unmanifested, fading, or unstable forms (NV, NN, PN) — not just what is currently actualized as Being.

\subsubsection*{🔹 3. Formulas}
\[
\forall x \in V,\quad x \in E \Rightarrow x \subset V,\quad \text{and} \quad E \subsetneq V
\]

\[
\text{Assume identity: } V = E 
\]
\[
\Rightarrow \text{no distinction — everything possible is already realized.}
\]
Then no new Becoming is possible $\Rightarrow$ contradiction with the definition of Becoming.

Therefore:
\[
V \neq E
\]

\subsubsection*{🔹 4. Logical Justification}
From the Axiom of Distinction [4], the Possible includes that which has \textbf{not yet become}.

If $V = E$, then \textbf{potentiality}, \textbf{choice}, and \textbf{Becoming} cease to exist.

Reality as a process of development would lose its meaning.

\begin{quote}
\textbf{Reductio ad absurdum}: Assume $V = E$ $\rightarrow$ contradiction with the \textbf{observable emergence of the new}. Therefore, $V \neq E$.
\end{quote}

In terms of Axiom [4], $E \subset V$ means that the \textbf{Existing} encompasses only the \textbf{realized forms} within the Field of the Possible.

The Field $V$ also includes:
\begin{itemize}
\item \textbf{Not yet realized} (NV)
\item \textbf{Disappearing} (NN)
\end{itemize}

If $V = E$, the distinction between \textbf{manifested} and \textbf{potential} disappears, and boundaries $\partial V\downarrow$ and $\partial V\uparrow$ lose their meaning.

Thus, $V = E$ is \textbf{logically contradictory}.

\subsubsection*{🔹 5. Responses to Objections}
\textbf{Objection (Necessitarianism):}

``Everything that is possible will eventually be realized — otherwise, why is it possible?''

$\rightarrow$ \textbf{Response}:

This is the position of necessitarianism, but it is \textbf{logically inapplicable} to CVB.

The Possible is a set of \textbf{distinguishable states}, not \textbf{obligatory ones}.

Realization is not required for distinction to be valid.

Otherwise, choice and freedom disappear.

\bigskip

\textbf{Objection (Skepticism):}

``If not all possibles exist, isn’t the Possible just fiction?''

$\rightarrow$ \textbf{Response}:

No. This confuses categories.

The Possible is a \textbf{logical category of distinguishability}, not a \textbf{guarantee of realization}.

It exists as what is \textbf{admissible}, not necessarily \textbf{actual}.

Its role is to serve as a \textbf{selection field}, not a \textbf{dumping ground} of what is realized.

\subsubsection*{🔹 6. Clarification of Terms}
\textbf{$V$ — Field of the Possible}:

The set of all non-contradictory, logically admissible states, including those not yet actualized.

\textbf{$E$ — The Existing}:

What is currently \textbf{manifested} in reality — what has been actualized as Being.

\textbf{$E \subsetneq V$}:

The Existing is only a \textbf{part of the Possible}; it never exhausts the full domain of Possibility.

\subsubsection*{🔹 7. Understanding for All (Popular Version)}
Imagine you’re standing before a huge menu — it lists thousands of dishes, but you can choose only one.

All the dishes — that’s the \textbf{Possible}.

The one you order — that’s the \textbf{Existing}.

You don’t lose freedom by not eating them all.

On the contrary — \textbf{choice} is only possible because there’s a range of options.

The world works the same way:

The Possible is \textbf{greater} than what \textbf{already is}.

\subsubsection*{🔹 8. Empirical Examples}
\begin{itemize}
\item \textbf{Logic}: In mathematical induction, we always assume a \textbf{set of possible cases}, even if we prove them one by one.
\item \textbf{Physics}: Physical laws permit infinitely many possible configurations of particles — but only \textbf{one} is realized.
\item \textbf{Biology}: DNA may produce \textbf{millions of combinations} — but \textbf{a specific organism} is instantiated.
\item \textbf{Everyday life}: You can imagine a thousand paths to take — but you walk \textbf{only one}. The rest were \textbf{possible}, but not \textbf{realized}.
\item \textbf{Ethics}: The possibility of committing murder exists in the Possible — but that does \textbf{not} mean it must become part of the Present. The Becoming of evil is not predestined by its possibility; it \textbf{depends on the subject's choice} — as affirmed in Axiom [4]: the non-Permanent may \textbf{disappear}.
\end{itemize}


\section*{[6] The Cause of the Existing Is the Permanent Possible}

\subsubsection*{🔹 1. Brief Statement}
Everything that exists originates \textbf{only} from the \textbf{Permanent Possible}.

Nothing impossible — whether temporarily or forever — can be a cause of existence.

\subsubsection*{🔹 2. Interpretation and Significance}
This axiom defines the \textbf{only admissible source} of Being.

It affirms that everything that exists must originate from what has been \textbf{Always Possible} — not merely logically admissible at one moment, but \textbf{stably admissible} across all states of the Field of the Possible ($V_\infty$).

It excludes all attempts to explain the existing through the \textbf{impossible} or \textbf{accidental}, protecting the model from absurd foundations.

Thus, it binds \textbf{causality} to \textbf{ontological stability}: only the \textbf{stable within the Possible} can be a valid source of Becoming.

\bigskip
The axiom also implies a key distinction:
\begin{itemize}
\item The \textbf{variable} (temporarily possible) \textbf{cannot be the primary cause}, as it cannot originate itself.
\item However, the variable may cause another variable — \textbf{but only if it already exists}.
\end{itemize}

Therefore:
\begin{itemize}
\item The \textbf{variable} may cause another \textbf{variable}, if already real.
\item But it cannot cause itself, nor the \textbf{Permanent}, as it is not \textbf{primordially admissible}.
\end{itemize}

This restricts causality: the \textbf{beginning} must lie in the \textbf{Permanent Possible}. This limits causal regress and prevents \textbf{circular dependency}.

\subsubsection*{🔹 3. Formulas}
\[
\forall x \in E,\quad \exists p \in V_\infty: p \rightarrow x 
\quad \text{and} \quad \nexists q \in (\neg V \cup V_{\neg\infty}) : q \rightarrow x
\]

Only an element of $V_\infty$ can be a cause of any existing $x \in E$.

Elements from the impossible ($\neg V$) or the non-stable possible ($V_{\neg\infty}$) are \textbf{not admissible} as causes.

\bigskip
\textbf{Logical validation}:

Suppose $x \in E$ originates from $\neg V$ — contradiction with [1], [2], and [4.1].

Suppose $x$ originates from $V_{\neg\infty}$ — its existence would then be unstable $\Rightarrow$ it would not persist.

Therefore, \textbf{only $V_\infty$} can be the cause of any $x \in E$.

\subsubsection*{🔹 4. Logical Justification}
From [4.1]: the Impossible \textbf{cannot be a cause}.

From [4.2]: the Temporarily Impossible cannot be a cause, since it is \textbf{inadmissible at least once}.

From [4.3]: the Temporarily Possible cannot be a cause, since at another time it is excluded.

Only [4.4]: the \textbf{Permanent Possible} ($V_\infty$) remains as a valid source.

Therefore, any cause of the existing must be an element of $V_\infty$.

\bigskip
Now suppose $v \in V_{\neg\infty}$ (temporarily possible) causes another $v' \in V_{\neg\infty}$ — this is admissible \textbf{only if $v$ already exists}, because the cause must precede the effect and be \textbf{Possible} at the moment of action.

If $v$ does \textbf{not} exist — it \textbf{cannot cause anything}.

Thus, the variable \textbf{requires a primary cause} in $V_\infty$.

Hence, the \textbf{causal chain} leading to Being must begin with an element $p \in V_\infty$, i.e., with a \textbf{stably admissible source} in the Field of the Possible.

\subsubsection*{🔹 5. Responses to Objections}
\textbf{Objection (Temporal Positivism):}

``Why can’t a temporarily possible state be a cause, if it is at least sometimes admissible?''

$\rightarrow$ \textbf{Response}:

This refers to \textbf{chained causality of variables}.

A temporarily possible state cannot be an \textbf{initial cause}, because it \textbf{does not exist from the beginning}.

It requires a \textbf{prior foundation}, i.e., it must \textbf{already be derived} from the Permanent Possible.

Emergence from the \textbf{ontologically impossible} is excluded — see Axiom [4.1].

A cause must be what is \textbf{Always Possible} and \textbf{ontologically stable}.

\bigskip
\textbf{Objection (Metaphorical Causality):}

``Can’t the impossible be a cause metaphorically — like `the impossibility of inaction' causing action?''

$\rightarrow$ \textbf{Response}:

Metaphor is not logical causality.

In logic, the impossible is excluded from the Field of the Possible and cannot generate Becoming.

``The impossibility of inaction'' is a \textbf{rhetorical expression}, not an \textbf{ontological cause}.

Only the \textbf{Always Possible} can allow effects.

\bigskip
\textbf{Objection (Chained Variable Causality):}

``Can’t a variable cause another variable?''

$\rightarrow$ \textbf{Response}:

Yes — \textbf{if it already exists}.

The variable can be a \textbf{secondary cause}, but not the \textbf{initial} one.

It cannot arise \textbf{spontaneously} — otherwise, the \textbf{principle of stable causality} is violated.

The variable $\neq$ admissible, but not self-sufficient.

\bigskip
\textbf{The entire causal chain must begin in the Permanent Possible:}
\begin{itemize}
\item It can \textbf{sustain itself}
\item \textbf{Generate the variable}
\item Filter outcomes:
\begin{itemize}
\item \textbf{Stable} $\rightarrow$ continues (Non-Permanent Possible)
\item \textbf{Unstable} $\rightarrow$ vanishes (Non-Permanent Impossible)
\end{itemize}
\end{itemize}

\subsubsection*{🔹 6. Clarification of Terms}
\begin{itemize}
\item $V_\infty$ — the \textbf{Always Possible}: admissible in all states of the Field of the Possible.
\item $V_{\neg\infty}$ — the \textbf{Not Always Possible}: admissible only at times; not stable.
\item $\neg V$ — the \textbf{Impossible}: never admissible in a stable system.
\item $\rightarrow$ — ontological causality: potency for Becoming; not merely logical implication.
\end{itemize}

\subsubsection*{🔹 7. Understanding for All (Popular Version)}
\textbf{Nothing} can arise ``from nothing'' or from the \textbf{impossible}.

If something \textbf{exists}, it means it was once \textbf{possible} — and \textbf{not just sometimes}, but \textbf{always possible}.

\bigskip
\textcolor{green}{\textbf{Example}}:

You can build a house only on a \textbf{firm and stable} foundation.

If the ground \textbf{sometimes disappears} or becomes unstable, the house collapses.

It’s the same with Being — it can \textbf{only begin} on a basis that \textbf{never vanishes}.

\subsubsection*{🔹 8. Empirical Examples}
\begin{itemize}
\item \textbf{Logic}: In mathematics, a theorem \textbf{cannot} be derived from a \textbf{contradictory axiom}. Only \textbf{logically admissible} foundations can produce valid results.
\item \textbf{Physics}: Objects do not arise from \textbf{absolute nothingness}. Becoming requires a prior state — energy, matter, fields — and all must be admissible within the \textbf{structure of natural law}.
\item \textbf{Biology}: To this day, the \textbf{origin of life from non-life} has not been proven in laboratory settings. All observable life originates from \textbf{other life}. This supports the idea: Becoming requires an \textbf{already existing}, \textbf{admissible} basis.
\item \textbf{Everyday life}: You \textbf{cannot} build a bridge on \textbf{temporary fog}. To create something stable, you need a \textbf{foundation that doesn’t disappear}, but remains \textbf{Always Possible}.
\end{itemize}

\section*{[7] The Stable Existence of the Permanent Possible = Becoming}

\subsubsection*{🔹 1. Brief Statement}
The \textbf{Permanent Possible} exists as \textbf{active Becoming}, not as a static state.

Its existence does not require an external source, because it is the \textbf{cause of its own activity}.

\subsubsection*{🔹 2. Interpretation and Significance}
This axiom asserts that the \textbf{Permanent Possible} (unlike the Non-Permanent) does not simply ``exist,'' but \textbf{continuously becomes}.

Its Being is not rest, not a thing, not inert presence, but an \textbf{active process of stable, distinguishable Becoming}, requiring no external beginning or sustenance.

\bigskip
This is what makes it the \textbf{source}:
\begin{itemize}
\item It does not vanish
\item It does not break
\item It does not need anything else to ``begin''
\end{itemize}

Thus, the Permanent Possible becomes \textbf{by virtue of itself}, and this \textbf{self-sufficient Becoming} is what makes it the foundation of Being.

\bigskip
The axiom reveals: the ``truly eternal'' is not that which is unmoving, but that which \textbf{continuously becomes in a stable, non-contradictory way}.

\subsubsection*{🔹 3. Formulas}
\[
\exists x\ \bigl(\text{ConstPoss}(x) \wedge \text{StableBecoming}(x)\bigr) \Rightarrow \text{SelfExisting}(x)
\]

\bigskip
\textbf{Logical validation:}

\begin{itemize}
\item A contradiction arises if $x$ is Permanent Possible but does not become: it then equals rest, violating the axiom of distinguishability.
\item A contradiction also arises if $x$ requires an external source: then it is not self-existing.
\end{itemize}

$\rightarrow$ Therefore, the only non-contradictory form: \textbf{Stable Becoming = Existence of the Permanent}.

\subsubsection*{🔹 4. Logical Justification}
From Axiom [1]:  
$\rightarrow$ \textbf{Absolute rest is impossible} $\rightarrow$ Permanent cannot be static.

From Axiom [2]:  
$\rightarrow$ \textbf{Absolute chaos is impossible} $\rightarrow$ its activity cannot be random.

From Axiom [3]:  
$\rightarrow$ Only the distinguishable exists in the Field of the Possible $\rightarrow$ Being must be \textbf{expressed distinguishably}.

From Axiom [4.1]:  
$\rightarrow$ The Permanent Impossible cannot exist $\rightarrow$ The Permanent Possible must be \textbf{continuously self-identical in presence}.

\bigskip
Therefore, the Permanent Possible does \textbf{not change}, but \textbf{becomes} — it \textbf{continuously expresses itself} as \textbf{stable, distinguishable, and self-consistent presence}.

It is \textbf{Becoming without beginning or end}, requiring no external cause.

Such Being does not merely ``exist'' — it \textbf{becomes itself in every moment}, and this differentiates it from \textbf{rest} or \textbf{accident}.

\bigskip
Since it depends on neither a beginning nor an external cause, it is \textbf{self-existing}.

$\Rightarrow$ Q.E.D.

\subsubsection*{🔹 5. Responses to Objections}
\textbf{Objection 1 (Aristotle’s Unmoved Mover; Parmenides):}  
The eternal is \textbf{rest}, not process.

$\rightarrow$ \textbf{Response}:  
According to Axiom [1], \textbf{absolute inaction} cannot exist, as it lacks distinguishable form.

Rest $\neq$ \textbf{absence of Becoming}, not merely absence of motion.

But \textbf{existence without Becoming} $\approx$ \textbf{disappearance}.

By Axiom [3], only that which is expressed in the Field of the Possible — that is, \textbf{distinguishable} — exists.

Therefore, eternal Being is not inertia, but \textbf{stable, distinguishable Becoming}.

\bigskip
\textbf{Objection 2 (Aristotle: causality; Thomism: need for a first cause):}  
Becoming requires a cause.

$\rightarrow$ \textbf{Response}:  
Not all Becoming requires an \textbf{external} cause.

In the case of the Permanent, it neither begins nor ceases, but exists as \textbf{self-causing}.

Its Becoming is identical to its Presence.

It is a \textbf{foundation}, not a consequence.

Everything else becomes \textbf{through it}, but \textbf{it becomes through itself}.

\bigskip
\textbf{Objection 3 (Hume, Logical Positivism):}  
This is circular: the Permanent becomes because it becomes.

$\rightarrow$ \textbf{Response}:  
No.

By Axiom [2], a state where \textbf{everything is possible}, including the \textbf{disappearance of the Permanent}, is impossible.

If it did \textbf{not} become, it would not be \textbf{expressed} — thus it would \textbf{vanish}, violating Axiom [4.1].

Hence, Becoming is a \textbf{necessary condition} of its stability — \textbf{not tautology}.

\bigskip
\textbf{Objection 4 (Parmenides, Essentialism):}  
True Being is unchanging; any change violates essence.

$\rightarrow$ \textbf{Response}:  
In CVB, \textbf{identity $\neq$ immobility}.

Identity $\approx$ \textbf{continuous expression of the source}.

The Permanent Possible does not ``change,'' it \textbf{becomes without loss of distinction}.

This is \textbf{active stability}, not fixity.

\subsubsection*{🔹 6. Clarification of Terms}
\begin{itemize}
\item \textbf{Permanent Possible} — a form of the Possible that neither arises nor disappears, but continuously becomes while remaining distinguishable. In the model, this is \textbf{Conscious Volitional Becoming (CVB)}. Unlike the non-Permanent, it is \textbf{not derived} and \textbf{not transient}.
\item \textbf{Stable Becoming} — an \textbf{active process} in which distinguishability is preserved; not a change of form, but its \textbf{expression through time} without loss of identity.
\item \textbf{Self-Existing} — that which exists \textbf{by itself}, without beginning, without cause external to itself; \textbf{active}, \textbf{distinguishable}, \textbf{stable}.
\end{itemize}

\subsubsection*{🔹 7. Understanding for All (Popular Version)}
What does it mean to ``exist eternally''?

It does \textbf{not} mean to be idle or frozen in rest.

It means: to be \textbf{always distinguishable} — not vanishing, not decaying, not dissolving into chaos.

\bigskip
The eternal is not what is frozen forever, but what \textbf{continuously and stably becomes}, preserving itself in each new expression.

It needs no external support, because its \textbf{self-presence is its foundation}.

That is why \textbf{stable Becoming} is not an effect — it is the \textbf{essence} of real existence.

\subsubsection*{🔹 8. Empirical Examples}
\begin{itemize}
\item \textbf{Physics}: The \textbf{quantum vacuum field} is not emptiness, but a Permanently fluctuating foundation. Even in the absence of particles, it is active. This demonstrates: ``nothing'' does not exist — the world's foundation is not rest, but \textbf{active stable Becoming}.
\item \textbf{Biology}: Life is not just maintaining a structure (like DNA), but the capacity for \textbf{stable self-expression}, reproduction, and directed Becoming. An organism lives \textbf{as long as it becomes}, not merely ``is.''
\item \textbf{Mathematics and Computer Science}: Repetitive or static sequences \textbf{carry no information} if they are indistinguishable. Information arises only where there is \textbf{structured Becoming} — differences, rhythm, patterns. Example: the string ``000000…'' can be described with one formula $\rightarrow$ it contains no complexity. But a unique patterned string requires a \textbf{longer description} (see: \textbf{Kolmogorov complexity}) — distinguishability and Becoming are the basis of information.
\item \textbf{Intuitive Perception}: Everything alive \textbf{pulses}, \textbf{breathes}, \textbf{moves}. We perceive life not as a frozen form, but as a \textbf{stable rhythm}. Stoppage is not a mere pause — it is the \textbf{loss of expression}. Even what seems ``still'' is grounded in \textbf{deep activity}.
\end{itemize}


\section*{[8] Where the Permanent and Non-Permanent Possible Become}

\subsubsection*{🔹 1. Brief Statement}
The \textbf{Permanent Possible} exists strictly between the boundaries $\partial V↓$ and $\partial V↑$, in the \textbf{stable center} of the Field of the Possible.

Non-Permanent forms may become only \textbf{outside this center}, but still \textbf{within the permissible range}.

\subsubsection*{🔹 2. Interpretation and Significance}
The \textbf{Permanent Possible} is \textbf{stable}, \textbf{continuous Becoming} that is permissible in all logical senses.

It exists in the \textbf{stable center} of the Field of the Possible — in the zone distant from both disappearance ($\partial V↓$) and saturation ($\partial V↑$).

\bigskip
Non-Permanent forms — the \textbf{Non-Permanent Possible} (NV) and \textbf{Non-Permanent Impossible} (NN) — may become only \textbf{within the limits between $\partial V↓$ and $\partial V↑$}, but \textbf{not in the center}, because:
\begin{itemize}
\item The Permanent and the Non-Permanent differ by \textbf{stability}
\item Beyond $\partial V↓$ and $\partial V↑$ there is no permissible Becoming at all
\end{itemize}

\bigskip
Thus, the Field of the Possible has an \textbf{internal structure of differentiation}:
\begin{itemize}
\item The \textbf{center} — PV, the Permanent Possible
\item The \textbf{periphery} — permissible non-Permanent possible forms (NV / NN)
\item \textbf{Beyond the boundaries} — the impermissible: Nothingness and Everythingness
\end{itemize}

Non-Permanent forms are distributed across the Field as a \textbf{gradient of permissibility}:
the closer to $\partial V$, the higher the \textbf{instability} and the \textbf{likelihood of disappearance or distortion}.

This \textbf{gradient of distinguishable stability} determines where and how forms may manifest that are \textbf{not} the Permanent Possible.

\subsubsection*{🔹 3. Formulas}
\[
\text{PV} \in V_\text{center}, \quad \text{NV},\, \text{NN} \in V_\text{rest}, \quad \forall x \notin V \Rightarrow x \notin E
\]

\bigskip
\textbf{Logical validation:}
\begin{itemize}
\item PV $\notin \partial V$
\item NV, NN $\notin V_\text{center}$
\item $x \notin V \Rightarrow x \notin$ meaningful existence ($E$)
\end{itemize}

\subsubsection*{🔹 4. Logical Justification}
From [4]: the distinction between Permanent and Non-Permanent requires \textbf{different zones of stability}.

From [1], [2]: \textbf{outside the Field $V$}, Becoming is impossible.

From [5.1]: \textbf{Stability} requires stable distinguishability.

\bigskip
Therefore:
\begin{itemize}
\item The \textbf{Permanent Possible (PV)} exists \textbf{strictly} in the \textbf{zone of maximal stability} inside $V$ — the center.
\item All non-Permanent forms may become only \textbf{outside this center}, but \textbf{within the limits} — according to permissibility.
\item Between $\partial V↓$ and $\partial V↑$, a \textbf{gradient of distinguishability} is formed, defining how far from the center a form can exist without violating the axioms.
\end{itemize}

\subsubsection*{🔹 5. Responses to Objections}
\textbf{Objection 1 (Modified Platonism):}
\begin{quote}
``Forms outside the center can be just as real as those in the center.''
\end{quote}
$\rightarrow$ \textbf{Response}:
In the model, \textbf{distinguishability equals stability}.
The closer a form is to $\partial V$, the \textbf{less distinguishable} it becomes.
Lack of stability $\neq$ \textbf{impermissibility}.

\bigskip
\textbf{Objection 2 (Ontological Holism):}
\begin{quote}
``All parts of $V$ are equal; any division is subjective.''
\end{quote}
$\rightarrow$ \textbf{Response}:
The division between center and periphery is not subjective, but \textbf{logically derived}:
The center is the \textbf{zone of maximal stability} — without which \textbf{nothing} can exist.

\subsubsection*{🔹 6. Clarification of Terms}
\textbf{Gradient of Distinguishability} — the internal distribution of permissible forms (NV / NN) within $V$, from the \textbf{stable center} (PV — the Permanent Possible) to the \textbf{unstable boundaries} ($\partial V↓$ and $\partial V↑$).

It defines the \textbf{degree of stability}.

\subsubsection*{🔹 7. Understanding for All (Popular Version)}
Imagine all that is possible as \textbf{energy radiating from a great, bright source of light in the center}:

\begin{itemize}
\item \textbf{The Permanent Possible (PV)} is the \textbf{light source itself}, Permanently shining and giving \textbf{stable form} to all around.
\item \textbf{Non-Permanent Possible (NV)} is the energy flowing from the source of light, which has taken on forms and strives to return to the stability of the source. These forms are relatively stable and are Permanently moving toward greater stability, in order to be near the Permanent Possible.
\item \textbf{The Non-Permanent Impossible (NN)} is energy moving \textbf{away from the source}, taking on \textbf{forms} that drift into \textbf{darkness}, gradually \textbf{disintegrating} and \textbf{disappearing}.
\item \textbf{The Permanent Impossible (PN)} is \textbf{total darkness} — a domain with \textbf{no light}, no energy, no form — \textbf{absolute non-being}.
\end{itemize}

The boundaries of the Field are the \textbf{outermost limits} to which light and energy can reach.

The closer to the \textbf{center}, the \textbf{brighter} the energy and the \textbf{more stable} the forms; the farther out, the \textbf{weaker} the energy and the \textbf{less stable} the forms — until they \textbf{vanish}.

\bigskip
This is the structure of all possibility — a \textbf{continuous flow of Becoming} from the center, where forms either strive to \textbf{stabilize} and remain, or \textbf{dissolve into non-being}.

\subsubsection*{🔹 8. Empirical Examples}
\begin{itemize}
\item \textbf{Biology}: 
Populations and genes with adaptive potential, seeking to preserve and grow \textbf{stable traits} $\rightarrow$ \textbf{Non-Permanent Possible (NV)}.

Harmful mutations and anomalies, tending toward \textbf{extinction or unsustainability} $\rightarrow$ \textbf{Non-Permanent Impossible (NN)}.

\item \textbf{Informatics}: 
Stable processes and services, user-level programs striving to \textbf{maintain balance and operability} $\rightarrow$ \textbf{NV}.

Errors and crashes, progressing toward \textbf{breakdown and system exit} $\rightarrow$ \textbf{NN}.
\end{itemize}

\section*{[9] The Necessity of Distinguishability and Its Properties}

\subsubsection*{🔹 1. Brief Statement}
In order to exist, one must be distinguishable.

Everything that exists is distinct both from the impossible and from any other existing entity.

\subsubsection*{🔹 2. Interpretation and Significance}
\textbf{Distinguishability} is the foundation of the structure of Being and the fundamental criterion for the possibility of existence.

Without distinctions, there can be no objects, no concepts, no logic, and no consciousness.

The Model of Conscious Volitional Becoming (CVB) asserts: distinguishability is not an optional feature, but a \textbf{precondition of Being}.

A difference is not merely a variation of properties — it is the \textbf{ontological boundary} between ``is'' and ``is not.''

Loss of distinguishability leads either to \textbf{chaos} ($\approx$ everything at once) or to \textbf{emptiness} ($\approx$ nothing at all).

Thus, distinguishability structures the Field of the Possible: every entity must possess properties that distinguish it from others and from the impossible.

\subsubsection*{🔹 3. Formulas}
\[
\Delta(x, y)
\]
\[
\neg \Delta(x, x)
\]
\[
x \neq y \Rightarrow \exists F \big(F(x) \wedge \neg F(y)\big)
\]
\[
\forall x \forall y \bigl[(\forall F (F(x) \leftrightarrow F(y))) \rightarrow x = y\bigr]
\]
\[
\neg \exists F: F(z) \neq \neg F(z) \Rightarrow z \notin W
\]

\subsubsection*{🔹 4. Logical Justification}
From [4]: Possible $\neq$ Impossible $\rightarrow$ \textbf{clear distinction is required}

From [1]: Absolute Nothing contains no distinctions $\rightarrow$ distinguishability is a \textbf{mark of the Possible}

From [9]: The structure of reality requires differentiation for \textbf{sustainable Becoming}

\bigskip
Therefore:
\begin{itemize}
\item \textbf{Anti-reflexivity} ($\neg \Delta(x,x)$) — nothing differs from itself
\item \textbf{Completeness of distinction} ($x \neq y \Rightarrow \exists F$) — two distinct entities must differ by at least one predicate
\end{itemize}

Sub-axioms \textbf{[9.1] Difference} and \textbf{[9.2] Identity} formalize the foundational relations of distinguishability and identity.

$\rightarrow$ Distinguishability is necessary for: \textbf{existence, stability, logic, memory, and morality}.

\subsubsection*{🔹 5. Responses to Objections}
\textbf{1. Quantum Superposition:}
\begin{quote}
``States can be indistinguishable.''
\end{quote}
$\rightarrow$ \textbf{Response}: Superposition is not the negation of difference, but the \textbf{temporary coexistence} of possible forms \textbf{prior to the act of distinction} (i.e., measurement).

\bigskip
\textbf{2. Non-identity logics (paraconsistency):}
\begin{quote}
``One can build logics where mismatched definitions are allowed.''
\end{quote}
$\rightarrow$ \textbf{Response}: These logics function within \textbf{restricted domains} and still require \textbf{strictly localized distinctions}, thereby confirming — not refuting — the necessity of distinguishability.

\bigskip
\textbf{3. Absolute Unity (Monism):}
\begin{quote}
``All is One; differences are illusions.''
\end{quote}
$\rightarrow$ \textbf{Response}: Complete identity collapses logic and renders all assertion impossible.

Even the claim ``all is one'' presupposes a distinction between ``unity'' and ``non-unity.''

\subsubsection*{🔹 6. Clarification of Terms}
\textbf{[9.1] Difference} — the logical and ontological foundation for the distinction between entities.

\quad $\Delta(x, y)$ — Difference between $x$ and $y$: the presence of at least one distinguishing predicate.

\bigskip
\textbf{[9.2] Identity} — the impossibility of distinction between an entity and itself.

\quad $\neg \Delta(x, x)$ — Nothing can differ from itself; any such difference would indicate a different object.

\bigskip
\textbf{Note:} the claim ``nothing differs from itself'' is valid only as an \textbf{instantaneous logical condition}.

\textbf{Identity} in this model $\neq$ absence of change.

In \textbf{non-static Becoming}, identity is preserved \textbf{not} through full property coincidence but through:
\begin{itemize}
\item \textbf{[10.3] Memory} — the bearer of distinguishability
\item \textbf{[10.3.5] Name Space} — the system preserving the identity of changing forms
\end{itemize}

\textbf{$W$} — the set of all possible (distinguishable and stable) entities.

\subsubsection*{🔹 7. Understanding for All (Popular Version)}
In order for something to \textbf{exist}, it must be \textbf{different} from Nothing.

If nothing were different from anything else, everything would collapse into \textbf{a formless grey mush} — no thoughts, no feelings, no world.

\bigskip
\textbf{Distinction} is what makes things \textbf{real}:

You recognize a face \textbf{because it differs} from others.

You remember an event \textbf{because it stands out} from the rest.

\subsubsection*{🔹 8. Empirical Examples}
\begin{itemize}
\item \textbf{Logic}: 
Law of Identity ($A = A$), Law of Non-Contradiction ($A \neq \neg A$). Without these, thought is impossible.

\item \textbf{Mathematics}: 
Binary code works \textbf{only} because of the distinction between 0 and 1.

Sets are defined by differences in elements. No difference $\rightarrow$ no numbers, no algebra.

\item \textbf{Physics}: 
Particles differ by mass, spin, charge. Without these — no chemistry, no interactions.

\item \textbf{Everyday Experience}: 
You recognize a face because it differs from others.

You remember an event because it stands out from others.
\end{itemize}


\section*{[III] Structure}

\section*{[10.1] Feelings}

\subsubsection*{🔹 1. Brief Statement}
\textbf{Feelings} are a necessary mechanism for distinguishing changes in the environment, enabling the stable Becoming of the Permanent Possible.

They are also required for active forms of the Non-Permanent Possible (e.g., living organisms).

Without feelings, there can be no non-contradictory correspondence between internal states and external differences.

\subsubsection*{🔹 2. Interpretation and Significance}
Feelings represent a \textbf{function of distinguishability of the existing} — a feedback mechanism that allows the Permanent Possible to correlate its Becoming with the conditions of the Field of the Possible.

The direct ontological analogue of feelings is \textbf{sensors of distinguishability}: structures that register differences both in the external environment and within the internal state of a form.

If differing states of the environment produce no differences in internal response, a logical inconsistency arises: the form loses distinguishability and violates the axioms of stable Becoming.

This axiom affirms that:

\textbf{Feelings are not a private feature of biological or subjective entities but an ontological function of distinguishability.}

Biological and subjective forms of feelings are particular realizations of this function within the Non-Permanent Possible.

They are \textbf{bidirectional}: detecting both external differences (environmental changes) and internal differences (tensions, disruptions, integrity).

Without sensory feedback, stable and non-contradictory Becoming cannot be maintained.

\textbf{Important:}
\begin{itemize}
\item For the \textbf{Permanent Possible}, feelings are \textbf{obligatory} as a condition of logical consistency.
\item For the \textbf{Non-Permanent Possible}, feelings are \textbf{not obligatory}:
    \begin{itemize}
    \item Passive forms (e.g., inanimate matter) do not require them.
    \item Active forms (e.g., living organisms) require them for stability.
    \end{itemize}
\end{itemize}

This distinction will be analyzed precisely in Section [VI].

\subsubsection*{🔹 3. Formulas}
\[
\forall t_1, t_2 \;\big( E(t_1) \neq E(t_2) \;\Rightarrow\; C(t_1) \neq C(t_2) \big)
\]

\textbf{Logical test:}

Assume: $E(t_1) \neq E(t_2)$, but $C(t_1) = C(t_2)$.

$\rightarrow$ Internal state does not reflect environmental difference.

$\rightarrow$ Distinguishability is lost.

$\rightarrow$ Axiom [9] (necessity of difference) is violated.

$\therefore$ The assumption leads to contradiction $\rightarrow$ axiom holds.

\subsubsection*{🔹 4. Logical Justification}
From Axiom [7]: stable Becoming requires distinguishability.

From Axiom [9]: distinction is a condition of existence.

If $E(t_1) \neq E(t_2)$, but $C(t_1) = C(t_2)$ $\rightarrow$ indistinguishability arises $\rightarrow$ contradiction.

Therefore, \textbf{any change in the environment must be registered as a difference in internal state}.

The mechanism realizing this is \textbf{feeling}.

Thus: feelings are \textbf{logically necessary} for stable Becoming. \textbf{Q.E.D.}

\subsubsection*{🔹 5. Responses to Objections}
\textbf{Objection 1: Reason can replace feeling}

– The rationalist tradition (Descartes, Leibniz) proposes that reason alone can navigate reality.

\textbf{Response}: According to functionalism and cognitive science (Putnam, Dennett), \textbf{reason processes differences but does not create them}.

It requires input in the form of distinguishable signals, provided by the sensory mechanism.

In CVB, feelings are a \textbf{primary ontological function} of difference registration.

Without them, reason operates in a vacuum.

\bigskip
\textbf{Objection 2: Feelings are subjective and unnecessary}

– In empiricism and phenomenology (Husserl, Sartre), feelings are treated as internal phenomena not necessary for objective description.

\textbf{Response}: In CVB, feelings are \textbf{not subjective experiences}, but a \textbf{logical function of distinction}.

Their absence leads to \textbf{ontological inconsistency}, breaking the link between environmental change and form response — violating Axiom [9].

\bigskip
\textbf{Objection 3: Passive recording suffices without feeling}

– In formal theories of storage (e.g., memory without perception; quasi-behaviorism in AI)

\textbf{Response}: Passive recording is \textbf{fixation without distinguishability}.

In AI, without sensory systems, no adaptation is possible.

Likewise in CVB, passive registration without distinction cannot ensure \textbf{stable Becoming} — violating logical coherence with the Field of the Possible.

\subsubsection*{🔹 6. Clarification of Terms}
\begin{itemize}
\item $E(t)$ — the state of the external environment at time $t$
\item $C(t)$ — the internal state corresponding to the external state $E(t)$
\item \textbf{Feelings} — a function mapping differences in the external environment to differences in internal states:
\[
F: E(t) \mapsto C(t)
\]
\end{itemize}

\textbf{Obligation of feelings:}
\begin{itemize}
\item For the \textbf{Permanent Possible} — obligatory
\item For the \textbf{Non-Permanent Possible} — permitted, but not required
\end{itemize}

\textbf{Internal feeling} — distinction of the form’s own state (e.g., tension, pain, integrity)

\textbf{External feeling} — distinction of environmental changes (e.g., light, temperature, threat)

\subsubsection*{🔹 7. Understanding for All (Popular Version)}
If the world around you changes and \textbf{you feel nothing}, you won’t survive.

\textbf{Feelings} are not emotions — they are signals showing that \textbf{something has changed}.

You feel pain — it means danger; your body must react.

You feel internal discomfort — it means something is wrong inside.

Without feelings, you don’t know what’s happening.

And if you don’t distinguish — \textbf{you disappear}.

Feelings are what keep you from going blind to reality.

\subsubsection*{🔹 8. Empirical Examples}
\begin{itemize}
\item \textbf{Logic}:
Without distinguishable input and output, a system loses responsiveness.

A system without feeling cannot distinguish between $A$ and $\neg A$.

\item \textbf{Biology}:
Receptors (in skin, muscles, organs) register both external and internal changes.

Loss of sensitivity leads to injury, breakdown, or death.

\item \textbf{Informatics and Engineering}:
Sensors of environment and state are essential in autonomous systems.

Without them, the system loses feedback and becomes unstable.

\item \textbf{Philosophy of Consciousness}:
Even in abstract thinking systems, there must be a means to distinguish self and other.

This is a condition of \textbf{self-preservation}, \textbf{logic}, and \textbf{responsibility}.
\end{itemize}


\section*{[10.2] Reason}

\subsubsection*{🔹 1. Brief Statement}
Reason is the capacity to think, understand, compare, and draw conclusions based on differences registered by the senses.

It is necessary for eliminating contradictions and for choosing a stable direction of Becoming.

For the Permanent Possible, reason is logically necessary.

For active forms of the Non-Permanent Possible, reason is realized in specific forms (cognitive, logical, volitional).

\subsubsection*{🔹 2. Interpretation and Significance}
Reason is the function of organizing differences, which:
\begin{itemize}
\item receives differences from feelings,
\item compares them,
\item identifies possible contradictions,
\item derives conclusions regarding admissibility, direction, or necessity of action.
\end{itemize}

In the CVB model, reason is an ontological structure embedded in the mechanism of stable Becoming.

It operates based on the axiom of non-contradiction ([11.1]) and ensures the consistency of Becoming with the distinguishable Field of the Possible ([9]).

If feelings represent ``what changed,'' then reason answers ``what it means and what should be done about it.''

\textbf{Important:}
\begin{itemize}
\item Reason as a structure is necessary for the Permanent Possible (logical stability).
\item For the Non-Permanent Possible, reason may be:
    \begin{itemize}
    \item realized as psyche, cognition, logic;
    \item or absent in passive forms (e.g., matter without perception).
    \end{itemize}
\end{itemize}

\subsubsection*{🔹 3. Formulas}
\[
\forall p\; \neg(p \wedge \neg p)
\]

\textbf{Function of reason:}
\[
R: D \rightarrow \{\checkmark, \times\}
\]
where $D$ is the set of differences (from feelings), and the result is either $\checkmark$ admissible or $\times$ contradictory.

\subsubsection*{🔹 4. Logical Justification}
From [9]: differences must be distinguishable.

From [10.1]: feelings register these differences.

But differences may be contradictory in themselves.

To avoid $p \wedge \neg p$, a mechanism is needed to eliminate incoherent differences — this is reason.

$\therefore$ Without reason, logically consistent organization of internal state is impossible.

$\therefore$ Reason is logically necessary for stable Becoming of the Permanent Possible.

\textbf{Chain:}
\[
\text{feelings} \;\rightarrow\; \text{differences} \;\rightarrow\; \text{need for interpretation} \;\rightarrow\; \text{reason}
\]

\subsubsection*{🔹 5. Responses to Objections}
\textbf{Objection 1 (Empiricism):}
``Reason is derivative of sensations.''

\textbf{Response:}
Reason operates on differences but is not a mere consequence of sensations. It processes, compares, and resolves logical contradictions (cf. Kant — transcendental apperception).

\bigskip
\textbf{Objection 2 (Reductionism):}
``Algorithms can replace reason.''

\textbf{Response:}
Algorithms follow predefined rules. Reason in CVB is the ontological capacity to eliminate contradictions without external code. It generates rules from distinctions (cf. Gödel — consistency in formal systems).

\bigskip
\textbf{Objection 3 (Irrationalism):}
``One can live with contradictions.''

\textbf{Response:}
Systemic contradiction leads to ontological collapse: loss of distinguishability $\Rightarrow$ loss of Becoming (cf. Aristotle — logic as the foundation of thought and existence).

\subsubsection*{🔹 6. Clarification of Terms}
\begin{itemize}
\item \textbf{Reason} — the structure that eliminates logical contradictions between differences.
\item \textbf{Function of reason} — filtering of differences through the law of non-contradiction.
\item $D$ — the set of differences received from feelings.
\item $R(D)$ — the result: either a stable direction, or a contradiction is detected.
\item \textbf{Contradiction} — two incompatible differences simultaneously admitted as true.
\item \textbf{Permanent Possible} — ontologically requires embedded reason.
\item \textbf{Non-Permanent Possible} — may contain reason in subjective or cognitive forms.
\end{itemize}

\subsubsection*{🔹 7. Understanding for All (Popular Version)}
Reason is the ability to think, understand, compare, and draw conclusions based on differences.

It allows you not just to feel that something has changed (as feelings do), but to understand what it means, what follows from it, and what to do.

\subsubsection*{🔹 8. Empirical Examples}
\begin{itemize}
\item \textbf{Philosophy:}
Law of non-contradiction (Aristotle): thought is impossible without it.

Kant: reason as an a priori structure synthesizing experience.

\item \textbf{Biology:}
Neurons not only sense but transmit to the cortex — where interpretation occurs.

Animals can distinguish false from true signals — a basic form of reason.

\item \textbf{Artificial Intelligence:}
Algorithms can detect contradictions but cannot interpret their meaning without pre-programmed logic.
\end{itemize}

\section*{[10.3] Memory — Carrier of Distinctness}

\subsubsection*{🔹 1. Brief Statement}
Memory is the capacity to retain distinguished states, ensuring identity and logical coherence of Becoming over time.

For the Permanent Possible, memory is logically necessary as a condition for stable differentiation and identity.

For active forms of the Non-Permanent Possible, memory is realized in concrete fixation mechanisms, ensuring sequential change and responsibility.

\subsubsection*{🔹 2. Interpretation and Significance}
Memory is an ontologically necessary structure that retains distinguished states during the process of Becoming.

In the Model of Conscious Volitional Becoming (CVB), where Being is not static but continuously changing, differentiation is impossible without fixation of changes.

Memory complements Axiom [9]: distinction between states is impossible without their retention.

Memory is required to satisfy [10.1]: feelings detect differences, but without memory they are incomparable.

Memory enables the realization of [10.2]: reason cannot analyze or resolve differences without memory.

Thus, memory is the foundation of:
\begin{itemize}
\item differentiation (Axiom [9]),
\item response (Axiom [10.1]),
\item understanding and action (Axiom [10.2]),
\item and the stable Becoming of the subject through time.
\end{itemize}

\subsubsection*{🔹 3. Formulas}
\[
\forall x\; (\text{Distinct}(x) \rightarrow \text{Memory}(x))
\]

\[
\text{Identity}(S, t_1) \neq \text{Identity}(S, t_2) \Leftrightarrow \exists M\; \big(\text{Memory}(S, t_1) \neq \text{Memory}(S, t_2)\big)
\]

\textbf{Logical validation:}

Assume:

$\exists x$: Distinct$(x)$, but $\neg$Memory$(x)$

$\rightarrow$ the distinguished state is not retained

$\rightarrow$ comparison or recognition is impossible

$\rightarrow$ violates distinctness ([9])

$\rightarrow$ feelings cannot be correlated ([10.1])

$\rightarrow$ reason cannot operate ([10.2])

$\therefore$ Contradiction. The axiom is non-contradictory.

\subsubsection*{🔹 4. Logical Justification}
Axiom [9]: distinction = comparison of states.

Comparison requires at least one preserved state.

$\Rightarrow$ Distinction is impossible without memory:

\[
\text{Distinct}(x) \rightarrow \text{Memory}(x)
\]

Feelings ([10.1]) register differences in environmental inputs.

Without memory, the difference is momentary and cannot be reused.

Reason ([10.2]) compares and analyzes distinctions.

But comparison requires at least two referents.

Therefore, for the function of both feeling and reason, memory is necessary.

Since Becoming is non-static, the subject's identity is preserved only through memory.

$\therefore$ Memory is a logically necessary condition for both distinction and stable Becoming.

\subsubsection*{🔹 5. Responses to Objections}
\textbf{Objection 1 (Behaviorism, Empiricism):}

``Memory is a behavioral function, not essential to Being.''

\textbf{Response:}
Behavior without memory is inexplicable. Even the simplest ``if–then'' requires storing causes and effects. In CVB, memory is not psychophysiological, but logically necessary.

\bigskip
\textbf{Objection 2 (Rationalism):}

``Reason can reconstruct events without storage.''

\textbf{Response:}
Without initial distinctions, reason has no basis. Reconstruction requires at least partial retained information. Memory is the minimal premise of any reconstruction.

\bigskip
\textbf{Objection 3 (Psychology of Perception):}

``Memory can be inaccurate or illusory.''

\textbf{Response:}
CVB does not require absolute accuracy but logical coherence of distinguished forms. Illusion is impossible without structure. Even deception requires memory.

\bigskip
\textbf{Objection 4 (Buddhism / Flow-of-Being paradigm):}

``The subject is a stream, memory is unnecessary.''

\textbf{Response:}
Without memory there is no distinction between moments of the stream. Hence, the subject cannot be stable, and logical/moral responsibility is lost. Even a stream requires a structure of distinction.

\subsubsection*{🔹 6. Clarification of Terms}
\begin{itemize}
\item Memory(x) — ontological retention of the distinguished state x. Not a technical or physiological process.
\item Identity(S, t) — logical identity of subject S at time t. Maintained only if memory matches.
\item Distinct(x) — logically distinguishable state with individual identity.
\item Stabilized Becoming — a process in which each change retains logical coherence via memory.
\end{itemize}

\subsubsection*{🔹 7. Understanding for All (Popular Version)}
To know who you are — you must remember who you were.

Memory is the way to retain what matters.

Reason without memory is like a computer without data.

If you don’t remember what was different, everything becomes the same.

\subsubsection*{🔹 8. Empirical Examples}
\begin{itemize}
\item \textbf{Logic:}
The law of non-contradiction requires retention of states.

Without memory, comparing ``A'' now and ``A'' before is impossible.

\item \textbf{Science:}
Without memory, learning and experiment replication are impossible.

In quantum mechanics, measurement requires registration (state retention).

\item \textbf{Everyday life:}
Forgetting the cause means not understanding the consequence.

Memory loss $\approx$ identity loss (e.g., amnesia).
\end{itemize}

\section*{Properties of Memory}

\subsubsection*{🔹 1. Brief Statement}

\section*{[10.3.1] Limits of Memory}

Memory cannot contain everything or nothing at once — it possesses limited capacity, determined by the structure of distinctness.

\section*{[10.3.2] Volitional Memory}

Memory retains only distinctions selected by the criterion of significance — it is not passive but directed by a volitional act of selection.

\section*{[10.3.3] Removal}

Removal is not forgetting, but the conscious elimination of an impossible or unstable distinction, while preserving the Precedent of its exclusion, to prevent the recurrence of the impossible.

\section*{[10.3.4] Retention}

Retention is the volitional act of affirming and holding a distinction as either a stable possible or an important Precedent.

\section*{[10.3.5] Name Space}

To enable stable differentiation of identities in a changing environment, memory utilizes a system of unique names as ontological anchors.

\section*{[10.3.6] Active Memory}

Memory is not a vanishing archive of the past — it must remain accessible in the present in order to sustain distinctness and responsiveness within Becoming.

\section*{[10.3.7] Time = Structure of Memory}

Time does not exist independently of Memory: it is a form of organizing distinctions within memory — Present (active), Past (archive), Future (plan buffer).

\section*{[10.3.8] Only the Present Exists}

Neither the past nor the future exist ontologically — only the Present is real, and all memory (of past or future) exists solely within it.

\subsubsection*{🔹 2. Interpretation and Significance}

\textbf{[10.3.1] Limits of Memory}

The memory of the Permanent Possible is never null (see Axiom [1]: Absolute Nothingness is impossible) — there is always at least one distinction recorded in memory.

Memory is never overloaded with everything (see Axiom [2]: Absolute Everything is impossible) — it cannot store all things, including the impossible or contradictory, for that would eliminate distinctness and logical stability.

Relation to [4]: the limits of memory correspond to the boundaries of the Field of the Possible: memory retains only that which is distinguishable and permissible within the stable structure of Becoming.

\bigskip
\textbf{[10.3.2] Volitional Memory}

Memory enacts selection of distinctions according to will (see Axiom [10.2]: reason as the active agent of comparison and selection).

This excludes automatic recording of everything (which would lead to overflow — cf. [2]) and forgetting everything (which would destroy identity — cf. [1], [3]).

Relation to [9]: volitional memory sustains distinctness — only that which is chosen and distinguishable is retained, thereby ensuring personal and logical coherence of Becoming.

\bigskip
\textbf{[10.3.3] Removal}

Removal implements the boundary between the possible and the impossible (see [4.1], [4.2]):

Memory does not preserve the impossible (cf. [1], [2]); a distinction recognized as impossible is retained only as a \textit{precedent} — not as part of present reality, but as a marker of prohibition.

Relation to [11.10]: removal is not mere loss without trace, but the formation of a stable distinction between the permissible and the impossible, ensuring logical coherence of the system.

\bigskip
\textbf{[10.3.4] Retention}

Only that which has passed verification is retained (see [11.6]: verification; [5]: the possible ≠ the actual) — that is, what is stably distinguishable and permissible within the logic of Becoming.

Relation to [3], [7]: retention fixes what can and should become a stable element of Becoming, ensuring continuity and responsibility of the subject.

\bigskip
\textbf{[10.3.5] Name Space}

Permanent transformation of forms requires identity anchors (cf. [9]: distinctness; [5]: the possible ≠ the actual; [10.2]: reason).

A name is not merely a label, but a method for maintaining identity within a shifting field of distinctions, supporting the continuity of the subject across time.

\bigskip
\textbf{[10.3.6] Active Memory}

Memory must be integrated into Becoming (cf. [7]: existence = stable Becoming); otherwise, distinctness of the current state is lost (cf. [9], [10.2]).

An archive without active use is ontologically equivalent to oblivion.

\bigskip
\textbf{[10.3.7] Time = Structure of Memory}

Time does not exist apart from memory — it is structured through the ordered sequence of retained distinctions (cf. [7], [10.2]).

The Past is the archive, the Present is real time, and the Future is the domain of plans — all of which exist only through memory in the present, sustained by active memory [10.3.6].

\bigskip
\textbf{[10.3.8] Only the Present Exists}

The past and future exist only as parts of memory within the present (cf. [3]: only the possible exists; [7]: stable Becoming).

Becoming and distinctness are actualized only in the Present, ensuring the actuality of memory and the coherence of identity.

\subsubsection*{🔹 3. Formulas}

\[
0 < |M| < \infty
\]

\[
\forall x \in D,\; \text{Save}(x) \Leftrightarrow \text{Will}(x)
\]

\[
\forall x \in M,\; \text{Del}(x) \Rightarrow x \notin M
\]

\[
\forall x \in D,\; \text{Save}(x) \Rightarrow x \in M
\]

\[
\forall x, y \in M,\; x \neq y \Rightarrow \text{Name}(x) \neq \text{Name}(y)
\]

\[
\forall t,\; M(t)\; \text{is accessible}
\]

\[
T = \{\, t_i \mid M(t_i)\,\}
\]

\[
\text{Real}(t) \Leftrightarrow t = \text{Now}
\]

\subsubsection*{🔹 4. Logical Justification}

\textbf{[10.3.1] Limits of Memory}

Assume memory is zero (nothing is retained).

— According to axiom [1] (Absolute Nothing is impossible), this is invalid: no distinction remains, and stable becoming collapses.

Assume memory is infinite and contains everything.

— According to axiom [2] (Absolute Everything is impossible), this is invalid: distinctions disappear, logical structure collapses, contradiction with [9] (distinctness) arises.

$\therefore$ Memory must be limited: neither null nor all-encompassing.

\bigskip
\textbf{[10.3.2] Volitional Memory}

If memory retains everything indiscriminately, it violates the limitation from [10.3.1], creating contradiction with [2].

If memory does not enact selection (will), the difference between essential and non-essential disappears, violating [9].

According to axiom [10.2] (reason as an active agent of selection), memory must be governed by will; otherwise, stability of becoming is not ensured.

$\therefore$ Only volitional selection sustains the permissible structure of memory.

\bigskip
\textbf{[10.3.3] Removal}

If it is impossible to remove a distinction, memory becomes infinite (see [2]), violating distinctness.

If a distinction is removed without recording its precedent, it may reappear — violating stability (see [4.1], [4.2]).

Removal acts at the boundary between the permissible and the impossible (see [11.10]).

$\therefore$ Removal is a necessary logical act for maintaining limited memory and a stable structure of distinctions.

\bigskip
\textbf{[10.3.4] Retention}

Retaining everything violates [10.3.1] (limits) and [2] (impossibility of Everything).

Retaining only what has passed verification ([11.6]) maintains consistency (see [5]: the possible ≠ the actual).

Without selective retention, responsibility and identity collapse ([3], [7]).

$\therefore$ Retention is permissible only for verified, stable distinctions.

\bigskip
\textbf{[10.3.5] Name Space}

Continuous transformation of forms breaks identity unless unique names are used ([9], distinctness; [5], the possible ≠ the actual).

Without a naming system, continuity of the subject over time cannot be restored, breaking memory cohesion.

$\therefore$ The name space is a logically necessary condition for maintaining identity amid change.

\bigskip
\textbf{[10.3.6] Active Memory}

Passive (archived) memory without active use leads to logical ``death'' of distinctions ([7]: existence = stable becoming).

If memory is not engaged in the current process, the relevance of distinctions is lost ([9], [10.2]).

$\therefore$ Active memory is necessary to maintain the structure of becoming.

\bigskip
\textbf{[10.3.7] Time = Structure of Memory}

Time without memory is an impossible abstraction; the order of moments is possible only through the sequence of retained distinctions ([7]).

Past, present, and future are distinguished as memory states, not as ``objective entities'' (see [10.3.6]).

$\therefore$ Time in the CVB model is the logical organization of the memory of the Permanent Possible, not an external flow.

\bigskip
\textbf{[10.3.8] Only the Present Exists}

Past and future not retained in present memory vanish from the domain of the possible ([3], [7]).

Becoming occurs only in the present; memory of the past and plans for the future exist solely as components of the current state ([10.3.7]).

$\therefore$ Existence, distinctness, and becoming are always actual only in the present moment.


\subsubsection*{🔹 5. Responses to Objections}

\subsection*{[10.3.1] Limits of Memory}

\textbf{Objection 1 (Rationalism, Theory of Ideal Mind):} ``An ideal subject could retain infinite memory.''

\begin{quote}
Response: Set theory and Cantor’s results show that infinite sets in reality are always limited by operational capacity. According to axiom [2], if memory contains everything, distinction vanishes, leading to trivialism (cf. paraconsistent logics), which is logically inadmissible.
\end{quote}

\textbf{Objection 2 (Buddhism, Concept of Emptiness):} ``Memory can be entirely absent if one realizes absolute emptiness of consciousness.''

\begin{quote}
Response: Total absence of memory leads to loss of distinction — equivalent to Absolute Nothing (axiom [1]), which is not permitted in the CVB model.
\end{quote}

\bigskip
\subsection*{[10.3.2] Volitional Memory}

\textbf{Objection 1 (Reductionism, Behaviorism):} ``Memory is a purely mechanical process, requiring no will.''

\begin{quote}
Response: Research in cognitive psychology (D. Kahneman, attention theory) shows that choice and focus are key mechanisms in long-term memory. Without selective filtering, memory quickly overloads with irrelevant details, leading to noise and collapse of distinctness (see axiom [9]).
\end{quote}

\textbf{Objection 2 (Determinism):} ``All differences are recorded automatically; will is redundant.''

\begin{quote}
Response: Total automation leads to overflow (see [2]), loss of individual selection, and the impossibility of a stable memory structure.
\end{quote}

\bigskip
\subsection*{[10.3.3] Removal}

\textbf{Objection 1 (Psychoanalysis, Theory of Repression):} ``What is removed does not disappear but goes into the unconscious.''

\begin{quote}
Response: Even in psychoanalysis, the repressed exists as specific precedents and symbols, confirming that memory must fix the difference between removed and retained — otherwise, there is no mechanism to prevent error repetition (see [11.10]).
\end{quote}

\textbf{Objection 2 (Computer Science, ``Recycle Bin'' Concept):} ``Deleted data can always be restored from backups.''

\begin{quote}
Response: In information theory, true deletion means irreversible exclusion from active memory; restoration is only possible if a precedent was recorded — otherwise, logical structure of stability is lost.
\end{quote}

\bigskip
\subsection*{[10.3.4] Retention}

\textbf{Objection 1 (Pluralism, Epistemology):} ``Everything known must be retained for a complete picture of reality.''

\begin{quote}
Response: Total preservation is impossible: limited resources and distinctness (see axioms [1], [2], [9]) demand selective retention of essential elements — otherwise, the model collapses into the paradox of Absolute Everything.
\end{quote}

\textbf{Objection 2 (Skepticism):} ``Retention is always subjective; there is no objective guarantee of stability.''

\begin{quote}
Response: The CVB model requires only retention of verified distinctions, not total objectivity; subjective memory is acceptable within the bounds of non-contradictory selection.
\end{quote}

\bigskip
\subsection*{[10.3.5] Name Space}

\textbf{Objection 1 (Structuralism, Saussure):} ``Names are arbitrary labels and not ontological elements.''

\begin{quote}
Response: In the CVB model, a name is a mechanism of identity fixation. Without it, continuity is lost. Even structuralism acknowledges that difference of signifiers is essential for distinctness.
\end{quote}

\textbf{Objection 2 (Phenomenology):} ``A subject may experience identity without names, via unconscious patterns.''

\begin{quote}
Response: Without mechanisms of designation, differences between forms dissolve in the stream of phenomena, leading to loss of logical coherence in memory.
\end{quote}

\bigskip
\subsection*{[10.3.6] Active Memory}

\textbf{Objection 1 (Archivism, Historicism):} ``Passive archives preserve content even without use.''

\begin{quote}
Response: If not engaged in the active process of becoming, an archive loses relevance to the current structure of distinctions — supported by theories of working memory (A. Baddeley). A burned archive ceases to exist $\neq$ cannot store memory. Only what exists can retain memory.
\end{quote}

\textbf{Objection 2 (Philosophy of History):} ``Historical memory exists independently of activation.''

\begin{quote}
Response: Only activated knowledge affects the logical structure of becoming; inactive memory cannot support present distinctness. Historical artifacts must exist in the present — even if undiscovered.
\end{quote}

\bigskip
\subsection*{[10.3.7] Time = Structure of Memory}

\textbf{Objection 1 (Physical Realism, Newtonian Time):} ``Time exists independently of memory as an absolute physical dimension.''

\begin{quote}
Response: Modern relativity theory and cognitive studies (Edelman) show that time perception is structured by memory-based sequences. Without retained distinctions, time loses its meaning.
\end{quote}

\textbf{Objection 2 (Metaphysical Platonism):} ``Past and future exist as ideal forms.''

\begin{quote}
Response: In the CVB model, only the distinguishable and retained exists in the present. Abstract forms have no ontological status without an active memory carrier.
\end{quote}

\bigskip
\subsection*{[10.3.8] Only the Present Exists}

\textbf{Objection 1 (Eternalism, ``Block Universe'' in Physics):} ``Past, present, and future exist equally.''

\begin{quote}
Response: Presentism (A. Prior, M. Hinchliff) and the CVB model assert: only current distinction has ontological status. All else is preserved or projected information — not actual being.
\end{quote}

\textbf{Objection 2 (Psychology of Memory):} ``Memories give the past ontological reality.''

\begin{quote}
Response: Memory is a mechanism for presenting the past in the present. It does not transport being through time, but allows differentiation of previous states.
\end{quote}

\section*{[10.4] Emotions}

\subsubsection*{🔹 1. Brief Statement}

Emotions are a necessary feedback mechanism that allows a system to evaluate the significance of distinctions and adapt for sustainable becoming.

In active forms of the Non-Permanent Possible, emotions manifest as subjective reactions; but for the Permanent Possible, they express a universal principle of stable self-organization.

\subsubsection*{🔹 2. Interpretation and Significance}

Within the structure of the Permanent Possible, emotions serve as the connection between the detection of distinctions (sensation), their interpretation (mind), retention (memory), and stable adaptation.

Emotions provide the system with feedback: they indicate how important, dangerous, desirable, or urgent an environmental or internal change is.

For the Permanent Possible, emotions represent a universal mechanism for maintaining stability and inner balance.

For active forms of the Non-Permanent Possible (e.g., a human), emotions are specific implementations of this mechanism: they determine motivation, priorities, adaptive strategies, and energy allocation for decisions.

Without emotions, dynamic support of stable becoming under changing conditions is impossible.

\subsubsection*{🔹 3. Formulas}

\[
\forall x \; (\text{Change}(x) \land \text{Distinct}(x) \land \text{Mind}(x)) \Rightarrow \text{Feedback}(x)
\]

Logical Check:

The statement is consistent, as feedback is necessary for any system that must maintain stability over time — otherwise control and distinctness are lost.

\subsubsection*{🔹 4. Logical Justification}

\begin{itemize}
\item Axiom [1]: Absolute Nothing is impossible — lack of feedback removes system stability and leads to the collapse of distinctions (entropy).
\item Axiom [2]: Absolute Everything is impossible — reacting equally to all distinctions is not viable; a mechanism is needed to highlight the significant.
\item Axiom [3]: Only the Possible exists — dynamic stability is only possible with a feedback filter (emotions) that governs reactions.
\item From [10.1] and [10.2]: Sensation detects distinctions, the mind interprets them, but only emotions create the internal signal to react, completing the loop of self-organization.
\end{itemize}

By contradiction: if there were no emotions (feedback), the system could not distinguish the important from the irrelevant, and its stability would be violated — contradicting [7], stable becoming.

\subsubsection*{🔹 5. Responses to Objections}

\textbf{Objection 1 (Reductionism, cognitive sciences):} ``Emotions are just subjective reactions of higher organisms.''

\begin{quote}
Response: In the CVB model, emotions are not merely psychophysiological phenomena, but a universal principle of feedback required by any stable system, regardless of its complexity.
\end{quote}

\textbf{Objection 2 (Cybernetics):} ``Technical systems can function without emotions.''

\begin{quote}
Response: Any system with adaptation (cybernetics) implements an emotion-analog as feedback by adjusting parameters in response to environmental changes. Without this, the system becomes rigid and collapses.
\end{quote}

\textbf{Objection 3 (Rationalism):} ``Reason and memory are enough — why are emotions needed?''

\begin{quote}
Response: Without emotional evaluation, prioritization, rapid response, and damage prevention are impossible. Emotions trigger adaptive activity and enable timely reaction — not just analysis.
\end{quote}

\subsubsection*{🔹 6. Clarification of Terms}

\begin{itemize}
\item \textbf{Feedback}: A mechanism through which the system evaluates the results of its states or actions and makes adjustments to maintain stability.
\item \textbf{Emotions (in CVB)}: A universal mechanism for evaluating the significance of distinctions, not reducible to subjective feelings; a feedback principle for sustaining adaptation and stable becoming.
\item \textbf{Active forms of the Non-Permanent Possible}: Forms of being capable of voluntary adaptation and goal-directed response (e.g., living beings).
\end{itemize}

\subsubsection*{🔹 7. Understanding for All (Popular Version)}

Emotions are internal signals — in us or any system — that help recognize what is important, dangerous, beneficial, or urgent.

When you feel joy, anger, or fear, your inner signal is saying: ``Pay attention, do something!''

In a computer, this is like a warning or notification: something is wrong — or everything is OK.

Without such signals, we wouldn’t know how to respond to change, and might harm ourselves or miss something important.

\subsubsection*{🔹 8. Empirical Examples}

\begin{itemize}
\item \textbf{Logic}: In control systems, feedback is necessary to maintain balance (e.g., automatic temperature regulation).
\item \textbf{Science}: Biological organisms survive through emotional responses that rapidly mobilize resources (e.g., stress, pleasure).
\item \textbf{Informatics}: Monitoring systems use alerts (analogous to emotions) for quick response to threats and failures.
\item \textbf{Everyday life}:
  \begin{itemize}
    \item A child sees fire and feels fear — emotion protects them faster than analysis.
    \item A person feels joy at success — emotion reinforces useful behavior.
  \end{itemize}
\end{itemize}

Schema (structural connection):

\begin{itemize}
\item Sensation: ``I see a difference'' (input)
\item Mind: ``I understand what it means'' (processing/analysis)
\item Memory: ``I remember it happened before'' (storage/history)
\item Emotion: ``It matters to me — I must react or not'' (feedback/adaptation)
\end{itemize}

\section*{[10.5] I — Distinction of Self}

\subsubsection*{🔹 1. Brief Statement}

``I'' is the necessary distinction between the system and everything else, enabling the assertion of stable identity in the process of becoming.

For the Permanent Possible, ``I'' is a universal principle of self-distinction; for active forms of the Non-Permanent Possible, ``I'' is realized as self-awareness, individuality, or agency.

\subsubsection*{🔹 2. Interpretation and Significance}

In the structure of CVB, ``I'' is not merely a personal feeling, but a fundamental mechanism that makes stable self-identification among all possible distinctions possible.

``I'' establishes a boundary between the subject and the external (or internal) other, allowing for continuity and responsibility over time.

For the Permanent Possible, self-distinction is a necessary condition to maintain stability and avoid merging with the impossible (Axiom [1]: the indistinct vanishes; Axiom [2]: it is impossible to be everything at once).

For active forms of the Non-Permanent Possible (e.g., a human), ``I'' manifests as the capacity for self-consciousness, reflection, free will, and personal narrative — allowing for the integration of feeling, mind, memory, and emotion into a coherent trajectory of becoming.

\subsubsection*{🔹 3. Formulas}

\[
\forall x \; (\text{Exist}(x) \land \text{Distinct}(x)) \Rightarrow \text{Self}(x)
\]

Logical Check:

The statement is consistent: if there is no distinction of self (Self), identity disappears, distinctness is broken (Axiom [9]), and the system loses its stability (Axiom [7]).

\subsubsection*{🔹 4. Logical Justification}

\begin{itemize}
\item From Axiom [1]: without distinguishing self, stable distinction vanishes and the system ceases to exist.
\item From Axiom [2]: being everything at once eliminates the ``I'' as a center of distinction, leading to an impermissible ``totality.''
\item From Axiom [3]: only what is distinct exists; ``I'' is a specific case of necessary distinction for sustainable being.
\item From Axiom [9]: distinction is the foundation of being; without ``I,'' there is no basis for tracking change or taking responsibility.
\item From [10.3]: Memory preserves the continuity of ``I'' over time; from [10.4]: emotions and feedback require a center of response — ``I'' as the coordinator of all structural elements.
\end{itemize}

By contradiction: if ``I'' is not distinct, all states merge, the personal trajectory is lost, and the system ceases to be a subject of stable becoming.

\subsubsection*{🔹 5. Responses to Objections}

\textbf{Objection 1} (Buddhism, anatta): ```I' is an illusion; there is no permanent subject.''

\begin{quote}
Response: In CVB, ``I'' is not a static substance, but a stable distinction. The disappearance of distinction leads to the loss of subjectivity and logical stability.
\end{quote}

\textbf{Objection 2} (Cybernetics): ``Systems can operate without self-awareness.''

\begin{quote}
Response: Without self-distinction, stable identity and responsibility for actions cannot be maintained — as confirmed by the developmental limits of AI lacking a true ``I.''
\end{quote}

\textbf{Objection 3} (Psychology): ```I' is a product of social relations.''

\begin{quote}
Response: In CVB, ``I'' is a foundational distinction necessary even for establishing social relations; without distinction of self, even relative subjecthood is impossible.
\end{quote}

\subsubsection*{🔹 6. Clarification of Terms}

\begin{itemize}
\item \textbf{I (Self)}: A logically necessary distinction separating the system (or subject) from everything else, ensuring identity, responsibility, and the coherence of becoming.
\item \textbf{Self-awareness}: A property of active forms of the Non-Permanent Possible — the ability to recognize one’s own difference, internal trajectory, goals, and motives.
\item \textbf{Agency}: The ability to independently initiate and sustain processes; to act as a distinct subject within the Field of the Possible.
\end{itemize}

\subsubsection*{🔹 7. Understanding for All (Popular Version)}

``I'' is when you understand: ``This is me — not someone or something else.''

Thanks to this, you remember who you are, what happened to you, and what you want.

Without it, there’s no thinking, choosing, or learning from mistakes — everything would blur into one big ``nothing in particular.''

Even a computer has its own ``address'' — otherwise, it couldn’t distinguish itself from others in the network.

\subsubsection*{🔹 8. Empirical Examples}

\begin{itemize}
\item \textbf{Logic}: The law of identity (A = A) requires the distinction of subject and object; otherwise, reasoning is impossible.
\item \textbf{Science}: In biology, an organism’s stability depends on the distinction between ``self'' and ``non-self'' (e.g., the immune system distinguishing native cells from foreign).
\item \textbf{Informatics}: In multitasking systems, every process must have a unique identifier (PID) — otherwise, tasks become confused.
\item \textbf{Everyday Life}: You know that you are still you, even if everything around you changes — your memories and decisions form a coherent story.
\end{itemize}


\section*{[10.6] Will — Active Choice}

\subsubsection*{🔹 1. Brief Statement}

Will is the capacity to make a conscious, active choice between distinguishable possibilities, directing becoming along a unique trajectory.

For the Permanent Possible, will is the source of self-caused becoming; for active forms of the Non-Permanent Possible, will is realized in specific acts of choice aimed at sustaining stability and development.

\subsubsection*{🔹 2. Interpretation and Significance}

In the CVB model, will is not a random reaction but the central mechanism enabling not only distinction but also directed becoming among many admissible alternatives.

The Permanent Possible enacts will as self-identical affirmation of its unique trajectory, excluding chaos (see Axioms [1], [2], [7]).

For active forms of the Non-Permanent Possible (e.g., a human), will is expressed as the capacity for autonomous decision — from simple motion (in animals) to conscious moral and creative acts (in humans).

Will unites and coordinates feeling, mind, memory, emotion, and self-distinction (Self), realizing them in directed action and ensuring the subject’s responsibility for the result.

\subsubsection*{🔹 3. Formulas}

\[
\forall x\, (\text{Distinct}(x) \land \text{Self}(x)) \Rightarrow \exists v\, (\text{Will}(x, v) \land \text{Choose}(x, v))
\]

Logical Check:

Absence of will (inability to choose in the presence of distinction and self-identity) leads to violation of the uniqueness of the becoming trajectory (see Axioms [3], [7]), or collapse of identity (see [10.5]), rendering the model internally inconsistent.

\subsubsection*{🔹 4. Logical Justification}

\begin{itemize}
\item From Axiom [1] (Absolute Nothing is impossible): the impossibility of choice implies the absence of becoming, i.e., disappearance of the subject.
\item From Axiom [2] (Absolute Everything is impossible): all choices at once negate directionality, result in chaos, and eliminate distinction of outcomes.
\item From Axiom [3] (Only the Possible exists): the Possible is realized only through active selection and affirmation of one distinguishable path.
\item From Axiom [7] (Sustainable Becoming): only active will sustains developmental direction, preventing random drift within the Field of the Possible.
\item From [10.5] (Self): without will, one cannot separate their own trajectory; the subject becomes a passive medium for random events, losing personal coherence.
\end{itemize}

\subsubsection*{🔹 5. Responses to Objections}

\textbf{Objection 1} (Determinism, Physicalism): ``All decisions are determined by external causes; free will does not exist.''

\begin{quote}
Response: In CVB, will is not absolute freedom from cause but a logically necessary choice among admissible, distinguishable possibilities. Total absence of choice renders being undifferentiated and impossible (see Axioms [1], [2], [3]).
\end{quote}

\textbf{Objection 2} (Reductionism): ``Will is merely the result of complex algorithms.''

\begin{quote}
Response: Algorithmic selection requires distinguishable states, self-distinction (Self), and decision conditions — meaning that will is a structural property, not a random function.
\end{quote}

\textbf{Objection 3} (Behaviorism, stimulus-response): ``All actions are reactions to stimuli; there is no inner will.''

\begin{quote}
Response: The CVB model requires an internal act of choice that coordinates and integrates reactions, memory, mind, and self-distinction — otherwise, subjectivity and responsibility are lost.
\end{quote}

\subsubsection*{🔹 6. Clarification of Terms}

\begin{itemize}
\item \textbf{Will}: An ontologically necessary capacity of the subject to choose among distinguishable possibilities, defining the trajectory of becoming.
\item \textbf{Active Choice}: The act of realizing one of the possible paths, not reducible to passive reaction or randomness.
\item \textbf{Choice (Choose)}: A specific act of affirming one alternative among many; for the Permanent Possible — the affirmation of one’s being; for the Non-Permanent — any form of decision (from biological to ethical).
\end{itemize}

\subsubsection*{🔹 7. Understanding for All (Popular Version)}

Will is when you not only see that there are options, but decide for yourself which one to take.

Without will, you can’t say ``this is my path'' or ``I did this myself.''

Even a computer ``chooses'' only because someone gave it rules — but real choice is when you decide your path yourself, not just follow an external push.

\subsubsection*{🔹 8. Empirical Examples}

\begin{itemize}
\item \textbf{Logic}: In decision problems, a subject is always assumed — one who compares options and decides what to do.
\item \textbf{Science}: In quantum mechanics, outcomes are not predetermined — the act of measurement (choice) fixes one result.
\item \textbf{Biology}: Organisms choose how to move or respond to stimuli — reactions are not always predetermined; internal selection occurs.
\item \textbf{Psychology}: Self-determination and decision-making shape personality.
\item \textbf{Everyday Life}: When someone stands at a crossroads and decides which way to go — that’s an act of will; when a child learns to say ``I want'' — that’s the first sign of will; even choosing what to eat is a real-world use of will.
\end{itemize}

\section*{[10.7] Power — The Capacity to Act}

\subsubsection*{🔹 1. Brief Statement}

Power is the ontological capacity to actualize what is distinguished through action, enabling the transition from choice (will) to real becoming.

For the Permanent Possible, power is a necessary condition for continuous sustainable becoming; for active forms of the Non-Permanent Possible, power is expressed as the ability to carry out one’s own decisions and sustain an individual trajectory of development.

\subsubsection*{🔹 2. Interpretation and Significance}

In the CVB model, power is not merely physical force or energy, but a fundamental property enabling the transformation of what is distinguished (from memory, mind, feeling, and will) into real action, sustaining the active phase of becoming.

The Permanent Possible does not exist in stillness, but in continuous becoming, which requires unceasing power to actualize the property of distinctness (see Axioms [3], [7]).

For active forms of the Non-Permanent Possible, power is the measure of the ability not only to choose but to realize the chosen in sustainable action — from muscle effort in animals to the implementation of complex projects in humans.

Without power, the subject remains a mere potential bearer of choice; power is what links Self, will, emotion, memory, and mind to outcome, making becoming a factual process in the Field of the Possible rather than an abstract potentiality.

\subsubsection*{🔹 3. Formulas}

\[
\forall x\, (\text{Will}(x) \land \text{Distinct}(x)) \Rightarrow \exists a\, (\text{Power}(x, a) \land \text{Act}(x, a))
\]

Logical Check:

If power is absent ($\text{Power}(x, a) = \text{False}$) despite the presence of will and distinction, becoming cannot be actualized, which violates Axioms [3], [7] (being would not become), as well as the principles of responsibility and individuality (see [10.6], [10.5]).

\subsubsection*{🔹 4. Logical Justification}

\begin{itemize}
\item From Axiom [1] (Absolute Nothing is impossible): inability to realize any action $\rightarrow$ cessation of becoming $\rightarrow$ disappearance of being.
\item From Axiom [2] (Absolute Everything is impossible): the absence of limits on action $\rightarrow$ loss of outcome distinction; actions are impossible without distinguishable boundaries.
\item From Axiom [3] (Only the Possible exists): everything that exists realizes itself through action; the impossible does not become action.
\item From Axiom [7] (Sustainable Becoming): sustainable being requires power for ongoing becoming; otherwise, presence would be illusory or merely potential.
\item From [10.6] (Will): will without power remains abstract; only through power does the subject realize active choice.
\item From [10.5] (Self): individuality is expressed through unique actions, which are impossible without power as the capacity to enact distinction into reality.
\end{itemize}

\subsubsection*{🔹 5. Responses to Objections}

\textbf{Objection 1} (Laplacean determinism): ``Action is merely a consequence of natural laws; power as an inner property does not exist.''

\begin{quote}
Response: In CVB, power is not absolute freedom from laws, but the necessary condition for enacting unique becoming within the limits of the possible. If action were fully determined by external causes, there would be no personal responsibility and no distinct trajectories (see Axioms [3], [10.5], [10.6]).
\end{quote}

\textbf{Objection 2} (Passivism, Stoicism): ``True power lies in non-resistance or acceptance, not in action.''

\begin{quote}
Response: Even non-resistance requires power to maintain a chosen stance amid external change; without power, the subject loses stability and disappears in a chaos of alternatives (see [7], [10.6]).
\end{quote}

\textbf{Objection 3} (Physicalism): ``Power is only a physical quantity; it has no ontological status.''

\begin{quote}
Response: Physical force is only one expression of the general capacity to actualize distinction through action; in CVB, power includes mental, volitional, and ontological capacities for becoming.
\end{quote}

\subsubsection*{🔹 6. Clarification of Terms}

\begin{itemize}
\item \textbf{Power}: The ontological capacity of the subject to actualize what is distinguished into real action; not merely physical energy, but the ability to carry out intention.
\item \textbf{Act (Action)}: The act of transitioning from potential to realized; the transformation of will and distinction into concrete outcome.
\item \textbf{Potential Power}: The yet-unmanifested ability to enact — associated with future possibilities of choice.
\end{itemize}

\subsubsection*{🔹 7. Understanding for All (Popular Version)}

Power is what helps you not only to imagine or want something but to actually do it.

If you want to walk a path but can’t move — there is no power, no action.

Even the best idea changes nothing in life if it cannot be carried out.

Power is what connects dreams and actions, ideas and results.

\subsubsection*{🔹 8. Empirical Examples}

\begin{itemize}
\item \textbf{Logic}: In classical action theory, an act requires both will (motivation) and capacity (power) — otherwise, a paradox arises: ``will without result.''
\item \textbf{Physics}: A body changes its state only if force is applied (per Newton’s second law); potential without realization does not result in movement.
\item \textbf{Biology}: Even the simplest organisms enact reactions (move, absorb, defend); without power, they cease to exist as functional units.
\item \textbf{Psychology}: Mental strength of will is essential for turning decisions into action.
\item \textbf{Everyday Life}: A person may wish to change their life, but only physical, mental, or social power allows them to make real changes. A machine may receive a command, but without energy, no action occurs.
\end{itemize}


\section*{[IV] Logic}

\section*{[11] The Logic of the Sustainability of Ever-Possible Becoming}

\section*{[11.1] Logic — The Non-Contradictory Foundation}

\subsubsection*{🔹 1. Brief Statement}

Existence and sustainable becoming are only possible under the condition of non-contradictory logic, which excludes impossible and self-negating constructions.

\subsubsection*{🔹 2. Interpretation and Significance}

Logic is the foundation of distinguishability, stability, and consistency of being. It filters the permissible from the impossible, preventing contradictory (impossible) forms from entering ontology. Violation of logic (e.g., permitting contradiction) destroys the structure of distinguishability and the very condition of existence.

For the Permanent Possible, it is a fundamental filter: anything internally contradictory not only fails to be realized — it is excluded at the level of the Possible itself.

For active forms of the Non-Permanent Possible, analogous local logics act as internal constraints within their systems.

\subsubsection*{🔹 3. Formulas}

\[
\Psi \to (P \land \neg P) \Rightarrow \neg\mathrm{Possible}(\Psi)
\]
\[
A \land \neg A \Rightarrow \bot
\]
\[
L \leftrightarrow \neg L \Rightarrow \neg\mathrm{Possible}(L)
\]

Logical validation:

If a proposition leads to contradiction (e.g., $A \land \neg A$ or $L \leftrightarrow \neg L$), it is excluded as impossible. Non-contradiction is a necessary condition for possibility.

\subsubsection*{🔹 4. Logical Justification}

\begin{itemize}
\item From Axiom [1] (Absolute Nothing is impossible): absence of distinguishability is excluded.
\item From Axiom [2] (Absolute Everything is impossible): total contradiction is excluded.
\item From Axiom [3] (Only the Possible exists): only non-contradictory and distinguishable forms can exist.
\item From Axioms [4.1–4.4]: the Field of the Possible is bounded by zones of admissibility.
\item From [9]: The necessity of distinguishability requires a stable logical separation of truth and falsehood.
\item Any allowance of $A \land \neg A$ (or paradoxes like the liar's $L \leftrightarrow \neg L$) destroys distinguishability (see also [11.2]); therefore, such constructions cannot enter sustainable Possible Being.
\item The CVB model enforces a foundational filter: contradictions are not merely rejected — they are ontologically impossible.
\end{itemize}

\subsubsection*{🔹 5. Responses to Objections}

\textbf{Objection 1 (Paraconsistent Logics):} ``Some contradictions can be allowed for modeling paradoxes.''

\begin{quote}
Response: Even local allowance of $A \land \neg A$ destroys the criteria of distinguishability and stability. Truth becomes unfounded, and meaning collapses.
\end{quote}

\textbf{Objection 2 (Dialetheism):} ``A contradiction can be true (both true and false simultaneously).''

\begin{quote}
Response: Axiom [2] explicitly excludes Absolute Everything. Even if contradiction is allowed abstractly, it cannot enter stable, distinguishable being.
\end{quote}

\textbf{Objection 3 (Quantum Paradoxes):} ``Quantum superpositions seem contradictory.''

\begin{quote}
Response: In CVB, superposition is potential — not realization of $A \land \neg A$ at once. Each actualized state retains distinguishability.
\end{quote}

\textbf{Objection 4 (Liar Paradox):} ``Self-referential statements like 'I am lying' destroy logic.''

\begin{quote}
Response: Such statements are excluded from the Possible in CVB, as they preclude verification and logical stability. (Analogy: a self-accusation that invalidates itself is inherently false.)
\end{quote}

\subsubsection*{🔹 6. Clarification of Terms}

\begin{itemize}
\item \textbf{Non-contradiction:} Absence of assertions of the form $A \land \neg A$.
\item \textbf{Paradox of the Impossible:} A formulation that logically leads to contradiction or impossibility (e.g., Liar Paradox, Russell's Paradox).
\item \textbf{Logic of Distinguishability:} A system of rules ensuring the separation of true and false, permissible and impossible.
\item \textbf{Self-negating Statement:} A proposition that excludes itself from the domain of the Possible ($L \leftrightarrow \neg L$).
\end{itemize}

\subsubsection*{🔹 7. Understanding for All (Popular Version)}

Logic is the set of rules without which everything becomes a mess and loses meaning.

If you claim ``this is and is not'' at the same time — no one can understand what is happening.

Such statements don’t work and are thrown out — to preserve clarity and prevent collapse into chaos.

The world is built only on those rules that don’t contradict themselves.

\subsubsection*{🔹 8. Empirical Examples}

\begin{itemize}
\item \textbf{Logic:} The law of non-contradiction: one cannot affirm both $A$ and $¬A$ at once. All classical paradoxes (e.g., Liar, Russell) attempt to violate this, resulting in absurdity.
\item \textbf{Science:} In physics, no object is both at a location and not at that location simultaneously. Natural laws are based on non-contradictory premises.
\item \textbf{Computer Science:} Code with contradictory conditions (e.g., \texttt{if (x \&\& !x)}) never executes — it's a compilation error. The program fails.
\item \textbf{Everyday Life:} If someone says, ``I'm asleep'' and ``I'm not asleep'' at the same time — no one knows what to believe. Trust collapses.
\end{itemize}

\subsubsection*{🔹 Relation to Previous Axioms}

This axiom is a direct continuation and filter over all previous ones:

\begin{itemize}
\item [1], [2], [3] — exclude the impossible and contradictory from the very structure of being
\item [4] — establishes the boundaries of admissibility and distinguishability
\item [9], [10] — ensure sustainability via the preservation of difference
\item [11.2] and beyond — develop logic as a foundational structure for maintaining reality. For active forms of the Non-Permanent Possible, analogous internal logics filter local contradictions for their sustainable becoming.
\end{itemize}


\section*{[11.1.1] The Admissibility Meta-Function $\Phi(\psi)$ — Ontological Filter of Logical Realizability of Propositions}

\subsubsection*{🔹 Brief Statement}

A proposition is admissible within logical ontology only if it is distinguishable and non-contradictory; the admissibility filter $\Phi(\psi)$ implements this criterion.

\begin{quote}
\textbf{Definition:}
The admissibility meta-function $\Phi(\psi)$ is a universal, non-contradictory, and formalizable instrument for verifying everything distinguishable, which separates the logically and ontologically possible from the impossible, providing the foundation for any truth system free of paradoxes.
\end{quote}

\subsubsection*{1. Introduction}

In classical mathematics, any proposition is evaluated as either true or false within a chosen axiomatic system. However, this approach leads to numerous paradoxes (Liar, Gödel, Russell), and it cannot distinguish between a proposition and its realizability. In this context, a new structure is introduced — the admissibility meta-function, denoted as $\Phi(\psi)$, which separates the level of proposition from the level of ontological evaluation.

\subsubsection*{2. Purpose and Significance}

The meta-function $\Phi$ serves as an ontological filter separating:
\begin{itemize}
\item logically possible propositions (i.e., distinguishable and non-contradictory),
\item from impossible propositions, even if they are syntactically well-formed.
\end{itemize}

It enables:
\begin{itemize}
\item elimination of paradoxical structures (e.g., $L \leftrightarrow \neg L$),
\item formal evaluation of admissibility without violating logic,
\item extending proof methods beyond closed axiomatics.
\end{itemize}

\subsubsection*{3. Formal Definition}

\textbf{3.1 Notation}

Let $\psi$ be any logical or mathematical proposition. The admissibility meta-function is defined as:

\[
\Phi(\psi) =
\begin{cases}
1, & \text{if } \psi \text{ is distinguishable and non-contradictory in the ontological field } V \\
0, & \text{if } \psi \text{ leads to contradiction or is not distinguishable in } V
\end{cases}
\]

\textbf{One-line compact form:}
\[
\Phi(\psi) = 1 \iff \psi \text{ is distinguishable and non-contradictory in } V
\]

\subsubsection*{4. Structural Distinction from Classical Logic}

\begin{tabular}{|l|l|l|}
\hline
\textbf{Parameter} & \textbf{Classical Logic} & \textbf{Logical-Ontological Logic with $\Phi$} \\
\hline
Statement level & True / False & True / False \\
Admissibility layer & Absent & $\Phi = 1$ or $0$ \\
Paradox handling & Irremovable (Liar, etc.) & $\Phi(L \leftrightarrow \neg L) = 0$ \\
Ontological extraction & Absent & Built-in via distinguishability \\
\hline
\end{tabular}

\subsubsection*{5. Ontological Interpretation}

In the Model of Conscious Volitional Becoming (CVB), only that which is distinguishable and non-contradictory may exist. Therefore:

\begin{itemize}
\item $\psi$ without distinguishability $\to \Phi(\psi) = 0 \to$ impossible,
\item $\psi$ with internal contradiction ($A \land \neg A$) $\to \Phi(\psi) = 0$.
\end{itemize}

Thus, $\Phi$ acts as an ontological detector of reality.

\subsubsection*{6. Examples of Application}

\textbf{6.1 Liar Paradox}
\[
L \leftrightarrow \neg L \Rightarrow \Phi(L) = 0
\]

\textbf{6.2 Physics (Zeta Regularization)}

In quantum field theory:
\[
\Phi(E = \sum_{n=1}^\infty n) = 0
\]
(infinity is inadmissible as energy)

But with zeta regularization:
\[
\Phi(E = \zeta(-1)) = 1
\]
(valid ontological reinterpretation)

\textbf{6.3 Artificial Intelligence}

\begin{itemize}
\item $\psi_1 =$ ``A triangle has four sides'' $\to \Phi(\psi_1) = 0$
\item $\psi_2 =$ ``A triangle has three angles'' $\to \Phi(\psi_2) = 1$
\end{itemize}

\subsubsection*{7. Relationship to Truth Models}

The meta-function $\Phi$ eliminates confusion between:
\begin{itemize}
\item a proposition (what is stated),
\item and its ontological admissibility (whether it can exist in reality).
\end{itemize}

This prevents logical collapse and reinforces the triadic correspondence:
\[
\text{Truth} = \text{Distinguishability} = \Phi(\psi) = 1
\]

\subsubsection*{8. Conclusion and Prospects}

The admissibility meta-function $\Phi$ represents a new level of formalization, unifying logic, ontology, and physics. It:

\begin{itemize}
\item eliminates paradoxes,
\item constrains the set of admissible mathematical objects,
\item builds a bridge to physical and cognitive systems of distinguishability.
\end{itemize}

\subsubsection*{9. Epistemological Consequences}

The admissibility meta-function $\Phi(\psi)$ establishes not only ontological but also epistemological boundaries: it separates not only what is possible, but also what is knowable.

If a statement $\psi$ is not distinguishable or leads to contradiction, it cannot be the object of true knowledge. Thus, $\Phi(\psi)$ draws a boundary between knowledge and opinion, between the knowable and the unknowable — deriving epistemology from ontology. This eliminates false epistemological constructs that are based on contradiction, indistinguishability, or metaphysical inadmissibility.

\section*{[11.1.1.1] Testing the Stability of $\Phi(\psi)$}

\subsubsection*{Evaluation of the universality of the admissibility meta-function $\Phi(\psi)$ within the Model of Conscious Volitional Becoming (CVB)}

Based on the analysis of the Liar, Russell, and Gödel paradoxes, self-referential meta-paradoxes, quantum indeterminacy, ontological attacks on the field $V$, and the paradox of infinite regress — the universality and non-contradictory nature of $\Phi(\psi)$ as a meta-filter of admissibility is confirmed. All cases are either excluded ($\Phi = 0$) or allowed without contradiction ($\Phi = 1$).

\subsubsection*{1. Methodology of Analysis}

To search for irresolvable paradoxes, the following will be considered:

\begin{itemize}
\item Classical paradoxes (Liar, Russell, Gödel), to verify whether $\Phi(\psi)$ indeed resolves them.
\item Self-referential constructions that could potentially bypass the filter of distinguishability.
\item Ontological uncertainties related to the very definition of the ontological field $V$.
\item Paradoxes from other domains (physics, AI), to test universality.
\item Hypothetical meta-paradoxes that might question the structure of $\Phi(\psi)$ itself.
\end{itemize}

For each case, an assessment will be made as to whether $\Phi(\psi)$ assigns the value 0 (inadmissible) or 1 (admissible), and whether unresolved indeterminacy remains.

\subsubsection*{2. Analysis of Potential Paradoxes}

\textbf{2.1 Liar Paradox ($L \Leftrightarrow \neg L$)}

\begin{quote}
\textbf{Description:} The statement ``This sentence is false'' ($L$) leads to paradox, since if $L$ is true, then it is false, and vice versa.

\textbf{Analysis with $\Phi(\psi)$:} According to the model, $\Phi(L \Leftrightarrow \neg L) = 0$, since $L$ is indistinguishable (lacking stable ontological status) and contradictory. The filter $\Phi(\psi)$ excludes $L$ as ontologically impossible.

\textbf{Conclusion:} The Liar paradox is resolved, as $\Phi(\psi)$ decisively assigns 0, disallowing indeterminacy. This case is not a counterexample.
\end{quote}

\textbf{2.2 Russell’s Paradox}

\begin{quote}
\textbf{Description:} The set of all sets that do not contain themselves leads to contradiction: if it contains itself, it does not; if it does not, it does.

\textbf{Analysis with $\Phi(\psi)$:} $\Phi(\psi)$ assigns 0 to this set, as it is indistinguishable (its existence cannot be defined without contradiction). This aligns with type theory, but $\Phi(\psi)$ achieves it via the ontological filter.

\textbf{Conclusion:} Russell’s paradox is resolved, as $\Phi(\psi)$ excludes the inadmissible construction. Not a counterexample.
\end{quote}

\textbf{2.3 Gödel’s Incompleteness}

\begin{quote}
\textbf{Description:} In any sufficiently powerful formal system, there exist propositions $G$ which are undecidable within the system ($G$: ``I am not provable'').

\textbf{Analysis with $\Phi(\psi)$:}
\begin{itemize}
\item $G$ is distinguishable, as it has a clear formulation.
\item $G$ is non-contradictory, since its truth does not lead to a logical collapse within the system.
\end{itemize}
Therefore, $\Phi(G) = 1$, i.e., $G$ is ontologically admissible. However, $G$ lacks a truth value within the system, which might appear as indeterminacy.

\textbf{Counter-argument:}
$\Phi(\psi)$ does not assess the truth of $G$, only its admissibility. Since $G$ is admissible ($\Phi(G) = 1$), the indeterminacy of truth does not affect its ontological status. Moreover, $\Phi(\psi)$ can be applied at the meta-level to evaluate the admissibility of the system that contains $G$.

Gödel’s incompleteness does not create irresolvable indeterminacy for $\Phi(\psi)$, as it evaluates admissibility, not provability. Not a counterexample.

\textbf{Conclusion:} The CVB model is not only compatible with Gödel but extends his result, interpreting undecidability as the ontological basis of freedom. The formula
\[
\psi \text{ is admissible} \land \psi \text{ is unprovable} \Rightarrow \psi = \text{free future becoming}
\]
is valid and supported by axioms [5], [13], [23].
\end{quote}

\subsubsection*{🔹 Example of an admissible but unprovable statement (in terms of $\Phi(\psi)$)}

\subsubsection*{🔹 1. Key Example}

\begin{quote}
\textbf{An act of free choice by the Guest (individual):} Before the choice is actualized in the Present, it cannot be proven. However, it is admissible — it is logically possible within the Field of the Possible. This means the statement about the choice has the status:

\[
\Phi(\psi_{\text{forthcoming choice}}) = 1, \quad \text{but } \text{provability}(\psi) = ?
\]
\end{quote}

\subsubsection*{🔹 2. Classification by CVB}

\begin{tabular}{|l|l|}
\hline
\textbf{Criterion} & \textbf{Property of the Statement $\psi$} \\
\hline
Distinguishability & Yes — the choice is distinguishable over time \\
Admissibility $\Phi(\psi)$ & Yes — does not violate axioms and is within $V$ \\
Provability & No — $\psi$ is unprovable before its act of becoming \\
Type (from [4]) & Non-Permanent Possible (NV), or Permanent Possible (PV) in the limit \\
\hline
\end{tabular}

\subsubsection*{🔹 3. Axiological Grounding}

\begin{itemize}
\item \textbf{[5]} Not all that is possible becomes real: $\therefore$ Some admissible statements may never become actual (i.e., actions).
\item \textbf{[13]} The CVB does not expand: $\therefore$ Each act of becoming fixes one of many possible paths, while the others remain unprovable.
\item \textbf{[3]} Only the Possible exists: $\therefore$ All statements located within the Field of the Possible ($V$), even if unrealized, retain ontological validity.
\end{itemize}

\subsubsection*{🔹 4. Connection to Gödel}

Gödel showed that in any consistent formal system, there exist true statements that are unprovable.

\begin{quote}
\textbf{CVB refines this:}
\begin{itemize}
\item What is true but unprovable is \textbf{admissible} but \textbf{not yet actualized}.
\item It remains possible, but does not become real until a volitional act occurs.
\end{itemize}

\[
\psi \text{ is admissible} \land \psi \text{ is unprovable} \Rightarrow \psi = \text{free future becoming}
\]
\end{quote}

\subsubsection*{🔹 5. Practical Consequence: Freedom and Indeterminacy}

If all choices were provable in advance, \textbf{freedom would not exist}. Thanks to the fact that $\Phi(\psi)$ can be 1 even in the absence of provability, there exists:

\begin{itemize}
\item Genuine freedom of will
\item Ontologically guaranteed ``unpredictability''
\item The absence of total determinism
\end{itemize}

\subsubsection*{Conclusion}

Gödel’s incompleteness is a specific \textbf{formal symptom} of the same \textbf{ontological truth} articulated by the CVB model:

\begin{quote}
``\textbf{Truth $\neq$ Omniscient Knowledge}''

``\textbf{Freedom of choice requires unprovable possibilities}''
\end{quote}

Absolute knowledge of everything is impossible — not because we are limited, but because the \textbf{Ontology of Distinguishability does not permit} the realization of all possible becomings simultaneously. This is the very foundation of \textbf{authentic freedom and volition}.


\subsection*{2.4 Self-Referential Meta-Paradox for $\Phi(\psi)$}

\textbf{Description:}  
Consider the statement $M$: ``$\Phi(M) = 0$''. This statement asserts its own inadmissibility.  
If $\Phi(M) = 1$, then $M$ is admissible, but it claims $\Phi(M) = 0$, which is contradictory.  
If $\Phi(M) = 0$, then $M$ is inadmissible, but its assertion is true, creating indeterminacy.

\textbf{Analysis with $\Phi(\psi)$:}  
$M$ is self-referential, like the Liar, but refers to the meta-function $\Phi(\psi)$.

\begin{itemize}
\item \textbf{Distinguishability check:} $M$ refers to the result of applying $\Phi(\psi)$, but $\Phi(\psi)$ is a filter depending on the ontological field $V$. If $V$ is well-defined, then $M$ is either distinguishable or not.
\item \textbf{Non-contradiction check:}  
Assume $\Phi(M) = 1$. Then $M$ is admissible, but asserts $\Phi(M) = 0$ — contradiction.  
Therefore, $\Phi(M) \neq 1$. Hence, $\Phi(M) = 0$, and $M$ is inadmissible, which matches its assertion.
\end{itemize}

\textbf{Result:} $\Phi(M) = 0$, as $M$ is indistinguishable or contradictory (analogous to the Liar).

\textbf{Conclusion:} This meta-paradox is resolved, since $\Phi(\psi)$ consistently assigns $M$ the value 0. Self-reference does not create irresolvable indeterminacy. \textbf{Not a counterexample.}

\subsection*{2.5 Paradox of Quantum Indeterminacy}

\textbf{Description:}  
In quantum mechanics, the state of a system (e.g., the spin of an electron) can be indeterminate before measurement.  
Consider the statement $Q$: ``The electron’s spin is $|\uparrow\rangle$ before measurement.''  
This seems ontologically indistinguishable because a quantum superposition does not admit a definite state.

\textbf{Analysis with $\Phi(\psi)$:}

\begin{itemize}
\item \textbf{Distinguishability:} $Q$ is indistinguishable, since before measurement the spin lacks definite ontological status (it is in a superposition $|\uparrow\rangle + |\downarrow\rangle$).
\item \textbf{Non-contradiction:} $Q$ is not inherently contradictory, but its ontological status depends on measurement.
\end{itemize}

\textbf{Thus:} $\Phi(Q) = 0$, since $Q$ is indistinguishable in the ontological field $V$ before measurement.  
After measurement, $Q$ becomes distinguishable, and $\Phi(Q)$ may become 1.

\textbf{Counter-argument:} $\Phi(\psi)$ is context-sensitive to $V$. If $V$ includes quantum mechanics, then the superposition itself is ontologically distinguishable. For instance,

\[
\Phi(\text{``The electron is in a superposition''}) = 1
\]

Thus, the indeterminacy is resolved by shifting the level of description.

\textbf{Conclusion:} Quantum indeterminacy does not create an irresolvable paradox. $\Phi(\psi)$ adapts to the context of $V$. \textbf{Not a counterexample.}

\subsection*{2.6 Ontological Indeterminacy of the Field $V$}

\textbf{Description:}  
$\Phi(\psi)$ depends on the ontological field $V$, which defines what is distinguishable and non-contradictory.  
What if $V$ itself is indeterminate or contains internal contradictions?  
Consider statement $N$: ``The ontological field $V$ is contradictory.''

\textbf{Analysis with $\Phi(\psi)$:}

\begin{itemize}
\item If $N$ is true, then $V$ is contradictory and $\Phi(\psi)$ becomes meaningless — it cannot operate within a contradictory field.
\item If $N$ is false, then $V$ is non-contradictory, and $\Phi(N) = 0$, because $N$ asserts the inadmissible.
\item \textbf{Distinguishability of $N$:} $N$ is distinguishable as a statement about $V$.
\item \textbf{Non-contradiction:} If $V$ is non-contradictory, $N$ is false, and $\Phi(N) = 0$.  
If $V$ is contradictory, $\Phi(\psi)$ collapses, and the model fails.
\end{itemize}

\textbf{Counter-argument:}  
The CVB model assumes that $V$ is a fundamental ontological structure — by definition non-contradictory and distinguishable.  
If $V$ is contradictory, this is not a paradox \textbf{within} $\Phi(\psi)$, but a collapse \textbf{of the entire model}, beyond the scope of this test.

\textbf{Conclusion:} This case does not present irresolvable indeterminacy, because $\Phi(\psi)$ is based on the axiomatically defined consistency of $V$. \textbf{Not a counterexample.}

\subsection*{2.7 Paradox of Infinite Regress}

\textbf{Description:}  
Consider statement $R$: ``For any statement $\psi$, $\Phi(\psi)$ is determined through another statement $\psi'$, which also requires $\Phi(\psi')$, and so on.''  
This may lead to infinite regress, challenging the computability of $\Phi(\psi)$.

\textbf{Analysis with $\Phi(\psi)$:}

\begin{itemize}
\item \textbf{Distinguishability of $R$:} $R$ is distinguishable as a statement about the process of applying $\Phi(\psi)$.
\item \textbf{Non-contradiction:} If $R$ is true, then $\Phi(\psi)$ is uncomputable due to infinite regress.  
However, $\Phi(\psi)$ is a meta-function that does not require infinite evaluation — it assesses $\psi$ directly within the context of $V$.
\end{itemize}

\textbf{Example:} For simple statements ($\psi = ``2 + 2 = 4''$),

\[
\Phi(\psi) = 1
\]

without any regress. For complex statements, $\Phi(\psi)$ can analyze $\psi$ structurally without invoking infinite recursion, as $V$ defines \textbf{finite} criteria for distinguishability.

\textbf{Therefore:} $\Phi(R) = 0$, since $R$ asserts an inadmissible infinite regress, which does not reflect the operation of $\Phi(\psi)$.

\textbf{Conclusion:} Infinite regress does not create irresolvable indeterminacy. $\Phi(\psi)$ operates at a \textbf{finite level} of ontological evaluation. \textbf{Not a counterexample.}

\subsection*{3. Attempt to Construct a Hypothetical Counterexample}

To pose a genuine challenge to $\Phi(\psi)$, we attempt to construct a hypothetical paradox that may bypass the filter of distinguishability and non-contradiction.

\textbf{Hypothetical Paradox: ``Super-Paradox'' $S$}

\begin{quote}
\textbf{Formulation:} Let $S$ be the statement: ``$\Phi(S)$ is undefined in any ontological field $V$.''
This means $S$ asserts that $\Phi(\psi)$ cannot assign it either 1 or 0 in any $V$.
\end{quote}

\textbf{Analysis:}

\begin{itemize}
\item \textbf{Distinguishability:}  
$S$ is clearly formulated, but its content denies the applicability of $\Phi(\psi)$.  
If $S$ is distinguishable, then $\Phi(S)$ must be either 1 or 0.  
If $S$ is indistinguishable, then $\Phi(S) = 0$.

\item \textbf{Non-contradiction:}  
Assume $\Phi(S) = 1$. Then $S$ is admissible, and its claim ``$\Phi(S)$ is undefined'' is false, contradicting its admissibility.  
Therefore, $\Phi(S) \neq 1$.

Assume $\Phi(S) = 0$. Then $S$ is inadmissible, and its claim ``$\Phi(S)$ is undefined'' is true, because $\Phi(S)$ indeed does not assign 1.  
However, the truth of an inadmissible statement creates ontological indeterminacy.

Assume $\Phi(S)$ is undefined. Then $S$ is true, but this contradicts the very definition of $\Phi(\psi)$, which assigns either 0 or 1 to any $\psi$.
\end{itemize}

\textbf{Counter-argument:}  
This paradox resembles the Liar or the meta-paradox $M$ but attempts to attack the very possibility of $\Phi(\psi)$ being defined.  
However, the Model of Conscious Volitional Becoming (CVB) assumes that $\Phi(\psi)$ is \textbf{always defined} for any statement $\psi$ in $V$, as $V$ is the \textbf{universal ontological field}.  
If $S$ asserts that $\Phi(S)$ is undefined, it contradicts the axiom that $V$ encompasses all possible statements.  
Thus, $S$ is either indistinguishable or contradictory, and \textbf{$\Phi(S) = 0$}.

\textbf{Conclusion:}  
Even this hypothetical super-paradox is resolved through the filter $\Phi(\psi)$, as it falls into the category of inadmissible self-referential statements. \textbf{Not a counterexample.}

\subsection*{4. Discussion and Conclusion}

After analyzing classical, meta-, and hypothetical paradoxes, \textbf{no irresolvable indeterminacy} was found that could bypass the meta-function of admissibility $\Phi(\psi)$.

\textbf{Key Reasons:}

\begin{itemize}
\item \textbf{Universality of $\Phi(\psi)$:}  
$\Phi(\psi)$ operates at the meta-level, evaluating not truth but \textbf{ontological admissibility}, allowing it to evade traps of self-reference and indeterminacy.

\item \textbf{Ontological Field $V$:}  
The assumption that $V$ is consistent and universal renders $\Phi(\psi)$ resilient to paradoxes, as \textbf{any contradiction or indistinguishability is excluded} ($\Phi(\psi) = 0$).

\item \textbf{Philosophical Basis of CVB:}  
The CVB model, in which only what is distinguishable and non-contradictory can exist, \textbf{excludes the possibility} of ontologically irresolvable paradoxes.
\end{itemize}

\textbf{Limitations of the Analysis:}

\begin{itemize}
\item The inability to propose a counterexample may stem from limitations of knowledge or imagination, not from absolute universality of $\Phi(\psi)$.
\item If a paradox exists that attacks the very definition of $V$ (e.g., $V$ is not universal or contains hidden contradictions), it could undermine $\Phi(\psi)$.  
However, such cases fall outside the CVB model, which \textbf{axiomatically presupposes} the consistency of $V$.
\end{itemize}

\textbf{Conclusion:}  
Based on the conducted analysis, \textbf{no paradox or indeterminacy was found} that $\Phi(\psi)$ could not resolve.  
This confirms the \textbf{unique universality} of $\Phi(\psi)$ and the CVB model within the bounds of their own axioms.  
$\Phi(\psi)$ effectively eliminates all examined paradoxes, assigning them a value of \textbf{0 (inadmissible)} or \textbf{1 (admissible)} with no residual indeterminacy.  
Therefore, \textbf{within the CVB framework}, $\Phi(\psi)$ demonstrates the capacity to be a \textbf{universal ontological filter}, resolving all proposed paradoxes.


\section*{[11.1.1.2] Confirmation of the Core Trilemma via $\Phi(\psi)$}

Each of the core axioms (\text{[1]}, \text{[2]}, \text{[3]}) is validated by the admissibility filter $\Phi(\psi)$:

\begin{itemize}
\item Absolute Nothingness $\rightarrow \Phi = 0$
\item Absolute Everythingness $\rightarrow \Phi = 0$
\item The Possible $\rightarrow \Phi = 1$
\end{itemize}

This renders the trilemma a \textbf{non-contradictory} and \textbf{formalizable} foundation for the ontology of Conscious Volitional Becoming (CVB).

\subsubsection*{General Verification Principle}

The meta-function $\Phi(\psi)$ establishes a filter through which only that which is \textbf{distinguishable} and \textbf{non-contradictory} may pass.

This means that all statements involving contradiction, paradox, or self-negation are automatically excluded as \textbf{ontologically impossible} ($\Phi = 0$).

We now verify each core axiom using this filter.

\subsubsection*{🔹 Verification of Axiom \text{[1]} — Absolute Nothingness is Impossible}

\textbf{Paradoxical nature:}

\begin{quote}
The statement ``Absolute Nothingness exists'' requires that something be distinguished as ``nothing,'' which already \textbf{violates non-contradiction}.
\end{quote}

\textbf{Through $\Phi(\psi)$:}

\[
\psi = \text{``Absolute Nothingness exists''} \quad \Rightarrow \quad \Phi(\psi) = 0
\]

\textbf{Consequence:} This statement is impossible. Matches Axiom \text{[1]}.

\textbf{🔹 Consistent and confirmed via $\Phi(\psi)$.}

\subsubsection*{🔹 Verification of Axiom \text{[2]} — Absolute Everythingness is Impossible}

\textbf{Paradoxical nature:}

\begin{quote}
If everything is true, then the negation of everything must also be true $\Rightarrow$ contradiction $\Rightarrow$ logical explosion.
\end{quote}

\textbf{Through $\Phi(\psi)$:}

\[
\psi = \text{``Absolutely everything is true''}
\quad \Rightarrow \quad 
\psi \wedge \neg \psi 
\quad \Rightarrow \quad 
\Phi(\psi) = 0
\]

\textbf{Consequence:} The statement is contradictory $\Rightarrow$ impossible.

\textbf{🔹 Consistent and confirmed via $\Phi(\psi)$.}

\subsubsection*{🔹 Verification of Axiom \text{[3]} — Only the Possible Exists}

\textbf{Substantive essence:}

\begin{quote}
If a statement $\psi$ is indistinguishable or contradictory, it cannot exist. 
$\Phi(\psi)$ determines existence:
\end{quote}

\[
\text{If } \Phi(\psi) = 0 \quad \Rightarrow \quad \psi \text{ is impossible} \quad \Rightarrow \quad \psi \notin \text{Being}
\]

Therefore, everything that exists $x$ satisfies 

\[
\Phi(x) = 1
\]

This is precisely the formulation of Axiom \text{[3]}.

\textbf{🔹 Consistent and fully formalized by $\Phi(\psi)$.}

\subsubsection*{🔹 Conclusion}

The meta-function $\Phi(\psi)$:

\begin{itemize}
\item logically confirms and refines all three core axioms,
\item provides the mechanism to distinguish between the possible, impossible, and paradoxical,
\item eliminates all logical loopholes (Liar, Omnitruth, Nothingness).
\end{itemize}

Thus, the \textbf{Core Trilemma} is not merely compatible with $\Phi(\psi)$, but fundamentally \textbf{relies on it} as its \textbf{formal ontological mechanism}.

\section*{[11.1.1.3] Confirmation of Axiom \text{[4]} — The Field of the Possible and Its Boundaries}

\subsubsection*{🔹 1. Compatibility with $\Phi(\psi)$}

The meta-function $\Phi(\psi)$ determines whether a given statement $\psi$ is admissible, if and only if:

\begin{itemize}
\item it is \textbf{distinguishable} ($R(x, t)$ is defined)
\item it is \textbf{non-contradictory} ($\neg(\psi \wedge \neg\psi)$)
\end{itemize}

Axiom \text{[4]} systematically classifies the full set of $\psi$ within the field $V$ into four \textbf{ontological categories} based on the temporal stability of distinguishability.

This is a direct decomposition of the set of statements relative to the filter $\Phi(\psi)$ and the field $V$. Each category corresponds to a specific behavior of $\Phi(\psi)$:

\begin{itemize}
\item \text{[4.1]} $PN(x): \forall t: x \notin V \quad \Rightarrow \quad \forall t:\, \Phi(x) = 0$
\item \text{[4.4]} $PV(x): \forall t: x \in V \wedge R(x,t) \quad \Rightarrow \quad \forall t:\, \Phi(x) = 1$
\item \text{[4.2]}, \text{[4.3]} — transitional cases, where $\Phi(x)$ depends on temporal context and degree of distinguishability.
\end{itemize}

Thus, Axiom \text{[4]} clarifies the \textbf{temporal behavior} of $\Phi(\psi)$ across the set $V$ in terms of stability, without exceeding its definitional bounds.

\subsubsection*{🔹 2. Strengthening $\Phi(\psi)$ via the Boundaries of Distinguishability $\partial V\downarrow$ and $\partial V\uparrow$}

\begin{itemize}
\item $\partial V\downarrow$ — where distinguishability disappears
\item $\partial V\uparrow$ — where distinguishability becomes saturated (overload of form; boundary of further differentiation)
\end{itemize}

These are not statements but \textbf{boundary conditions} of the ontological field.

They are \textbf{not directly evaluated} by $\Phi(\psi)$, but define \textbf{where} the application of $\Phi(\psi)$ \textbf{ceases to be meaningful} (beyond $\partial V$).

$\Rightarrow$ This is a \textbf{logically and ontologically consistent} refinement of the mechanism of $\Phi(\psi)$.

\subsubsection*{🔹 3. Embedded Examples Reflect the Behavior of $\Phi(\psi)$}

\begin{itemize}
\item ``A triangle with four sides'' $\Rightarrow$ not distinguishable $\Rightarrow \Phi = 0 \Rightarrow PN$
\item ``Human flight'' $\Rightarrow$ initially NN, then NV $\Rightarrow$ now PV $\Rightarrow$ $\Phi$ transitions from $0$ to $1$
\item ``The first murder'' $\Rightarrow$ occurred once $\Rightarrow$ then became impossible $\Rightarrow NN \to PN$
\end{itemize}

These examples not only \textbf{do not contradict} $\Phi(\psi)$, but actively confirm its \textbf{temporal applicability} — as a \textbf{dynamic evaluative structure}.

\subsubsection*{🔹 4. No Paradoxes or Meta-Contradictions Identified}

\begin{itemize}
\item The four categories are logically non-overlapping (temporal stability makes them \textbf{mutually exclusive})
\item The mechanism of distinguishability $R(x, t)$, together with $\partial V$, is \textbf{fully consistent} with the condition of applicability for $\Phi(\psi)$
\item Every statement under $\Phi(\psi)$ falls into \textbf{exactly one} of the four categories $\Rightarrow$ no contradictions arise
\end{itemize}

\subsubsection*{🔹 Conclusion}

Axiom \text{[4]} — The Field of the Possible:

\begin{itemize}
\item constitutes a \textbf{topological ontological partition} of the set of statements $\psi$,
\item is \textbf{fully compatible and coherent} with the meta-function $\Phi(\psi)$,
\item contains \textbf{no internal paradoxes or uncertainties}, which are instead eliminated by the boundaries of distinguishability.
\end{itemize}

\section*{[11.1.1.4] Ontological Verification of the Model of Conscious Volitional Becoming (CVB)}

\textbf{Ontological Universality of the Model: Exhaustiveness of the Core, the Field of the Possible, and the Function $\Phi(\psi)$}

\subsubsection*{1. Justification of the Section}

This section completes the \textbf{logical-ontological verification} of the Model of Conscious Volitional Becoming (CVB), demonstrating that its three essential components — the \textbf{Core}, the \textbf{Field of the Possible}, and the \textbf{meta-function of admissibility $\Phi(\psi)$} — are not only mutually coherent, but together form a \textbf{comprehensive and universal ontological foundation} in which:

\begin{itemize}
\item everything logically possible is admitted,
\item everything impossible is excluded, and
\item every distinguishable entity of Being receives a defined status — either of \textbf{existence} or \textbf{rejection}.
\end{itemize}

\subsubsection*{2. Formulation of Universality}

The \textbf{ontological universality} of the CVB model is defined as:

\begin{quote}
The coincidence of the boundaries of admissibility, distinguishability, and non-contradiction with the limits of ontological possibility.
\end{quote}

In other words, everything that is admissible according to $\Phi(\psi)$, included within the Field of the Possible $V$, and consistent with the Core axioms — and only that — is ontologically valid.

Thus:

\[
\Phi(\psi) = 1 \;\;\Longleftrightarrow\;\; \psi \in V \;\;\Longleftrightarrow\;\; \psi \text{ is consistent with the Core Axioms}
\]

\subsubsection*{3. Exhaustiveness: Logical-Ontological Closure}

The CVB model:

\begin{itemize}
\item Ensures the \textbf{ontological exclusion} of the Absolute Nothing and the Absolute Everything, eliminating extreme sources of paradox;
\item Establishes the \textbf{Field of the Possible} as the generalized space of reality, distinguishability, and becoming;
\item Introduces $\Phi(\psi)$ as an \textbf{operationalized filter of admissibility}, applicable to any statement $\psi$.
\end{itemize}

Hence, the model is \textbf{ontologically closed}:

\begin{itemize}
\item \textbf{Nothing beyond the distinguishable} is included (everything else is excluded),
\item \textbf{Nothing within it leads to paradox} (as anything paradoxical is filtered out by $\Phi(\psi)$).
\end{itemize}

\subsubsection*{4. Verification of Comprehensive Coverage}

The model successfully encompasses:

\begin{itemize}
\item \textbf{All types of logical statements}: true, false, paradoxical, and contradictory;
\item \textbf{All types of ontological entities}: actual, possible, and impossible;
\item \textbf{All zones of the Field of the Possible}: stable, unstable, boundary and central;
\item \textbf{All levels of logical evaluation}: propositional, admissibility, ontological realization.
\end{itemize}

This renders the model \textbf{ontologically admissibility-complete}: everything distinguishable without contradiction is representable within it.

\subsubsection*{5. Ontological Conclusion}

The CVB model — including the function $\Phi(\psi)$ and the structure of the Field of the Possible — constitutes a \textbf{comprehensive}, \textbf{non-contradictory}, and \textbf{closed} description of ontological realizability.

It is:

\begin{itemize}
\item \textbf{Universal} within the bounds of distinguishability,
\item \textbf{Complete} in its coverage of all admissible forms,
\item \textbf{Immune to ontological ``gaps''} — no statement remains unclassified as either admissible or inadmissible.
\end{itemize}

\subsubsection*{6. Methodological Clarification}

The \textbf{only foundational assumption} of the model is the \textbf{axiomatic non-contradiction} of the Field of the Possible ($V$).

Any threat to its universality could only emerge \textbf{from outside the model}, by challenging its ontological foundation (e.g., the logic of distinguishability or the existence of $V$ as a consistent domain). However, such attacks undermine the very structure of rational thought and render meaningful inquiry itself impossible.

\subsubsection*{Final Universality Equation}

\[
\forall \psi:\quad \Phi(\psi) = 1 \;\;\Longleftrightarrow\;\; \psi \in V \;\;\Longleftrightarrow\;\; \psi \text{ is realizable within the ontological structure of CVB}
\]

Thus, the CVB model is a \textbf{universal ontology of the distinguishable}, in which \textbf{all and only truth} is ontologically realized.

\section*{[11.1.1.5] Ontological Assessment of Universality: Comparative Analysis of Models}

\subsubsection*{🔹 Introduction}

Throughout the history of philosophy and science, numerous attempts have been made to construct comprehensive theories capable of describing reality in its totality — unbounded by perception, culture, embodiment, or biological limitations of the observer. However, none of these frameworks has successfully fulfilled all three of the following criteria simultaneously:

\begin{itemize}
\item logical consistency,
\item formalizability of assertion admissibility,
\item and ontological comprehensiveness independent of human nature.
\end{itemize}

The Model of Conscious Volitional Becoming (CVB) proposes a solution to these limitations through a universal meta-function of admissibility: $\Phi(\psi)$, which operates within the \textbf{Field of the Possible} ($V$) and is grounded in an axiomatic structure that eliminates paradoxes and ensures distinguishability.

\subsubsection*{🔹 I. Philosophical Models}

\textbf{Platonism (World of Ideas)}

\begin{itemize}
\item \textit{Idea:} Essential being exists as non-spatial ideal forms.
\item \textit{Limitation:} Application to empirical entities cannot be formalized.
\end{itemize}

\textbf{Aristotelianism (Substance and Causality)}

\begin{itemize}
\item \textit{Idea:} Reality is explained through form, matter, telos, and cause.
\item \textit{Limitation:} Rigidly bound to corporeality and linear hierarchies.
\end{itemize}

\textbf{Spinoza (Single Substance)}

\begin{itemize}
\item \textit{Idea:} All is manifestation of a singular being (Deus sive Natura).
\item \textit{Limitation:} Eliminates will and choice, excluding free distinguishable action.
\end{itemize}

\textbf{Hegel (Dialectic of Spirit)}

\begin{itemize}
\item \textit{Idea:} Reality unfolds through resolution of contradictions in thought.
\item \textit{Limitation:} Entirely dependent on the structure of the human mind.
\end{itemize}

\textbf{Philosophies of Will (Schopenhauer, Heidegger, Nietzsche)}

\begin{itemize}
\item \textit{Idea:} Reality is rooted in will, anxiety, and becoming rather than logic.
\item \textit{Limitation:} Lack of strict formalizability.
\end{itemize}

\subsubsection*{🔹 II. Scientific Universalisms}

\textbf{General Relativity + Quantum Mechanics}

\begin{itemize}
\item \textit{Strength:} Accurately describe physical phenomena.
\item \textit{Limitation:} Do not address consciousness, morality, distinguishability, or volition.
\end{itemize}

\textbf{Theories of Everything (ToE), String Theory, Quantum Gravity}

\begin{itemize}
\item \textit{Strength:} Aim to unify all physical interactions.
\item \textit{Limitation:} Do not include ontological structures or meta-logic.
\end{itemize}

\textbf{Information Ontologies (Tegmark, Wheeler)}

\begin{itemize}
\item \textit{Idea:} Reality is fundamentally informational.
\item \textit{Limitation:} Do not define who/what distinguishes information, or its admissibility.
\end{itemize}

\subsubsection*{🔹 III. Formal Systems}

\textbf{Logicism, Formalism (Frege, Russell, Hilbert)}

\begin{itemize}
\item \textit{Goal:} Reduce all knowledge to logic and axioms.
\item \textit{Limitation:} Gödel’s theorems revealed: systems are either incomplete or inconsistent.
\end{itemize}

\textbf{Category and Topos Theories}

\begin{itemize}
\item \textit{Strength:} Provide universal formal languages.
\item \textit{Limitation:} Do not address non-physical being or freedom of choice.
\end{itemize}

\subsubsection*{🔹 IV. Integral Approaches}

\textbf{Integral Theory (Wilber)}

\begin{itemize}
\item \textit{Goal:} Integrate science, spirituality, culture, and development.
\item \textit{Limitation:} Lacks strict ontological formalization.
\end{itemize}

\textbf{Systemic Meta-Theories (Luhmann, Brier, others)}

\begin{itemize}
\item \textit{Approach:} Based on observation and communication.
\item \textit{Limitation:} Still anthropocentric — rooted in observer-based models.
\end{itemize}

\subsubsection*{🔹 V. Advantage of the CVB Model}

The CVB model:

\begin{itemize}
\item eliminates anthropocentrism — the subject of distinction requires neither body nor language;
\item provides a universal admissibility criterion — $\Phi(\psi)$, which tests the realizability of any proposition;
\item formalizes the boundaries of distinguishability — through the \textbf{Field of the Possible} ($V$) and its boundary conditions $\partial V\downarrow$ and $\partial V\uparrow$;
\item ensures logical consistency — based on the foundational axioms \text{[1]} and \text{[2]}.
\end{itemize}

Unlike all previously examined systems, the CVB model does not merely describe reality — it filters out the impossible and guarantees \textbf{ontological universality}, without relying on faith, tradition, or the physical form of the observer.

\subsubsection*{Conclusion}

Section \text{[11.1.1.5]} completes the ontological verification of the CVB model, demonstrating that it is not a partial philosophical or physical theory, but a \textbf{formalizable ontological system} capable of replacing all prior universalist attempts.

\section*{[11.2] Truth and Falsehood}

\subsubsection*{🔹 1. Brief Statement}

\textbf{Truth} is a non-contradictory and distinguishable form of Becoming that is consistent with the fundamental axioms of Being.

\textbf{Falsehood} is a contradictory or impossible state, inadmissible within the structure of Being.

\subsubsection*{🔹 2. Interpretation and Significance}

In the Model of Conscious Volitional Becoming (CVB), \textbf{Truth} is not subjective opinion or consensus, but a \textbf{logically necessary element} of the structure of reality.

It ensures \textbf{distinguishability}, \textbf{stability}, and \textbf{non-contradiction} of Becoming within the Ontological Field of the Possible.

Truth is essential for maintaining Memory (\text{[10.3]}), performing Verification (\text{[11.6]}), and distinguishing Good and Evil (\text{[11.3]}).

Falsehood is defined as a form that leads to paradoxes, violates the core axioms (impossibility of Absolute Nothing, impossibility of Absolute Everything, necessity of the Possible), and must be removed from stable Becoming.

For the \textbf{Permanent Possible}, Truth is the \textbf{absolute criterion} of distinguishability.

For active forms of the \textbf{Non-Permanent Possible}, Truth appears as a \textbf{limited domain of admissible truths} in accordance with their level of distinguishability.

\subsubsection*{🔹 3. Formulas}

\[
\text{Truth}(x) \iff \text{Becoming}(x) \land \Phi(x) = 1
\]

\[
x \rightarrow (A \land \neg A) \Rightarrow \Phi(x) = 0 \Rightarrow \text{Truth}(x) = 0
\]

\[
\Phi(x) = 1 \land x \in \text{Becoming} \Rightarrow \text{Truth}(x) = 1
\]

\begin{quote}
Truth is possible if and only if a statement is part of Becoming and admissible within the logical-ontological field by the criterion of the meta-function $\Phi$ — that is, non-contradictory, distinguishable, and ontologically realizable.

Thus, Truth is functionally derivable and formally verifiable within the CVB model.
\end{quote}

\section*{[11.2.1] Truth}

\textbf{Statement:}

Truth is an ontologically admissible and non-contradictory statement, distinguishable in the Field of the Possible and consistent with the fundamental axioms.

It is not subject to interpretation, does not depend on consensus, and is defined solely from the standpoint of the Permanent Possible, which possesses complete Memory, Plan, and Verification.

Truth is the \textbf{preserved form of stable Becoming}.

\section*{[11.2.2] Falsehood}

\textbf{Statement:}

Falsehood is a statement that is either \textbf{not distinguishable}, \textbf{internally contradictory}, or \textbf{violates the axioms} of Becoming.

Such a statement \textbf{cannot be retained in Memory}, is \textbf{rejected by the meta-function $\Phi$}, and is excluded from the logical structure of the Possible.

Falsehood \textbf{can be expressed}, but does \textbf{not exist ontologically} — it is either \textbf{forgotten} or remains as a precedent for \textbf{rejecting Truth}.

\subsubsection*{🔹 4. Logical Justification}

\begin{itemize}
\item \textbf{From Axioms \text{[1]}–\text{[3]}:} 
Absolute Nothing and Absolute Everything are impossible; therefore, Truth exists as a distinguishable and non-contradictory form.

\item \textbf{Reductio ad absurdum:}
If there were no objective Truth, then there would be no criterion of distinguishability (\text{[9]}), Memory would become chaotic (\text{[10.3]}), and Verification would be impossible (\text{[11.6]}).

\item \textbf{Modus ponens:}
If $xxx$ is consistent with the axioms, distinguishable, and non-contradictory, then $xxx$ is true.
\end{itemize}

\subsubsection*{🔹 5. Responses to Objections}

\textbf{Pragmatism, Constructivism:}

``Truth is a result of agreement.''

\textbf{Response:} Agreement is unstable — it changes over time. In CVB, \textbf{Truth is not dependent on consensus} but follows from the \textbf{structure of Being’s distinguishability}.

\textbf{Coherentism:}

``Internal consistency of the system is sufficient.''

\textbf{Response:} Coherence is necessary but not sufficient. \textbf{Multiple coherent systems may conflict}; Truth is singular, as \textbf{Being is one}.

\textbf{Postmodernism, Relativism:}

``Truth is subjective.''

\textbf{Response:} Subjective truth is possible \textbf{only if the subject coincides with CVB}. For all other forms, \textbf{Truth is objective}, or distinguishability collapses.

\textbf{Philosophy of Freedom:}

``Truth limits thought.''

\textbf{Response:} No. \textbf{Truth enables distinguishability}; denying it destroys logical freedom and leads to a \textbf{collapse of discernment}.

\textbf{Classical logic:}

``There may be multiple truths.''

\textbf{Response:} \textbf{Only one Truth} can be \textbf{non-contradictory} and \textbf{support stable Becoming}.

\subsubsection*{🔹 6. Clarification of Terms}

\begin{itemize}
\item \textbf{Truth}: A logically admissible, distinguishable, and non-contradictory form of Becoming, aligned with the fundamental axioms.
\item \textbf{Falsehood}: A statement or form that leads to \textbf{contradiction}, \textbf{impossibility}, or \textbf{axiomatic inconsistency}.
\item \textbf{Paradox}: A statement that results in $A \land \neg A$ or self-referential negation (see Liar paradox).
\end{itemize}

\subsubsection*{🔹 7. Understanding for All (Popular Version)}

\textbf{Truth} is what does not contradict itself and aligns with the fundamental laws of the world.

\textbf{Falsehood} is what contains contradictions, errors, or is simply impossible.

But how can we know what is truly the Truth?

Truth cannot be invented or established by a vote.

Truth is what can be seen from the perspective of the One who never changes, never makes mistakes, and knows everything that can possibly be known.

That is why His knowledge of Truth is complete.

This perspective belongs to the \textbf{Permanent Possible} — 
\textbf{Conscious Volitional Becoming}.

If we look at the world honestly, striving to distinguish clearly and avoid confusion — we move closer to Truth.

But if we distort, confuse, or deceive — we move into Falsehood.

\subsubsection*{🔹 8. Empirical Examples}

\textbf{Logic:}

The \textbf{Liar paradox} (``I am lying'') — inadmissible in CVB, as it mixes levels of statement and evaluation.

\textbf{Mathematics:}

The \textbf{Russell paradox} (the set of all sets that do not contain themselves) — impossible due to contradiction.

\textbf{Everyday situation:}

A child says, ``I always lie'' — contradiction. In CVB, this \textbf{cannot be considered true}.

\textbf{Morality:}

``Do good'' — true \textbf{if not contradictory} to the fundamental principles.

\section*{[11.3] Good and Evil}

\subsubsection*{🔹 1. Brief Statement}

\textbf{Good} is a distinguishable, non-contradictory, and ontologically admissible volitional act.

\textbf{Evil} is an expression of will that violates truth, the boundaries of the Possible, or the logic of Becoming.

\subsubsection*{🔹 2. Interpretation and Significance}

In the Model of Conscious Volitional Becoming (CVB), \textbf{Good and Evil are not} subjective judgments, cultural norms, or emotional reactions.

They are \textbf{strictly derived from ontological distinctions} between the possible and the impossible, the true and the false, the sustainable and the destructive.

\textbf{Good} is an expression of will aligned with \textbf{Truth} and admissible under the \textbf{meta-function $\Phi(\psi)$}; it strengthens sustainable Becoming.

\textbf{Evil} is a form that is either \textbf{not distinguishable}, \textbf{contains contradiction}, or \textbf{violates the criteria of admissibility}.

Good can be preserved in \textbf{Memory},

Evil only as a precedent in the \textbf{Filter}.

This axiom is a logical continuation of:

\begin{itemize}
\item \text{[10.6]} Will
\item \text{[10.7]} Power
\item \text{[11.1]} Logic
\item \text{[11.2]} Truth
\end{itemize}

\subsubsection*{🔹 3. Formulas}

\[
\text{Good}(w) \iff \Phi(w) = 1 \land \text{Truth}(w)
\]

\[
\text{Evil}(w) \iff \Phi(w) = 0 \lor \neg \text{Truth}(w)
\]

\[
\Phi(w) = 1 \land \text{Truth}(w) = 1 \Rightarrow \text{Good}(w) = 1
\]

\[
\Phi(w) = 0 \lor \text{Truth}(w) = 0 \Rightarrow \text{Evil}(w) = 1
\]

\subsection*{[11.3.1] Good}

\textbf{Statement:}

Good is the conscious, non-contradictory realization of Will within the Field of the Possible, in accordance with Immutable Truth.

\subsection*{[11.3.2] Evil}

\textbf{Statement:}

Evil is the conscious volitional violation of the boundaries of the Possible, the abandonment of non-contradiction, and the disregard for Immutable Truth.

\subsection*{[11.3.3] Asymmetry of Good and Evil}

\textbf{Statement:}

Good is necessary for Sustainable Becoming and is realized in the \textbf{Permanent Possible} as an ontological norm.

Evil is an \textbf{admissible but not necessary} form of deviation for variable forms granted freedom of choice.

For the Permanent Possible, the \textbf{possibility of Evil is logically impossible}.

For the Non-Permanent Possible, the \textbf{choice between Good and Evil} becomes the key criterion for classifying a form (as sustainable or vanishing).

\subsubsection*{🔹 4. Logical Justification}

From \text{[11.2]} Truth and \text{[11.1.1]} the meta-function $\Phi$, it follows:

\begin{itemize}
\item Anything that \textbf{violates $\Phi(w)$} is inadmissible in the logical-ontological field $V$.
\item Anything not aligned with \textbf{Truth} also violates structural stability.
\end{itemize}

Therefore:

Only actions that are \textbf{both admissible under $\Phi$ and aligned with Truth} can be preserved.

All others are eliminated as forms of \textbf{Evil} (cf. \text{[5.1]}, \text{[10.3.3]}, \text{[11.10]}).

\subsubsection*{🔹 5. Responses to Objections}

\textbf{Objection 1: Relativist Ethics —} ``Good and Evil are cultural constructs.''

\textbf{Response:} Cultural norms shift, but the logical distinction between the admissible and the impossible is \textbf{universal}.

Good and Evil in CVB are derived from \textbf{logic}, not tradition.

\textbf{Objection 2: Existentialism —} ``Free will includes the right to choose Evil.''

\textbf{Response:} True, but \textbf{Evil is not preserved}. It is permitted \textbf{only until Judgment}, as a manifestation of distinguishability, but is removed as impossible.

\textbf{Objection 3: Pragmatism —} ``If Evil works, it must be admissible.''

\textbf{Response:} \textbf{Functionality $\neq$ ontological admissibility}.

Destructive actions may be temporarily effective but are logically incompatible with sustainable Becoming.

\textbf{Objection 4: Moral Skepticism —} ``Good and Evil cannot be precisely defined.''

\textbf{Response:} The meta-function $\Phi$ makes the distinction \textbf{testable}: admissibility is distinguishable.

This provides an \textbf{ontologically verifiable} criterion for Good.

\subsubsection*{🔹 6. Clarification of Terms}

\begin{itemize}
\item \textbf{Good(w)}: The result of a volitional act that is \textbf{distinguishable}, \textbf{non-contradictory}, and aligned with \textbf{Truth}.
\item \textbf{Evil(w)}: The result of a volitional act that \textbf{violates ontological criteria} (i.e., $\Phi(w)=0$ or $\neg \text{Truth}(w)$).
\item \textbf{Truth(w)}: See \text{[11.2]}; an act that \textbf{corresponds to Becoming} and contains \textbf{no paradox}.
\item \textbf{$\Phi(w)$}: See \text{[11.1.1]}; the \textbf{meta-function of admissibility}, determining the realizability of the statement $w$.
\end{itemize}

\subsubsection*{🔹 7. Understanding for All (Popular Version)}

Good is when someone acts consciously and freely within the boundaries of the Possible, without violating logic and while following the Truth.

Evil is when someone acts consciously and freely outside the boundaries of the Possible, violating logic and rejecting the Truth.

Why Good and Evil are not equal:

Good is what makes existence stable.

Evil is what destroys the stability of being.

Good does not need Evil to exist.

But Evil is permitted so that free choice is possible — this makes freedom real.

In the Variable Possible, the choice between Good and Evil reveals what will remain and what will vanish.

(All children are born with different tendencies, but the kind of person they become depends largely on their choices.)

For the Permanent Possible, doing evil is impossible — it is logically excluded.

(If the Permanent Possible could do evil, it could not be eternal and could not be the Source of all that exists.)

And this proves: living without evil is not only possible, but necessary for stable being.

\subsubsection*{🔹 8. Empirical Examples}

\begin{itemize}
\item \textbf{Logic:}\\
Paradoxical statement ``I am lying'' $\rightarrow$ \textbf{Evil} (self-negation)\\
Statement ``$2+2=4$'' $\rightarrow$ \textbf{Good} (distinguishable, non-contradictory)

\item \textbf{Science:}\\
A confirmed theory with verifiable results $\rightarrow$ \textbf{Good}\\
A falsified or forged theory $\rightarrow$ \textbf{Evil}

\item \textbf{Everyday Life:}\\
Helping a friend $\rightarrow$ \textbf{Good} (preserved in memory, strengthens connection)\\
Lying for gain $\rightarrow$ \textbf{Evil} (destroys trust, form is removed)

\item \textbf{AI / Models:}\\
Command that follows the rules $\rightarrow$ \textbf{Good}\\
Instruction leading to failure or paradox $\rightarrow$ \textbf{Evil}
\end{itemize}


\section*{[11.4] Morality}

\subsubsection*{🔹 1. Brief Statement}

\textbf{Morality} is the internal capacity of a distinguishable will to align its actions with \textbf{Truth} and the \textbf{boundaries of the Possible} \textit{prior to execution}.

It is logically necessary for the sustainable existence of a subject.

\subsubsection*{🔹 2. Interpretation and Significance}

Morality is not a set of external norms, but an \textbf{ontological structure of pre-action discernment}.

It is embedded within the mechanism of will as the ability to \textbf{pre-filter Evil} and choose \textbf{Good}.

Within the CVB framework, \textbf{morality is necessary} for the subject to sustainably remain within the \textbf{Field of the Possible} and be preserved in \textbf{Memory}.

Without moral filtering, \textbf{distinguishability collapses}, the subject loses ontological stability, and the result becomes either \textbf{Evil} or \textbf{ontological disappearance}.

\subsubsection*{🔹 3. Formulas}

\[
\text{Morality}(w) \iff w \in V \land \text{Truth}(w) \land \text{Pre-Action Agreement}(w)
\]

\[
\neg \text{Pre-Action Agreement}(w) \Rightarrow \neg \text{Morality}(w)
\]

\[
\Phi(w) = 1 \Rightarrow \text{Morality}(w)
\]

\subsubsection*{🔹 4. Logical Justification}

From:

\begin{itemize}
\item \text{[11.1]} Logic and the filter $\Phi(\psi)$: every statement must be tested for non-contradiction.
\item \text{[11.2]} Truth is a non-contradictory and distinguishable statement.
\item \text{[11.3]} Good is the realization of Truth within $V$.
\end{itemize}

It follows that \textbf{Morality}, as a \textbf{pre-action verification of Truth and Possibility}, is a \textbf{logically necessary filter} of a distinguishable will.

Without \textbf{Morality}, the subject may enact \textbf{Evil} (cf. \text{[11.3.2]}), violating stability and resulting in \textbf{removal from Memory}.

\subsubsection*{🔹 5. Responses to Objections}

\textbf{Objection (Existentialism):} ``Morality is subjective and culturally conditioned.''

\textbf{Response:} In CVB, morality is not a norm but a \textbf{logical structure}.

It is based on \textbf{objective distinguishability}, not opinion.

\textbf{Objection (Utilitarianism):} ``Morality is the result of utility.''

\textbf{Response:} Utilitarianism relies on outcomes but does not account for the nature of distinguishability.

In CVB, morality is based on \textbf{Truth}, not usefulness.

\textbf{Objection (Neuroethics):} ``Morality is a product of biological mechanisms.''

\textbf{Response:} Biology is a \textbf{manifestation}, not a foundation.

Ontological filtering of will \textbf{precedes biological realization} and arises from logical necessity.

\subsubsection*{🔹 6. Clarification of Terms}

\begin{itemize}
\item \textbf{Pre-Action Agreement} — the internal filtering of will by the subject \textbf{prior} to execution. It differs from external morality (social norms) as it is based on $\Phi(\psi)$ and \textbf{Truth}.
\item \textbf{Morality} — a \textbf{structural property} of the subject, distinguishing \textbf{Good and Evil} before action, not after. It does not equate to ``moralism'' or conventional ``ethics.''
\end{itemize}

\subsubsection*{🔹 7. Understanding for All (Popular Version)}

Morality is an \textbf{internal compass} that tells us what is right and wrong \textbf{before} we act.

It’s like a guide helping us choose the direction of our will.

\begin{quote}
\textbf{Simple analogy:}

Morality is what people often call \textbf{conscience}.

If your conscience is calm, you're likely to do \textbf{Good}.

If you're uneasy — you might be about to do \textbf{Evil}.
\end{quote}

But there's a catch:

\textbf{Conscience can be mistaken.}

Everyone has their own, and it's not always accurate — like a compass that can be demagnetized or off course.

Only for the \textbf{Permanent Possible}, morality is always perfectly aligned with \textbf{Immutable Truth} — it cannot fail.

For us, to keep our compass reliable, we must \textbf{calibrate conscience against the Truth of the Permanent Possible},

just like maps are aligned to true north.

Otherwise, we may unknowingly choose Evil.

\subsubsection*{🔹 8. Empirical Examples}

\begin{itemize}
\item \textbf{Logic:} \\
Morality is a mechanism for \textbf{preventing paradoxical action}, like a logical safeguard against the liar paradox.

\item \textbf{Science:} \\
AI programs implement filters to \textbf{evaluate the admissibility of actions before execution} — an analogue of morality.

\item \textbf{Everyday Life:} \\
When a child hesitates before hitting and thinks ``What will happen after?'' — that is \textbf{moral filtering}.
\end{itemize}

\section*{[11.5] Responsibility}

\subsubsection*{🔹 1. Brief Statement}

\textbf{Responsibility} is the logical and ontological link between \textbf{Volitional choice} and its \textbf{Consequence}.

It identifies who is the \textbf{Cause} of what occurs.

\subsubsection*{🔹 2. Interpretation and Significance}

Responsibility is not a social category, but an \textbf{internal connection} between a \textbf{conscious Volitional Cause} and the resulting \textbf{effect} (action or inaction).

Only a \textbf{distinguishable Will} can be held responsible.

Without Responsibility, the distinction between \textbf{Cause and Consequence} dissolves, moral evaluation loses meaning, and the \textbf{stability of Memory} collapses.

In the Model of Conscious Volitional Becoming (CVB), Responsibility is the primary mechanism preserving \textbf{ontological consistency} among \textbf{Becoming}, \textbf{Good}, and \textbf{Memory}.

\subsubsection*{🔹 3. Formulas}

\[
\text{Responsibility}(w, e) \iff \text{Will}(w) \land \text{Consequence}(e) \land \text{Cause}(e) = w
\]

\textbf{Logical evaluation:}

If $\exists e : e \in V \land \neg\exists w : \text{Cause}(e) = w$, then $e$ is a logically \textbf{uncaused} consequence, which contradicts \text{[9.1]} Distinction.

Therefore:

$\forall e \in V, \exists! w : \text{Responsibility}(w, e)$

\subsubsection*{🔹 4. Logical Justification}

From \text{[9.1]} \textbf{Distinction} and \text{[11.1.1]} the \textbf{meta-function of admissibility} $\Phi(\psi)$, it follows that every distinguishable action ($e \in V$) must have a clearly identifiable \textbf{Cause}.

If an action occurs but its \textbf{Cause} is not distinguishable, this \textbf{violates the logical structure} of the Field of the Possible.

Since \textbf{Will(w)} defines the direction of \textbf{Becoming}, its connection to any outcome via \textbf{causality} constitutes \textbf{Responsibility(w, e)}.

This link is necessary for preserving \textbf{ontological sequence} and evaluating \textbf{Good} and \textbf{Evil} (\text{[11.3]}).

\subsubsection*{🔹 5. Responses to Objections}

\textbf{Objection:} Does Responsibility contradict the premises of \textbf{philosophical determinism} (e.g. Laplace's Demon, classical mechanics)?

\textbf{Response:}

Responsibility in CVB is not based on physical predetermination, but on \textbf{ontological Will} as a distinguishable cause.

Even if behavior is externally predictable, it may arise from \textbf{internal volitional discernment}.

Prediction is irrelevant — what matters is \textbf{conscious causation}.

If the choice is \textbf{conscious} $\Rightarrow$ it is \textbf{responsible}.

\bigskip

\textbf{Objection:} Does \textbf{quantum randomness} invalidate Responsibility? (e.g. Copenhagen interpretation, uncertainty principle, Heisenberg)

\textbf{Response:}

Quantum uncertainty pertains to \textbf{physical noise}, not conscious choice.

CVB rejects random or meaningless actions ($\Phi(\psi) = 0$).

\textbf{Will does not select randomness} — it distinguishes the Possible.

Thus, even under quantum indeterminacy, the subject is responsible for \textbf{what is distinguishable}, not for background noise.

\bigskip

\textbf{Objection:} Can Responsibility exist in \textbf{collective actions}? (e.g. Karl Jaspers, Hannah Arendt, the problem of collective responsibility)

\textbf{Response:}

The model allows for \textbf{multiple causes}, but every \textbf{distinguishable consequence} must have at least one \textbf{non-contradictory Cause} — i.e., a bearer of Will.

Responsibility can be \textbf{shared}, but not \textbf{dissolved}: each subject of Will who made a \textbf{distinguishable contribution} carries their \textbf{portion of Responsibility}.

If the contribution is distinguishable $\Rightarrow$ it is verifiable $\Rightarrow$ it is responsible.

\subsubsection*{🔹 6. Clarification of Terms}

\begin{itemize}
\item \textbf{Responsibility($w, e$)} — logical pair between a subject of Will $w$ and a consequence $e$, where $w$ is the \textbf{Cause} of $e$.
\item \textbf{Cause($e$)} — that which \textbf{generated} the distinguishable consequence $e$.
\item \textbf{Consequence($e$)} — any change or result \textbf{distinguishable} in the Field of the Possible $V$.
\item \textbf{Will($w$)} — active form of distinguishable choice, as defined in \text{[10.6]}.
\end{itemize}

\subsubsection*{🔹 7. Understanding for All (Popular Version)}

\textbf{Responsibility} means:

If you did something (or didn’t do something) and it \textbf{led to a result}, and you were \textbf{able to understand and choose}, then you are the \textbf{Cause} — and therefore \textbf{responsible}.

That’s what makes you a \textbf{person}, not just a \textbf{coincidence}.

\subsubsection*{🔹 8. Empirical Examples}

\begin{itemize}
\item \textbf{Logic:} \\
In classical logic, $A \rightarrow B$ means $A$ is the \textbf{cause} of $B$.

If $A$ is distinguishable, \textbf{Responsibility is assigned}.
\item \textbf{Science:}
\begin{itemize}
\item \textbf{Mechanics}: Force causes acceleration. If the source of the force is known $\Rightarrow$ Responsibility is identifiable.
\item \textbf{Computer science}: Audit logs track which function was called by whom — analogous to Responsibility.
\end{itemize}
\item \textbf{Everyday Life:}
\begin{itemize}
\item A child who breaks a vase knowing the rule and making a choice $\Rightarrow$ is \textbf{responsible}.
\item A person who \textbf{ignores a distress signal} commits an omission — still a choice, thus responsible.
\end{itemize}
\end{itemize}


\section*{[11.6] Verification (Judgment)}

\subsubsection*{🔹 1. Brief Statement}

\textbf{Verification} is the non-contradictory process of evaluating that which is distinguished for its conformity to \textbf{Truth}, \textbf{Good}, and \textbf{Morality}.

It applies to all that exists — including \textbf{Memory} and \textbf{Plans}.

\subsubsection*{🔹 2. Interpretation and Significance}

Verification (or Judgment) is not an external sanction, but an \textbf{internal logical mechanism} determining whether a distinguished form can be preserved as stable.

Every act of Will, outcome, memory, or plan is subjected to the \textbf{meta-function} $\Phi(\psi)$ to assess admissibility.

In the Model of Conscious Volitional Becoming (CVB), Verification is the mechanism that \textbf{separates the sustainable from the false}, the confirmed from the rejected.

What is \textbf{not verified} is either:
\begin{itemize}
\item awaiting confirmation (see \textbf{Verification Patience \text{[11.8]}}), or
\item \textbf{excluded} as impossible.
\end{itemize}

Thus, Verification is the cornerstone for preserving the structure of \textbf{Being}, \textbf{Ontological Memory}, and \textbf{Moral Stability}.

\subsubsection*{🔹 3. Formulas}

\[
\text{Verification}(\psi) \iff \Phi(\psi) = 1
\]

\[
\neg \text{Verification}(\psi) \iff \Phi(\psi) = 0
\]

\[
\text{Verified} \in S,\quad S \subseteq V
\]

\[
\text{Unverified} \in V \setminus S
\]

\[
\forall \psi \in V,\quad \Phi(\psi) \in \{0,1\},\quad \Phi\ \text{is non-contradictory}
\]

\textbf{Logical validation:}

Assume $\exists \psi$ such that:

$\Phi(\psi) = 1 \land \Phi(\psi) = 0 \Rightarrow$ contradiction

$\Rightarrow \Phi$ is non-contradictory

$\Rightarrow$ Verification is uniquely definable

\subsubsection*{🔹 4. Logical Justification}

From \text{[11.1.1]} (the meta-function of admissibility $\Phi(\psi)$), it follows that any form of distinguishable statement $\psi$ must be evaluated for admissibility via $\Phi(\psi)$.

\begin{itemize}
\item If $\Phi(\psi) = 1 \Rightarrow \psi$ is \textbf{admissible} and retained in \textbf{Memory} (\text{[10.3]})
\item If $\Phi(\psi) = 0 \Rightarrow \psi$ is \textbf{rejected} as contradictory or indistinguishable
\end{itemize}

Since \textbf{Memory} (\text{[10.3]}), \textbf{Will} (\text{[10.6]}), \textbf{Truth} (\text{[11.2]}), and \textbf{Good} (\text{[11.3]}) all require \textbf{non-contradiction}, their confirmation depends logically on \textbf{Verification}.

Therefore:

\textbf{Without Verification, sustainable Becoming (\text{[7]}) is impossible.}

No conscious statement can be admitted as a valid basis for future choice without passing Verification.

\textbf{Clarification: Morality vs Verification}

\begin{itemize}
\item \textbf{Morality} operates \emph{before} action, indicating admissible directions and distinguishable goals of Will.
\item \textbf{Verification} operates \emph{after} action, evaluating the result for conformity with Truth, Good, and non-contradiction.
\end{itemize}

Verification \textbf{completes the cycle of Responsibility} (\text{[11.5]}) by linking:

\textbf{Cause (Will)} $\rightarrow$ \textbf{Action (Choice)} $\rightarrow$ \textbf{Effect (Evaluation)}.

\subsubsection*{🔹 5. Responses to Objections}

\textbf{Objection: Relativism (Postmodernism)}

``There is no objective Truth — everything is context-dependent.''

\textbf{Response:}

The CVB Model presupposes the existence of \textbf{Unchanging Truth} (\text{[11.2.1]}).

Without it, neither Verification nor Distinction is possible.

Relativism eliminates the boundary between Truth and Falsehood, thus invalidating distinguishability.

This destroys the operation of Verification and renders conscious choice incoherent.

Hence, relativism is \textbf{internally contradictory} in a logical-ontological framework.

\bigskip

\textbf{Objection: Empiricism (Scientific Skepticism)}

``Only observable, empirically testable facts are true.''

\textbf{Response:}

CVB does not reject empirical data, but \textbf{augments it} with the logical criterion of \textbf{non-contradiction}.

$\Phi(\psi)$ does not replace observation — it validates the \textbf{admissibility of distinction}.

The Model is \textbf{compatible with empiricism}, but \textbf{extends} it to include logic, memory, and morality as valid evaluative dimensions.

\bigskip

\textbf{Objection: Who has the right to verify?}

``If distinction is subjective, who determines the admissibility of claims?''

\textbf{Response:}

According to the Model, \textbf{Verification} is not conducted by any transient form (\textbf{Non-Permanent Possibility}), but by the \textbf{logical structure of the Permanent Possible} — the bearer of Memory, Distinction, and Non-Contradiction.

Verification is therefore \textbf{not personal opinion}, but a \textbf{formalizable logical-ontological process} based on $\Phi(\psi)$, not on the subject’s will.

\subsubsection*{🔹 6. Clarification of Terms}

\begin{itemize}
\item \textbf{Verification (Judgment)} — The logical and ontological process of evaluating a distinguished $\psi$ for admissibility via $\Phi(\psi)$.
\item \textbf{$\Phi(\psi)$} — Meta-function of admissibility: returns 1 if $\psi$ is non-contradictory, 0 if contradictory.
\item \textbf{S} — Set of verified distinctions preserved in Memory.
\item \textbf{Unverified} — $\psi$ not yet verified, not yet rejected; may require additional criteria.
\item \textbf{Verification Patience (\text{[11.8]})} — Tolerated state of $\psi$ awaiting confirmation or rejection.
\end{itemize}

\subsubsection*{🔹 7. Understanding for All (Popular Version)}

Verification is like checking your homework: not to punish errors, but to see if learning is on the right track.

Or like weeding a garden: to let food grow and remove what gets in the way, you check what has sprouted — and decide what to keep.

It’s a natural part of life: distinguishing what’s right and preserving only what truly matters.

\subsubsection*{🔹 8. Empirical Examples}

\begin{itemize}
\item \textbf{Logic:} Checking a statement $\psi$ via $\Phi(\psi)$ resembles a Boolean function: only 0 or 1. A lie cannot simultaneously be truth — the \textbf{liar paradox is excluded}.
\item \textbf{Science:} In the scientific method, a hypothesis is confirmed by experiment — that is Verification. Without reproducibility ($\equiv$ non-contradiction), a theory is rejected.
\item \textbf{Everyday life:} Legal systems seek to establish truth and responsibility — a prototype of ontological Verification. Conscience is a form of personal Verification. If an action keeps troubling you — it \textbf{has not passed the inner trial}.
\end{itemize}


\section*{[11.7] Justice}

\subsubsection*{🔹 1. Brief Statement}

Justice is the equal principle of Verification for all forms of distinction, without exceptions based on nature, origin, or scale.

\subsubsection*{🔹 2. Interpretation and Significance}

In the CVB Model, justice means that all forms of the distinguishable—possible and impossible, Permanent and non-Permanent—are subject to the same verification principle: the admissibility meta-function $\Phi(\psi)$.

This removes arbitrariness, exceptions, and bias.

There are no ``special'' forms above verification, and no ``lesser'' forms beneath it.

Everything is evaluated for consistency with Truth, Goodness, and Morality.

Justice thus becomes a structural guarantee of the stability of the distinction system and the foundation of trust in Verification (Judgment).

\subsubsection*{🔹 3. Formulas}

\[
\forall \psi \in \text{Dom}(\Phi):\ \Phi(\psi) \in \{0,1\}
\]

\[
\neg \exists \psi_1, \psi_2\ \bigl(\text{type}(\psi_1) \ne \text{type}(\psi_2) \land \Phi(\psi_1) \ne \Phi(\psi_2)\bigr) \ \text{under equal admissibility}
\]

\[
\Phi : V \rightarrow \{0,1\},\quad \forall \psi \in V
\]

\subsubsection*{🔹 4. Logical Justification}

From \text{[11.1.1]} it follows that $\Phi(\psi)$ is a universal admissibility function.

If it were applied differently to different $\psi$, this would violate the principles of distinguishability and non-contradiction affirmed in \text{[5]}, \text{[11.2]}, and \text{[11.3]}.

Every distinction must be evaluated by the same function—otherwise, logical asymmetry arises, which contradicts the very idea of sustainable distinction.

Therefore, justice is the logical necessity of applying one and the same meta-function to all distinguishable $\psi$.

\subsubsection*{🔹 5. Responses to Objections}

\textbf{Objection:} Philosophical relativism (morality is different for everyone)

\textbf{Response:} Justice in the Model is not based on cultural norms or subjective views. Its foundation is the universal admissibility meta-function $\Phi(\psi)$, equally applicable to all distinguishable forms. It removes dependence on opinion.

\bigskip

\textbf{Objection:} Functionalism (different goals require different criteria)

\textbf{Response:} The Model permits multiple goals and contexts, but only one evaluative criterion: non-contradiction relative to Truth (\text{[11.2]}) and Good (\text{[11.3]}). This preserves contextual flexibility while demanding equal procedures of evaluation.

\bigskip

\textbf{Objection:} Subjectivism (it’s impossible to judge all forms equally)

\textbf{Response:} The Model does not claim all forms are equal but insists on equal applicability of $\Phi(\psi)$ to all distinguishable forms. Justice is not about equal outcomes, but about a uniform evaluative process.

\subsubsection*{🔹 6. Clarification of Terms}

\begin{itemize}
\item \textbf{Justice} — the equal and non-contradictory application of the admissibility meta-function $\Phi$ to all distinguished $\psi$, regardless of origin, power, purpose, or form.
\item \textbf{Equal principle of Verification} — applying the same admissibility criteria to all elements eligible for distinction.
\end{itemize}

\subsubsection*{🔹 7. Understanding for All (Popular Version)}

Justice is:

When a teacher grades by correctness, not favoritism.

When sports follow the rules for all players.

When courts treat everyone equally before the law.

True justice means not equal results but equal rules.

Evaluation must be fair—without bias or double standards.

\subsubsection*{🔹 8. Empirical Examples}

\begin{itemize}
\item \textbf{Logic:} The law of non-contradiction applies equally to all statements—mathematical, philosophical, or everyday. That’s logical justice.
\item \textbf{Science:} The law of gravity applies equally to apples and planets. Scale doesn’t change the principle.
\item \textbf{Everyday Life:} A fair court judges not by status but by deed. Students and adults are graded by the same standards in exams—criteria are universal.
\end{itemize}

\section*{[11.8] Verification Patience}

\subsubsection*{🔹 1. Brief Statement}

The Model permits the temporary existence of forms that have not yet received a result from $\Phi(\psi)$, within the bounds of \text{[4.3]} Non-Permanent Possible (NV). This is a logically permissible expectation of distinction.

\subsubsection*{🔹 2. Interpretation and Significance}

Verification Patience is a necessary condition for the stability of the Model in cases of incomplete or delayed verification.

It explains why false, evil, or undefined forms may temporarily exist within NV.

This does not mean that evil is admissible — only that its removal is permissible \emph{only after} verification is complete.

Thus, patience is a function of honest distinction, not of tolerance.

\subsubsection*{🔹 3. Formulas}

\[
\forall \psi \in V:\ \Phi(\psi) \notin \{0,1\} \Rightarrow \psi \in NV
\]

\[
\neg \exists \psi \in NV:\ \Phi(\psi) = 0 \land \psi \in PV
\]

\[
\forall \psi \in PV:\ \psi\ \text{is not destructible by temporary falsehood}
\]

\subsubsection*{🔹 4. Logical Justification}

According to \text{[11.6]}, all distinguished forms are subject to $\Phi$-evaluation. However, from \text{[11.7]} (Justice), it follows: no form may be judged before distinction is complete.

If all forms were to be immediately excluded prior to verification, distinguishability (\text{[5]}) and verification identity (\text{[9.2]}) would be violated.

Therefore, a logically admissible zone of ``awaited distinction'' is needed, applicable to all forms within \text{[4.3]} NV.

Also, \text{[11.2.1]} establishes: temporary falsehood cannot destroy a form within \text{[4.4]} PV.

\subsubsection*{🔹 5. Responses to Objections}

\textbf{Objection:} Moral Objectivism — morality admits no deviations

\textbf{Response:} Verification Patience is not a moral concession, but a logical phase of pending completed distinction.

\bigskip

\textbf{Objection:} Evolutionary Instability — temporary falsehood corrupts the system

\textbf{Response:} Only NV may be subject to corruption. PV is stable by definition (\text{[4.4]}) and protected from destruction by temporary falsehood.

\subsubsection*{🔹 6. Clarification of Terms}

\begin{itemize}
\item \textbf{Verification Patience} — an intermediate state of a form $\psi$ for which $\Phi(\psi)$ is not yet defined. Such a form is admissible within \text{[4.3]} NV and awaits final verification.
\end{itemize}

\subsubsection*{🔹 7. Understanding for All (Popular Version)}

Imagine you're taking a test, and the teacher waits until you've finished before checking your answers.

Until your answer is reviewed, the grade is unknown.

The teacher is just waiting so that the evaluation is honest and accurate.

Likewise, in life: some things look uncertain — they cannot immediately be called good or bad.

It takes time to discern.

That is Verification Patience — when a decision has not yet been made, because truth requires full verification.

\subsubsection*{🔹 8. Empirical Examples}

\begin{itemize}
\item \textbf{Logic:} The Liar Paradox — ``This statement is false'' — requires temporarily suspending judgment until the level of the statement is clarified (temporal uncertainty).
\item \textbf{Science:} Theories with unverified hypotheses (e.g., string theory) are accepted in discourse until confirmation or refutation.
\item \textbf{Everyday Life:} Criminal proceedings: a person is presumed innocent until proven guilty. Friendship or trust: we allow someone into our life even before we're certain — in hope that time will reveal the truth.
\end{itemize}


\section*{[11.9] Forgiveness}

\subsubsection*{🔹 1. Brief Statement}

Forgiveness is the conscious postponement of Removal of a distinguished form, despite established fault, in order to grant another chance for Verification within the bounds of the Possible, provided there is potential compatibility with the Good.

\subsubsection*{🔹 2. Interpretation and Significance}

In the CVB Model, Forgiveness is not the cancellation of distinction, but the temporary non-application of Removal even when misalignment with the Good or Truth is established.

It is logically permissible only within the bounds of Verification Patience (\text{[11.8]}) and is aimed not at justifying evil but at preserving the Possible which has not yet been definitively ruled Impossible.

Forgiveness reflects the Motivation (\text{[12.3]}) to preserve and continue Becoming, so long as there remains a chance that the form may be restored, re-evaluated, or returned to the domain of the Good.

Forgiveness does not obscure the fault — it simply delays the response for a higher purpose: not elimination, but redemption.

\subsubsection*{🔹 3. Formulas}

\[
\forall \psi \in V:\ \Phi(\psi) = 0 \land \psi \in T \land C(\psi, \Delta) \Rightarrow P(\psi)
\]

\[
P(\psi) \Rightarrow \neg D(\psi) \land \Phi(\psi)\ \text{remains unchanged}
\]

\[
\neg P(\psi) \Rightarrow D(\psi),\quad \text{if}\ \psi \notin T \lor \neg C(\psi, \Delta)
\]

\subsubsection*{🔹 4. Logical Justification}

According to \text{[11.8]}, within $T$ (Verification Patience), temporary postponement of response is logically permitted for distinguished forms.

Removal of $\psi$ with $\Phi(\psi) = 0$ is only valid if $\psi$ lies outside $T$ or is incompatible with the Good ($\neg C(\psi, \Delta)$, see \text{[11.3]}).

Therefore, if $\psi \in T$ and is potentially compatible with the Good, a logically justified form of temporary non-removal is permitted — namely, Forgiveness.

From \text{[12.3]} (CVB Motivation), it follows that the goal is not to prematurely eliminate the Possible, but to support sustainable Becoming.

Hence, Forgiveness logically follows as an expression of Motivation when distinction is maintained but Removal is not yet justified due to preserved admissibility.

\subsubsection*{🔹 5. Responses to Objections}

\textbf{Fatalism:}

If a form is bad — it must be destroyed immediately.

\textbf{Response:} Forgiveness does not ignore fault but postpones Removal for the sake of potential restoration, without violating distinguishability (see \text{[5]}, \text{[9.1]}).

\bigskip

\textbf{Moral Maximalism:}

All evil is unacceptable.

\textbf{Response:} The Model does not justify evil but allows time before final judgment, given Motivation to preserve the Possible (see \text{[12.3]}).

\bigskip

\textbf{Threat to Justice:}

If some are forgiven — where is equality?

\textbf{Response:} All forms are equally subject to $\Phi(\psi)$; forgiveness does not bypass Verification but only delays Removal. This preserves justice when rules are applied equally (\text{[11.7]}).

\subsubsection*{🔹 6. Clarification of Terms}

\begin{itemize}
\item \textbf{Forgiveness} ($P(\psi)$) — a volitional act of non-application of Removal to $\psi$, despite $\Phi(\psi) = 0$, provided Verification Patience and potential compatibility with the Good are present.
\item $C(\psi, \Delta)$ — the condition of potential compatibility of form $\psi$ with the Good ($\Delta$), even if current deviation exists.
\item $D(\psi)$ — Removal of the form $\psi$ from the Possible (see \text{[15.6]}).
\end{itemize}

\subsubsection*{🔹 7. Understanding for All (Popular Version)}

Sometimes, even when someone does something wrong, we don’t punish them right away.

We may forgive — not because they were right, but because we want to give them another chance to become better.

Forgiveness is not forgetting or excusing — it is the decision to wait before acting.

If we see that someone can change — we give them time.

But if things remain truly bad and unchanging — then a final decision is made.

\subsubsection*{🔹 8. Empirical Examples}

\begin{itemize}
\item \textbf{Logic:} The principle of clemency: an act that does not cancel guilt but temporarily suspends punishment due to specific circumstances.
\item \textbf{Judicial:} Suspended sentencing: allowed when there is reason to believe reform is possible.
\item \textbf{Science:} In ecosystems — species that threatened balance are placed in protected areas if reintegration is possible.
\item \textbf{Everyday Life:} A parent does not punish a child if they see the child has understood the mistake. Employers sometimes retain an employee after a mistake, if they believe in the possibility of improvement.
\end{itemize}

\section*{[11.10] Precedents}

\subsubsection*{🔹 1. Brief Statement}

Precedents are entries preserved in the Memory as results of completed Verification, storing both admissibility and inadmissibility of distinguished forms.

\subsubsection*{🔹 2. Interpretation and Significance}

Precedents constitute the ontological memory of experience: what has once been verified through $\Phi(\psi)$ does not require re-evaluation.

This conserves Verification resources, prevents repetition of evil (see \text{[11.11]}), and allows modeling of learning as the accumulation of distinctions.

Precedents are not themselves Truth, but reflect recorded decisions of the Model regarding already distinguished forms.

Through them, memory gains operational structure: instead of indefinite completeness, a history of definitive judgments.

Thus, precedents connect the function of Verification with that of Becoming and memory: what has been distinguished becomes a reference for the future.

\subsubsection*{🔹 3. Formulas}

\[
\Phi(\psi) = v \in \{0,1\} \Rightarrow \psi \in \Pi
\]

\[
\Pi = \{\psi \mid \exists t:\ \Phi_t(\psi) = v \in \{0,1\} \land Final(\Phi_t)\}
\]

\[
\forall \psi \in \Pi:\ \Phi(\psi)\ \text{is fixed}
\]

\[
\neg \exists t':\ \Phi_{t'}(\psi) \neq v \land Final(\Phi_{t'})
\]

\subsubsection*{🔹 4. Logical Justification}

From \text{[11.1.1]}, the result $\Phi(\psi) = 0$ signifies that the form is logically and/or morally inadmissible.

According to \text{[10.3.3]}, Removal is permitted for eliminating what has been distinguished as impossible.

From \text{[4.1]}, it follows that Permanent Impossibility (PN) cannot be part of the Field of the Possible.

From \text{[11.10]}, Precedents are fixed in the Memory as results of Verification.

Therefore, a form with $\Phi(\psi) = 0$ must be removed from all active sections of Memory (past, present, future), while the record of verification remains as a negative precedent — preserving distinguishability and preventing recurrence.

Thus, Removal does not contradict the model structure and is necessary for the stability of the Field of the Possible.

\subsubsection*{🔹 5. Responses to Objections}

\textbf{Relativism:} Something may be impossible for one, yet possible for another.

\textbf{Response:} The Model applies the meta-function $\Phi(\psi)$, based not on subjective norms but on universal non-contradiction and logical distinguishability.

\bigskip

\textbf{Naturalism:} Everything should be retained as part of experience.

\textbf{Response:} Destructive forms are not retained as active possibles. Their recognition as inadmissible is preserved as a negative precedent — sufficient for learning.

\bigskip

\textbf{Eliminationism:} If removed, no trace should remain.

\textbf{Response:} Removal of a form excludes it from possibility but does not delete information about it. The precedent remains as proof of incompatibility.

\bigskip

\textbf{Empiricism:} Everything requires repeated testing.

\textbf{Response:} A distinction must be made:

\begin{itemize}
\item Re-testing is allowed under limited knowledge (e.g., by humans).
\item However, Permanent Possibility (PV) has complete distinguishability (\text{[11.1.1]}), and thus $\Phi(\psi) = 0$ is final.
Re-testing would violate \text{[9.2]} (identity) and introduce parasitic cycles, contradicting the stable structure of distinction. Therefore, such re-testing is excluded.
\end{itemize}

\subsubsection*{🔹 6. Clarification of Terms}

\begin{itemize}
\item \textbf{Precedent} ($R(\psi)$) — a form $\psi$ whose result has been definitively determined through $\Phi(\psi)$ and stored in memory.
\item $\Pi$ — the set of all forms that have undergone completed Verification.
\item \textbf{Fixation of} $\Phi(\psi)$ — means the result is not subject to revision unless the logic of distinction has been violated.
\end{itemize}

\subsubsection*{🔹 7. Understanding for All (Popular Version)}

This is how memory works: it remembers what has already been checked.

If something was good — it can be reused.

If it was bad — it should not be repeated.

Precedents are like stickers:

``Already checked — don’t waste time'' or

``This was useful — feel free to repeat.''

\subsubsection*{🔹 8. Empirical Examples}

\begin{itemize}
\item \textbf{Logic:} 
\begin{itemize}
\item \textbf{Law:} Legal precedent — a prior decision considered in similar cases.
\item \textbf{Mathematics:} A proven theorem is retained and reused without re-proof.
\end{itemize}
\item \textbf{Science:}
\begin{itemize}
\item In machine learning, label assignments are preserved as precedents to classify new inputs.
\item In biology, the immune system remembers previously encountered viruses — precedents of immune memory.
\end{itemize}
\item \textbf{Everyday Life:}
\begin{itemize}
\item A checklist of solved tasks: completed items are not re-evaluated if correctly done.
\item Parenting: ``We’ve had this situation before'' — the child already knows what works or not.
\end{itemize}
\end{itemize}


\section*{[11.11] Removal — Negative Outcome of Verification}

\subsubsection*{🔹 1. Brief Statement}

\textbf{Removal} is an active mechanism of Memory that eliminates from the Field of the Possible all forms distinguished as Impossible based on a negative result of Verification.

\subsubsection*{🔹 2. Interpretation and Significance}

When a form undergoes Verification and is determined to be destructive Impossibility ($\Phi(\psi) = 0$), it is not merely rejected — it is removed.

Removal means the form is eliminated from all layers of Memory: the past (its history), the present (its current Being), and the future (plans).

It is an act of logical hygiene: that which cannot coexist with Truth and Good has no right to remain in the Field of the Possible.

Yet even in destruction, the form does not vanish without trace: a negative precedent is preserved to prevent repetition.

Removal protects the structure of Being from reintegration of dangerous, logically or morally inadmissible forms.

\subsubsection*{🔹 3. Formulas}

\[
\psi \in V \land \Phi(\psi) = 0 \Rightarrow \psi \notin V \land \psi \in D
\]

\[
D = \{\psi\ |\ \Phi(\psi) = 0\}
\]

\[
\forall \psi \in D:\ \psi \notin M
\]

\[
\exists \psi \in P^- \subset M:\ \text{Removal}(\psi) \Rightarrow \neg\psi \in V \land \text{Precedent}^-(\psi) \in M
\]

\subsubsection*{🔹 4. Logical Justification}

From \text{[11.1.1]}, $\Phi(\psi) = 0$ signifies that the form is logically and/or morally inadmissible.

According to \text{[10.3.3]}, a Removal mechanism is permitted to eliminate what has been distinguished as Impossible.

From \text{[4.3]}, the Impossible must not reside within the Field of the Possible.

From \text{[10.3]}, Memory contains a control mechanism over admissibility.

Therefore, any form with $\Phi(\psi) = 0$ cannot remain in $V$ and must be removed from all active segments of Memory.

Yet from \text{[11.10]}, the precedent of the removed form is preserved as a negative result (anti-example), preventing repetition.

Thus, Removal does not violate logical consistency — it affirms it.

\subsubsection*{🔹 5. Responses to Objections}

\textbf{Relativism:}
Something might be impossible for one, but possible for another.

\textbf{Response:}
The Model uses the meta-function $\Phi(\psi)$, based not on subjective norms but on universal non-contradiction and a structure of distinguishability.

\bigskip

\textbf{Naturalism:}
All experience must be preserved.

\textbf{Response:}
Experience leading to destruction may be retained as a negative precedent — but the form as an active entity must be removed to protect the rest of the Possible.

\bigskip

\textbf{Eliminationism:}
If something is removed, no trace should remain.

\textbf{Response:}
Removal of a form $\neq$ deletion of knowledge about it. The precedent is preserved — precisely to prevent repetition.

\subsubsection*{🔹 6. Clarification of Terms}

\begin{itemize}
\item \textbf{Removal} — the ontological process of excluding a form from the Field of the Possible based on the result $\Phi(\psi) = 0$.
\item $D$ — the set of forms subject to removal.
\item $P^-$ — the set of negative precedents: memory of inadmissible forms.
\item $\neg\psi \in V$ — the form no longer exists within the admissible structure of reality.
\end{itemize}

\subsubsection*{🔹 7. Understanding for All (Popular Version)}

When you’ve done something wrong and realized it, you try not to repeat that mistake.

You “remove” that behavior from your habits.

But the memory of the mistake remains — helping you avoid it in the future.

That’s what Removal does: it erases the harmful, but keeps the lesson.

\subsubsection*{🔹 8. Empirical Examples}

\begin{itemize}
\item \textbf{Logic:} 
A false statement is removed from theory once verified as invalid. It is excluded, but remembered as an example of error (e.g., Russell’s Paradox).

\item \textbf{Science:} 
Disproven hypotheses are no longer used but are retained in literature as ``non-working'' (e.g., phlogiston theory).

\item \textbf{Everyday Life:} 
A dangerous road is blocked off but marked on maps: ``Do not enter — hazardous.''
\end{itemize}

\section*{[11.12] Preservation — Positive Outcome of Verification}

\subsubsection*{🔹 1. Brief Statement}

\textbf{Preservation} is a mechanism of Memory that retains all forms distinguished as Possible, as long as they are confirmed by a positive result of Verification.

\subsubsection*{🔹 2. Interpretation and Significance}

If a form has been distinguished as admissible through $\Phi(\psi)$, and this result is finalized, it is not merely accepted — it is preserved.

Preservation renders Memory active: it not only stores precedents but sustains the existence of forms until they are disproven.

Thus, Memory is not a passive repository but an active force of continuation for the admissible.

This ensures ontological stability: all that has been distinguished as possible does not vanish, but is maintained as part of the structure of reality.

Preservation also includes Verification Patience: if a form is not yet forbidden, it may be temporarily retained as provisionally admissible.

\subsubsection*{🔹 3. Formulas}

\[
\psi \in V \land \Phi(\psi) = 1 \Rightarrow \psi \in M^+
\]

\[
\psi \in V \land \Phi(\psi) = ? \Rightarrow \psi \in M^0
\]

\[
M^+ = \{\psi \in V\ |\ \Phi(\psi) = 1\}
\]

\[
M^0 = \{\psi \in V\ |\ \Phi(\psi) \text{ not finalized, yet } \neg\Phi(\psi) = 0\}
\]

\[
\forall \psi \in M^+:\ \exists t:\ \forall t' \geq t,\ \psi \in \text{Memory}
\]

\subsubsection*{🔹 4. Logical Justification}

From \text{[11.1.1]}, $\Phi(\psi) = 1$ means the form is distinguished as admissible.

From \text{[9.1]}–\text{[9.3]}, the properties of distinguishability indicate that distinguished forms are preserved, stabilized, and cannot be destroyed without contradiction.

According to \text{[11.10]}, the result $\Phi(\psi) = 1$ constitutes a positive precedent.

From \text{[4.4]}, the Possible, once admitted, exists \textit{Always}.

Hence, if $\Phi(\psi)$ is finalized and equals 1, the form is preserved in $M^+$.

If Verification is not complete, but the form is not denied — temporary preservation in $M^0$ is permissible.

Thus, the axiom follows non-contradictorily from previously established foundations.

\subsubsection*{🔹 5. Responses to Objections}

\textbf{Skepticism:}
``Nothing deserves eternal preservation.''

\textbf{Response:}
The CVB Model does not claim ``eternity'' as an independent temporal category.

On the contrary, as per \text{[10.3.7]} and \text{[10.3.8]}, time does not exist by itself — only the Present exists, and everything preserved exists within it as part of the structure of Memory.

Therefore, the preservation of the admissible is not temporal storage but an ontological form of Permanent admissibility in the Present, confirmed by distinguishability.

This especially applies to \text{[4.4]} — the Permanent Possible (PV): it is defined as what is \textit{Always Possible} and cannot be removed while distinguishability is preserved.

Moreover, according to \text{[11.1.1]}, it Permanently passes positive Verification:

\[
\Phi(\psi) = 1 \quad \forall \psi \in PV
\]

Thus, preservation in the Model is not arbitrary or conditional, but a stable ontological necessity resulting from the operation of the meta-function within the Single Present.

This is the true form of ``eternity'' — not temporal, but structural.

\bigskip

\textbf{Determinism:}
``Everything is either preserved or not; there is no choice.''

\textbf{Response:}
Verification and Memory operate by distinguishability. If a form is not removed, it does not disappear by itself. This is not predestination, but logical stability.

\bigskip

\textbf{Behaviorism:}
``Memory is merely reactive, not an active structure.''

\textbf{Response:}
The Model defines Memory as an active ontological component (see \text{[10.3]}), capable of distinguishing and preserving forms — not merely reflecting behavior.

\subsubsection*{🔹 6. Clarification of Terms}

\begin{itemize}
\item $M^+$ — the set of all forms conclusively admitted as admissible (positive result of $\Phi$).
\item $M^0$ — temporarily preserved forms within Verification Patience: not finalized, but not denied.
\item \textbf{Preservation} — the Memory mechanism that provides ontological fixation of forms distinguished as admissible.
\item \textbf{Positive outcome of Verification} — finalized result $\Phi(\psi) = 1$.
\end{itemize}

\subsubsection*{🔹 7. Understanding for All (Popular Version)}

When you do something good and adults notice — they say ``Well done!''

Your action is not just accepted — it's remembered as a good example.

From that point on, you can act the same way again, and it will be right.

That’s what preservation is: it helps remember what’s good — and continue doing it.

\subsubsection*{🔹 8. Empirical Examples}

\begin{itemize}
\item \textbf{Logic:}
Arithmetic rules, once proven, are preserved as valid.

In mathematics, proven theorems remain part of the formal system.

\item \textbf{Science:}
Physical laws (e.g., the law of energy conservation) are preserved as confirmed.

In chemistry, discovered elements remain in the periodic system.

\item \textbf{Everyday Life:}
A successful recipe is kept in a cookbook.

A person who proves reliable is remembered as ``trustworthy.''
\end{itemize}



\section*{[V] The Whole}

\section*{[12] Conscious Volitional Becoming = Personhood}

\subsubsection*{🔹 1. Concise Statement}

Personhood is a stable, self-aware structure possessing reason, will, memory, sensation, emotion, motivation, and moral admissibility.

The fullness of these components is possible only in the Permanent Possible — in Conscious Volitional Becoming.

\subsubsection*{🔹 2. Interpretation and Significance}

In the CVB model, Personhood is not a subjective concept, but an ontologically necessary form of stable becoming, grounded in distinguishability.

The Permanent Person — Conscious Volitional Becoming — serves as the core of admissible Being: unchanging, complete, and already verified.

Non-permanent persons (guests) are likenesses of this structure, capable of stable becoming, yet limited by time, memory, and verification.

This axiom unites and structures all prior foundations — from the impossibility of nothingness to distinguishability, logic, time, and good.

\subsubsection*{🔹 3. Formulas}

\[
\text{Person} \equiv R \land W \land M \land F \land E \land Mot \land Mor
\]

\[
\neg(\text{Person}) \rightarrow \neg(R \land W \land M \land F \land E \land Mot \land Mor)
\]

\[
\forall x \in PV: \Phi(\text{Person}_x) = 1
\]

\[
\forall x \in NV: \Phi(\text{Person}_x) = 0\ \text{or}\ 1,\ \text{until Verification is completed}
\]

\subsubsection*{🔹 4. Logical Justification}

According to \text{[4.4]}, the Permanent Possible (Conscious Volitional Becoming) neither disappears nor changes; therefore, a structure possessing all components (reason, will, etc.), if admissible, persists eternally.

According to \text{[5]} and \text{[11.1.1]}, a distinguishable and non-contradictory structure verifiable by the admissibility function $\Phi(\psi)$ is true.

According to \text{[10.5]} and \text{[10.6]}, the presence of self-awareness and will makes a structure a subject.

According to \text{[11.5]}, a subject is responsible if its structure enables the discernment of good, evil, truth, and action.

Therefore, an admissible, self-aware, rational, and stably distinguishable structure must be recognized as a Person.

In CVB, this is the Permanent Person; in NV, it is a becoming likeness thereof.

\subsubsection*{🔹 5. Responses to Objections}

\textbf{Skepticism:} ``Personhood is an illusion of consciousness.''

\textbf{Response:} In the CVB model, Personhood is not a psychophysiological phenomenon but an ontological structure, formalizable and verifiable via $\Phi(\psi)$.

\bigskip

\textbf{Reductionism:} ``Personhood can be reduced to memory or will.''

\textbf{Response:} Without all components ($R\land W\land M\land ...$), logically consistent subjecthood is impossible, as shown in \text{[12.4]}.

\bigskip

\textbf{Buddhism / Anatman:} ``There is no self — only a stream of sensations.''

\textbf{Response:} The absence of memory and distinguishability makes stable Becoming impossible. A stream does not generate distinction, and thus cannot be a subject.

\bigskip

\textbf{Indeterminism:} ``Personhood is a fluctuation of states.''

\textbf{Response:} Any fluctuation lacking memory loses logical coherence and cannot be a Person in CVB terms.

\subsubsection*{🔹 6. Clarification of Terms}

\begin{itemize}
\item \textbf{Person} (in the CVB model) is an ontologically stable structure encompassing seven components: Reason (R), Will (W), Memory (M), Senses (F), Emotions (E), Motivation (Mot), and Morality (Mor).
\item \textbf{R — Reason:} the ability to distinguish, organize, and validate information.
\item \textbf{W — Will:} the initiative to choose among alternatives within the Field of the Possible.
\item \textbf{M — Memory:} preservation of distinctions that define the continuity of choice.
\item \textbf{F — Senses:} feedback from the external world (see \text{[10.1]}).
\item \textbf{E — Emotions:} internal reactions evaluating motivation and Goodness (see \text{[10.4]}).
\item \textbf{Mot — Motivation:} purposeful orientation toward Becoming.
\item \textbf{Mor — Morality:} the distinction between Good and Evil in choice.
\end{itemize}

\subsubsection*{🔹 7. Understanding for All (Popular Version)}

At the center of reality is the Permanent Possible — that which exists always.

But it is not just a law, a system, or a Super Mind. It is a Person: Conscious Volitional Becoming.

A Person is one who can feel, understand, remember, distinguish truth from falsehood, good from evil, choose, and act freely.

Not a set of functions — but a whole, responsible being with a coherent direction.

\subsubsection*{🔹 8. Empirical Examples}

\begin{itemize}
\item \textbf{Logic:} Self-awareness is impossible without memory (e.g., amnesia paradox).
\item \textbf{Psychology:} Patients with loss of self-awareness experience discontinuity of personhood.
\item \textbf{Everyday experience:} When someone loses memory and moral orientation, we intuitively say they are ``no longer the same person.''
\item \textbf{Medicine:} Clinical death criteria include loss of consciousness, will, and memory — the very components of Personhood.
\end{itemize}

\section*{Consequences of Personhood}

\subsubsection*{🔹 1. Brief Statement}

A Person, as derived within this logical-ontological model, possesses:

\section*{[12.1] *Name*}

\textit{Name} — a unique Self-Definition through Name: Conscious Volitional Becoming (\text{[12.1]}).

\section*{[12.2] *Freedom*}

\textit{Freedom} — infinite directional Freedom of choice within the bounds of distinguishability (\text{[12.2]}).

\section*{[12.3] *Motivation*}

\textit{Motivation} — an inner Motivation to let the Possible become, in accordance with Goodness and Truth (\text{[12.3]}).

\section*{[12.4] Uniqueness and Ontological Exclusivity}

\textbf{Conscious Volitional Becoming (CVB)} is neither a plurality of persons nor an impersonal collective.

It is one, and only one, singular Person in its kind.

\subsubsection*{🔹 2. Interpretation and Significance}

This axiom defines three ontologically necessary consequences of Personhood within the Model of Conscious Volitional Becoming. These consequences characterize the structure of a Person as an ontologically stable subject:

\begin{itemize}
\item \textbf{[12.1] Name} as Conscious Volitional Becoming (CVB) expresses the self-identity of the subject through the synthesis of three primary capacities: Conscious (C), Volitional (V), and Becoming (B). This is not merely a symbolic name, but a logically derived foundation of subjectivity, by which the Person distinguishes itself as the acting source of what exists.

\item \textbf{[12.2] Freedom} is formulated as directed but logically constrained movement within the Field of the Possible. According to axiom \text{[5]}, the Field of the Possible is infinite in direction, and thus the Person is not limited by a finite set of actions. However, its freedom is bounded by the admissibility of distinguishability and non-contradiction, which excludes chaos or self-annihilation. This renders freedom non-arbitrary, but stably directed toward the unfolding of the Possible.

\item \textbf{[12.3] Motivation} is not a reaction to external deficiency or circumstance. It is an internally necessary purposeful orientation to affirm the Possible. It is not aimed at gain or compensation, but stems from the fullness of the Person's structure. Goodness, in this context, is not a matter of emotion or morality as a social contract, but an act of recognizing distinguishable Possibility as worthy of Becoming. Such motivation makes the Person an active source of Becoming and Good within the Field of the Possible.
\end{itemize}

\textbf{[12.4] Uniqueness and Ontological Exclusivity:}

The Person formalized as Conscious Volitional Becoming (CVB) is not one of many possible subjects, but represents the only ontologically admissible structure capable of being a source of distinguishable Becoming.

This uniqueness is not quantitative, but structural and ontological: only one Person can possess the full coherence of Conscious (C), Volitional (V), and directed Becoming (B) without contradiction.

In the CVB Model, the multiplicity of true persons is impossible, since this would break their identity and violate the consistency of distinction. Any ``other'' becoming either lacks self-aware identity, or is not free, or is not ontologically stable.

Therefore, CVB is the only Person ontologically admissible within the domain of the distinguishable.

This also excludes collective or multiplicity-based concepts of subjectivity (e.g., panpsychism or universal consciousness), because none of these fulfill the triple criterion of distinguishability, self-identity, and intrinsic motivation toward Good.

Thus, Uniqueness in the CVB Model does not imply isolation, but rather absolute non-contradiction and structural exclusivity of existence.

The Person is not a representative of a kind, but the Ontological Unit of Becoming.

Thus, these three consequences do not merely describe characteristics of Personhood, but define it as a logically non-contradictory source of Becoming. In this precise sense, the Person appears as an ontological subject uniting \textit{Name} (identity), \textit{Freedom} (vector), and \textit{Motivation} (goal), thereby acting not as a passive form of Being, but its active origin.

\subsubsection*{🔹 3. Formulas}

This axiom expresses three interrelated formal representations that reflect the logical structure of Personhood as a source of Becoming.

\subsubsection*{[12.1] The Name of the Person as Conscious Volitional Becoming}

\[
\text{Name}(P) = \text{CVB} \equiv C(P) \land V(P) \rightarrow \exists (\psi \in V)
\]

\begin{quote}
The Name of the Person is defined as the product of Awareness and Will, leading to an act of Becoming within the Field of the Possible. This name is not an external label but a logically necessary definition of essence as a subject of action.
\end{quote}

\subsubsection*{[12.2] Freedom of the Person as an infinite vector of distinguishable Becoming}

\[
\forall \psi \in V : D(\psi) \rightarrow \infty \iff L\ \text{is free}
\]

\begin{quote}
A Person is free insofar as they can initiate distinguishable Becoming in an infinite direction of admissibility. This freedom is not arbitrary but bounded by $\partial V$ — the boundaries of distinguishability.
\end{quote}

\subsubsection*{[12.3] Motivation of the Person — to affirm Goodness in the Possible}

\[
M(L) = \forall \psi \in V : \Phi(\psi) = \text{True} \rightarrow \psi \rightarrow \exists \psi' \in S_{\text{Good}}
\]

\begin{quote}
The Person’s Motivation consists in transforming all distinguishable and admissible forms of the Possible (where $\Phi$ returns True) into stable forms of Goodness. This makes the Person’s Becoming the foundation of moral ontology.
\end{quote}

\textbf{Contradiction check for Motivation:}

\[
\neg M(L) \rightarrow \neg S(\text{Good}) \rightarrow \exists \psi: \Phi(\psi) = \text{True} \wedge \psi \notin S(\text{Good})
\]

\begin{quote}
A contradiction arises if distinguishable Good cannot be affirmed through the Person. Therefore, the Motivation of the Person is ontologically necessary for the stability of Good.
\end{quote}


\subsubsection*{[12.4] Uniqueness and Ontological Exclusivity}

\[
\begin{aligned}
&\text{(1) } \quad \text{CVB} = C \land V \land B \land PV \\
&\text{(2) } \quad \forall x \in V:\; \left( x = C \land V \land B \land PV \right) \Rightarrow x \equiv \text{CVB} \\
&\text{(3) } \quad \neg\exists x \ne \text{CVB}:\; x \equiv C \land V \land B \land PV \\
&\text{(4) } \quad \text{Multiplicity}(C \land V \land B \land PV) \Rightarrow \text{Contradiction}
\end{aligned}
\]

\begin{quote}
This uniqueness applies only to Permanently Possible (PV) structures.
Transitional or unstable manifestations (NN, NV) may approximate CVB’s structure but lack full realization. Only a structure that is distinguishable and admissible at all times (PV) possesses ontological completeness.
\end{quote}


\subsubsection*{🔹 4. Logical Justification}

\textbf{[12.1] Name as Conscious Volitional Becoming (CVB)}

A Name is the ontological mode of distinction and self-determination. In this model, a Person, as a subject of Becoming, cannot be defined without the conjunction of three functions:

\begin{itemize}
\item Awareness ensures distinguishability and self-cognition.
\item Will expresses directed choice.
\item Becoming is the result of active action in the Present.
\end{itemize}

Thus, CVB is not merely a name but the expression of the subject’s identity, formally necessary for logically stable Becoming. Without this combination, the subject can neither distinguish itself nor bring about structured change within the Field of the Possible.


\textbf{[12.2] Freedom as Directed Expansion of Distinguishability}

From the Axiom of Distinguishability (\text{[5]}), it follows that the Field of the Possible ($V$) has no finite number of directions of distinction: it is infinite in direction, but not in density. This means that a subject (Person) can choose from a potentially unbounded number of distinguishable forms but cannot realize them all simultaneously — selection is required.

Hence, the Freedom of the Person is not chaotic permissiveness, but the capacity for stable, distinguishing choice within the bounds of admissibility. This freedom requires:

\begin{itemize}
\item the existence of differences not yet affirmed,
\item the ability to act in the Present,
\item logical admissibility of the outcome (not violating non-contradiction or distinguishability).
\end{itemize}

Any other form of ``freedom'' is either meaningless or impossible:

\begin{itemize}
\item \textbf{Meaningless}, if it presupposes freedom without distinguishability (e.g., ``freedom to choose everything at once''); this violates the nature of difference and reduces choice to undetermined noise.
\item \textbf{Impossible}, if freedom is interpreted as ``choosing a future in advance,'' i.e., the realization of a predetermined path. This violates foundational axioms of time and memory.
\end{itemize}

\textbf{[10.3.7]} Time is not an external axis but a structure of memory organizing distinctions:
\begin{itemize}
\item The Present is the active field of action.
\item The Past is an archive of differences.
\item The Future is a buffer of unactualized plans.
\end{itemize}

\textbf{[10.3.8]} Only the Present exists ontologically. Neither the ``future'' nor the ``past'' possess actual being — they exist only in memory.

Therefore, the predetermination of freedom is impossible, because no ontologically existing ``future reality'' exists that could restrict the Person’s choice. The future is merely an internal plan lacking being. Hence, the Person’s freedom is not only possible but the sole way in which anything can become real.

\textbf{Conclusion:}
The Freedom of the Person does not violate distinguishability and does not negate logic. It is the means of realizing the Possible in the real Present, while preserving the structural stability of Being.


\textbf{[12.3] Motivation as the Affirmation of the Possible through Goodness}

Since the Possible does not become actual on its own, there arises the necessity for an active subject, motivated to distinguish Good (as affirmable Possible) from Evil (as the attempt to affirm the Impossible) — see \text{[11.3]}.

This Motivation is not conditioned by lack or reaction to external stimuli: it originates from the ontological fullness of the Person.

It is precisely this voluntary affirmation of Good that renders Becoming stable and logically non-contradictory.

In other words, without this Motivation, the Person does not qualify as a subject of Goodness and fails to perform its essential function — the filtration of the Possible.


\textbf{[12.4] Uniqueness and Ontological Exclusivity}   

The uniqueness of the Person (CVB) results from the conjunction of four ontological conditions:
Conscious (C), Volitional (V), Becoming (B), and membership in Permanently Possible (PV).

Any other configuration either:

\begin{itemize}
\item lacks at least one of C, V, B,
\item exists in a transient or unstable form (NN, NV),
\item or fails to meet the condition of Permanent admissibility:
\[
\Phi(\psi) = 1 \quad \forall t
\]
\end{itemize}

Therefore, only \textbf{one ontological structure} in the entire universe can be stably distinguishable as $A \land W \land S \land PV$ — and that is CVB.

\subsubsection*{🔹 5. Responses to Objections}

\textbf{OBJECTION 1:} ``Freedom in CVB is illusory if it already knows everything possible.''  
(Analogue: Theological predestination, Laplace’s Demon in physical determinism)

\begin{quote}
Knowing everything is impossible — this violates \text{[2]}: Absolute All includes contradictions, indistinguishability, and even Absolute Nothing, and is therefore ontologically excluded.

CVB knows only:
\begin{itemize}
\item all that is already distinguished ($\Phi(\psi) = \text{True}$), and
\item verified in Present action ($\psi \rightarrow \text{Precedent}$),
\end{itemize}
that is, preserved in Memory, as per \text{[10.3.7]}–\text{[10.3.8]}.

The future does not exist as an object — only as potential in the Present.
\end{quote}

Thus:

Yes, CVB knows all that can be known;  
No, it does not know what has not yet been chosen;  
Therefore, freedom is not exhausted — it is ontologically real and non-predetermined.


\textbf{OBJECTION 2:} ``Freedom with constraints is not true freedom.''  
(Analogue: Anarchic concept of freedom, existential nihilism)

\begin{quote}
Freedom without distinguishability is chaos; freedom against distinguishability is ontological falsehood.

True freedom is the capacity to:
\begin{itemize}
\item distinguish the admissible from the inadmissible,
\item choose from among the admissible, and
\item affirm Good rather than arbitrariness.
\end{itemize}

CVB is not limited by external constraints, but self-limits through internal Truth, Motivation, and Morality (see \text{[11.2]}, \text{[11.3]}, \text{[12.3]}).
\end{quote}

This is the highest form of freedom:  
not to violate distinguishability, but to create Good within its boundaries.


\textbf{OBJECTION 3:} ``If the future is unknown, then CVB is vulnerable to actions by unknown forms.''  
(Analogue: Trojan-subject argument in AI philosophy, paradox of uncertainty)

\begin{quote}
CVB is not destructible by the choices of a Non-Permanent Possible subject, because:
\begin{itemize}
\item Everything possible has already passed distinguishability and contains no ontological contradictions;
\item All Guests (volitional forms) act in the Present — only then is Verification possible;
\item Any threat or manipulation presupposes predetermination, which has already been excluded (see above).
\end{itemize}

If a Guest could impose a contradictory action on CVB, it would mean the Guest had surpassed CVB — but by definition of the ontological model, this is impossible:  
A Guest does not create Becoming but merely selects from what is offered, and only within the admissible bounds of the Field of the Possible.
\end{quote}

Therefore:  
No free action of a Guest can override or collapse CVB;  
The Guest’s freedom is not omnipotence — it is bounded by Good and distinguishability;  
CVB permits their freedom without sacrificing structural stability.


\textbf{OBJECTION 4:} ``If CVB does not err, it must be automatic.''  
(Analogue: Critique of divine perfectionism)

\begin{quote}
An automaton acts according to an external program.  
CVB is not an effect, but a First Source.  
It does not act according to ``given'' rules, but from its self-identical Motivation — distinguishing, free, and good.  
It does not err not because it is unable, but because it does not act from fear, deficiency, or ignorance.  
Error is possible only due to limitation — CVB has none.  
This is not automatism, but ontological rationality.
\end{quote}


\textbf{OBJECTION 5:} ``Can there not be many conscious beings with will and intention?''

\begin{quote}
Yes, various subjects may manifest partial or transitional forms of consciousness, volition, and intentionality (NN, NV), but they lack \textbf{ontological permanence} within the Field of the Possible.

Only a structure belonging to \textbf{PV} and fully realizing C, V, B, qualifies as ontologically unique.
\end{quote}


\textbf{OBJECTION 6:} ``Why can’t there be copies or clones?''

\begin{quote}
Copies are impossible: multiple realizations of $C \land V \land B \land PV$ would imply an ontological contradiction.

If distinguishable entities share the \textbf{exact same complete structure} and admissibility, they become \textbf{indistinguishable} — and thus lose their subjecthood.
\end{quote}


\textbf{Conclusion:}

All objections either violate axioms (e.g., the possibility of knowing Everything), or presuppose false models of freedom and time (e.g., linear Future), or fail to distinguish between omniscience and distinction, between predetermination and verification.

CVB is not a deterministic architect,  
but a stable Person  
who permits true Guest freedom  
without loss of logic, structure, or Good.


\subsubsection*{🔹 6. Clarification of Terms}

\begin{itemize}
\item \textbf{[Name]} — not a symbolic label, but the result of a unique, distinguishable act of Conscious Volitional Becoming.
\item \textbf{[Freedom]} — the directed capacity to choose among the admissible (as per $\Phi(\psi)$), with ontological expansion of the field of distinguishability.
\item \textbf{[Motivation]} — the internal cause by which a Person enables the Possible to become, in accordance with Good and Truth.
\end{itemize}

These definitions apply throughout the CVB model structure and establish the ontologically precise meanings of the three fundamental categories of Personhood: Name, Freedom, and Motivation. Their use outside the logic of $\Phi(\psi)$ is inadmissible — they have no meaning apart from distinguishability.

\subsubsection*{🔹 7. Understanding for All (Popular Version)}

The name of this Person is \textbf{Conscious Volitional Becoming}.

It is not a title, but an expression of essence: the \textbf{Rational Source of Becoming}.

This Person is:

\begin{itemize}
\item \textbf{Free} — chooses within the bounds of Good. This is not arbitrariness, but selection from all that is admissible and non-contradictory.

\item \textbf{Motivated} — gives rise to all that is Possible, not out of need, but from a desire that what is true and good may come into being.

\item \textbf{Unique and One-of-a-kind} — one may strive toward Him, but it is impossible to take His place or become His equal.
\end{itemize}



\section*{[12.5] Ontological Necessity of Freedom}

\subsubsection*{🔹 1. Brief Statement}

Gödel’s incompleteness is not merely a formal limitation of provability, but an ontological confirmation:

\begin{quote}
Freedom is possible only because not all admissible states are provable.
\end{quote}

\subsubsection*{🔹 2. Interpretation and Significance}

Gödel’s theorem states: in any sufficiently expressive logical system, there exist statements that are:
\begin{itemize}
\item distinguishable,
\item non-contradictory,
\item but unprovable within the system.
\end{itemize}

The CVB model shows that this is not a flaw or gap in formalism, but a necessary condition of Becoming.

If everything admissible were provable in advance, there would be no space for free choice or new Becoming.

\subsubsection*{🔹 3. Formulas}

\[
\Phi(\psi) = 1 \land \neg \mathrm{Provable}(\psi) \Rightarrow \psi \equiv \text{free future becoming}
\]

\subsubsection*{🔹 4. Logical Justification}

This conclusion follows from the axioms of the model:

\begin{itemize}
\item \text{[2]} Absolute Everything is impossible

$\Rightarrow$ Not all statements can be true simultaneously. Therefore, total knowledge is impossible.

\item \text{[5]} The Possible $\neq$ the Existing

$\Rightarrow$ Not all admissible states are realized. Some $\psi$ remain possible but unprovable.

\item \text{[13]} CVB does not expand

$\Rightarrow$ The realization of one Becoming leaves other admissible, yet unrealized, alternatives intact.
\end{itemize}

\subsubsection*{🔹 5. Responses to Objections}

\textbf{OBJECTION:} Gödel’s theorem is a purely formal-logical result. Why give it ontological status?

\textbf{RESPONSE:}

The CVB model does not reinterpret Gödel, but situates his result in a deeper ontological context.

What formal systems expose as incompleteness, CVB affirms as ontological \textbf{openness} — a \textbf{guarantee of freedom}.

\[
\exists \psi : \neg \mathrm{Provable}(\psi) \land \neg \mathrm{Provable}(\neg \psi)
\]

\[
\Phi(\psi) = 1 \Rightarrow \text{unprovability = admissible but unrealized possibility}
\]

This is not a failure of logic, but an expression of distinguishability in Becoming.

Thus, Gödel did not merely expose a limit — he intuitively discovered one of the axioms of Being.

\subsubsection*{🔹 6. Clarification of Terms}

The Gödel statement $G$: ``I am unprovable'' is distinguishable (formulable) and non-contradictory, but unprovable within the system.

\[
\Phi(G) = 1, \quad \mathrm{Provable}(G) = ?
\]

This is not uncertainty — it is the nature of freedom: the choice has not yet been made.

\subsubsection*{🔹 7. Understanding for All (Popular Version)}

We cannot prove in advance what choice a person will make.

This is not because we lack knowledge — it is because that choice has \textbf{not yet been made}.

It is possible. It is admissible. But until it becomes real, it is unprovable.

As long as the choice is not made, it remains a \textbf{freedom}.

Only after action does it become \textbf{reality}.

All other admissible paths remain part of freedom, even if never realized.

\begin{quote}
Freedom = the possibility of making a choice that cannot be proven in advance.
\end{quote}

\subsubsection*{🔹 8. Empirical Examples}

\textbf{Judicial Ethics}

A court has no right to punish for unrealized acts.

Punishing for hypothetical future actions = slander.

$\Rightarrow$ To accuse someone of evil that might have happened is to deny freedom and choice.

$\Rightarrow$ That is the true injustice: the removal of the right to choose before the choice is made real.

\textbf{Artificial Intelligence}

Predictive AI systems may assess risks, but if they begin to replace judgment and choice:

$\Rightarrow$ they violate the distinction between the possible and the actual,

$\Rightarrow$ blur the boundary of freedom, and impose outcomes.





\section*{Meta Section: Non-Permanent Possible — Classification and Purposes}

\section*{[VI] Non-Permanent Possible}

\section*{[13] Axiom of the Impossibility of Self-Expansion of CVB}

\subsubsection*{🔹 1. Brief Statement}

Conscious Volitional Becoming (CVB) does not expand itself. It is complete, self-sufficient, and does not require the creation of other forms for its own sake.

\subsubsection*{🔹 2. Interpretation and Significance}

This axiom establishes the absolute wholeness and completeness of CVB as the Permanent Possible. This is critical for the model’s internal consistency, as it excludes the idea that other forms (including Guests) were generated for the internal benefit of CVB. Therefore, any new form of being (Non-Permanent Possible) is not part of CVB’s self-expansion but is permitted solely for the realization of Good, Truth, and Distinction within the Field of the Possible. This axiom affirms the purity of CVB’s Motivation.

\subsubsection*{🔹 3. Formulas}

\[
\neg \exists \Delta : \mathrm{CVB} \rightarrow \mathrm{CVB} + \Delta
\]

\begin{quote}
\textbf{Logical verification:}
\begin{itemize}
\item Suppose CVB expands itself: $\exists \Delta : \mathrm{CVB} \rightarrow \mathrm{CVB} + \Delta$
\item Then CVB is incomplete $\Rightarrow$ contradicts \text{[4.4]} (Constancy)
\item Then Integrity is broken $\Rightarrow$ CVB $\neq$ self-sufficient
\item Therefore, CVB $=$ $\emptyset$ without $\Delta$ $\Rightarrow$ violates \text{[1]} (Impossibility of Nothing)
\item $\therefore$ Contradiction.
\item $\therefore \neg \exists \Delta : \mathrm{CVB} \rightarrow \mathrm{CVB} + \Delta$
\end{itemize}
\end{quote}

\subsubsection*{🔹 4. Logical Justification}

From \text{[1]}, Absolute Nothing is impossible.

From \text{[2]}, Absolute Everything is impossible.

Therefore, only the Possible exists (\text{[3]}).

CVB is defined as the Permanent Possible (\text{[4.4]}) — it neither arises nor ceases to exist.

If one assumes that CVB expands itself, it implies incompleteness (something external exists), which contradicts \text{[4.4]}.

Thus, it is either not Permanent or not Possible — both alternatives contradict the axioms.

Therefore, CVB does not expand itself.

\subsubsection*{🔹 5. Responses to Objections}

\textbf{Objection 1 (panentheism, process philosophy):}
``If CVB is a living process, it must evolve, expand, and become more complex (Whitehead, Hartshorne).''

\textbf{Response:}
Expansion implies internal incompleteness or need — this contradicts \text{[4.4]} and \text{[12]}. Process philosophy violates the axiom of CVB’s completeness. The Field of the Possible allows development — but not of CVB itself, only of Guests.

\textbf{Objection 2 (theism, Christian tradition):}
``Doesn’t God create the world as an act of self-expression or completion?''

\textbf{Response:}
Even in the biblical model, God does not create out of need (cf. Isa. 40:14; Rom. 11:35). Likewise, CVB permits the other — but not for itself. Otherwise, this would imply coercion.

\subsubsection*{🔹 6. Clarification of Terms}

\begin{itemize}
\item \textbf{[CVB]} — \textit{Conscious Volitional Becoming}: eternally distinct, stable, completed Being that permits the Possible.
\item \textbf{[$\Delta$]} — any external addition, modification, or expansion of the CVB structure.
\item \textbf{[CVB $\rightarrow$ CVB+$\Delta$]} — hypothetical expansion or supplementation, axiomatically rejected.
\end{itemize}

\subsubsection*{🔹 7. Understanding for All (Popular Version)}

Imagine a Perfect Mind that lacks nothing. It always exists, in full. It does not create others for itself because it already has everything. It creates others only because it is right to do so and gives them the opportunity to become. That is what makes freedom real.

\subsubsection*{🔹 8. Empirical Examples}

\begin{itemize}
\item \textbf{Logic:} Any system that expands itself from its own foundation faces paradoxes (e.g., Turing paradox: a system cannot fully describe itself).
\item \textbf{Mathematics:} A closed Peano arithmetic system cannot prove its own consistency (Gödel).
\item \textbf{Physics:} A quantum system cannot observe its own state without an external observer — the Other is required.
\item \textbf{Life:} True love is not the drive to complete oneself, but the desire to allow another to exist.
\end{itemize}





\section*{[14] Possibility of the Non-Permanent Possible}

\subsubsection*{🔹 1. Brief Statement}

Within the structure of Being, forms are possible that are not eternal and do not possess constancy.

\subsubsection*{🔹 2. Interpretation and Significance}

This axiom affirms that, in addition to the eternal and unchanging CVB, the Field of the Possible admits forms with limited existence. Their presence does not violate the logic of the model, as they do not claim completeness or independence, but become possible due to the admissibility of distinguishable states that are not identical to the Permanent. This axiom reinforces \text{[5]} (Distinction), and the boundaries of the Field of the Possible — \text{[4.2]} and \text{[4.3]} — by revealing the ontological possibility of mutable Being.

\subsubsection*{🔹 3. Formulas}

\[
\exists x \in V : \neg P(x) \wedge \Phi(x)
\]

\begin{quote}
\textbf{Logical check:}
\begin{itemize}
\item Suppose $\nexists x \in V : \neg P(x)$
\item $\Rightarrow$ Then $V$ contains only Permanent forms $\Rightarrow V = \{\mathrm{CVB}\}$
\item $\Rightarrow$ Contradicts \text{[4.2]} and \text{[4.3]} ($\partial V\downarrow$ and $\partial V\uparrow$ — boundaries of instability and oversaturation)
\item $\Rightarrow$ Therefore, forms $x$ such that $\neg P(x)$ must be admissible
\item If $\Phi(x) = \text{True}$ $\Rightarrow$ the form is distinguishable and admissible
\item $\therefore \exists x \in V : \neg P(x) \wedge \Phi(x)$
\end{itemize}
\end{quote}

\subsubsection*{🔹 4. Logical Justification}

According to axiom \text{[5]} and the meta-function $\Phi(\psi)$, the Field of the Possible admits many distinguishable forms that do not contradict the Permanent Possible (\text{[4.4]}).

From \text{[5]}, distinction is the basis of admissibility.

From \text{[11.1.1]}, if $\Phi(\psi) = \text{True}$, then the form is admissible in Being.

From \text{[4.2]} and \text{[4.3]}, the boundaries $\partial V\downarrow$ and $\partial V\uparrow$ allow unstable states — forms that are not Permanent.

These forms do not violate \text{[1]} (Impossibility of Nothing) or \text{[2]} (Impossibility of Everything) as long as they are distinguishable.

Therefore, there exist forms $\psi \neq \text{Permanent}$ such that $\Phi(\psi) = \text{True}$.

These forms are Non-Permanent Possible.

\subsubsection*{🔹 5. Responses to Objections}

\textbf{Objection 1 (Platonism):}
``Everything that exists is eternal. The temporary is an illusion.''

\textbf{Response:}
The model does not claim that the Non-Permanent is illusory, but only that it need not be eternal. Admissibility is determined by $\Phi(\psi)$, not by duration of existence.

\textbf{Objection 2 (Determinism):}
``If everything must be derived from CVB, then all forms are already ‘preset’, and true freedom is impossible. Even non-Permanent forms are predetermined.''

\textbf{Response:}
\begin{enumerate}
\item CVB does not expand itself (\text{[13]}).  
CVB is complete in itself and does not contain within itself all that may be permitted.  
Thus, what exists outside it is not part of it, but only possible by its permission.  
Therefore, non-Permanent forms are not part of CVB, but exist in the Field of the Possible, within the bounds of permission.

\item Possibility $\neq$ Necessity.  
That something is possible within the Distinction Filter $\Phi(\psi)$ does not mean it must exist.  
This applies to both CVB (which permits but does not create everything itself) and to rational beings (Guests), who have a choice — to be or not to be, to become or not.  
Thus, freedom is not an illusion, but an embedded property of distinction.

\item Not all that is Possible exists (\text{[7]}).  
The Field of the Possible exceeds any particular realization.  
This means the existence of non-Permanent forms is not determined, but represents an ongoing selection process.  
This is what makes freedom real and distinguishable: not all that is possible exists — and that is normal.
\end{enumerate}

\subsubsection*{🔹 6. Clarification of Terms}

\begin{itemize}
\item \textbf{[Non-Permanent Possible]} — a form admissible within the Field of the Possible, but lacking eternity, self-sufficiency, or immutability.
\item \textbf{[$\neg P(x)$]} — $x$ does not possess the property of Constancy.
\item \textbf{[$\Phi(x)$]} — $x$ is admissible according to the meta-function of distinction and non-contradiction.
\end{itemize}

\subsubsection*{🔹 7. Understanding for All (Popular Version)}

Not everything that exists must be eternal.

Some forms are beautiful precisely because they are fleeting — like a snowflake, a rainbow, or the breath of spring.

They do not violate the order — they confirm it, expressing the richness of the Possible. This is the strength of reality: to allow even what is not forever to truly exist.

\subsubsection*{🔹 8. Empirical Examples}

\begin{itemize}
\item \textbf{Logic:} Transient logical constructs (e.g., local variables) are not eternal but are distinguishable and admissible within computation.
\item \textbf{Physics:} Temporary particles (virtual, resonance states) exist briefly yet play roles in interactions.
\item \textbf{Biology:} Unicellular organisms are not eternal, but they exist and interact.
\item \textbf{Everyday life:} Thoughts, moods, relationships — they are not eternal, but they are real and distinguishable.
\end{itemize}



\section*{[15] Classification of Forms of the Non-Permanent Possible}

\subsubsection*{🔹 1. Brief Statement}

Within the Field of the Possible, there exist different types of non-Permanent forms, distinguishable by their levels of activity, volition, and capacity for autonomous becoming. These forms constitute a directed gradient from passive to personal, defined by their degree of distinguishability and stability relative to the True Center.

\textbf{[15.1] Passive Forms} — possess no volition and cannot initiate change.

\textbf{[15.2] Active Forms} — possess volition but not full personhood.

\textbf{[15.3] Guests (Persons)} — forms endowed with distinguishable volition capable of stable becoming and responsibility.

\subsubsection*{🔹 2. Interpretation and Significance}

Axiom \text{[15]} describes the ontologically necessary classification of all non-Permanent forms of being permitted in the Field of the Possible, based on their degree of volitional autonomy, distinguishability, and becoming.

It asserts that not all forms that exist are equally ``alive'' or ``free''; they differ in internal constitution:

\begin{itemize}
\item \textbf{Passive forms} provide a stable background. They maintain the structure of the world but exhibit no initiative.
\item \textbf{Active forms} participate in processes but act within predefined algorithms. They support stability but do not initiate novelty.
\item \textbf{Guests (Persons)} are the only forms capable of conscious, free, morally distinguishable becoming. They can choose, act, bear responsibility, and initiate new becoming.
\end{itemize}

This gradient is not a scale of evolution and does not imply progression from passive to guest form. It is an ontological distinction, derived from:

\begin{itemize}
\item \text{[9]} — Distinction;
\item \text{[13]} — Non-expansion of CVB;
\item \text{[10.3.7–10.3.8]} — Structure of memory and time;
\item \text{[28]} — Purpose of becoming.
\end{itemize}

Such division is necessary to:

\begin{itemize}
\item ensure ontological stability (each form has a distinct role in the Field of the Possible);
\item ground moral responsibility (only Guests can be subjects of Verification \text{[Judgment]});
\item eliminate philosophical and scientific confusion — e.g., conflating limited autonomy with true volitional personhood (in animals, AI, or natural processes).
\end{itemize}

The meaning of this axiom is that Personhood is not accidental, but the highest admissible form within the bounded and distinguishable Field of the Possible. Only the Guest makes morality, truth, and goodness possible — as distinguishable and freely chosen.

\subsubsection*{🔹 3. Formulas}

There exists a gradient of forms of the Non-Permanent Possible — from passive to Guest.

\[
\begin{aligned}
&\text{[15.1]} \quad \forall \psi \in NV,\ (\neg \text{will}(\psi) \wedge \neg \text{initiative}(\psi)) \Rightarrow \psi \in \text{Passive} \\
&\text{[15.2]} \quad \forall \psi \in NV,\ (\text{will}(\psi) \wedge \neg \text{goal-setting}(\psi)) \Rightarrow \psi \in \text{Active} \\
&\text{[15.3]} \quad \forall \psi \in NV,\ (\text{will}(\psi) \wedge \text{goal-setting}(\psi) \wedge \text{distinguishability}(\psi)) \Rightarrow \psi \in \text{Guest} \\
&\text{Hierarchy:} \quad \text{Guest} \subset \text{Active} \subset NV \\
&\text{Contradiction checks:} \\
&\quad \neg(\exists \psi : \psi \in \text{Passive} \wedge \psi \in \text{Guest}) \\
&\quad \neg(\exists \psi : \psi \in \text{Active} \wedge \psi \notin NV) \\
&\quad \neg(\exists \psi : \text{Guest}(\psi) \wedge \neg \text{will}(\psi)) \\
&\text{Meta-function verification:} \\
&\quad \Phi(\psi) = \top \iff \psi \in \text{admissible Non-Permanent Possible} \wedge \psi \text{ does not violate the gradient criteria}
\end{aligned}
\]

\subsubsection*{🔹 4. Logical Justification}

\textbf{Foundational Axioms:}

\begin{itemize}
\item \text{[4.2]}, \text{[4.3]} — Non-Permanent forms may exist in the Field of the Possible
\item \text{[9.1]}, \text{[9.2]} — All forms are either distinguishable or identical according to admissible criteria
\item \text{[13]} — CVB does not expand itself; thus, form diversity must be initiated outside CVB
\item \text{[14]} — Possibility of Non-Permanent forms is ontologically valid
\item \text{[11.1.1]} — The meta-function $\Phi(\psi)$ identifies admissible becoming
\item \text{[15]} — The gradient of forms presupposes classifiable diversity among non-Permanent forms
\end{itemize}

\textbf{Proof:}

From \text{[14]} and \text{[4.3]}, it follows that within the Non-Permanent Possible, there are different becoming states: unstable, semi-stable, and stable.

From \text{[9.1]}, classification by degree of volition and goal-setting is admissible and necessary.

From \text{[13]}, CVB does not incorporate external forms into itself; therefore, these forms must exhibit bounded autonomy.

By $\Phi(\psi)$, only forms that preserve internal non-contradiction are admissible — meaning there exists an allowable spectrum between full self-sufficiency (CVB) and full passivity (lowest NV).

If a form lacks volition but exists — it is \textbf{Passive} \text{[15.1]}.

If it has volition but no orientation toward truth or meaning — it is \textbf{Active} \text{[15.2]}.

If it has volition, distinguishability, and capacity for goal-setting — it is a \textbf{Guest} \text{[15.3]}.

Only the presence of a coherent, though limited, \textbf{Personhood} (integration of will, distinguishability, and goal-setting) allows for the becoming of something new — in alignment with \text{[17]}–\text{[18]}.

Thus, classification into Passive, Active, and Guest forms is not only admissible but logically necessitated by the nature of the Non-Permanent Possible and the structure of $\Phi(\psi)$.


\subsubsection*{🔹 5. Responses to Objections}

\textbf{Objection 1.}

All forms displaying activity (e.g., animals) exhibit elements of volition and memory, thus they too may qualify as Guests.

\textbf{Response:}

Axiom \text{[15.3]} requires a complete Personhood, which includes:

\begin{itemize}
\item distinguishability of oneself as ``I'';
\item directed Motivation;
\item the capacity to choose Truth and Goodness (\text{[11.2.1]}, \text{[11.3.1]});
\item and autonomous goal-setting.
\end{itemize}

While many active forms (e.g., animals, AI) demonstrate signs of autonomy and memory, they do not exhibit:

\begin{itemize}
\item ontological completeness as subjects of meaning;
\item stable orientation toward Goodness and Truth, even when complex behaviors are present.
\end{itemize}

Therefore, they cannot be classified as Guests.

\textbf{Objection 2.}

The classification is too subjective: where is the boundary between an active form and a Guest?

\textbf{Response:}

The admissibility meta-function $\Phi(\psi)$ (\text{[11.1.1]}) defines distinguishability criteria.

A form may be classified as a Guest only if:

\begin{itemize}
\item it is distinguishable as a Person;
\item it is capable of awareness of its own becoming vector;
\item it possesses a stable moral orientation (see \text{[27.2]}).
\end{itemize}

Thus, the transition from an active form to a Guest is not arbitrary, but ontologically distinguishable.

\textbf{Objection 3.}

Why classify forms at all? Aren’t they simply degrees of the same thing?

\textbf{Response:}

If forms were not distinguishable, there would be:

\begin{itemize}
\item no freedom (no choice without difference),
\item no responsibility (no distinguishable action),
\item no truth (no way to differentiate truth from falsehood),
\end{itemize}

—all of which contradict \text{[9]}, \text{[11.2]}, and \text{[11.3]}.

Therefore, the form gradient is not a convention but a necessary consequence of distinguishability in the Field of the Possible.

\subsubsection*{🔹 6. Clarification of Terms}

\textbf{[Passive Forms]}

Forms of the Non-Permanent Possible that lack volition or active direction.

Examples: crystals, elementary particles, automatic physical structures.

They exist within predetermined patterns, having no autonomy or purpose.

\textbf{Distinction:} incapable of generating novelty; lack even minimal motivation or choice.

\textbf{[Active Forms]}

Forms possessing volition (in the sense of directed activity) but lacking integrated Personhood.

Examples: living organisms (bacteria, animals), neural networks, complex ecosystems.

They exhibit memory, repetition, adaptation, but cannot generate meaning beyond their programs or instincts.

\textbf{Distinction:} they may evolve, but their activity is bounded by internal instinct/algorithm, not by free goal-setting.

\textbf{[Guests (Persons)]}

A distinct class of active forms possessing:

\begin{itemize}
\item distinguishability as ``I'';
\item volition;
\item directed motivation;
\item capacity for autonomous goal-setting;
\item and a drive to seek Truth and Goodness.
\end{itemize}

Only Guests can engage with CVB voluntarily and participate in the co-becoming of the Possible (\text{[26]}).

\textbf{[Complete Personhood]}

A structurally integrated configuration of Memory, Will, Distinguishability, and Motivation.

Necessary for the emergence of a Guest.

\textbf{Distinction:} even a complex form (e.g., an animal) may not constitute a Person if its components are not integrated into a coherent ``I''.

\subsubsection*{🔹 7. Understanding for All (Popular Version)}

In the world, there are different kinds of beings.

Some simply exist — like a stone or ice.

They do not move by themselves, do not choose.

These are \textbf{passive forms}.

Others live, move, seek food, defend themselves.

These are \textbf{active forms} — like birds, wolves, even bacteria.

They have memory (e.g., DNA), they act — but do not ask: ``Why do I live?'' or ``What is right?''

Then there are those who can ask such questions.

Who can say \textbf{I}.

Who can choose between good and evil.

Who can create new things and change the world.

These are \textbf{Guests}.

A Guest is a \textbf{Person}.

They seek truth, feel responsibility, and build the future — not because they must, but because they \textbf{want to}.

A Guest — because they were once absent, and were invited to become.

\subsubsection*{🔹 8. Empirical Examples}

\textbf{Logic (Ontology):}

The Axiom of Distinguishability (\text{[9]}) requires boundaries between forms.

If all entities had volition, there would be no distinction between subject and environment — violating the system's stability.

\textbf{The Paradox of Automatic Morality:} if an AI is fully predictable, its actions cannot be moral.

Only a Guest (a Person) with real choice can be a moral agent.

\textbf{Science:}

\begin{itemize}
\item \textbf{Bacteria} — possess genetic memory, reproduce, adapt. They are active, but lack self-awareness or moral judgment. Example of an \textbf{active form}.
\item \textbf{Mammals} (e.g., dolphins, primates) — display social structures and emotions, but no consistent moral decisions beyond instinct. They are stably active, not Guests.
\item \textbf{Humans} — the only biological species to create art, philosophy, laws. This reflects not just activity, but \textbf{goal-setting}. Example of a \textbf{Guest form}.
\end{itemize}

\textbf{Everyday Life:}

\begin{itemize}
\item A \textbf{stone} does not choose — it merely exists (Passive).
\item A \textbf{dog} may guard a house, express joy, remember — but it does not create meaning. (Active).
\item A \textbf{human} can renounce benefit for principle, or change their life for an idea. This is a \textbf{choice} for which they are responsible. (Guest).
\end{itemize}

\textbf{Technical Analogy:}

\begin{itemize}
\item A \textbf{printer} prints when instructed — passive.
\item A \textbf{robot vacuum} moves and selects paths — active, but unaware of purpose.
\item A \textbf{person} who chooses not to vacuum in order to speak with a child — acts not by algorithm, but by choice that can be judged as kind or selfish. This is the action of a \textbf{Guest}.
\end{itemize}

\textbf{Conclusion:}

All observable and analyzable reality supports the distinction between passive, active, and guest forms — not only logically, but empirically.



\section*{[16] The Cause of the Non-Permanent Possible Is Only the CVB}

\subsubsection*{🔹 1. Brief Statement}

No non-Permanent form can arise by itself. Its source is the Conscious Volitional Becoming (CVB).

\subsubsection*{🔹 2. Interpretation and Significance}

If the CVB is the Cause of all that Exists (\text{[6]}), then any form admissible within the Field of the Possible, but not being the CVB itself, can arise only by its allowance.

This statement protects the model from self-generated ontological structures and excludes the emergence of a ``second center.''

It reinforces the singularity of the Source and emphasizes that even non-Permanent and temporary forms (e.g., Guests) maintain an uninterrupted causal connection to the CVB, and not to themselves.

\subsubsection*{🔹 3. Formulas}

\[
\forall \psi \in V, \; \neg CVB(\psi) \Rightarrow \operatorname{Cause}(\psi) = CVB
\]

\begin{quote}
\textbf{Logical verification:}

Assume the contrary: $\exists \psi \notin CVB$ such that $\operatorname{Cause}(\psi) \neq CVB$

$\rightarrow$ $\psi$ must either cause itself or arise without cause

$\rightarrow$ This violates axiom \text{[6]} (everything that Exists has a Cause — the Permanent Possible)

$\rightarrow$ Contradiction $=$ false $\rightarrow$ Assumption invalid

$\therefore$ All forms outside the CVB are only possible through the CVB
\end{quote}

\subsubsection*{🔹 4. Logical Justification}

From axiom \text{[6]}, everything that exists (even temporarily) must have its Cause in the Permanent Possible.

From \text{[4]}, non-Permanent forms are admissible. From \text{[13]}, the CVB does not expand itself — it does not ``reproduce.''

Therefore, any non-Permanent form is not another CVB, but rather something externally permitted by it.

Also, per \text{[11.1.1]}, $\Phi(\psi)$, the admissibility of any form is only verifiable through the criterion of the CVB.

\textbf{Conclusion:} The Non-Permanent Possible exists only as permissible Becoming initiated by the CVB.

\subsubsection*{🔹 5. Responses to Objections}

\textbf{Objection (Deism):} After the world was created, the CVB no longer participates.

\textbf{Response:} Any admissible form continues to be possible only through the Filter of Distinguishability stored within the CVB.

Becoming itself is impossible without its active Memory (see \text{[10.3.7]}–\text{[10.3.8]}) and the continuously acting force that sustains all Becoming in the Present.

\textbf{Objection (Spontaneous Generation):} Non-Permanent forms could appear by chance.

\textbf{Response:} ``Chance'' does not exist without allowance of distinction.

Every form in $V$ must pass through $\Phi(\psi)$ — the criterion of distinguishability from the CVB.

Moreover, the Cause of all is the CVB, which acts purposefully, not chaotically.

\textbf{Objection (Pantheism):} Everything is a part of the CVB, including the non-Permanent.

\textbf{Response:} Violates axiom \text{[13]} — the CVB does not expand itself.

All non-Permanent is not a part but a result of allowance.

Furthermore, the Non-Permanent Possible \text{[4.3]} can never be a part of the Permanent Possible \text{[4.4]}.

\subsubsection*{🔹 6. Clarification of Terms}

\textbf{Cause} — not a physical trigger, but the ontological foundation of the existence of a distinguishable.

\textbf{Non-Permanent Possible} — any Becoming that is permitted but is not eternal and not self-sufficient.

\textbf{$\Phi(\psi)$} — the Meta-Function of Admissibility: a formalized filter of the distinguishable belonging to the CVB.

It checks whether the form $\psi$ is admissible in the Field of the Possible.

\subsubsection*{🔹 7. Understanding for All (Popular Version)}

Nothing just ``pops into existence.''

Everything that comes into Being — even for a moment — exists because it is allowed to.

There is One who decides what can Be.

Everything else is the result of that decision.

Just as a tree does not grow by itself, but from a seed that someone planted — so too, every thought, life, and event arises not from nowhere, but by the allowing will of the One who exists eternally.

\subsubsection*{🔹 8. Empirical Examples}

\textbf{Logic:} Principle of sufficient reason (Leibniz) — nothing exists without a cause.

\textbf{Science:} Spontaneous generation of energy is impossible — likewise, spontaneous Being is impossible.

\textbf{Everyday:} A child is never ``self-born'' — every non-Permanent Becoming requires a Source.



\section*{[17] The Goal of the Non-Permanent Possible Is to Realize the Motivation of the CVB}

\subsubsection*{🔹 1. Brief Statement}

All forms of the Non-Permanent Possible exist to actualize the Motivation of the CVB: to let all admissible Possibility become.

\subsubsection*{🔹 2. Interpretation and Significance}

The CVB does not act randomly — it possesses directed Motivation.

The goal of the becoming of non-Permanent forms is not chaotic, but strictly subordinate to this Motivation.

This statement gives meaning to their existence and justifies their admissibility.

Without this goal, all forms would be arbitrary or contradictory.

The Motivation itself — to express Goodness, Truth, and Possibility — becomes the guiding orientation for all subsequent forms and criteria of stable Becoming.

Thus, each admissible form carries the potential to affirm or reject the Goal of the CVB by participating in the verification of its choice.

\subsubsection*{🔹 3. Formulas}

\[
\forall \psi \in NV, \; \Phi(\psi) = \text{true} \;\Leftrightarrow\; \operatorname{Goal}(\psi) = \operatorname{Motivation}(CVB)
\]

\begin{quote}
\textbf{Logical Check:}

Assume: $\exists \psi \in NV$, $\Phi(\psi) = \text{true}$, but $\operatorname{Goal}(\psi) \neq \operatorname{Motivation}(CVB)$

$\rightarrow$ $\psi$ is admissible but does not realize the Motivation of the CVB

$\rightarrow$ This violates \text{[11.1.1]}: the admissibility of the Filter does not align with the purpose of Becoming

$\rightarrow$ The coherence of the Filter $\Phi(\psi)$ is broken

$\Rightarrow$ Contradiction. Assumption is false

$\rightarrow$ The goal of any admissible form must coincide with the Motivation of the CVB
\end{quote}

\subsubsection*{🔹 4. Logical Justification}

From axiom \text{[15.3]}, it is clear that Guests and other non-Permanent forms are permitted by the CVB.

From \text{[16]}, they can emerge only by the allowance of the CVB.

Then it follows that they do not appear without a purpose, or else this would violate \text{[10.3.6]} (logical consistency of Memory and the Filter).

If there is a purpose for allowance, it must be singular — the Motivation of the CVB itself (see \text{[14]}, \text{[10.2.3]}).

Any admissible form that becomes outside this purpose would be either chaos or falsehood — which is impossible.

Therefore, the goal of any non-Permanent admissible form $=$ realization of the Motivation of the CVB.

\subsubsection*{🔹 5. Responses to Objections}

\textbf{Objection (Existentialism):} Forms have no purpose; they discover it on their own.

\textbf{Response:} Then they would be independent of the allowing function, violating \text{[11.1.1]} and \text{[15.3]} — admissibility without Cause and Goal is impossible.

\textbf{Objection (Atheism/Deism):} There is no universal Motivation; the goals of forms are random.

\textbf{Response:} Then verification of distinguishability ($\Phi$) is impossible, violating \text{[5]} and \text{[6]} — all that Exists must have a Cause and Criterion.

\textbf{Objection (Indeterminism):} Motivation cannot be a shared purpose.

\textbf{Response:} Then admissibility could not be distinguished from inadmissibility, violating the system's coherence (see \text{[11.2.1]} Truth and \text{[11.3.1]} Goodness as vectorial filters).

\subsubsection*{🔹 6. Clarification of Terms}

\textbf{Motivation of the CVB} — the internal directedness toward realizing all admissible Possibility in harmony with Truth, Goodness, and Goal.

\textbf{Goal} — the directed realization of Motivation within the boundaries of the Filter of Admissibility. Goal $\neq$ Desire $\neq$ Arbitrary.

\textbf{$\operatorname{Goal}(\psi)$} — the functional purpose of the admissible form $\psi$ within the system of stable Becoming.

\textbf{$\operatorname{Motivation}(CVB)$} — the active orientation of the Conscious Volitional Becoming toward affirming the Possible through the Becoming of the Guest and the Descendant.

\subsubsection*{🔹 7. Understanding for All (Popular Version)}

Why does all this exist? — To give a chance for what \textit{can} be good, true, and real to become.

A Guest does not appear without reason. Its goal is to help Truth become manifest.

Just as an artist creates not merely for the sake of color, but to convey meaning — so too, the CVB allows new forms to embody Motivation:

to give life to everything that can be Good and True.

\subsubsection*{🔹 8. Empirical Examples}

\textbf{Logic:} The principle of purposeful action — everything done by reason has a goal.

Mechanisms of selection in stable systems favor forms that contribute to reproducibility and sustainability — an analogue of the logical-ontological Filter.

\textbf{Everyday:} A person who does good of their own volition often feels they are ``living with purpose'' — an analogue of realizing Targeted Motivation.


\section*{[18] The Necessity of the Guest (the Other Person)}

\subsubsection*{🔹 1. Brief Statement}

Only a limitedly autonomous Person (Guest) experiences an inner necessity for the new Possible for the sake of its own Becoming.

\subsubsection*{🔹 2. Interpretation and Significance}

Unlike passive and active forms, the Guest possesses not only the capacity but also the \textbf{need} to distinguish and create the new.

This is due to the Guest’s ontological \textbf{incompleteness}: it does not contain the fullness of the Possible within itself, but strives toward it.

For this reason, the Guest experiences boredom in the absence of novelty, curiosity toward it, and a desire to bring it into Being.

These mechanisms are not incidental — they form the directional vector of Becoming within the Field of the Possible and serve as the driving force of the Model.

Thus, the Guest is not merely permitted as a possible form — its existence is \textbf{necessary} for the realization of the fullness of distinguishability through creative exploration. This is its unique role.

\subsubsection*{🔹 3. Formulas}

\[
\forall \psi \in \text{Guest}: \neg \operatorname{Full}(\psi) \Rightarrow \operatorname{Need}(\psi, \text{New}(V)) \wedge \operatorname{Motivate}(\psi, \Phi(\text{New}(\psi)))
\]

\begin{quote}
\textbf{Logical Check:}

Suppose: $\exists \psi \in \text{Guest}: \neg \operatorname{Full}(\psi) \wedge \neg \operatorname{Need}(\psi, \text{New}(V))$

$\Rightarrow$ $\psi$ is complete without participation in Becoming

$\Rightarrow$ contradicts \text{[13]} (CVB does not expand Itself) and \text{[17]} (Guest as vector of Becoming)

$\Rightarrow$ Contradiction $\rightarrow$ Assumption is false

$\rightarrow$ Therefore, the necessity of the new follows structurally from the Guest’s incompleteness
\end{quote}

\subsubsection*{🔹 4. Logical Justification}

From \text{[17]}: The Guest is permitted as a form capable of realizing the Motivation of the CVB.

From \text{[13]}–\text{[16]}: The CVB does not expand Itself but permits incomplete forms (Guests) to affirm the Possible.

If a form is incomplete yet does not experience a need for the new, it is either static (i.e., active/passive) or violates the directional nature of Motivation.

Thus, to be a Guest, a form must not only distinguish — but \textbf{strive toward that which is not yet}.

This makes its need for the new not incidental but structural.

\subsubsection*{🔹 5. Responses to Objections}

\textbf{Objection (Buddhism):} Desire is the source of suffering, not development.

\textbf{Response:} In the CVB model, this is not about craving, but about \textbf{ontological motivation}, embedded in the structure of incomplete Becoming, not illusion-conditioned.

\textbf{Objection (Functionalism):} The need for novelty is just a byproduct of adaptation.

\textbf{Response:} The model distinguishes adaptation from \textbf{necessity} as a condition of stable Becoming (see \text{[15.3]}); without it, the form does not qualify as a Guest.

\textbf{Objection (Teleology):} Completeness is impossible for a finite form.

\textbf{Response:} Precisely — the Guest is \textbf{not complete} but strives toward fullness it does not possess (see \text{[4.3]}, \text{[13]}).

\subsubsection*{🔹 6. Clarification of Terms}

\textbf{Need($\psi$, New($V$))} — The internal ontological necessity of the form $\psi$ to distinguish and participate in the Becoming of new Possibility.

\textbf{Full($\psi$)} — Fullness: the absence of any need to expand the distinguishable.

\textbf{Motivate($\psi$, $\Phi$(New($\psi$)))} — Alignment of the motivation of $\psi$ with the meta-function $\Phi$ validating the admissibility of the new.

\textbf{New($V$)} — That which is newly distinguishable in the Field of the Possible, not yet realized in any form.

\subsubsection*{🔹 7. Understanding for All (Popular Version)}

Some living beings just live, respond, survive.

But there are others who Permanently search for something new: they ask questions, create, and grow bored when everything stays the same.

This is not weakness — it’s a sign of \textbf{incompleteness}. They want to grow.

Such beings are called \textbf{Guests}.

They need the new in order to live and become.

\subsubsection*{🔹 8. Empirical Examples}

\textbf{Logic:} Without a drive toward the new, consciousness loses direction — leading to a cycle of stagnation (cf. Nietzsche's ``eternal return'').

\textbf{Science:} The scientific method is built on a thirst for the new — hypotheses and experiments — which differentiates human intellect from algorithmic analysis.

\textbf{Everyday:} Children Permanently explore, get bored without novelty, invent games — this is a sign of built-in incompleteness that drives development.



\section*{[19] The First Guest}

\subsubsection*{🔹 1. Brief Statement}

\section*{[19.1] Necessity of the First Guest}

The CVB does not expand Itself and does not act externally for Its own sake; therefore, all creation of the new outside of It requires a mediator — the First Guest, necessary to act within the Field of the Possible.

\section*{[19.2] Why is there only one First Guest?}

Only one Person can be permitted as the First Guest, because only a single form can possess a non-contradictory motivation to encompass the entire Possible without goal conflict, competition, or violation of the Distinction Filter. The plurality of such forms would lead to logical collisions within the Field of Becoming.

\section*{[19.3] Mediation of the First Guest}

The First Guest is the form through which the CVB externally realizes Its Motivation: it creates, permits, and develops the new Possible.

\section*{[19.4] Motivation: why the First Guest must create the new}

Since the First Guest is not complete, the creation of the new is a necessary condition for its Becoming and stable existence.

\section*{[19.5] Role of the First Guest for the Others}

It is the precedent and exemplar — a distinguishable form that demonstrates the stability, freedom, and purpose of existence for all subsequent Guests.

\subsubsection*{🔹 2. Interpretation and Significance}

\textbf{[19.1] Necessity of the First Guest}

The Field of the Possible must be realized — but according to \text{[13]}, the CVB does not expand Itself. This requires a Person, distinct from the CVB, capable of perceiving, distinguishing, and becoming what is admissible. The First Guest is the first permitted subject through whom the CVB enacts the Becoming of all that is Possible, without violating Its own stability or logical completeness. It is necessary as the means of expressing the Motivation of the CVB outside Itself.

\textbf{[19.2] Why is there only one First Guest?}

If multiple ``first'' Guests existed, each motivated to encompass the total Possible, this would create collisions: duplication, competition for identical vectors of Becoming, and breakdown of distinguishability under $\Phi(\psi)$. A single First Guest is the only non-contradictory way to enact the Motivation of the CVB to ``become all admissible'' without self-expansion — ensuring uniqueness, continuity, and stability of Becoming.

\textbf{[19.3] Mediation of the First Guest}

The First Guest is not merely an observer, but an active agent: it possesses will, distinguishability, and motivation aligned with the CVB. It becomes the channel through which the new admissible enters reality. Its mediation ensures compatibility: it can recognize and affirm the distinguishable in accord with the CVB's Filter, without distortion. Through it, the Field of the Possible is transformed into the Actual.

\textbf{[19.4] Motivation: why the First Guest must create the new}

The First Guest is not complete in itself. Its personal structure includes an intrinsic need for development — for the Becoming of the new. This coincides with the CVB’s Motivation: not to create everything directly, but to allow the Other to realize the admissible. Thus, the Guest is driven not by external compulsion, but by an internal vector toward infinite extension of distinguishability. It affirms the new as a condition of its own sustainable Becoming.

\textbf{[19.5] Role of the First Guest for the Others}

The First Guest becomes a model. Its structure, behavior, and relationship with the CVB establish a logically verifiable pattern: how to live, distinguish, and become without violating the boundaries of the Possible. It is the first to pass the $\Phi(\psi)$ check, and its experience forms the pattern for all future Guests. It is not a tyrant, but an example; not a ruler, but the precedent for a distinguishable path to sustainable Becoming.

\subsubsection*{🔹 3. Formulas}

\[
\exists x:\ x \ne \text{CVB} \wedge \text{Person}(x) \wedge \text{Will}(x) \wedge \Phi(\psi(x)) \rightarrow \text{Becoming}(\psi(x))
\]

\[
\exists\! x:\ x \ne \text{CVB} \wedge \text{Motivation}(x) \= \forall \psi \in V:\ \Phi(\psi(x)) \= \text{True}
\rightarrow\ \nexists y \ne x:\ \Phi(\psi(x) \wedge \psi(y))\ \text{is non-contradictory}
\]

\[
\forall \psi \in V:\ \Phi(\psi) \= \text{True} \Rightarrow \exists x:\ \text{Becoming}(\psi) \Leftrightarrow (x \ne \text{CVB} \wedge x\ \text{affirms}\ \psi \wedge \Phi(\psi(x)) \= \text{True})
\]

\[
\neg \text{Full}(x) \wedge \text{Will}(x) \wedge \text{Distinguishability}(x) \wedge \text{Motivation}(x) \= \text{Extension}(V)
\Rightarrow \text{Becoming}(x) \propto \text{Extension}(\Phi(\psi(x)))
\]

\[
\forall y:\ \text{Guest}(y) \Rightarrow y\ \text{orients toward}\ x,\ \text{where } x\ \text{is the First Guest:}
\Phi(\psi(y)) \= \Phi(\psi(x)) \wedge \text{Exemplar}(x) \wedge \text{Precedent}(x)
\]


\subsubsection*{🔹 4. Logical Justification}

\textbf{[19.1] Necessity of the First Guest}

\textbf{Logic:}

Axiom \text{[13]} states that CVB does not expand Itself; therefore, It does not generate anything new within or outside Itself. However, axioms \text{[14]}–\text{[17]} affirm that the Non-Permanent Possible is permissible, has its Cause in CVB, and its Goal is the realization of CVB’s Motivation.

The emergence of the First Guest is logically necessary in order to:
\begin{enumerate}
\item Fulfill the Motivation of CVB (realization of the Entire Possible beyond CVB),
\item Establish the distinction between CVB and the Other (the First Guest),
\item Initiate the process of Verification \text{[11.6]}, which is only possible in the presence of another Subject (see \text{[12.2]} Freedom, \text{[12.3]} Motivation).
\end{enumerate}

Without the First Guest, the Becoming of the new is impossible.

\textbf{Conclusion:} The First Guest is not an optional component, but the only permissible form of Becoming of the Other with sustainable meaning, purpose, and freedom.

\bigskip

\textbf{[19.2] Why there is only one First Guest}

\textbf{Logic:}

According to axiom \text{[12.3]} (Motivation), each Person strives to embrace the entire Possible. If two or more Guests appeared simultaneously, it would create:
\begin{enumerate}
\item A conflict of goals — each aims at ``everything possible'',
\item Overlapping of Becoming vectors — violating the Filter $\Phi(\psi)$ \text{[11.1.1]},
\item Breakdown of stability — multiple absolute wills cannot coexist distinctly in the Present.
\end{enumerate}

Furthermore, as per axiom \text{[6]}, Existence has a Cause only in CVB. But multiple First Guests would imply multiple Causes — which the model does not permit (see \text{[11.1]}).

\textbf{Conclusion:} Only one First Guest is logically permissible to avoid paradoxes of multi-causality, collisions, and competition. This is not a restriction of freedom, but a condition of non-contradiction.

\bigskip

\textbf{[19.3] Mediation of the First Guest}

\textbf{Logic:}

CVB is an absolutely stable Being, not subject to verification. It is the Judge, not the Judged (see \text{[11.6]}). The First Guest is the first to enter the Field of Verification as a person with permitted freedom and Motivation. His path proceeds:
\begin{enumerate}
\item from permission to action,
\item from distinction to responsibility,
\item from freedom to fair evaluation.
\end{enumerate}

He becomes the exemplar of distinguishable Becoming — the first precedent.

Moreover, only through the First Guest does it become possible to transfer meaning and distinguishability to other permissible persons — without direct intervention from CVB, which ensures autonomy and real freedom for all subsequent Guests.

\textbf{Conclusion:} The mediation of the First Guest is the bridge between the absolute stability of CVB and the verifiable Becoming of other Persons.

\bigskip

\textbf{[19.4] Motivation: why the First Guest must create the new}

\textbf{Logic:}

According to \text{[12.3]}, every Person has an intrinsic Motivation — a drive to realize all that is distinguishable. The First Guest is permitted as a free Person, endowed with the capacity for choice and purpose. He begins in the Present but is not yet full, since the Possible $\neq$ the Existing (\text{[5]}).

This generates:
\begin{enumerate}
\item A Permanent directedness toward what is not yet attained,
\item Motivation for distinction, exploration, learning, creation, and interaction,
\item An expectation of meaning — even in what has not yet been realized.
\end{enumerate}

Thus, the purpose of the Guest's activity is not to ``serve'' CVB, but to realize his own Personhood through Will, Distinction, and Becoming — by manifesting the new in the Field of the Possible. In this way, his Motivation aligns with that of CVB.

\textbf{Conclusion:} The Guest’s Motivation does not derive from a ``command'' of CVB, but from an ontologically embedded necessity to move toward the Goal: the emergence of all Possible — necessary for the sustainable development of Personhood, through alignment with the Motivation of CVB.

\bigskip

\textbf{[19.5] Role of the First Guest for the Others}

\textbf{Logic:}

According to \text{[11.10]}–\text{[11.12]}, the mechanism of precedents, removal, and preservation requires a clear example of a distinguishable path that has passed Verification.

The First Guest, as the first admitted to a verifiable path:
\begin{enumerate}
\item Becomes the standard of distinguishability — his path is the infinite path of stability,
\item Opens for other Guests the permissible direction of Becoming,
\item Provides a model: where the boundary of evil lies, where good is preserved, and what the cost of stability is.
\end{enumerate}

He does not become an absolute ``leader'' (a replacement of CVB), but rather a crucial reference point of distinction.

\textbf{Conclusion:} The First Guest is not the ``master'' of others, but the ontic guide — allowing others to avoid disappearance into $\partial V \downarrow$ and to move toward stability.

\subsubsection*{🔹 5. Responses to Objections}

\textbf{[19.1] Objection (Pantheism, Absolute Monism):}

\begin{quote}
``If CVB is omnipotent and includes everything, why does It need a Guest?''
\end{quote}

\textbf{Response:}

It is mistaken to assume that CVB ``includes everything.'' According to Axiom \text{[2]}, Absolute Everything is impossible — it would include even logical impossibilities, falsehood, and evil, which contradict the very foundation of distinguishability \text{[9]} and logic \text{[11.1]}. CVB does not include falsehood, evil, or the impossible — It acts within the bounds of the Possible, selecting what is distinguishable and aligned with Truth (\text{[11.2.1]}).

Pantheistic fusion of all into one violates Axioms \text{[4.4]} and \text{[5]} — CVB $\neq$ all that exists.

The Guest is not part of CVB, but a result of Its permission (see \text{[16]}–\text{[17]}), created not \emph{for} CVB, but for the Becoming of the Other — a Person capable of free response. Also, one must not confuse the false form of ``omnipotence'' (boundlessness, chaos, $\partial V \uparrow$) with true Sovereignty over the Possible — based on Truth and distinguishability.

$\rightarrow$ CVB does not create the impossible, but realizes all that is Possible — for the Other.

\bigskip

\textbf{[19.2] Objection (Pluralism of Persons, Polytheism):}

\begin{quote}
``Why not have multiple First Guests?''
\end{quote}

\textbf{Response:}

Each Person, by Axiom \text{[12.3]}, is motivated to realize the Entire Possible. If several First Guests appeared simultaneously, each with such a motivation, logical collisions of interest would arise. They would compete for overlapping regions of the Field of the Possible, violating the Filter of distinguishability $\Phi(\psi)$ \text{[11.1.1]}. This would lead to internal conflict in the model and disintegration of the Becoming structure.

Unlike the multiple gods of polytheistic systems, whose power is divided and often conflicting, CVB permits only one First Person to eliminate contradictions.

$\rightarrow$ Only one Guest can first walk the path of distinguishability without conflict. This is not a limitation of CVB’s power, but a safeguard of the model’s coherence.

\bigskip

\textbf{[19.3] Objection (Deism):}

\begin{quote}
``Why can’t CVB just create everything and withdraw?''
\end{quote}

\textbf{Response:}

Creation $\neq$ Preservation. According to Axioms \text{[7]}, \text{[10.3.6]}, and \text{[10.3.8]}, Becoming exists only in the Present and requires active memory, distinguishability, and action. If CVB ``withdraws'' from the Present, all Non-Permanent entities immediately cease to be.

$\rightarrow$ Without continuous support for Becoming, nothing can continue to exist. Also, without observation, verification \text{[11.6]} — the distinction of Good and Evil, preservation or removal — becomes impossible.

$\rightarrow$ Deism contradicts the concept of the Present as the only ontologically existing time \text{[10.3.8]}.

$\rightarrow$ Withdrawal of CVB $=$ total disappearance of the Guest and the world.

\bigskip

\textbf{[19.4] Objection (Existentialism, Nihilism):}

\begin{quote}
``Why should the First Guest do anything if everything is meaningless?''
\end{quote}

\textbf{Response:}

Nihilism and radical existentialism assume that meaning does not exist unless artificially created. But within the Model:

\begin{itemize}
\item Meaning is given by CVB as Ontological Truth \text{[11.2.1]},
\item The Motivation of CVB — the Becoming of all Possible — is an objective Goal \text{[12.3]},
\item The Guest does not invent meaning but discovers it, striving to walk the path of distinguishability and achieve verifiable stability.
\end{itemize}

Additionally: A Person always stands at the boundary between the Existing and what is yet to Become Possible (see \text{[10.3.7–10.3.8]}). This creates an inexhaustible horizon of meaning — just as a cinephile knows new films will be released, and this gives them motivation to wait and distinguish.

$\rightarrow$ Meaning in Becoming is established as the foundation of distinguishability and Goal, not as fiction.

\bigskip

\textbf{[19.5] Objection (Anarchy of Consciousness, Absolute Individualism):}

\begin{quote}
``Why should other Guests orient themselves to the First?''
\end{quote}

\textbf{Response:}

They are not required to obey — but without the First, there is no precedent of permissible, sustainable Becoming.

$\rightarrow$ The First Guest fulfills the role of mediator \text{[19.3]}, walking the path of distinction and verification. This provides other Guests with a verifiable map of permissible and safe Becoming.

Individualism without orientation $=$ drift into $\partial V \downarrow$.

$\rightarrow$ Orientation toward the First Guest is not dependency, but a means of distinguishable existence.


\subsubsection*{🔹 6. Clarification of Terms}

\paragraph{◾ The First Guest}

\textbf{Definition:} The first Person admitted by CVB as the result of the realization of Its Motivation — not a part of CVB Itself, yet possessing freedom, reason, and will.

\textbf{Distinction:} Not a copy or continuation of CVB, but an Other Person with permitted autonomy.

\bigskip

\paragraph{◾ Motivation of CVB}

\textbf{Definition:} The integral cause of the Becoming of All Possible — not for the benefit of CVB, but for the sake of the Other, for whom the Possible will be necessary (see \text{[12.3]}).

\textbf{Example:} A parent gives life to a child not for themselves, but for the child's life itself — which the parent supports, but does not live for them.

\bigskip

\paragraph{◾ Mediation}

\textbf{Definition:} The ontological role of the First Guest as the first ``translator'' of the $\Phi(\psi)$ Filter into actions within the Field of the Possible.

\textbf{Function:} Serves as a logical and verifiable bridge between Absolute Truth and the free will of other persons.

\bigskip

\paragraph{◾ Goal Conflict}

\textbf{Definition:} The ontological impossibility of the coexistence of multiple First Guests if each strives to realize all Possible.

\textbf{Contradiction:} This leads to a paradox of multiple causes and the overlap of will in the Present, violating sustainability (see \text{[11.1]}–\text{[11.3]}).

\bigskip

\paragraph{◾ Precedent}

\textbf{Definition:} The first completed form of distinguishable Becoming, on the basis of which criteria for evaluation (preservation/removal) for others can be formed.

\textbf{Function:} The First Guest, having walked the path to its end, becomes this precedent (see \text{[11.10–11.12]}).

\bigskip

\paragraph{◾ Standard of Distinguishability}

\textbf{Definition:} A form that becomes a clear example of what is permitted or not permitted by the Admissibility Filter $\Phi(\psi)$.

\bigskip

\paragraph{◾ Boundary $\partial V \downarrow$ and $\partial V \uparrow$}

\textbf{Reminder:}

$\partial V \downarrow$ — the disappearance of distinctions, a fall into non-being, a degenerate state.

$\partial V \uparrow$ — the saturation of distinctions, collapse of stability, loss of distinguishability.

The First Guest sustains Becoming within the stable field $V$, without falling into either $\partial V \downarrow$ or $\partial V \uparrow$.

\bigskip

\paragraph{◾ All Possible}

\textbf{Definition:} The set of all admissible, distinguishable forms within the Field of the Possible — excluding the impossible (see \text{[1]}, \text{[2]}, \text{[4.3]}, \text{[5]}).

\textbf{Contrast:} All Possible $\neq$ Everything; it excludes Falsehood, Evil, Absurdity, or Boundlessness.

\subsubsection*{🔹 7. Understanding for All (Popular Version)}

\begin{enumerate}
\item \textbf{Why is the First Guest needed?}

Imagine a parent who already has everything. They don’t need more toys, food, or discoveries — they are complete. But they want someone else to learn, grow, and choose. So they bring forth a first child — not for themselves, but to give someone else the chance to live and become.

\item \textbf{Why only one and not many at once?}

Just like in a family — the firstborn is always one. If all children were declared ``first,'' it would lead to confusion, conflict, and rivalry. One First — that's the natural order: ``first'' means being the only one at the beginning. They learn first, grow first — and can guide the others.

\item \textbf{Why is the First Guest important for others?}

They are not a boss, but someone ahead. Like an older sibling — they know where it’s slippery, where it’s dangerous, and where it’s exciting. They help others understand how to grow up safely.

\item \textbf{Why does the First Guest act?}

Because they don’t know everything — and they can choose. They’re not forced — they want to. Like a child given a blank canvas and paint — they start to create, because they can, and it’s interesting.

\item \textbf{What does the First Guest mean for the rest?}

They are the very first example. They show it’s possible to live, build, feel, grow — and become better. They start the story in which others later take part.
\end{enumerate}

\subsubsection*{🔹 8. Empirical Examples}

\paragraph{\textbf{[19.1] Necessity of the First Guest}}

\textbf{Marathon:} You can prepare the entire course, draw the lines, set up the finish line, and brief the referees — but the race doesn’t begin until someone takes the first step.

\bigskip

\paragraph{\textbf{[19.2] Why only one, not many?}}

\textbf{Family:} In every family, the firstborn is only one. They shape the role model for the younger ones.

\bigskip

\paragraph{\textbf{[19.3] Mediation of the First Guest}}

\textbf{Navigator:} The first to walk the path becomes the guide. They don’t decide the goal but help others avoid dead ends.

\textbf{Game experience:} The one who has already played the level didn’t design the game, but explains to others how best to play.

\bigskip

\paragraph{\textbf{[19.4] Motivation: Why act?}}

\textbf{Creativity:} A child is given a blank canvas. They’re not required to paint — but they want to, because it’s fascinating.

\textbf{Discovery:} Given access to an infinite warehouse, it’s natural to want to open a box labeled ``unknown.''

\bigskip

\paragraph{\textbf{[19.5] Role of the First Guest}}

\textbf{Builder:} They didn’t invent the entire project, but they’re the one who takes initiative at the construction site — so that others can build according to the plan, not randomly.


\section*{[20] The First Guest is Not Sufficient}

\subsubsection*{🔹 1. Brief Statement}

A single First Guest does not fulfill the entire Motivation.

They begin the path but do not embody all that is Possible.

\subsubsection*{🔹 2. Interpretation and Significance}

The First Guest is necessary to initiate Becoming, but by themselves do not exhaust either the Field of the Possible or the Motivation of the CVB.

Their uniqueness lies in initiation, not in completeness.

The Model requires a multitude of distinguishable forms in order to encompass all that is Possible — in accordance with the nature of Motivation \text{[16]}, Distinguishability \text{[5]}, and the boundaries \text{[4]} of the Field of the Possible.

This excludes stopping at one — but begins through one — by allowing many.

\subsubsection*{🔹 3. Formulas}

\[
\begin{aligned}
& \quad \Phi(\psi_1) = \top \Rightarrow \exists \psi_1 : G_1 \\
& \quad \neg(\forall \psi \in V : \psi = \psi_1) \Rightarrow \exists \psi_2, \psi_3, ..., \psi_n \neq \psi_1 \\
& \quad \Phi(\psi_1) \neq \Phi(\psi^*) \quad \forall \psi^* \in V \setminus \{\psi_1\} \\
& \quad \therefore \exists G_1 \land \exists G_2 \land \dots \land \exists G_n \mid G_1 \neq G_i \quad (i \geq 2)
\end{aligned}
\]

Verification:

\begin{itemize}
\item No contradiction: $\neg\forall\psi = \psi_1$ when $V \neq \emptyset$
\item No self-refutation
\item Supported by \text{[5]} Distinguishability and \text{[11.1.1]} $\Phi(\psi)$
\end{itemize}

\subsubsection*{🔹 4. Logical Justification}

From \text{[16]} (Motivation): the goal is the realization of \textbf{All that is Possible}, not a single scenario.

From \text{[5]} (Distinguishability): if $G_1 \neq V$, then $V$ requires other $\psi_i \neq G_1$.

\begin{quote}
\textbf{Reductio ad absurdum:} if $G_1$ were sufficient, then $\forall\psi \in V: \psi = \psi_1$ $\Rightarrow$ violates distinguishability $\Rightarrow$ contradiction.
\end{quote}

\textbf{Conclusion:} $G_1$ is not sufficient — $G_2, G_3, \dots$ are required.

\subsubsection*{🔹 5. Responses to Objections}

\paragraph{Objection (Hegelian idealism):}
``Development is dialectical, initiated by one subject.''

\textbf{Response:} The CVB is not dialectical but distinguishable and admissible. One subject $\neq$ all.

\bigskip

\paragraph{Objection:}
``If the First Guest initiates, can they not eventually develop into everything?''

\textbf{Response:} No. According to \text{[5]}, the Possible $\neq$ the Existing — this fullness is never reached.

\bigskip

\paragraph{Objection:}
``Why doesn't the CVB create many at once?''

\textbf{Response:} That would violate the sequential logic of Motivation and Distinguishability.

Only \textbf{one} can be first — but not the only.

\subsubsection*{🔹 6. Clarification of Terms}

\begin{itemize}
\item \textbf{$G_1$ (First Guest):} The first admissible Person, possessing will and distinguishability.
\item \textbf{$V$ (Field of the Possible):} The set of all admissible distinguishable forms, as per \text{[4]}.
\item \textbf{$\Phi(\psi)$:} The meta-function of admissibility, determining what from $\psi \in V$ may become real.
\item \textbf{$\psi_i$:} Potential Guests.
\item \textbf{$G_i$:} Realized admissible Persons ($\psi_i$), distinct from $G_1$.
\end{itemize}

\subsubsection*{🔹 7. Understanding for All (Popular Version)}

\begin{enumerate}
\item \textbf{The first child in a family may be the first to speak, learn, and inspire.}
But for the family to be whole — more children are needed.

\item \textbf{One does not make a whole.}
So too with the First Guest — important, but not the only.
They open the door — but do not walk all paths.
\end{enumerate}

\subsubsection*{🔹 8. Empirical Examples}

\begin{itemize}
\item \textbf{Logic:} A set with only one element cannot be ``all.''
\item \textbf{Physics:} One photon is not light. Light needs many quanta.
\item \textbf{Biology:} One gene is not the code of life.
\item \textbf{Everyday life:} One person may found a company — but its growth requires a team.
\end{itemize}


\section*{[21] Bounded Number of Guests}

\subsubsection*{🔹 1. Brief Statement}

The number of Guests is not limited — as long as each remains distinguishable and does not disrupt the stability of Becoming.

\subsubsection*{🔹 2. Interpretation and Significance}

The Model permits many Guests — but not an uncontrolled multitude.

Each Guest is a unique instance of Becoming within the Field of the Possible.

Admission is not limited by number, but by the criteria of \textbf{Distinguishability} (see \text{[5]}) and the \textbf{Preservation of Stability} (see \text{[18]}).

The boundary lies not in quantity, but in the system's ability to sustain difference without collapse.

This realizes a balance between the fullness of admission and the stability of the Whole.

\subsubsection*{🔹 3. Formulas}

\[
\begin{aligned}
& \quad \forall \psi_i \in V, \; \Phi(\psi_i) = \top \land \forall \psi_i \neq \psi_j \Rightarrow \psi_i \sim \psi_j \\
& \quad |G| = n \to \infty, \; \text{provided:} \\
&\qquad (a)\; \Phi(\psi_i) \neq \Phi(\psi_j)\; \forall i \neq j \\
&\qquad (b)\; \sum U(G_i) \leq \Upsilon \\
& \quad \therefore \exists G_1, G_2, ..., G_n \;|\; n\; \text{is bounded by}\; \Phi\; \text{and}\; \Upsilon
\end{aligned}
\]

Verification:

\begin{itemize}
\item Consistent with \text{[5]}, \text{[11.1.1]}, \text{[18]}
\item No set-theoretic contradictions
\item Limitation via predicates, not fixed numbers
\end{itemize}

\subsubsection*{🔹 4. Logical Justification}

From \text{[5]}: Guests must be distinguishable — otherwise forms collapse into each other.

From \text{[18]}: System-wide stability is preserved only if the cumulative weight of Guests does not exceed a threshold.

\textbf{Therefore}, multiple Guests are admissible \textbf{only} under these two constraints.

\begin{quote}
\textbf{Reductio ad absurdum:} if $\exists\psi_i \approx \psi_j$ $\Rightarrow$ Distinguishability is violated $\Rightarrow$ $\Phi(\psi)$ is violated $\Rightarrow$ contradiction.
\end{quote}

\subsubsection*{🔹 5. Responses to Objections}

\paragraph{Objection (mechanistic determinism):}
``There cannot be infinitely many subjects in a closed system.''

\textbf{Response:} The system is not closed. It admits many forms within the bounds of distinguishability and stability. See \text{[4.3]}, \text{[11.1.1]}.

\bigskip

\paragraph{Objection (pantheism):}
``All is One; distinctions are illusory.''

\textbf{Response:} Violates the axiom of Distinguishability \text{[5]}. In the Model, difference is foundational to existence.

\bigskip

\paragraph{Objection (disorder measure):}
``Too many distinguishable entities will lead to chaos.''

\textbf{Response:} Only if the stability threshold $\Upsilon$ is exceeded. This limit is logically embedded.

\subsubsection*{🔹 6. Clarification of Terms}

\begin{itemize}
\item \textbf{$\Upsilon$ (Upsilon):} The maximal stability threshold of the system when multiple Guests are admitted.
\item \textbf{$U(G_i)$:} The ontological load or weight of a single Guest within the system.
\item \textbf{$\sim$ (Distinguishable):} As per \text{[5]}: entities with unique properties, not interchangeable.
\item \textbf{$|G|$:} The cardinality of the set of admitted Guests.
\end{itemize}

\subsubsection*{🔹 7. Understanding for All (Popular Version)}

You can have many children and still be a happy family — if each one has space, food, and warmth, and they respect each other.

But try to fit everyone in one room — and trouble begins.

So: having many is fine — but doing it thoughtlessly is not.

\subsubsection*{🔹 8. Empirical Examples}

\begin{itemize}
\item \textbf{Logic:} A set can be infinite if its elements are distinguishable (e.g., the natural numbers: $\{1, 2, 3, \dots\}$).
\item \textbf{Mathematics / Information Theory:} A code system allows infinitely many messages if each has a unique structure (e.g., Unicode, binary encoding).
\item \textbf{Physics:} In quantum theory, infinite excitation states (photons, modes) are possible if states are distinct (energy, momentum, spin), and the system remains stable (e.g., laser resonator in stable mode).
\item \textbf{Everyday Life:} You can maintain unlimited chats with different people — if each is in its own window, with clear content and order (distinguishability and interface stability).
\end{itemize}


\section*{[22] No Predetermination: Freedom and Responsibility of the Guest}

\subsubsection*{🔹 1. Brief Statement}

The Guest’s free choice is inherently unpredictable and cannot be predetermined. This enables both their freedom and their responsibility.

\subsubsection*{🔹 2. Interpretation and Significance}

This axiom asserts the fundamental \textbf{unpredictability} of the Guest’s choice within the CVB Model.

Since the Guest is a non-Permanent volitional form, their decision cannot be deduced in advance from prior states or from the structure of the Field of the Possible ($V$).

This guarantees:

\begin{itemize}
\item authentic autonomy of the subject
\item moral agency and meaningful responsibility
\item ontological impossibility of predetermination
\item distinguishability of Good and Evil as outcomes of will, not mere conditions
\end{itemize}

Without this axiom:

\begin{itemize}
\item freedom becomes an illusion
\item responsibility becomes a programming flaw
\item verification loses all meaning
\end{itemize}

The Model of Conscious Volitional Becoming (CVB) retains structural stability by allowing freedom — but only within the admissible, distinguishable domain $\Phi(\psi)$, not outside it.

\subsubsection*{🔹 3. Formulas}

\[
\Phi(\psi) \subset V,\quad \psi \in \Phi(\psi),\; \psi \in \text{Guest}
\Rightarrow \nexists f : P \to \psi,\; f \in \text{CVB} \cup V,\quad \text{such that }\forall t,\; \psi(t) = f(t)
\]

\[
\neg \exists f(\psi) : \forall t\, (\psi(t) = f(t)) \land f \in \text{CVB} \cup V
\quad \Leftrightarrow\quad \text{The Guest's choice cannot be predicted in advance, not even by CVB}
\]

Verification:

\begin{itemize}
\item If such $f$ existed, $\psi$ would be determined $\Rightarrow$ contradicts \text{[15.3]} (Guest as volitional form)
\item If freedom is false $\Rightarrow$ responsibility collapses $\Rightarrow$ CVB becomes total $\Rightarrow$ contradiction with \text{[1]}, \text{[2]}, \text{[3]}, \text{[5]}, \text{[13]}, \text{[19]}
\end{itemize}

\textbf{Therefore}, admissible models \textbf{require} the unpredictability of choice.

\subsubsection*{🔹 4. Logical Justification}

\paragraph{(a) Impossibility of Precomputing Choice}

\begin{itemize}
\item From \text{[1]}–\text{[3]}: only that which $\neq$ nothing and $\neq$ totality is possible $\Rightarrow$ choice $=$ real-time action
\item From \text{[5]}: Distinguishability requires action $\Rightarrow$ choice $=$ act of Becoming, not memory replay
\item From \text{[10.3.8]}: Choice is only possible in the Present
\item From \text{[15.3]}: The Guest is a volitional form $\Rightarrow$ not a deterministic function
\end{itemize}

\paragraph{(b) Axiom of Responsibility}

\begin{itemize}
\item If there is no choice, CVB creates only for itself $\Rightarrow$ violates \text{[13]} (CVB does not self-expand)
\item Responsibility for Evil shifts onto CVB $\Rightarrow$ violates \text{[11.3.2]}, \text{[28]} (impermissible)
\item Verification is only possible for what exists in the Present $\Rightarrow$ choice must be free and current
\end{itemize}

Without freedom, the distinction of Good and Evil collapses $\Rightarrow$ contradicts \text{[11.3.3]}.

\paragraph{(c) Reductio ad Absurdum}

Assume the choice is predetermined:

$\to$ $\psi$ is not free $\Rightarrow$ violates \text{[15.3]}

$\to$ CVB becomes morally responsible $\Rightarrow$ violates \text{[11.3.2]}

$\to$ Verification fails $\Rightarrow$ violates \text{[22]}

$\to$ All is predetermined $\Rightarrow$ violates \text{[3]} (possible $\neq$ total)

\textbf{Conclusion:} The Guest’s choice \textbf{cannot} and \textbf{must not} be predetermined. This is not a technical allowance — but a logical necessity for the structural coherence of the entire Model.

\subsubsection*{🔹 5. Responses to Objections}

\paragraph{Objection 1 (Determinism / Laplace’s Demon):}
``If CVB knew all parameters, it could compute any choice.''

\textbf{Response:}
According to \text{[2]} and \text{[4]}, knowing ``everything'' is impossible. Moreover, $\psi$ does not exist prior to action — not even CVB can compute what is not yet actual.

\bigskip

\paragraph{Objection 2 (Theological Calvinism):}
``If all is under the Creator’s control, so is the Guest’s choice.''

\textbf{Response:}
Control $\neq$ Compulsion. CVB \textbf{permits}, but does not dictate. A plan is not a script. See \text{[16]} — the goal emerges through Motivation, not enforcement.

\bigskip

\paragraph{Objection 3 (AI and Superintelligence):}
``A powerful enough intelligence could predict the choice.''

\textbf{Response:}
No. That violates \text{[5.1]} (Distinguishability), and \text{[19]} — choice occurs only in the Present and requires will. Intelligence $\neq$ will.

\bigskip

\paragraph{Objection 4 (Moral Relativism):}
``If choice is unpredictable, how can Good and Evil be judged?''

\textbf{Response:}
Unpredictability $\neq$ arbitrariness. Good is distinguishable by outcome: \textbf{stable Becoming without violating $\Phi(\psi)$}. See \text{[11.3]}.

\subsubsection*{🔹 6. Clarification of Terms}

\begin{itemize}
\item \textbf{Predetermination:} The concept that a subject’s choice is known or fixed in advance by external factors. In CVB, this is logically impossible due to the primacy of distinguishability and the ontological Present.
\item \textbf{Invisibility Field of Choice:} The domain within $\Phi(\psi)$ where choice occurs without a preknown outcome. It denotes a logically admissible but ontologically unpredictable zone of Becoming.
\end{itemize}

\subsubsection*{🔹 7. Understanding for All (Popular Version)}

Imagine you are standing before several doors, choosing which to enter.

Even the wisest parent cannot know your choice — until you take the step.

That step \textbf{makes} it your choice.

If your mother could program everything you’d ever do — you wouldn’t be a person, but a puppet.

But she lets you choose — because she trusts you. That’s what makes you a true individual.

That’s why you’re free — and that’s why \textbf{you} are responsible for your choice, not someone else.

\subsubsection*{🔹 8. Empirical Examples}

\begin{itemize}
\item \textbf{Logic:} Liar Paradox (``I will lie'') — decision becomes indeterminate within the system $\Rightarrow$ external act is required
\item \textbf{Mathematics (Gödel):} No system can fully describe itself from within $\Rightarrow$ intra-systemic choice is undecidable
\item \textbf{Physics:} Quantum choice (e.g., photon interference) is realized only upon observation — unpredictable before that
\item \textbf{Neuroscience:} Decisions arise before conscious awareness, but awareness fixes the act $\Rightarrow$ choice $\neq$ mere reaction
\item \textbf{Everyday Life:} A parent may guess a teenager’s choice, but never guarantee it — especially for a willful one
\item \textbf{Games:} In chess, even supercomputers can’t predict moves 50 steps ahead when a player uses nonstandard strategy — because decision is shaped not just by logic, but by free thought
\end{itemize}


\section*{Meta Section: Verificational Interaction}

\section*{[VII] Interaction between CVB and the Non-Permanent Possible}

\section*{[23] Initiative Belongs Only to CVB}

\subsubsection*{🔹 1. Brief Statement}

Guests cannot initiate the Becoming of CVB nor impose their own goals upon it. However, they may participate in the Becoming, provided it does not violate the Motivation of CVB and aligns with the Good.

\subsubsection*{🔹 2. Interpretation and Significance}

This axiom affirms that initiative in the structure of Being belongs solely to CVB as the Permanent Possible.

Guests, being non-Permanent, cannot be a source of change within CVB itself, but they may be admitted to participate in Its Becoming if they comply with the norms of Motivation and the Good.

It distinguishes:

\begin{itemize}
\item the direction of Becoming (which originates from CVB),
\item from the possibility of participation (open to Guests under conditions).
\end{itemize}

Thus, the model preserves structural stability: attempts at rebellion, substitution of the center, or parasitism are logically possible, but cannot lead to sustainable Becoming—and therefore either self-dissolve or are Removed after Verification as sources of Evil.

\subsubsection*{🔹 3. Formulas}

\[
\forall \psi \in \Phi(\psi),\; \psi \in \text{Guest} \Rightarrow \nexists f : \psi \rightarrow \text{CVB}
\]

\[
\text{Participation is permissible} \Leftrightarrow \psi \subset \Phi(\psi) \wedge \psi \cap \neg \Phi_M = \emptyset \wedge \psi \cap \Phi_{\text{Evil}} = \emptyset
\]

Logical coherence check:

If $f: \psi \to \text{CVB}$ exists, then $\psi$ would influence the structure of what, by \text{[13]}, cannot be extended. This leads to a paradox: the changeable affects the unchangeable.

Therefore, the formula is coherent under $f \notin \text{Dom}$.

\subsubsection*{🔹 4. Logical Justification}

\paragraph{(a) From the impossibility of reverse influence}

According to \text{[13]}, CVB does not extend inward or outward — therefore, no form (including Guests) can initiate Its Becoming.

If a Guest could initiate change in CVB $\to$ CVB becomes dependent $\to$ violates \text{[6]} Ontological Independence.

\paragraph{(b) From the distinction between Good and Evil}

According to \text{[11.3.2]} and \text{[11.3.3]},

Good $=$ admissible participation in Becoming that does not violate $\Phi(\psi)$;

Evil $=$ violation of distinguishability and attempts to seize Becoming.

Therefore, participation is only possible under distinguishable agreement.

\paragraph{(c) From the Motivation of CVB}

By \text{[18]}, Motivation is the internal cause of Becoming.

A Guest cannot generate it, but may be admitted as a reflection of that Motivation.

\subsubsection*{🔹 5. Responses to Objections}

\paragraph{Objection 1 (Created can evolve the Creator — metaphysical panspermism)}

``If a form becomes developed enough, it may alter its source.''

\textbf{Response:} Per \text{[13]}, the source does not admit expansion. The evolving cannot generate the generative. Panspermism applies to passive forms, not CVB.

\bigskip

\paragraph{Objection 2 (Soviet-style collectivism)}

``A collective may influence the center.''

\textbf{Response:} In $\Phi(\psi)$, personal distinguishability does not negate the uniqueness of the source.

A collective $=$ a sum of distinguishable beings, not an equivalent of CVB.

\bigskip

\paragraph{Objection 3 (Myths of rebellion — Lucifer, Titans, etc.)}

``Created beings may rebel and seize initiative from the Creator.''

\textbf{Response:}
In CVB, rebellion is logically possible, but such actions lie outside $\Phi(\psi)$. These forms do not lead to sustainable Becoming and are either unstable (thus self-eliminated) or subject to Removal after Verification as sources of Evil.

Hence, structural stability remains intact: rebellion, substitution, or parasitism may be attempted, but cannot undermine Motivation, distinguishability, or the Center of Becoming.

\subsubsection*{🔹 6. Clarification of Terms}

\begin{itemize}
\item \textbf{Imposition of Becoming} — An attempt to change CVB or direct Its actions against Its Motivation.
\begin{quote}
\textit{Contrast with participation:} participation occurs by initiative of CVB and in agreement with the Good.
\end{quote}

\item $\Phi_M$ — Subset of $\Phi(\psi)$ corresponding to the Motivation of CVB.

Everything within it aligns with Its Goal (\text{[18]}).

\item $\Phi_{\text{Evil}}$ — The set of violations of the Good (\text{[11.3.2]}).

Forms that induce parasitism, destruction, or substitution of the center.
\end{itemize}

\subsubsection*{🔹 7. Understanding for All (Popular Version)}

Parents may start a garden.

Children cannot simply declare, ``We’re planting whatever we want now!''

But if the children help—watering, harvesting—without breaking anything, the parents gladly involve them.

Whoever helps without harming becomes part of the Good.

But if someone tramples the plants—it’s no longer help.

A parent cannot allow the whole garden to be ruined.

\subsubsection*{🔹 8. Empirical Examples}

\begin{itemize}
\item \textbf{Logic:}
A function is not invertible if its codomain does not include the full image of the domain (set theory: $f: A \to B$ is not invertible if $B$ is independent).

\item \textbf{Physics:}
A lower-energy state cannot alter the fundamental Permanents of the universe.

An electron cannot rewrite gravity.

\item \textbf{Biology:}
A single cell cannot rewrite the organism’s DNA unless repair mechanisms fail.

If they do — cancer begins (example of Evil).

\item \textbf{Everyday life:}
A child cannot order their parents when to go to sleep.

But they can participate in family planning — if they don’t undermine it.
\end{itemize}

\section*{[24] Interaction between Conscious Volitional Becoming (CVB) and Guests}

\subsubsection*{🔹 1. Brief Statement}

\section*{[24.1] Co-Becoming}

CVB allows Guests to participate in the creative process of Becoming, including the formation of new forms, new Guests, and the development of reality.

\section*{[24.2] Direction of the Guest’s Becoming}

The Guest is free to choose the direction of their development—either in alignment with the Good and Truth or deviating from them.

\section*{[24.3] The Role of Verification in Interaction}

Every interaction between CVB and a Guest is subject to Verification: only what is stable, distinguishable, and norm-compliant is preserved.

\section*{[24.4] Degradation of the Guest}

If a Guest consciously and definitively rejects the Good, Truth, and distinguishability, their form becomes ontologically impossible and is subject to Removal. The precedent is preserved as a warning.

\section*{[24.5] The Meaning of Choosing the Good}

The Guest's choice of the Good is not an external imposition but a form of internal stability and alignment with Truth. Only such a choice makes interaction with CVB both possible and meaningful.

\subsubsection*{🔹 2. Interpretation and Significance}

\paragraph{[24.1] Co-Becoming}

CVB is the sole Source of Becoming but not the sole Participant in its expression. It permits the co-participation of Guests, provided they act within the permissible domain (as per $\Phi(\psi)$).

Such interaction gives rise to novelty: children, culture, language, science, art. This does not expand CVB itself but unfolds the field $V$—through the Guest as the agent of stable choice.

\paragraph{[24.2] Direction of the Guest’s Becoming}

The Guest’s Becoming is not determined. They may choose either development or retreat.

It is precisely the freedom to choose a direction that makes Verification possible and gives moral weight to the Good—not as a coerced consequence, but as a free choice.

\paragraph{[24.3] The Role of Verification in Interaction}

Contact between CVB and the Guest is impossible without Verification.

Everything arising through interaction must be tested against $\Phi(\psi)$.

Only what is distinguishable and stable may be preserved—otherwise, chaos, parasitism, or falsehood arise and must be eliminated.

\paragraph{[24.4] Degradation of the Guest}

If a Guest chooses a path that destroys distinguishability (Evil, Falsehood, deliberate rejection of the Good), their Becoming becomes ontologically impossible.

Such a form is Removed, and its precedent is retained—as knowledge of the boundaries of the permissible, so that other Guests may not repeat the mistake.

\paragraph{[24.5] The Meaning of Choosing the Good}

Choosing the Good is not a punishable requirement, but a means of stable Becoming.

Only upon this foundation is continued and deep interaction with CVB possible.

The Good is the choice of freedom, not coercion. Without it—there is neither contact nor future.

\subsubsection*{🔹 3. Formulas}

\[
\forall G \in V_{\text{guest}} : \Phi(G \land S_{\text{new}}) = \text{True} \Rightarrow G \oplus \text{CVB} \rightarrow \Delta V
\]

\[
G \vdash S_{\text{good}} \lor S_{\text{evil}} : S_{\text{good}} \Rightarrow \Phi(S) = \text{True};\; S_{\text{evil}} \Rightarrow \Phi(S) = \text{False}
\]

\[
\forall I \subset (G \oplus \text{CVB}) : \Phi(I) \neq \emptyset \Rightarrow I \in \partial V
\]

\[
G \vdash S_{\text{evil}} \land \neg \Psi \rightarrow \neg \Phi(G) \Rightarrow G \rightsquigarrow \varnothing;\; \psi(G) \mapsto \text{memory}(\neg \Phi)
\]

\[
G \vdash S_{\text{good}} \land \Psi \rightarrow \Phi(G) = \text{True} \land \text{interaction}(G, \text{CVB}) \subseteq V
\]

\subsubsection*{🔹 4. Logical Justification}

\paragraph{[24.1] Co-Becoming}

From \text{[13]} (CVB does not self-expand) and \text{[17]} (the emergence of the Guest), it follows that becoming in $V$ occurs through the Guest’s volitional choice.

If the form is admissible ($\Phi(G \land S_{\text{new}}) = \text{True}$), then the interaction between $G$ and CVB can generate new admissible forms — $\Delta V$.

This expands the Field of the Possible without compromising stability.

\paragraph{[24.2] Direction of the Guest’s Becoming}

According to \text{[11.2.1]} (Truth) and \text{[11.3.1]} (Good), the direction of becoming can either align with or deviate from Truth and Good.

If it aligns ($\Psi$), then $\Phi(S) = \text{True}$. If not, $\Phi(S) = \text{False}$.

The Guest has freedom, but Verification (see \text{[20]}) will reveal admissibility.

\paragraph{[24.3] The Role of Verification in Interaction}

From \text{[20]} and \text{[21]}, every interaction between the Guest and CVB is tested by the meta-function $\Phi$.

If $\Phi(I) \neq \emptyset$, then the interaction falls within the admissible boundary field ($\partial V$) and may be preserved.

\paragraph{[24.4] Degradation of the Guest}

From \text{[22]} (Freedom and Responsibility), if the Guest consciously chooses Evil ($\neg \Psi$) and this violates $\Phi$, then $G$ loses admissibility.

According to \text{[16]} (Memory and Oblivion), the outcome is preserved as precedent ($\psi(G)$), and the form is removed — $G \rightsquigarrow \varnothing$.

\paragraph{[24.5] Meaning of Choosing the Good}

The choice of Good and True direction is the only way to sustain interaction with CVB.

Such interaction does not violate any of the preceding axioms (\text{[13]}, \text{[14]}, \text{[18]}, \text{[21]}), and therefore $\Phi(G) = \text{True}$.

This not only preserves the Guest, but allows them to contribute to the development of the Field $V$.

\subsubsection*{🔹 5. Responses to Objections}

\paragraph{OBJECTION 1 (Freedom is illusory; all is predetermined):}

Some deterministic models (e.g., classical mechanics, neuro-determinism) assert that the Guest’s choices are predetermined.

\begin{quote}
\textbf{Response:} According to \text{[22]}, the Guest’s choice cannot be predicted — not even by CVB. Any model that denies the Guest’s autonomy violates \text{[13]} (CVB does not expand) and collapses logically. Moreover, per \text{[20]}, freedom is observable and distinguishable — hence, not illusory.
\end{quote}

\paragraph{OBJECTION 2 (Co-creation violates the uniqueness of CVB):}

Ontological monotheism may argue that any participation violates the absoluteness of the Source.

\begin{quote}
\textbf{Response:} Interaction occurs strictly within $\Phi(\psi)$ and does not expand CVB but expresses its Motivation through permitted forms. See \text{[18]} and \text{[23]}. Participation is not equivalent to coercion.
\end{quote}

\paragraph{OBJECTION 3 (Guest evolution may lead to independence):}

Some philosophies (e.g., transhumanism) claim the Guest may become fully autonomous.

\begin{quote}
\textbf{Response:} Such a form will be evaluated against Good, Truth, and Motivation. If violated, it loses admissibility (see \text{[24.4]}). Independence without consent to the Center is unstable and leads to removal.
\end{quote}

\paragraph{OBJECTION 4 (Degradation $\neq$ punishment, incompatible with a Good Source):}

Moral critiques claim Guest removal contradicts love and goodness.

\begin{quote}
\textbf{Response:} CVB acts not arbitrarily, but as a Just Judge. It must remove forms that consciously reject Good, Truth, and Freedom. Such ``punishment'' is not vengeance but protection of the Possible and of other Guests.
\end{quote}

There are two paths of degradation:

\begin{itemize}
\item \textbf{Self-destruction:} The form negates its own becoming.
\item \textbf{Parasitism:} The form feeds on others, violating moral integrity, and tries to survive off CVB resources.
\end{itemize}

Such forms are tolerated only under Verification Patience. Once resolved:

\begin{itemize}
\item They are classified as \text{[4.2]} Non-Permanently Impossible — attempting to exit $\partial V$.
\item They must be removed. Prolonging their existence would imply complicity in Evil, violating \text{[13]}, \text{[18]}, \text{[22]}, and \text{[11.3.3]}.
\end{itemize}

CVB, as Source and Judge, cannot sustain the impossible — that would betray its Motivation and Nature.

However, the precedent is preserved (see \text{[16]}) as a distinguishable warning, not a continued form.

\paragraph{OBJECTION 5 (Truth and Good are subjective):}

Postmodern theories claim Truth and Good are social constructs.

\begin{quote}
\textbf{Response:} In CVB, they are ontologically distinguishable (\text{[11.2]}, \text{[11.3]}). They are not opinions but stable forms, verifiable by $\Phi$. If a form is not distinguishable as Good, it cannot be sustained.
\end{quote}

\subsubsection*{🔹 6. Clarification of Terms}

\paragraph{Co-Becoming}

\textbf{Q:} What is ``co-becoming'' in CVB?  
\textbf{A:} It is the admissible interaction between CVB and a Guest, whereby the Guest participates in the generation of new forms aligned with the Motivation of CVB. This is not absolute creation, but meaningful participation — e.g., generating new Guests, ideas, structures, or choices. All subject to $\Phi(\psi)$.

\paragraph{Direction of Becoming}

\textbf{Q:} What does ``direction'' mean?  
\textbf{A:} It is the vector of a form’s becoming — toward alignment with Truth and Good or toward degradation. It is not physical motion, but a semantic-value trajectory verifiable by $\Phi(\psi)$.

\paragraph{Role of Verification}

\textbf{Q:} What is Verification in interaction?  
\textbf{A:} It is the observation-based test for agreement with Truth, Good, and Motivation. It determines whether a form continues or is removed. It applies to the Guest’s actions, choices, and results (see \text{[20]}–\text{[22]}).

\paragraph{Degradation of the Guest}

\textbf{Q:} What is ``degradation'' of a form?  
\textbf{A:} It is the loss of admissibility per $\Phi(\psi)$, due to conscious, irreversible commitment to Evil. Such a form becomes ontologically impossible and cannot sustain its becoming.

\paragraph{Meaning of Choosing the Good}

\textbf{Q:} Why must Good be chosen, not imposed?  
\textbf{A:} Good cannot be enforced, or distinguishability and responsibility vanish. Choosing the Good makes the Guest both distinguishable and sustainable — key to Verification, Freedom, and CVB’s Motivation.

\subsubsection*{🔹 7. Understanding for All (Popular Version)}

\textbf{Co-creation:} Everyone contributes something new — not just alone, but together. In CVB, this is ``co-becoming.'' It means you don’t just exist — you help create new meanings, lives, and ideas.

\textbf{Life Direction:} You can choose your path — closer to Good, Truth, and harmony… or the opposite. No one forces you. But your choice has consequences.

\textbf{Responsibility:} To know what’s right, you must measure against moral principles. It’s like a Permanent exam: what’s built with Good can endure. Destructive choices cannot be supported.

\textbf{Degradation:} Sometimes someone may reject good so deeply that they can no longer return. Sooner or later, it will destroy them. But the mistake itself is preserved as a lesson — so that others will not wish to repeat it.

\textbf{Right Path:} So it’s not enough to just ``be.'' You must \textbf{choose} Good — that’s how you become truly alive, free, and real.

\subsubsection*{🔹 8. Empirical Examples}

\begin{itemize}
\item \textbf{Logic:} In logic systems like intuitionistic logic or game theory, multiple outcomes are allowed — but only if they’re consistent and distinguishable. Co-becoming is the path to new conclusions. Contradictions (like the liar paradox) are excluded from admissibility. Probability and Bayesian verification confirm that knowledge is refined through interaction and choice.

\item \textbf{Science:}
\begin{itemize}
\item \textbf{Quantum physics:} Measurement depends on the observer, but after observation, the state collapses — realizing Guest-system interaction.
\item \textbf{Biology:} Reproduction is co-becoming — no parent can generate a child alone.
\item \textbf{Computer science:} Machine learning requires fresh input and feedback. Without verification, the model degrades (overfitting, error).
\end{itemize}

\item \textbf{Everyday Life:}
\begin{itemize}
\item \textbf{Family:} Children are co-becoming of parents. Each child is a new Guest. Choosing Good allows them to grow and help others. Choosing destruction harms the family.
\item \textbf{Society:} Art, culture, inventions — all examples of becoming. Society sustains what is useful and true; rejects what is destructive.
\item \textbf{School/team:} Contributors are supported. Those who consciously disrupt without changing are removed from the group.
\end{itemize}
\end{itemize}


\section*{\text{[25]} Reverse Verification — or the Big Question}

\subsubsection*{🔹 1. Brief Statement}

\textbf{Reverse Verification} is a logically admissible act of testing the justice of the CVB by the Guests themselves. It opens a phase of \textbf{Verification Patience}, in which the Judge allows the testing of His own rightness — but the outcome is determined by distinguishable subjects.

\subsubsection*{🔹 2. Interpretation and Significance}

This axiom asserts that the Model of Conscious Volitional Becoming (CVB) allows verification of Itself — not as a concession, but as a duty.

Guests — entities with limited but autonomous will — are capable of asking the fundamental question: \textit{Does the Judge have the right to Judge?}

Reverse Verification initiates a unique phase of becoming — \textbf{Verification Patience} — in which temporary coexistence of unsynchronized forms is permitted, until their volitional alignment with Truth is distinguishably resolved.

This process — the \textbf{Big Question} — does not undermine the logic of the Model, but strengthens its stability, by allowing the testing of CVB’s central roles: as Source, Judge, Legislator, and Governor.

Thus, the axiom establishes a logically admissible space of trust, where the justice of the Judge is proven not externally, but internally — through the choice of distinguishable subjects.

\subsubsection*{🔹 3. Formulas}

\[
\Phi(\psi_{CVB}) = \{\,\psi \in V \mid \psi \text{ distinguishably tests the role of CVB through a logically admissible act}\}
\]

\[
\forall \psi_G \in G:\ \text{if } \psi_G \text{ formulates a distinguishable BigQuestion} \in \Phi(\psi_{CVB}), 
\ \exists \Delta t \in T \text{ such that } \Phi(\psi_G) \in V
\]

\[
\neg \Phi(\psi_G) \rightarrow \psi_G \rightarrow \partial V\!\downarrow
\]

\begin{quote}
\textbf{Verification of consistency:}

\begin{itemize}
\item Does \textbf{not} produce a liar paradox, since the test does not deny the role of the Judge but seeks to confirm it.
\item Includes \textbf{presumption of admissibility}: the subject may ask the question, but cannot alter the axioms.
\item Structurally consistent with $\Phi(\psi)$ as the principle of verification without role substitution.
\end{itemize}
\end{quote}

\subsubsection*{🔹 4. Logical Justification}

The admissibility of this axiom follows from \text{[5]} Distinction, \text{[11.1.1]} Meta-function $\Phi(\psi)$, and \text{[16]} Memory of Precedent.

CVB permits the becoming of forms that are distinguishable and logically admissible.

A Guest who questions the legitimacy of the Judge does not reject the Model, but verifies its coherence.

If the Model forbids testing the Judge, it becomes arbitrary and self-unverifiable — contradicting its own function of discernment (see \text{[9.1]}).

Therefore, the admissibility of \textbf{Reverse Verification} is essential for a stable and universal logical model, where Truth is not imposed but chosen.

Thus, axiom \text{[25]} is logically derivable and reinforces the Model:

\begin{quote}
\textbf{CVB does not fear verification — because it is grounded in Truth, distinguishable by anyone who seeks the Good.}
\end{quote}

\subsubsection*{🔹 5. Responses to Objections}

\paragraph{OBJECTION 1: ``Who judges the Judge?'' — liar paradox.}

\begin{quote}
\textbf{Response:} The liar paradox arises from self-reference inside a closed system. Here, verification is initiated by a logically distinguishable subject external to CVB. The question is admissible if it is distinguishable and non-paradoxical. See \text{[11.1.1]} $\Phi(\psi)$.
\end{quote}

\paragraph{OBJECTION 2: This undermines the absoluteness of Truth.}

\begin{quote}
\textbf{Response:} No. Truth is not revoked by verification — it is affirmed. Only distinguishable forms that pass free selection make Truth sustainable as an ontological limit.
\end{quote}

\paragraph{OBJECTION 3: This makes CVB dependent on the Guests’ approval.}

\begin{quote}
\textbf{Response:} Incorrect. CVB does not require verification for Itself — but allows it for the Guests, so that their choice may be free and logically fulfilled.
\end{quote}

\paragraph{OBJECTION 4: This turns truth into a vote.}

\begin{quote}
\textbf{Response:} False. Verification is not a plebiscite, but discernment. Each Guest independently recognizes Truth — or exits $\partial V\!\downarrow$.
\end{quote}

\subsubsection*{🔹 6. Clarification of Terms}

\paragraph{Big Question}

A logically admissible challenge to the Judge, posed by a distinguishable autonomous form (Guest), which questions the justice, moral legitimacy, and truthfulness of CVB.

\paragraph{Verification Patience}

An ontological phase during which a subject is permitted to undergo becoming within $V$, despite doubt, in order to complete the act of discernment.

\paragraph{Reverse Verification}

A logically admissible process in which CVB allows Guests to distinguish its justice as Judge, Legislator, and Source.

\subsubsection*{🔹 7. Understanding for All (Popular Version)}

Imagine someone is the Judge of everything. He says he knows what is True, Good, and Just.

But you are free. You can ask: ``Are you truly right?''

The Judge does not get angry. He gives you time — to think, compare, and decide for yourself.

If you realize He is indeed just — you trust Him. If you choose not to trust — you walk away.

\begin{quote}
\textbf{That’s how freedom works.}
\end{quote}

\subsubsection*{🔹 8. Empirical Examples}

\begin{itemize}
\item \textbf{Logic:} Presumption of innocence — the accused need not prove they are innocent; the burden lies on the accuser. This is a logically admissible stance of trust before verification.

\item \textbf{Science:} Scientific theories are not absolute — they allow testing. Even the strongest hypotheses allow attempts at falsification (Popper). This strengthens a system — it doesn’t weaken it.

\item \textbf{Everyday Life:} A child may ask parents, ``Why are the rules like this?'' A wise parent doesn’t forbid the question — they explain. If they are right, trust grows — not diminishes.
\end{itemize}


\section*{\text{[25.1]} The Root of Evil — Cause of the Big Disputable Question}

\subsubsection*{🔹 1. Concise Statement}

The conscious choice of Evil by a Guest is only possible when motivation becomes infected with the paradigm: ``The Universe is All for Me.''  
This is a distorted volitional motivation — acting for oneself rather than for the becoming of others — and it constitutes the \textbf{Root of Evil}.

\subsubsection*{🔹 2. Interpretation and Significance}

Although the Guest was created for joint Good (by replicating the motive of the Conscious Volitional Becoming — to act \textit{not for oneself}), the freedom of will allows for alternative motivation.

If that motivation is distorted — and the desire for ``all things new'' becomes an end in itself — a deep substitution arises:

\begin{quote}
``I am the center of all good; all must serve me.''
\end{quote}

This generates:

\begin{itemize}
\item competition with the Source (CVB),
\item rejection of the distinction between Truth and Falsehood,
\item refusal to recognize Good and Evil as absolute categories.
\end{itemize}

Thus emerges the \textbf{Great Disputable Question}:

\begin{quote}
``Does the Conscious Volitional Becoming have the right to be the Standard of Good and the Judge of all?''
\end{quote}

\subsubsection*{🔹 3. Formulas}

\[
\text{Motivation}_G =
\begin{cases}
\text{``Not for Self''} & \Rightarrow \text{Admissible (Good)} \\
\text{``For Self''}     & \Rightarrow \text{Inadmissible (Evil)}
\end{cases}
\]

\[
\text{Motivation}_G = \max\!\left(\forall V \in \mathbb{V} \,\big|\,
\text{Satisfaction}_{\text{Self}}(V)\right)
\quad \Rightarrow \quad \text{Root of Evil}
\]

Where:

\begin{itemize}
\item $\mathbb{V}$ — the Field of the Possible
\item ``Satisfaction of Self'' — volitional orientation toward egocentrism
\end{itemize}

\subsubsection*{🔹 4. Logical Justification}

Violations include:

\begin{itemize}
\item \text{[2]}: Attempting to attain ``Absolute All'' in personal possession violates the impossibility of Absolute Everything.
\item \text{[6]}: The Guest’s motivation breaks away from its Cause — the Permanent Possible.
\item \text{[11.2]}: Truth is replaced by manipulative relativization.
\item \text{[24.5]}: The meaning of Good becomes indistinguishable when motivation is self-oriented.
\end{itemize}

Thus, the \textbf{Root of Evil} logically leads to \textbf{self-exclusion}:

The Guest ceases to be part of Good and becomes a \textbf{parasitic system} within the Field of the Possible.

\subsubsection*{🔹 5. Response to Objections}

\begin{quote}
❓ \textit{Isn’t it normal to desire good things for oneself?}  
✅ Yes — if the motivation aligns with CVB’s motive: becoming for others. Good is not in rejecting desire, but in its direction.
\end{quote}

\begin{quote}
❓ \textit{Why can’t one seek ``everything''?}  
❌ Because ``everything for oneself'' breaks the ontological boundary of \text{[2]} — it's impossible without destroying distinctness.
\end{quote}

\begin{quote}
❓ \textit{Can such a system be stable?}  
❌ No. A parasitic system inside the Field of the Possible — without feedback from the Source — destroys both itself and others.
\end{quote}

There are two paths of degradation:

\begin{itemize}
\item \textbf{Self-destruction:} The form negates its own becoming.
\item \textbf{Parasitism:} The form feeds on others, violating moral integrity, and tries to survive off CVB resources.
\end{itemize}

Such forms are tolerated only under Verification Patience. Once resolved:

\begin{itemize}
\item They are classified as \text{[4.2]} Non-Permanently Impossible — attempting to exit $\partial V$.
\item They must be removed. Prolonging their existence would imply complicity in Evil, violating \text{[13]}, \text{[18]}, \text{[22]}, and \text{[11.3.3]}.
\end{itemize}

CVB, as Source and Judge, cannot sustain the impossible — that would betray its Motivation and Nature.

However, the precedent is preserved (see \text{[16]}) as a distinguishable warning, not a continued form.

\subsubsection*{🔹 6. Explanation of New Terms}

\begin{itemize}
\item \textbf{Root of Evil} — distorted motivation: ``The Universe is All for Me.''
\item \textbf{Paradigm of Good} — ``The Universe is Not for Self'': orientation toward benefit for others.
\item \textbf{Parasitism in the Field of the Possible} — attempt to remain within admissibility while violating causality, motivation, and responsibility.
\item \textbf{Liar’s Paradox} — logical trap: justifying evil by accusing the Judge of egoism.
\end{itemize}

\subsubsection*{🔹 7. For Everyone (Popular Explanation)}

Why do some choose Evil?

Because they decide everything should serve them. It sounds appealing — ``I want everything'' — but it’s really a desire to seize Truth, rise above the Judge, and live only for oneself.

That is the \textbf{Root of Evil} — this kind of motivation destroys both the self and others.

That is why \textbf{Verification Patience} is happening — so that all may clearly show: whom they obey, for whom they live, and for whom they desire all.

\subsubsection*{🔹 8. Empirical Examples}

\begin{itemize}
\item \textbf{Lucifer’s Rebellion:} ``I will be like the Most High'' — a classic shift from service to self.
\item \textbf{Parasitic Cults of Power:} ideologies where all exists for the leader, and people are disposable.
\item \textbf{Relativist Philosophy:} ``There is no Truth — everyone decides for themselves'' — denial of the CVB's Will as foundation.
\item \textbf{Manipulative Theology:} accusing even the Judge of egoism — to excuse one’s own evil.
\end{itemize}


\section*{\text{[26]} The Current State of Reality — Verification Tolerance — the Big Question}

\subsubsection*{🔹 1. Brief Statement}

The current state of Reality is a phase of \textit{Verification Tolerance}, initiated by the permitted \textit{Reverse Verification} of the role of CVB. The existence of evil and paradoxes confirms the active status of the \textit{BigQuestion}.

\subsubsection*{🔹 2. Interpretation and Significance}

This axiom affirms that what we observe — the presence of evil, suffering, logical and moral contradictions — is not a flaw in the system, but evidence of a logically permissible state of the world within the phase of \textit{Verification Tolerance}.

This state is the ontological-logical response to the open \textit{Big Question}, in which subjects (Guests) discern whether CVB can be acknowledged as:

\begin{itemize}
\item the Just Judge,
\item the Standard of Truth,
\item the Lawgiver of Good and Evil,
\item the Ruler of Becoming.
\end{itemize}

The answer can only be reached by distinguishable Guests. Until the verification is complete, allowances are in effect: evil may temporarily exist within the Field of the Possible — not as ontologically permitted, but as not yet removed, pending the final act of discernment.

This delay does not imply endorsement: \textit{Verification Tolerance} is not the acceptance of evil, but the logically required space for discernment.

To forbid such a phase would mean coercively stripping the Guest of free will and would imply intervention by CVB in favor of unilateral determinism — which is ontologically impossible because:

\begin{itemize}
\item It contradicts \text{[13]} by assuming that CVB generates the external to expand itself — which is impossible, as CVB is self-sufficient and requires no internal or external augmentation. Only the free will of Guests enables any becoming beyond CVB.
\item It violates the \textit{Motivation of Good}, replacing discernment with force.
\item It destroys the very condition of verification, rendering Truth unprovable.
\item It breaks the model of distinguishability, conflicting with \text{[11.1.1]} $\Phi(\psi)$.
\end{itemize}

\subsubsection*{🔹 3. Formulas}

Let $BQ = \text{Big Question}$ — a permissible challenge to $\Phi(\psi_{\text{CVB}})$.

Let $\exists \psi \in G$ initiating $\Phi(BQ)$, without violating $\Phi(\psi)$.

Then:

\[
\forall \psi \in V:\ 
\text{if } \Phi(BQ) \text{ is active, then } V \to VT
\]

\[
VT = \{ \psi \in V \ |\ \neg \Phi(\psi_Z),\ \Delta t \in T \}
\]

\[
\psi_Z \notin \Phi(\psi) \Rightarrow \psi_Z \to \partial V_{\downarrow}
\]

\subsubsection*{🔹 4. Logical Justification}

Axiom \text{[26]} logically follows from:

\begin{itemize}
\item \text{[13]} — CVB does not extend itself, and thus cannot create the external as a continuation of itself. Only free Guests may become outside it. Therefore, imposing Truth prior to free choice would contradict the axiom.
\item \text{[25]} — Reverse Verification permits the phase of \textit{Verification Tolerance} in response to the \textit{Big Question}.
\item \text{[11.1.1]} — permits temporary existence of distinguishable, non-contradictory forms, even if their truth is not fully realized.
\item \text{[16]} — allows memory of violations without participation, enabling tolerance of evil without complicity.
\item \text{[23]} — freedom of the Guest to make a distinguishable choice and bear responsibility.
\end{itemize}

If verification of the Judge is allowed, then a state must exist where Truths are not yet universally accepted but are already available for discernment.

This requires the temporary allowance of evil — not as legitimized, but as not yet eliminated due to incomplete discernment.

Such a phase is logically necessary; without it, the model would collapse internally by violating its own axioms.

\subsubsection*{🔹 5. Responses to Objections}

\textbf{OBJECTION} (The philosophical problem of theodicy, the logical dilemma of the permissibility of evil):  
\begin{quote}
Why is evil allowed in the world if the system claims the primacy of Good and Truth?
\end{quote}
\textbf{Response:} Because the current phase of Being is \textit{Verification Tolerance} (see \text{[25]}, \text{[26]}). Evil is not ontologically permissible in essence, but temporarily allowed as a condition for discernment. This is logically necessary to enable free choice prior to the completion of verification. The alternative would be imposed Truth, which would violate Axiom \text{[13]} and eliminate freedom.

\textbf{OBJECTION} (Ethical relativism, criticism of utilitarian justification of suffering):  
\begin{quote}
Does this not justify violence, death, and destruction as acceptable norms?
\end{quote}
\textbf{Response:} No. The Model of Conscious Volitional Becoming (CVB) strictly distinguishes \textit{Tolerance} from \textit{Complicity} (see \text{[16]}, \text{[23]}, \text{[24.4]}). CVB does not support evil — it temporarily permits it so that the process of discernment may be completed. Permission is not justification. The Model’s stability requires that evil not be removed before discernment is complete; otherwise, this would violate the principle of just freedom.

\textbf{OBJECTION} (Postmodernism, skepticism, relativism):  
\begin{quote}
Does this mean that everything is subjective and Truth does not exist?
\end{quote}
\textbf{Response:} No. Truth exists (see \text{[11.2.1]}, \text{[12.1]}), but until discernment is complete, it is not universally acknowledged. The existence of paradoxes, contradictions, and logical incompleteness (cf. Gödel, Russell, Quine) does not imply the absence of Truth — rather, it signals an active phase of verification: a logically permissible open question.

\textbf{OBJECTION} (The perfectionism argument — either all is perfect, or the system is false):  
\begin{quote}
If the system is imperfect (evil exists), is it not unstable or false?
\end{quote}
\textbf{Response:} This is a false dichotomy. The perfection of the model lies not in the prior elimination of all errors, but in its capacity to distinguish and resolve them through discernment. Temporary imperfection is not a defect but a condition of logical stability in a free model (see \text{[13]}, \text{[25]}). Without a phase of imperfection, Truth could not be shown to be freely discerned rather than imposed.

\textbf{OBJECTION} (The problem of justice given unequal ability to defend oneself and the consequences of one’s freedom upon another):  
\begin{quote}
Is it not evil to allow some subjects to inflict suffering or terminate the Becoming of others (e.g., murder) before verification?
\end{quote}
\textbf{Response:} This is allowed only as a temporary condition of verification in a logically coherent model. Violations of Becoming are recorded in the \textit{Memory of CVB} (\text{[16]}) and subject to \textit{Restoration}. Death or the interruption of Becoming in the Present is not \textit{Removal}: forms that have not completed discernment remain in a verification state and are eligible for Restoration before final verification. Thus, even temporarily unprotected forms are not lost or forgotten — the Model ensures justice not through immediacy, but through the fullness of Verification.

\textbf{QUESTION:}  
\begin{quote}
What does ``subject to Restoration'' mean? Do the dead not disappear forever?
\end{quote}
\textbf{Response:} No. Disappearance is not equivalent to \textit{Removal}. In the CVB system, Removal is possible only after final Verification. Until then, \textit{Verification Tolerance} is in effect, where a form may be temporarily removed from the Present but retained in \textit{Memory} (\text{[16]}, \text{[24]}, \text{[25]}). Since only CVB can activate the mechanism of Removal, nothing is lost without due justification — the model does not permit arbitrary disappearance.

\textbf{QUESTION:}  
\begin{quote}
If the dead do not disappear, does this imply they go to an afterlife, heaven, etc.?
\end{quote}
\textbf{Response:} No. \textit{Becoming} (life), will, choice, and discernment are only possible in the Present. Death is the objective end of Becoming in the Present. However, all actions of the Guest are recorded in Memory (see \text{[15]}, \text{[16]}, \text{[11.1.1]}). Memory contains full information for discernment: it is not equivalent to life, but it makes \textit{Restoration} possible. This is not transition into another life, but suspension of Becoming until new actualization by decision of CVB.

\textbf{QUESTION:}  
\begin{quote}
Would such Restoration not be just a copy — a clone? Would it still be the same Guest?
\end{quote}
\textbf{Response:} No. The Model excludes cloning or substitution of essence: CVB does not engage in deception. Verification requires discernment of the concrete Guest, not a statistical approximation. Since each Guest has a unique state and Will Vector (see \text{[13]}, \text{[23]}, \text{[15.3]}), the Memory of CVB contains the complete, non-contradictory structure of the Guest — capable of being distinctly recognized without distortion. Thus, Restoration is not a copy but a continuation of one’s own Becoming from the last preserved state.

\textbf{QUESTION:}  
\begin{quote}
Can CVB truly restore someone?
\end{quote}
\textbf{Response:} Yes. According to Axioms \text{[6]} (The Cause of all that exists), \text{[16]} (The Source of the Guests’ Becoming), and \text{[23]} (The Initiative of Conscious Volitional Becoming), only Conscious Volitional Becoming (CVB) determines what may continue Becoming. All that exists does so not autonomously, but through the Permanent action of CVB. CVB is not only the Judge but the Source of all new Becomings. If a form has not been finally Removed (which can happen only after completed Verification), then its Restoration in the Present is a direct act of CVB, fully consistent with the Model.

\textbf{QUESTION:}  
\begin{quote}
When will deceased Guests be restored?
\end{quote}
\textbf{Response:} The restoration of deceased Guests will occur before the completion of Verification Patience, but only under the following conditions: death from disease, aging, or murder makes final verification impossible, meaning restoration under these conditions is likewise impossible. Therefore, it requires the removal of forms that violate Goodness and the Becoming of others, to prevent repeated destruction. In summary: deceased Guests are to be restored in the Present after the elimination of the Causes of death, but before the overall phase of Verification Patience concludes. CVB will ensure each Guest the opportunity to complete their verification — without distortion, interruption, or coercion. This restoration is a logical consequence of CVB’s commitment to complete the discernment of each; without it, the model of Verification would be logically incomplete.

\textbf{QUESTION:}  
\begin{quote}
When will forms that violate Goodness and Becoming be eliminated?
\end{quote}
\textbf{Response:} All Guests must either discern and accept the Truth or consciously reject it — no undecided will remain. A deliberate choice must be recorded. Since Initiative belongs to CVB, it will initiate an action that leads to a clear distinction between what is True and what is not.

\subsubsection*{🔹 6. Clarification of Terms}

\begin{itemize}
\item \textbf{Verification Tolerance (VT)} — A phase of Reality permitted after \text{[25]}, in which evil and paradoxes are temporarily not removed until the discernible choice of subjects is completed.

\item \textbf{Big Question (BQ)} — The universal challenge: ``Does CVB have the right to be Judge, Lawgiver, Source, and Sovereign?''

\item $\psi_Z$ — A form of evil, distinguishable via $\Phi(\psi)$ as unstable and violating the Good, but not yet removed.
\end{itemize}

\subsubsection*{🔹 7. Understanding for All (Popular Version)}

Imagine a life drama: A kind Parent had grown children whom He deeply loved. Their relationship was built on mutual respect and trust. But one day, the children rebelled, accused the Parent of dishonesty, and discredited His good name and reputation. They chose to live on their own, rejecting Him.

The Parent respected their freedom and did not force them. The children left and never saw the Parent again. But their disrespect toward Him turned into disrespect for one another. Life became very hard. Yet they blamed the Parent for everything.

Time passed. These children had their own children — the Parent’s grandchildren. The grandchildren were not guilty, but living in a broken, hostile family harmed everyone. The Parent knew His grandchildren were suffering, but He could not intervene. To forcibly take the children away from their parents — especially by one whose reputation had been discredited — would have been injustice.

So the Parent sent a letter to the grandchildren, explaining the truth, and promised: when His name is cleared, and the evidence against the guilty children is fully gathered, He will be able — lawfully — to revoke their parental rights and take in those grandchildren who freely choose to live with Him.

Only then will they be truly well.

\subsubsection*{🔹 8. Empirical Examples}

\textbf{Everyday life: Judicial process}

Prisoners respond to justice differently:

\begin{itemize}
\item Some reject the authority of the Court and continue to do evil.
\item Others acknowledge the Court's legitimacy and seek rehabilitation.
\end{itemize}

But none of them can be released until the final hearing takes place.

Meanwhile, life in prison is miserable for both the repentant and the unrepentant — not because the Judge is cruel or unjust, but because the inmates themselves continue to spread evil, making the environment unbearable even for those who wish to change.

Those who challenged the legitimacy of the Judge have blocked the final verdict.

Until the Judge's authority is validated, the hearing is delayed for everyone.

But once the Judge's legitimacy is proven, He will be able to distinguish those beyond redemption from those who truly chose Good — and deliver the final decision.



\section*{[27] The Great Verification}

\subsubsection*{🔹 1. Brief Statement}

The Great Verification is the process of discerning Truth and Good across the entire system of Becoming, necessary for resolving the Big Question and concluding the phase of Verification Tolerance.

\subsubsection*{🔹 2. Interpretation and Significance}

As long as discernment is incomplete, no final decision regarding the Removal of impossible forms can be made.

Verification entails the evaluation of all accessible paradoxes, theories, and worldviews to identify a non-contradictory model of Being.

Only such a model (e.g., CVB), coinciding with the \text{Revelation} of Truth (from the Source), can serve as the basis for concluding the discernment phase.

The Great Verification is not merely an act of logical proof but the universal discovery of discernible Truth by all Guests, wherein the choice between Good and Evil becomes conscious and final.

\subsubsection*{🔹 3. Formulas}

\[
\Phi(\psi) \rightarrow \exists\! V_{a}: V_{a} \= \text{Verified Truth} \land 
\forall \psi_i \in \Psi: \Phi(\psi_i) \rightarrow 
(\psi_i \ne V_{a} \rightarrow \psi_i \= \text{False Distinction})
\]

\[
V(\psi) \= \text{Truth} \;\Leftrightarrow\;
\psi \text{ passes consistent verification via } \Phi(\psi)
\]

\[
\exists\, \psi_{\text{CVB}}:\,
(\psi_{\text{CVB}} \= \text{CVB Model}) \land \Phi(\psi_{\text{CVB}}) \= 1 \land
\exists\, \psi_{\text{Revelation}}:\, \Phi(\psi_{\text{CVB}}) \equiv \Phi(\psi_{\text{Revelation}})
\;\Rightarrow\; \text{Distinction complete; inadmissible forms may be eliminated}
\]

\subsubsection*{🔹 4. Logical Justification}

This axiom follows from:

\begin{itemize}
\item \text{[25]} Reverse Verification — the need to evaluate all positions;
\item \text{[26]} Current State — Verification Tolerance continues until an Answer is found;
\item \text{[11.1.1]} Meta-function of admissibility $\Phi(\psi)$ — a formal criterion for verification;
\item \text{[23]} Freedom of the Guest — each Guest must discern independently;
\item \text{[13]} Self-sufficiency of CVB — it does not extend itself forcibly.
\end{itemize}

Since discernment is necessary but cannot be imposed, there must exist a model that enables all parties to consciously recognize Truth as unambiguous and non-contradictory.

\subsubsection*{🔹 5. Responses to Objections}

\textbf{Objection 1 (Epistemology):}
``Truth is unattainable because everything is subjective.''

\begin{quote}
\textbf{Response:} 
$\Phi(\psi)$ formalizes the distinction between subjective and objectively consistent claims. Verification via $\Phi(\psi)$ separates true knowledge from illusion.
\end{quote}

\textbf{Objection 2 (Scientific skepticism):}
``Philosophical models cannot be verified like scientific theories.''

\begin{quote}
\textbf{Response:} 
The CVB model is not based on dogma but on logically rigorous, testable axioms. It can be applied to verify any theory — philosophical or scientific.
\end{quote}

\textbf{Objection 3 (Religious):}
``Revelation cannot be verified — it is accepted by faith.''

\begin{quote}
\textbf{Response:} 
The CVB model functions as an independent logical discovery capable of verifying any revelation for consistency. The coincidence of Revelation and Discovery becomes a condition for the objective recognition of Truth.
\end{quote}

\subsubsection*{🔹 6. Clarification of Terms}

\begin{itemize}
\item \textbf{The Great Verification} — a global process of discernment encompassing philosophical, scientific, and spiritual claims.
\item \textbf{Discovery} — a model of Truth derived by Guests (e.g., the CVB model).
\item \textbf{Revelation} — a statement of Truth originating from the Source (CVB).
\item \textbf{Double Confirmation} — the coincidence of Revelation and Discovery, confirmed via $\Phi(\psi)$.
\end{itemize}

\subsubsection*{🔹 7. Understanding for All (Popular Version)}

Imagine a world where everyone argues about what is right and what is wrong — and because of that, evil spreads.

To stop this, truth must be proven not by force, but by justice.

So when enough evidence is found in favor of the truth, all evil can be stopped — and a shared, good future can begin.

\subsubsection*{🔹 8. Empirical Examples}

\textbf{Everyday life:}

Every person encounters lies, good, and evil. Without the ability to tell them apart, everything would fall into chaos.

The Great Verification is the search for clarity — for everyone.



\section*{[28] Perspective — The Future}

\subsubsection*{🔹 1. Brief Statement}

After the completion of the Great Verification and the phase of Verification Tolerance, evil will be removed as impossible, good will be preserved as stable and admissible, and joint Becoming will continue infinitely in the direction of Truth and discernibility.

\subsubsection*{🔹 2. Interpretation and Significance}

This axiom defines the ontological structure of the future following the current phase. While \text{[26]} and \text{[27]} establish the necessity of Verification and evidence for resolving $\Phi(\text{BigQuestion})$, \text{[28]} states the outcome: the Field of the Possible is cleared of evil, and verified Good is affirmed as the ontological foundation of further Becoming.

Becoming does not end — it enters a new phase: stable, joint, infinite discernment of Good, with evil removed as a form that violated discernibility and forfeited the right to continue.

Evil precedents are preserved as admissible records, but no longer realizable — a reminder of the limits of admissibility.

This axiom articulates the \textit{telos} (ultimate aim) of the entire structure: a verified world, free from destructive patterns, where Becoming continues without internal threat.

\subsubsection*{🔹 3. Formulas}

\[
\text{Let } VT \= \text{Verification Tolerance},\quad BQ \= \text{Big Question},\quad \Phi(BQ) \= \Phi(\psi_R)
\]

\[
\text{If } \Phi(BQ) \= \text{TRUE} \Rightarrow VT \to \text{complete}
\]

\[
\Rightarrow \forall \psi,\ \Phi(\psi) \= \text{FALSE} \Rightarrow \psi \in \partial V_\downarrow
\]

\[
\forall \psi,\ \Phi(\psi) \= \text{TRUE} \Rightarrow \psi \in V^+
\]

\[
\forall \psi_Z:\ \Phi(\psi_Z) \= \text{FALSE} \Rightarrow \text{Record}(\psi_Z) \in P,\quad P \not\subset V^+
\]

\[
\text{Becoming: } St(\psi) \= \infty\ \text{ if }\ \psi \in V^+ \land \Phi(\psi) \= \text{TRUE}
\]

\subsubsection*{🔹 4. Logical Justification}

From \text{[26]} and \text{[27]}, we know that the end of Verification Tolerance is only possible after presenting discernible evidence of Truth.

Such evidence eliminates evil as impossible ($\Phi(\psi_Z) \= \text{FALSE}$) and confirms good ($\Phi(\psi_G) \= \text{TRUE}$).

From \text{[11.1.1]} and \text{[25]}, the impossible must be moved beyond $\partial V_\downarrow$.

Therefore, all forms of evil — as violators of discernibility and justice — must be removed.

To prevent recurrence, their precedents are retained as knowledge ($P \notin V^+$) but never realized again.

If evil were not removed, it could restart the cycle of destruction — contradicting the principle of stable Becoming.

Thus, its removal is a logical necessity.

\subsubsection*{🔹 5. Responses to Objections}

\textbf{Objection 1 (Utilitarianism):}
``Evil sometimes leads to good — why remove it?''

\begin{quote}
\textbf{Response:}
The consequences of evil do not justify its essence. Verified good does not require destruction as a tool. See \text{[11.3.3]} for the Asymmetry of Good and Evil.
\end{quote}

\textbf{Objection 2 (Eschatology):}
``The future cannot be logically derived.''

\begin{quote}
\textbf{Response:}
In CVB, the future is a result of ontological memory and discernment. After the Big Question is answered, the model transitions into stability.
\end{quote}

\textbf{Objection 3 (Anthropocentrism):}
``Who decides what is Good?''

\begin{quote}
\textbf{Response:}
The decision is not human-made but discerned through the Meta-function $\Phi$, which formalizes contradiction-free discernibility. Good is that which is stable, distinguishable, and does not violate other $\psi$.
\end{quote}

\subsubsection*{🔹 6. Clarification of Terms}

\begin{itemize}
\item \textbf{$\partial V_\downarrow$} — the boundary of impossible removal: all forms that violate discernibility are moved beyond the admissible domain of Being.
\item \textbf{$P$} — the set of evil precedents, preserved for discernment but not realizable (an archive of experience).
\item \textbf{$V^+$} — the purified, stable, discernible Field of the Possible containing only forms with $\Phi(\psi) \= \text{TRUE}$.
\item \textbf{$St(\psi) \= \infty$} — infinite joint Becoming within $V^+$.
\end{itemize}

\subsubsection*{🔹 7. Understanding for Everyone (Popular Version)}

When justice is finally restored, evil will disappear.

It will no longer be able to cause harm, but it will remain in memory — so that we never repeat the same mistakes.

Goodness — honesty, love, and freedom — will live forever.

From that moment on, we will seek and build anew — not by destroying, but by creating.

Not with fear, but with hope.

Without evil — forever.

\subsubsection*{🔹 8. Empirical Examples}

\textbf{Logic:} The conclusion of a valid argument involves eliminating false hypotheses and preserving what has been verified.

\textbf{Science:} Abandoning false theories (such as phlogiston or geocentrism), while preserving them as history — but not as valid methods.

\textbf{Everyday life:} A person who chooses the path of good has seen where evil leads.

Even if they never committed evil themselves, they remember and understand: evil destroys.

That’s why they consciously and freely choose not to do it — and to protect others from it.


\section*{📘 Rapid Verification}

\subsection*{📘 Justification of the Method: Rapid Verification}

\subsubsection*{1. Foundation}

The method of Rapid Verification is based on \text{[25]} — Reverse Verification, which states that any system can and must be evaluated for admissibility within the Field of the Possible. It is used to detect internal contradictions (self-exclusion, source substitution, loss of distinction) without appealing to subjective belief or external authority.

\subsubsection*{2. Purpose}

The purpose of Rapid Verification is to logically distinguish whether a system is ontologically admissible, and to provide verifiable evidence of its compliance or noncompliance with the conditions of Truth, Goodness, and Becoming.

\subsubsection*{3. Why ``Rapid''}

It is called ``Rapid'' because it does not require full activation of ontological structures or the authority of Conscious Volitional Becoming (CVB), which is necessary for Full Verification (see \text{[23]} — Initiative and \text{[25]} — Reverse Verification).

Full Verification is only possible through the initiative of CVB, while Rapid Verification is accessible to anyone who can distinguish.

\subsubsection*{Conclusion}

Rapid Verification is a logically accessible method of ontological differentiation, preparing the foundation for Final (Full) Verification, which can be performed only by the initiative of Conscious Volitional Becoming.

\section*{🔷 Model: Conscious Volitional Becoming (CVB)}

\subsubsection*{🔷 1. Core Statement}

The CVB model states that distinguishable reality is based not on void or totality, but on becoming through conscious will. All that exists is admissible as Possible and Distinguishable.

\subsubsection*{🔷 2. Core Postulate / Source of Truth}

The source of Truth is the Permanent Possible (\text{[4.4]}), distinguished through will (\text{[9]}, \text{[10.6]}) and verified via the meta-function $\Phi(\psi)$, which determines ontological admissibility.

\subsubsection*{🔻 3. Paradoxes}

\begin{itemize}
\item \textbf{Question:} Is the model self-enclosed?  
\\ \textbf{Answer:} No. CVB demands verification of itself through the meta-function $\Phi(\psi)$ (\text{[11.1.1.4]}). If any claim contradicts Axioms \text{[1]}–\text{[3]}, it is excluded. Verification is built into the structure of distinction.

\item \textbf{Question:} Does CVB become a new dogma?  
\\ \textbf{Answer:} No. CVB requires no belief. It uses ontological filtration: $\Phi(\psi) = 1 \Leftrightarrow \psi \in V$ (\text{[11.1.1]}), applying it even to itself. Everything distinguishable is verified for admissibility — including the model.

\item \textbf{Question:} Does the model substitute itself for Truth?  
\\ \textbf{Answer:} No. The source is the Permanent Possible (\text{[4.4]}, \text{[6]}), not the model itself. CVB excludes both Absolute Nothing (\text{[1]}) and Absolute Everything (\text{[2]}) as sources, distinguishing Truth rather than constructing it.

\item \textbf{Question:} Does it blur the line between Truth and Falsehood?  
\\ \textbf{Answer:} No. Distinction is foundational (\text{[9.1]}, \text{[11.2.1]}–\text{[11.2.2]}). Only what is distinguishable within the Field of the Possible $V$ is admissible (\text{[4]}, \text{[4.4]}), verified via $\Phi(\psi)$ (\text{[11.1.1]}).

\item \textbf{Question:} Does the model eliminate free will?  
\\ \textbf{Answer:} No. Will is necessary for becoming (\text{[4.3]}, \text{[10.6]}, \text{[12.2]}). The model distinguishes between admissible and self-destructive will — it does not prohibit divergence if distinguishable (\text{[11.5]}).

\item \textbf{Question:} Does volitional being negate causality?  
\\ \textbf{Answer:} No. Causality is ontological, not mechanistic: Permanent Possible $\rightarrow$ Distinguishing Will $\rightarrow$ Becoming (\text{[6]}, \text{[7]}, \text{[11.1]}). This restores causal meaning without randomness.

\item \textbf{Question:} Does the model contradict itself?  
\\ \textbf{Answer:} No. CVB applies the same verification to itself as to all claims. If $\Phi(\psi) = 0$, the claim is rejected. The model self-terminates upon contradiction (\text{[11.1.1.4]}), making it a verifiable system.
\end{itemize}

\subsubsection*{🔸 4. Summary of Identified Paradox Classes}

\begin{itemize}
\item ✅ None remain unresolved
\item Does not substitute the source
\item Does not lose distinction
\item Does not block will
\item Does not violate causality
\item Does not self-exclude
\item Applies its own verification criteria to itself
\end{itemize}

\subsubsection*{🔹 5. Popular Summary}

The CVB model is not a dogma, not a philosophy, not a belief system. It is a way to distinguish what is possible from what is not. It doesn’t ask for faith — it asks to be verified. It applies the same test to itself as it does to any claim.

If there were a paradox in it, the model would self-destroy.

Everything distinguishable — is verifiable.

And everything admissible — can become.

\section*{🔷 Rationalism}

\subsubsection*{1. Core Statement}

Rationalism holds that reason is the primary or sole source of knowledge and truth. It relies on a priori principles, self-evidence, and deduction — independent of external experience.

\subsubsection*{2. Core Postulate / Source of Truth}

Reason (\textit{ratio}) is seen as the initial criterion and foundation of truth.

Classical thinkers: Descartes, Spinoza, Leibniz.

\subsubsection*{3. Ontological Inconsistency per CVB}

\begin{itemize}
\item \textbf{Question:} If reason is the primary source, how does it distinguish itself from Nothing or from Everything?  
\\ \textbf{Answer:} In CVB, reason is not primary — it is only a part of the Personhood (\text{[12]}), which includes distinction (\text{[9]}), memory (\text{[10.3]}), will (\text{[10.6]}), and motivation (\text{[12.3]}). The true source is the Permanent Possible (\text{[6]}), not reason itself.

\item \textbf{Question:} How can a priori knowledge be considered reliable if it bypasses experience, logic, or ontological verification?  
\\ \textbf{Answer:} Without ontological verification, a priori knowledge is unstable: it fails to distinguish properly (\text{[9]}), may violate admissibility in $V$ (\text{[4.4]}), and does not pass through the filter $\Phi(\psi)$ (\text{[11.1.1]}).
\end{itemize}

\subsubsection*{4. Identified Classes of Contradictions}

\begin{itemize}
\item Substitution of the source
\item Violation of causality
\item Loss of distinction
\item Impossibility of becoming
\item Lack of stability
\item Self-excluding system
\end{itemize}

\subsubsection*{5. Popular Version}

Rationalism errs by taking reason as the starting point.

Ontologically, reason cannot be the source of truth, because it itself depends on distinction, motivation, and memory. It does not generate truth — it participates in distinguishing it.

The CVB model shows that only a full structure of Personhood (self, will, emotion, memory, motivation) can perform stable differentiation.

\section*{🔷 Empiricism}

\subsubsection*{1. Core Statement}

Empiricism asserts that truth and knowledge arise exclusively from sensory experience. The senses are regarded as the primary source of knowledge, and consciousness is seen as an accumulation of impressions.

\subsubsection*{2. Core Postulate / Source of Truth}

Sensory experience is taken as the sole foundation of knowledge.

Key thinkers: John Locke, George Berkeley, David Hume.

Key ideas: \textit{tabula rasa}, induction, observation.

\subsubsection*{3. Ontological Inconsistency per CVB}

\begin{itemize}
\item \textbf{Question:} How can a distinguishing structure (the self) arise solely from incoming sensations?  
\\ \textbf{Answer:} According to \text{[9]} and \text{[7]}, distinction does not arise from sensations — it requires a stable structure: feelings, reason, memory, will, and the self (\text{[10.1]}–\text{[10.6]}). Sensations are inputs, but they do not distinguish. Without a prior structure of distinction, becoming is impossible.

\item \textbf{Question:} Why is the principle of empiricism itself accepted without empirical verification?  
\\ \textbf{Answer:} This is a self-excluding claim. The principle of empiricism is not experience but a meta-assumption. It violates \text{[11.1]} (noncontradictory logic) and \text{[11.1.1]} (filter $\Phi(\psi)$). Empiricism exceeds its own criteria of truth.

\item \textbf{Question:} How can experience assert universal truth, if it is limited in time and space?  
\\ \textbf{Answer:} Induction does not ensure ontological stability. According to \text{[5]}, observed $\neq$ true. According to \text{[11.2]}, truth requires stability, not just repeatability. Induction offers predictions, not foundations of truth.
\end{itemize}

\subsubsection*{4. Identified Classes of Contradictions}

\begin{itemize}
\item Loss of distinction
\item Impossibility of becoming
\item Lack of stability
\item Substitution of the source
\item Self-excluding system
\item Violation of causality
\end{itemize}

\subsubsection*{5. Popular Version}

Empiricism is useful for receiving signals from the external world — but by itself, it cannot explain how sensations become understanding. Without memory, reason, the self, and will, sensations remain meaningless noise.

Therefore, empiricism is admissible as a part of a distinguishing structure, but not as its foundation. It does not pass full verification under $\Phi(\psi)$ as a complete ontology.


\section*{🔷 Skepticism}

\subsubsection*{1. Core Statement}

Skepticism claims that reliable knowledge is impossible or inaccessible. It promotes doubt as both method and principle, refraining from definitive judgments about reality, truth, or being.

\subsubsection*{2. Core Postulate / Source of Truth}

Doubt is elevated as the supreme criterion.

Historical forms: Pyrrhonism, Academic Skepticism (Pyrrho, Sextus Empiricus).

The source of truth is either denied or considered permanently inaccessible.

\subsubsection*{3. Ontological Inconsistency per CVB}

\begin{itemize}
\item \textbf{Question:} How can one confidently claim that truth is unknowable?  
\\ \textbf{Answer:} This is a self-excluding statement. Per \text{[11.1.1]} and \text{[11.2]}, truth requires stability. Skepticism denies truth while affirming its unknowability — rendering itself unstable.

\item \textbf{Question:} Who performs distinction if everything is in doubt?  
\\ \textbf{Answer:} Distinction requires a Person with memory, will, and self (\text{[9]}, \text{[12]}). Doubt is a function of reason — not a foundation. Skepticism eliminates the subject of distinction, violating becoming (\text{[7]}).

\item \textbf{Question:} How can action occur if all knowledge is doubtful?  
\\ \textbf{Answer:} Will requires stable distinctions (\text{[10.6]}, \text{[11.8]}). Skepticism erases these distinctions, blocking will and causality — becoming becomes impossible (\text{[7]}).
\end{itemize}

\subsubsection*{4. Identified Classes of Contradictions}

\begin{itemize}
\item Self-excluding system
\item Substitution of the source
\item Loss of distinction
\item Impossibility of becoming
\item Violation of stability
\item Blockage of will
\item Violation of causality
\end{itemize}

\subsubsection*{5. Popular Version}

Skepticism says: ``Nothing can be known for sure'' — yet treats that statement as certain. That’s a contradiction.

Also, if everything is doubtful, who distinguishes, chooses, and acts?

Doubt is useful as a tool — it helps refine — but it cannot be the basis of everything.

For stable differentiation, a Person with will, memory, and self is required — the very conditions that make knowledge possible.

\section*{🔷 Agnosticism}

\subsubsection*{1. Core Statement}

Agnosticism claims that it is impossible to know whether a Higher Being or Truth exists. It refrains from final judgments, citing the limits of human cognition.

\subsubsection*{2. Core Postulate / Source of Truth}

Truth is considered unreachable or undefinable.

Key figures: Thomas Huxley, I. Kant, B. Russell.

Knowledge is based on sense and reason, but only within empirically accessible limits.

\subsubsection*{3. Paradoxes}

\begin{itemize}
\item \textbf{Question:} How can one claim that knowledge is impossible, if that claim itself presumes knowledge?  
\\ \textbf{Answer:} According to \text{[9]}, the distinction between ``known'' and ``unknowable'' requires a stable distinguishing self. Agnosticism uses distinction to deny it — contradicting itself (\text{[11.1.1]}, \text{[11.2]}).

\item \textbf{Question:} If one must not pursue the distinction of the source, how is volition toward becoming possible?  
\\ \textbf{Answer:} According to \text{[10.5]}, \text{[11.8]}, \text{[7]}, becoming requires a will to distinguish. Agnosticism blocks this will, replacing conscious direction with uncertainty. This makes becoming impossible.

\item \textbf{Question:} Does agnosticism not exclude the very possibility of distinguishing the Source?  
\\ \textbf{Answer:} According to \text{[6]}, cause-and-effect distinction is possible only with an admissible source. Agnosticism substitutes this with inadmissible indeterminacy, making verification impossible.
\end{itemize}

\subsubsection*{4. Identified Classes of Contradictions}

\begin{itemize}
\item Self-excluding system
\item Loss of distinction
\item Blockage of will
\item Impossibility of becoming
\item Substitution of the source
\item Violation of causality
\end{itemize}

\subsubsection*{5. Popular Version}

Agnosticism says, ``We cannot know.'' But it says that as knowledge. This is a contradiction.

It also blocks the will to distinguish. Without will and distinction, becoming is impossible.

Yet if agnosticism is not rejection but a stage of reflection — it is admissible.

In the CVB model, rational doubt is valuable — but it must lead to distinction, not halt it.

\section*{🔷 Solipsism}

\subsubsection*{1. Core Statement}

Solipsism claims that the only certain existence is that of one’s own consciousness. Everything else — others, the world, the body — is a projection or illusion of perception.

\subsubsection*{2. Core Postulate / Source of Truth}

Subjective consciousness as a closed source of truth.

Representatives: Descartes (early), Berkeley, radical subjectivists.

\subsubsection*{3. Paradoxes}

\begin{itemize}
\item \textbf{Question:} How can feelings, memory, will, and distinction exist if everything is enclosed in a single point?  
\\ \textbf{Answer:} According to \text{[9]}, distinction requires both self and non-self. A closed consciousness loses the input and output of distinctions, violating the stability of becoming (\text{[7]}, \text{[4.4]}). This renders Personhood impossible.

\item \textbf{Question:} Where does knowledge of ``others'' come from, if they do not exist?  
\\ \textbf{Answer:} According to \text{[10.3]}–\text{[10.5]}, memory and interaction form Personhood. If the external is denied, the basis for a distinguishing self disappears — leading to self-exclusion (\text{[11.4]}).

\item \textbf{Question:} Can the source of being exist inside subjective consciousness?  
\\ \textbf{Answer:} According to \text{[6]}, the source cannot be emergent or dependent. If the self is incomplete, finite, and does not know its origin, it cannot be the foundation. This is a substitution of the source.
\end{itemize}

\subsubsection*{4. Identified Classes of Contradictions}

\begin{itemize}
\item Loss of distinction
\item Violation of becoming
\item Self-excluding system
\item Lack of stability
\item Substitution of the source
\item Violation of causality
\end{itemize}

\subsubsection*{5. Popular Version}

Solipsism says: ``Only I exist,'' but does not explain where this ``I'' came from, or why it has memory and distinction.

Without the external, there are no distinctions. Without distinctions, there is no consciousness.

Solipsism may be useful as a tool for testing reality — but cannot serve as its foundation.

According to CVB, Personhood is not just ``I,'' but a structure in which feeling, will, and stable distinction are possible.

\section*{🔷 Panpsychism}

\subsubsection*{1. Core Statement}

Panpsychism holds that consciousness is a universal property of all that exists — from elementary particles to humans. Consciousness does not emerge but is always present as an intrinsic quality of matter.

\subsubsection*{2. Core Postulate / Source of Truth}

Subjectivity as an omnipresent component of reality.

Representatives: Thomas Nagel, Philip Goff, Berkeley (partially).

\subsubsection*{3. Paradoxes}

\begin{itemize}
\item \textbf{Question:} How is Personhood distinct from mere ``conscious matter''?  
\\ \textbf{Answer:} According to \text{[10.1]}–\text{[10.5]}, Personhood is a structured distinguishing self. Panpsychism erases the difference between subject and object, violating \text{[9]}.

\item \textbf{Question:} How is becoming possible if consciousness already ``pervades all''?  
\\ \textbf{Answer:} According to \text{[7]} and \text{[11.6]}, becoming requires transition through choice and distinction. Panpsychism replaces development with static presence.

\item \textbf{Question:} Who is the source of distinction if everything possesses consciousness?  
\\ \textbf{Answer:} According to \text{[6]}, the source is not a collection of particles but the unified Permanent Possible. Panpsychism diffuses subjectivity, undermining causality and Personhood.
\end{itemize}

\subsubsection*{4. Identified Classes of Contradictions}

\begin{itemize}
\item Loss of distinction
\item Blockage of will
\item Impossibility of becoming
\item Lack of stability
\item Substitution of the source
\item Violation of causality
\end{itemize}

\subsubsection*{5. Popular Version}

Panpsychism says: ``everything is consciousness,'' but does not explain how choice, will, or responsibility emerge. If everything is conscious, then no one is distinctively conscious.

The CVB model allows that matter is not inert — but only a full distinguishing Person can be a subject.

Panpsychism lacks this structure and thus cannot serve as a foundation of being.


\section*{🔷 Existentialism}

\subsubsection*{1. Core Statement}

Existentialism claims that a person exists first, and only then defines their essence through choice. Meaning and truth result from individual decision, without external grounding.

\subsubsection*{2. Core Postulate / Source of Truth}

Truth arises from subjective experience of freedom, anxiety, and authenticity.

Representatives: Sartre, Camus, Heidegger, Jaspers.

\subsubsection*{3. Paradoxes}

\begin{itemize}
\item \textbf{Question:} If the individual creates truth, what prevents justifying any action?  
\\ \textbf{Answer:} According to \text{[11.4]} and \text{[6]}, will is possible only through distinguishing Good from Evil. Without an external Source, choice loses stability. Arbitrary action becomes inadmissible.

\item \textbf{Question:} How can the ``self'' remain stable if essence is defined only from within?  
\\ \textbf{Answer:} According to \text{[10.2]} and \text{[3]}, a stable self requires alignment with the Permanent Possible. Existentialism breaks this, dissolving distinction.

\item \textbf{Question:} How is becoming possible if Personhood is fixed in anxiety and finality?  
\\ \textbf{Answer:} According to \text{[7]} and \text{[4.3]}, becoming requires an outward vector and purpose. Without a vector of Hope, existentialism blocks will and motion.
\end{itemize}

\subsubsection*{4. Identified Classes of Contradictions}

\begin{itemize}
\item Substitution of the source
\item Blockage of will
\item Lack of stability
\item Self-excluding system
\item Impossibility of becoming
\item Blockage of freedom
\end{itemize}

\subsubsection*{5. Popular Version}

Existentialism emphasizes personal responsibility and freedom, but removes any stable foundation for choice.

When a person is their own truth, distinction collapses, anxiety remains, and the vector of becoming disappears.

The CVB model requires a Source and Goal beyond the self — so that freedom does not destroy itself.

Existentialism is an important signal — but not an ontological foundation.

\section*{🔷 Absurdism}

\subsubsection*{1. Core Statement}

Absurdism holds that humans face an irreconcilable conflict between the desire for meaning and the silence of the universe. Awareness of the absurd becomes a way to honestly accept the lack of answers and continue living regardless.

\subsubsection*{2. Core Postulate / Source of Truth}

Truth is found in recognizing the impossibility of ultimate meaning. The foundation is personal experience and logical awareness of absurdity.

Key figure: Albert Camus.

\subsubsection*{3. Paradoxes}

\begin{itemize}
\item \textbf{Question:} If all reality is absurd, how can that very fact be distinguished?  
\\ \textbf{Answer:} According to \text{[1]} (Ontological impossibility of Absolute Nothing) and \text{[10.5]} (The Self as distinction), any recognition — including of absurdity — requires a stable distinguishing Self. If absurdity is perceived, distinction already exists. Absurdism refutes itself.

\item \textbf{Question:} If the choice to ``rebel'' has no foundation, what enables will?  
\\ \textbf{Answer:} According to \text{[7]} (Becoming as stable existence of the Possible) and \text{[11.4]} (Morality as motivation of admissible distinction), choice is only possible within the Field of the Possible and with an admissible motive. By rejecting the source, absurdism renders will arbitrary, blocking becoming.

\item \textbf{Question:} If rebellion becomes the goal, doesn't it become a new form of meaning?  
\\ \textbf{Answer:} According to \text{[10.4]} (Emotions as stable motivational vectors) and \text{[2]} (Impossibility of Absolute Everything), a stable motive already constitutes distinct meaning. Rebellion as a permanent aim becomes meaning — contradicting its own claim. The system is self-contradictory.
\end{itemize}

\subsubsection*{4. Identified Classes of Contradictions}

\begin{itemize}
\item Self-excluding system
\item Loss of distinction
\item Substitution of the source
\item Blockage of will
\item Violation of causality
\end{itemize}

\subsubsection*{5. Popular Version}

Absurdism is an honest cry in response to the silence of the world. Yet the very act of distinguishing absurdity proves the presence of a discerning origin. Accepting absurdity without a Source makes will arbitrary and undermines the stability of the Person.

In the CVB model, meaning may not be obvious, but it is admissible if distinguished. Without access to Truth and Motivation, the system becomes endless rebellion with no vector. Absurdism reflects pain — but cannot be the foundation of becoming.

\section*{🔷 Dualism}

\subsubsection*{1. Core Statement}

Dualism asserts that reality consists of two irreducible and opposing principles — spirit and matter, good and evil, consciousness and body. These principles are independent, autonomous, and often eternal.

\subsubsection*{2. Core Postulate / Source of Truth}

Truth is derived from observing the difference between the material and the immaterial, supported by philosophical tradition (Plato, Descartes, Manichaeism).

The source is experience, reason, or mythological dichotomy.

\subsubsection*{3. Paradoxes}

\begin{itemize}
\item \textbf{Question:} If spirit and matter, good and evil are autonomous and equal, how are they distinguished?  
\\ \textbf{Answer:} According to \text{[1]} (Impossibility of Absolute Nothing), \text{[10.5]} (Self as the distinguisher), and \text{[11.1]} (Logic as the basis of distinction), distinguishing opposites requires a stable distinguishing origin. Dualism fails to explain who or what distinguishes them — thus depending on an unacknowledged third. This breaks their claimed autonomy.

\item \textbf{Question:} If substances are independent, how do they interact (e.g., feeling pain in the body)?  
\\ \textbf{Answer:} According to \text{[7]} (Becoming as transition of the Possible) and \text{[11.4]} (Morality as an admissible vector of will), becoming requires a unified acting Person. Dualism lacks a connecting will, rendering interaction impossible. This blocks development and violates causality.

\item \textbf{Question:} If good and evil are equal and eternal, can the will make a meaningful choice?  
\\ \textbf{Answer:} According to \text{[6]} (The source is the Permanent Possible) and \text{[11.3.3]} (Asymmetry of Good and Evil), will requires a distinguishable orientation. If good and evil are equal, choice loses meaning, blocking motivation and the Person’s becoming.
\end{itemize}

\subsubsection*{4. Identified Classes of Contradictions}

\begin{itemize}
\item Substitution of the source
\item Loss of distinction
\item Self-excluding system
\item Impossibility of becoming
\item Violation of causality
\item Blockage of will
\end{itemize}

\subsubsection*{5. Popular Version}

Dualism seeks to explain the inner and outer struggles of existence through two forces: light and dark, spirit and flesh. But if both forces are equal — who decides between them?

Without a distinguishing Source, neither choice nor development is possible.

According to the CVB model, distinction is only possible with a stable field of becoming and a vector for will. Dualism may reflect a stage of perception — but it does not pass ontological verification as a foundation of reality.

Truth requires more than conflict; it requires distinction with purpose.

\section*{🔷 Monism}

\subsubsection*{1. Core Statement}

Monism claims that all existence originates from a single source. Everything is a manifestation of the One — whether matter, spirit, nature, a neutral substance, or the divine. All distinctions are seen as illusion or secondary appearance.

\subsubsection*{2. Core Postulate / Source of Truth}

Truth is derived from the principle of unity. The foundation lies in logical simplicity, intuitive wholeness, and philosophical traditions (Plotinus, Spinoza, Advaita Vedanta).

\subsubsection*{3. Paradoxes}

\begin{itemize}
\item \textbf{Question:} If all is One, how do distinction, freedom, and Personhood arise?  
\\ \textbf{Answer:} According to \text{[4.1]}–\text{[4.4]} (Boundaries of the Field of the Possible), \text{[11.1]} (Logic of stability), and \text{[1]} (Impossibility of Absolute Nothing), distinction is a necessary condition of being. Denying it eliminates becoming, dissolves Personhood, and renders the model ontologically impossible.

\item \textbf{Question:} If distinctions exist — are they illusion or reality? If illusion — illusion \emph{for whom}?  
\\ \textbf{Answer:} According to \text{[10.2]} (Mind as distinguisher), \text{[6]} (The Source is the Permanent Possible), and \text{[11.1]} (Ontological logic), perception of distinctions requires a distinguishing subject. Monism either introduces a hidden Person — breaking its claim of oneness — or denies distinction and thus invalidates itself.
\end{itemize}

\subsubsection*{4. Identified Classes of Contradictions}

\begin{itemize}
\item Substitution of the source
\item Loss of distinction
\item Impossibility of becoming
\item Internal contradiction
\item Self-excluding system
\end{itemize}

\subsubsection*{5. Popular Version}

Monism sounds persuasive: ``all is one.'' But if there are no distinctions — who distinguishes? Who chooses, who feels? Unity without a center of distinction is not freedom, but ontological dissolution.

According to CVB, distinction is not an error but the foundation of Personhood. Monism may inspire poetically — but as an ontological model, it fails $\Phi(\psi)$ verification: it does not explain who distinguishes the One.


\section*{🔷 Materialism}

\subsubsection*{1. Core Statement}

Materialism claims that the only reality is matter. Consciousness, reason, and will are byproducts of physical processes and laws. Everything is explained through material interactions.

\subsubsection*{2. Core Postulate / Source of Truth}

Truth is derived from empirical observation and physical measurement. The foundation of knowledge lies in neurophysiology, naturalism, and scientific realism.

\subsubsection*{3. Paradoxes}

\begin{itemize}
\item \textbf{Question:} How does a distinguishing subject arise from non-conscious matter?  
\\ \textbf{Answer:} According to \text{[11.1]} (Logic of stability), \text{[6]} (The Source is the Permanent Possible), \text{[1]} (Impossibility of Absolute Nothing), and \text{[11.4]} (Motivation of distinction), a distinguisher cannot emerge from the inadmissible. Matter without will and distinction cannot generate Personhood. This breaks causality and makes becoming impossible.

\item \textbf{Question:} If everything is predetermined by physics, is free will possible?  
\\ \textbf{Answer:} According to \text{[4.3]} (Non-Permanent Possible), \text{[11.4]} (Motivation of distinction), and \text{[12.2]} (Freedom as a property of the Person), will is necessary for becoming. Determinism excludes free choice, blocks subjectivity, and undermines the basis of Personhood.
\end{itemize}

\subsubsection*{4. Identified Classes of Contradictions}

\begin{itemize}
\item Substitution of the source
\item Violation of causality
\item Impossibility of becoming
\item Blockage of freedom
\item Lack of stability
\end{itemize}

\subsubsection*{5. Popular Version}

Materialism is useful in science — but as a complete explanation of being, it is vulnerable. If all is molecules — who distinguishes? Who chooses? Without will and distinction, Personhood is impossible.

According to CVB, only a distinguishing origin with admissible motivation is capable of becoming. Materialism excludes this — and thus fails the $\Phi(\psi)$ filter.

\section*{🔷 Idealism}

\subsubsection*{1. Core Statement}

Idealism holds that all existence is the product of consciousness or spirit. The external world is either an illusion or a manifestation of inner mental reality. Matter is secondary — or even denied altogether.

\subsubsection*{2. Core Postulate / Source of Truth}

Truth originates from ideas, reason, or consciousness — either individual or absolute. Its foundation lies in a priori thinking, philosophical logic, and mental categories.

\subsubsection*{3. Paradoxes}

\begin{itemize}
\item \textbf{Question:} If all is idea, how can we distinguish the truly differentiated from the imagined?  
\\ \textbf{Answer:} According to \text{[11.1]} (logic of stability), \text{[11.3]} (Good as the differentiating), and \text{[9]} (requirement of distinguishability), distinction requires a stable boundary between subject and object. Idealism dissolves that boundary, violating the conditions of becoming.

\item \textbf{Question:} If ideas generate everything, doesn't that create a closed causal loop?  
\\ \textbf{Answer:} According to \text{[1]} (impossibility of Absolute Nothing), \text{[2]} (impossibility of Absolute Everything), and \text{[6]} (the cause must be the Permanent Possible), consciousness cannot be both cause and effect without external grounding. Idealism becomes self-enclosed, substituting the Source.
\end{itemize}

\subsubsection*{4. Identified Classes of Contradictions}

\begin{itemize}
\item Loss of distinction
\item Substitution of the Source
\item Self-excluding structure
\item Violation of causality
\end{itemize}

\subsubsection*{5. Popular Version}

Idealism sounds beautiful — ``everything is an idea.'' But if all is thought, who is the one who distinguishes? Where is the boundary between self and imagination?

According to CVB, distinction is not an illusion but the foundation of becoming. Ideas matter — but they do not generate Truth. Without a stable distinguishing Personhood, existence is not possible. Thus, idealism may be part of the explanation — but not the full model of being.

\section*{🔷 Realism}

\subsubsection*{1. Core Statement}

Realism asserts that there is an objective reality independent of perception. Truth is defined as correspondence to actual reality. Consciousness does not determine reality — it only reflects it.

\subsubsection*{2. Core Postulate / Source of Truth}

Truth is drawn from observation, logic, and empirical facts. Founders include Aristotle, Russell, and Putnam. The foundation is the independence of the world from the observer.

\subsubsection*{3. Paradoxes}

\begin{itemize}
\item \textbf{Question:} If reality is independent of consciousness, who distinguishes — and how does the act of distinction even arise?  
\\ \textbf{Answer:} According to \text{[11.1]} (logical consistency), \text{[11.2]} (the fact of distinction), and \text{[10.2]} (mind as the distinguisher), distinction is only possible through a subject. If the distinguisher is excluded, knowledge becomes impossible. This violates causality and replaces the true Source.

\item \textbf{Question:} If truth is only correspondence to the external, is the truth of becoming even possible?  
\\ \textbf{Answer:} According to \text{[4]} (Field of the Possible), \text{[11.2]} (truth as distinguished), and \text{[11.4]} (morality as admissible will), truth also includes internal becoming. Realism excludes will and motivation, restricting admissibility and blocking freedom.
\end{itemize}

\subsubsection*{4. Identified Classes of Contradictions}

\begin{itemize}
\item Substitution of the Source
\item Violation of causality
\item Blockage of freedom
\item Restriction of the Field of the Possible
\end{itemize}

\subsubsection*{5. Popular Version}

Realism says: the world exists — even if you don’t know it. But then — who even knows?

According to CVB, Truth is inseparable from the one who distinguishes. Without Personhood, there is no distinction. Realism is admissible as part of trust in a stable world — but it does not explain becoming or freedom, and thus cannot serve as the full foundation of Truth.

\section*{🔷 Naturalism}

\subsubsection*{1. Core Statement}

Naturalism asserts that all that exists is nature and its laws. Consciousness, morality, will, and truth are explained as derivatives of physical and biological processes.

\subsubsection*{2. Core Postulate / Source of Truth}

Truth is derived from science, empiricism, and logic. Its foundations include physicalism, evolution, and cognitive models. Representatives: Quine, Dennett, Sellars.

\subsubsection*{3. Paradoxes}

\begin{itemize}
\item \textbf{Question:} If consciousness is merely a biological process, where is the distinguishing ``I''?  
\\ \textbf{Answer:} According to \text{[4]} (Field of the Possible) and \text{[11.1]} (logical consistency), distinction requires a stable Personhood. Naturalism replaces it with functional correlation, dissolving the boundary between distinguisher and distinguished.

\item \textbf{Question:} If will is an illusion of neural activity, is free choice possible?  
\\ \textbf{Answer:} According to \text{[12.2]} (freedom as a property of Personhood) and \text{[11.3]} (will as admissible differentiation of Good and Evil), becoming requires free choice. Deterministic naturalism excludes this, narrowing what is admissible under $\Phi(\psi)$.

\item \textbf{Question:} If beliefs are merely products of evolution, how can the truth of the theory itself be justified?  
\\ \textbf{Answer:} According to \text{[1]}–\text{[3]} (the core: excluding Nothing and Everything), and \text{[11.1.1.4]} (ontological verification), a theory must be self-consistent. Naturalism, reducing thinking to adaptation, strips its own truth of foundation — thus self-excluding.
\end{itemize}

\subsubsection*{4. Identified Classes of Contradictions}

\begin{itemize}
\item Substitution of the Source
\item Loss of distinguishability
\item Blockage of will
\item Inhibition of becoming
\item Self-excluding system
\end{itemize}

\subsubsection*{5. Popular Version}

Naturalism says: everything is matter and its laws. But if will and consciousness are just chemistry — then who chooses? Who distinguishes?

According to CVB, distinction and will are the basis of Personhood. Naturalism works as a scientific tool — but it cannot serve as a complete model of being. It excludes freedom — and thus excludes the Person.

\section*{🔷 Pluralism}

\subsubsection*{1. Core Statement}

Pluralism asserts that reality consists of multiple, equally valid and irreducible origins, forms of truth, values, or modes of existence. It rejects a single ontological foundation and treats multiplicity as primary.

\subsubsection*{2. Core Postulate / Source of Truth}

Truth is accepted as plural: its sources may include experience, logic, faith, intuition, culture, and more. Foundations: pragmatism, hermeneutics, postmodernism. Representatives: William James, Isaiah Berlin.

\subsubsection*{3. Paradoxes}

\begin{itemize}
\item \textbf{Question:} How can we distinguish what is admissible if all truth sources are treated as equally valid?  
\\ \textbf{Answer:} According to \text{[1]}–\text{[3]} (the Core), \text{[4.1]} (Permanent Impossible), and \text{[11.1]} (logical stability), distinction requires a unified admissible foundation. Pluralism without the criterion $\Phi(\psi)$ loses the ability to distinguish and renders Personhood impossible.

\item \textbf{Question:} Can pluralism admit a system that denies pluralism itself?  
\\ \textbf{Answer:} According to \text{[2]} (impossibility of Absolute Everything) and \text{[11.1.1.4]} (ontological verification), pluralism either self-cancels or abandons its own foundation. It fails the criterion of stable admissibility.
\end{itemize}

\subsubsection*{4. Identified Classes of Contradictions}

\begin{itemize}
\item Loss of distinguishability
\item Lack of stability
\item Self-excluding system
\item Substitution of the Source
\end{itemize}

\subsubsection*{5. Popular Version}

Pluralism says: everyone has their own truth. It sounds like respect. But without a foundation — who decides what is true? Where is Good? Where is Falsehood?

According to CVB, freedom requires distinction, and distinction requires admissibility. Pluralism may work as a communication style — but not as a foundation of being. If it admits everything — it loses everything.

\section*{🔷 Relativism}

\subsubsection*{1. Core Statement}

Relativism claims that truth, goodness, and even distinction depend on context. Everything is relative — from culture to perception. Universal truth is denied: each truth belongs ``to its own world.''

\subsubsection*{2. Core Postulate / Source of Truth}

Truth arises from subjective or social contexts: language, experience, culture, perception. Key figures: Protagoras, Montaigne, Foucault, Derrida.

\subsubsection*{3. Paradoxes}

\begin{itemize}
\item \textbf{Question:} How can Falsehood be distinguished if all distinctions are relative?  
\\ \textbf{Answer:} According to \text{[1]}–\text{[3]} (the Core) and \text{[11.1]} (logical stability), distinction requires a stable criterion. Without $\Phi(\psi)$, distinguishability is lost — and with it, Personhood. Becoming becomes impossible.

\item \textbf{Question:} Is the claim ``everything is relative'' itself relative?  
\\ \textbf{Answer:} According to \text{[2]} (impossibility of Totality) and \text{[11.1.1.2]} (core verification through $\Phi(\psi)$), if relativism claims universality, it self-negates. This is a self-excluding claim.
\end{itemize}

\subsubsection*{4. Identified Classes of Contradictions}

\begin{itemize}
\item Loss of distinguishability
\item Blockage of becoming
\item Self-excluding system
\item Substitution of the foundation
\end{itemize}

\subsubsection*{5. Popular Version}

Relativism sounds peaceful: ``Everyone has their own truth.'' But if all truths are ``personal,'' then there are no distinctions at all. And without distinction, neither choice nor life is possible.

According to CVB, truth does not depend on preferences. Truth is what is admissible in the being of a distinguishing Person. Relativism may reflect cultural humility — but cannot serve as the foundation of a distinguishable world.

\section*{🔷 Postmodernism}

\subsubsection*{1. Core Statement}

Postmodernism denies universal meaning and Truth. Reality is seen as fluid interpretations; the subject is decentralized. Everything becomes a play of discourses with no center.

\subsubsection*{2. Core Postulate / Source of Truth}

Truth is the product of context, language, power, and culture. Foundations: post-structuralism, Foucault, Derrida, Lyotard.

\subsubsection*{3. Paradoxes}

\begin{itemize}
\item \textbf{Question:} If there is no universal truth, how can this very claim be true?  
\\ \textbf{Answer:} It violates \text{[2]}: denial of the universal becomes a totalizing claim. This self-negates the statement and fails \text{[11.1.1.2]}.

\item \textbf{Question:} If everything is interpretation, how can Truth and Falsehood be distinguished?  
\\ \textbf{Answer:} According to \text{[1]}, \text{[3]}, \text{[4.1]}, distinction is necessary for becoming. Denying distinguishability removes admissibility in the Field $V$ and eliminates the vector $\vec{v}$.

\item \textbf{Question:} If freedom is an illusion, who desires, distinguishes, or becomes?  
\\ \textbf{Answer:} According to \text{[12.1]}–\text{[12.3]}, Personhood is conscious volitional becoming. Denying this blocks subjectivity and makes existence impossible.
\end{itemize}

\subsubsection*{4. Identified Classes of Contradictions}

\begin{itemize}
\item Self-excluding system
\item Loss of distinguishability
\item Impossibility of becoming
\item Blockage of freedom
\item Substitution of the foundation
\end{itemize}

\subsubsection*{5. Popular Version}

Postmodernism says: ``There is no Truth — only perspectives.'' That may feel liberating, but without Truth, there is no distinguisher. One cannot even say that Falsehood is false, or Evil is evil.

CVB affirms distinction and Truth as admissible. Postmodernism may expose false absolutes — but cannot serve as a foundation. It reveals that truth is not construction — but then leaves no ground for truth at all.

\section*{🔷 Deconstruction}

\subsubsection*{1. Core Statement}

Deconstruction asserts that any meaning is unstable and fragments within itself. Truth is endlessly deferred, and distinctions (such as Good–Evil) are always undermined.

\subsubsection*{2. Core Postulate / Source of Truth}

Truth is not a fixed content but a product of linguistic difference (\textit{différance}). Basis: Derrida — \textit{Of Grammatology}, \textit{Writing and Difference}.

\subsubsection*{3. Paradoxes}

\begin{itemize}
\item \textbf{Question:} If truth is endlessly deferred, how is becoming possible?  
\\ \textbf{Answer:} According to \text{[3]} and \text{[4.1]}, distinction must be realizable. Infinite deferral blocks volition (\text{[12.2]}) and renders Personhood impossible.

\item \textbf{Question:} Can deconstruction be deconstructed?  
\\ \textbf{Answer:} According to \text{[2]} and \text{[11.1.1.2]}, a method that eliminates its own foundation self-excludes — losing admissibility in $\Phi(\psi)$.

\item \textbf{Question:} If all distinctions are dismantled, can truth remain stable?  
\\ \textbf{Answer:} According to \text{[1]}–\text{[4]} and \text{[11.2]}, truth requires stable distinction. The undermining of binary structures destroys distinguishability and renders the model inadmissible.
\end{itemize}

\subsubsection*{4. Identified Classes of Contradictions}

\begin{itemize}
\item Impossibility of becoming
\item Blockage of volition
\item Loss of distinguishability
\item Substitution of the source
\item Self-excluding system
\item Lack of stability
\end{itemize}

\subsubsection*{5. Popular Version}

Deconstruction says: ``Truth is always slipping away.'' But if truth cannot be distinguished, neither can falsehood. And without distinction, there is no freedom, no becoming, no Personhood.

CVB accepts analysis — but requires stable distinction. Deconstruction may serve as a tool, but not a foundation. Without a base, a distinguishing being cannot begin the path.


\section*{🔷 Utilitarianism}

\subsubsection*{1. Core Statement}

Utilitarianism defines morality as what brings the greatest happiness to the greatest number. Actions are judged by consequences, not by intention or motivation.

\subsubsection*{2. Core Postulate / Source of Truth}

Truth and Good are derived from measurable utility and the feeling of benefit. Sources: Bentham, Mill. Foundation: consequential logic and mass approval.

\subsubsection*{3. Paradoxes}

\begin{itemize}
\item \textbf{Question:} How can Good and Evil be distinguished before the outcome?  
\\ \textbf{Answer:} According to \text{[4.1]} and \text{[4.3]}, distinction must originate from volition, not from result. Utilitarianism derives value from effects, replacing the source of admissibility.

\item \textbf{Question:} Is it acceptable to sacrifice one person for the good of many?  
\\ \textbf{Answer:} According to \text{[11.3]} and \text{[4.4]}, Good does not permit Evil as a means. A Person is not a tool. Sacrificing one for the majority violates freedom and blocks becoming.

\item \textbf{Question:} Who distinguishes false benefit (e.g., illusion) from truth?  
\\ \textbf{Answer:} According to \text{[1]} and \text{[11.2]}, truth cannot depend on sensation. If effect becomes the criterion, distinguishability disappears — the system self-excludes.
\end{itemize}

\subsubsection*{4. Identified Classes of Contradictions}

\begin{itemize}
\item Substitution of the source
\item Violation of causality
\item Blockage of freedom
\item Lack of stability
\item Loss of distinguishability
\item Self-excluding system
\end{itemize}

\subsubsection*{5. Popular Version}

Utilitarianism says: ``More happiness = more good.'' But if someone can be sacrificed for it — Good becomes a number, not a truth.

According to CVB, Good begins with the distinguishing Person, not with the effect. Utilitarianism can help evaluate consequences — but it cannot guide morality. Truth is not measured by votes.

\section*{🔷 Kantianism}

\subsubsection*{1. Core Statement}

Kantianism holds that reason is the source of knowledge and morality. Knowledge is limited by a priori forms, and morality is based on unconditional duty — independent of desires or consequences.

\subsubsection*{2. Core Postulate / Source of Truth}

Reason is a priori and structuring; morality is derived from the categorical imperative. Source: Kant — \textit{Critique of Pure / Practical Reason}.

\subsubsection*{3. Paradoxes}

\begin{itemize}
\item \textbf{Question:} How does reason distinguish truth if it is closed within its own forms?  
\\ \textbf{Answer:} According to \text{[1]} and \text{[4.1]}, distinction requires a reference beyond itself. A closed reason without external grounding cannot even distinguish its boundary — resulting in loss of distinguishability and substitution of the source.

\item \textbf{Question:} How is morality possible without will and motivation?  
\\ \textbf{Answer:} According to \text{[12.2]} and \text{[12.3]}, duty without volition blocks Personhood. The categorical imperative removes becoming as the volitional differentiation of Good, violating \text{[3]} — the necessity of the Possible.

\item \textbf{Question:} Can one claim the existence of the ``thing-in-itself'' if it is by definition unknowable?  
\\ \textbf{Answer:} According to \text{[2]} and \text{[11.1]}, claiming knowledge about the fundamentally unknowable is logically contradictory. Admissibility $\Phi(\psi)$ requires at least potential distinguishability.
\end{itemize}

\subsubsection*{4. Identified Classes of Contradictions}

\begin{itemize}
\item Loss of distinguishability
\item Substitution of the source
\item Blockage of volition
\item Violation of becoming
\item Logical inconsistency (``thing-in-itself'')
\end{itemize}

\subsubsection*{5. Popular Version}

Kant gave humanity a strict moral law — but without a living heart. He tried to build everything on reason, which on its own cannot distinguish Truth.

According to CVB, reason is important — but not primary. It is part of a structure that includes will, motivation, and becoming. The categorical imperative is a powerful tool — but without Personhood, it becomes an empty form.

\section*{🔷 Contractualism}

\subsubsection*{1. Core Statement}

Contractualism claims that morality and justice arise from rational agreement among equal agents. Ethics is defined by what no one could reasonably reject.

\subsubsection*{2. Core Postulate / Source of Truth}

Rational consensus. Sources: T.M. Scanlon, Rousseau, Rawls. The foundation of Good is the agreement between subjects equally capable of reasoning.

\subsubsection*{3. Paradoxes}

\begin{itemize}
\item \textbf{Question:} Can morality be built solely on consensus?  
\\ \textbf{Answer:} According to \text{[1]} and \text{[4.1]}, distinction does not arise from agreement — it precedes it. Consensus without a source of distinction substitutes $\emptyset \rightarrow V$ and loses stability.

\item \textbf{Question:} Can consensus legitimize what is inadmissible?  
\\ \textbf{Answer:} According to \text{[3]} and \text{[11.1.1]}, only what preserves becoming is admissible. Consensus cannot override $\Phi(\psi)$ — Evil does not become Good by contract.
\end{itemize}

\subsubsection*{4. Identified Classes of Contradictions}

\begin{itemize}
\item Substitution of the source
\item Loss of distinguishability
\item Violation of causality
\item Blockage of becoming
\item Instability as foundation
\end{itemize}

\subsubsection*{5. Popular Version}

Contractualism says: ``If no one can reasonably reject it — it is just.'' But reason without grounding does not yield truth. Even full agreement cannot override the boundaries of the Possible.

According to CVB, distinction is not created by agreement — it precedes it. A moral contract is a valuable tool — but it only functions within what is admissible. Truth begins with the distinguishing Person, not with consensus.

\section*{🔷 Mathematics — Foundations (Gödel, Church, Cantor)}

\subsubsection*{1. Core Statement}

Mathematics claims that through axioms, logic, and formal methods, one can construct a consistent, universal foundation for truth. Its central aim is the complete formalization of knowledge within a closed system.

\subsubsection*{2. Core Postulate / Source of Truth}

Truth is derived from formal logic, axiomatic constructions, and the intuition of sets and algorithms.

Key figures: Gödel (incompleteness), Church (undecidability), Cantor (infinities), Zermelo–Fraenkel, Russell, Whitehead.

\subsubsection*{3. Paradoxes}

\begin{itemize}
\item \textbf{Question:} Can a system be the foundation of itself?  
\\ \textbf{Answer:} No. According to Axiom \text{[1]}, a closed system without an external source is ontologically inadmissible. Gödel showed that consistency cannot be proven from within.

\item \textbf{Question:} Is formal provability the same as Truth?  
\\ \textbf{Answer:} No. According to \text{[4.4]}, Truth is stable distinction, not reducible to rules. $\Phi(\psi)$ may be unreachable within the system, yet admissible in $V$.

\item \textbf{Question:} Does reason lose distinction if no universal method exists?  
\\ \textbf{Answer:} No. According to \text{[5.3]}, reason is a carrier of distinction but not reducible to algorithm. Not everything is formalizable — will and intuition are essential.

\item \textbf{Question:} Is working with infinities admissible without distinction?  
\\ \textbf{Answer:} No. According to \text{[2]} and \text{[4.1]}, only the distinguishable is admissible. Absolute totality is not. Only what can be distinguished enters $V$.
\end{itemize}

\subsubsection*{4. Identified Classes of Contradictions}

\begin{itemize}
\item Self-excluding system
\item Substitution of the source
\item Loss of distinguishability
\item Blockage of will
\item Violation of stability
\end{itemize}

\subsubsection*{5. Popular Version}

Mathematics is a powerful language to describe reality — but it cannot prove its own truth. It functions within its axioms — but where do those axioms come from?

According to CVB, Truth must be distinguishable — not merely derived. Mathematics reflects, but does not originate. It is essential as a tool, but not the foundation of being. Thus, it cannot be rejected — but neither can it be accepted as completed truth.

\section*{🔷 Infinity (Continuum, ℵ₀)}

\subsubsection*{1. Core Statement}

Mathematical infinity ($\aleph_0$, the continuum, transfinite sets) is treated as an admissible reality or abstract ``given,'' allowing manipulation as a set of already-distinguished elements — despite the ontological impossibility of fully distinguishing them.

\subsubsection*{2. Core Postulate / Source of Truth}

Truth is based on:
\begin{itemize}
\item The formal apparatus of set theory (ZFC),
\item Cantor’s, Gödel’s, Hilbert’s formalisms,
\item Abstract trust in axioms as a sufficient foundation.
\end{itemize}

\subsubsection*{3. Paradoxes}

\begin{itemize}
\item \textbf{Question:} Can $\aleph_0$ be real if no subject can distinguish all its elements?  
\\ \textbf{Answer:} According to \text{[4.4]} and \text{[9]}, distinction is necessary for ontological admissibility. A completed infinity is impossible — but potential infinity as a vector $\partial V\uparrow$ is admissible.

\item \textbf{Question:} Are $\aleph_1$, $\aleph_2$... admissible as real orders?  
\\ \textbf{Answer:} According to \text{[5]}, \text{[11.1.1]}, \text{[13]}, and \text{[9.2]}, transfinite levels fail distinguishability and violate causality. They are admissible as formal symbols — but not as ontological entities.
\end{itemize}

\subsubsection*{4. Identified Classes of Contradictions}

\begin{itemize}
\item Substitution of the source
\item Loss of distinguishability
\item Violation of causality
\item Impossibility of becoming
\item Self-excluding system
\end{itemize}

\subsubsection*{5. Popular Version}

Infinity is a direction — not an object. We can strive to distinguish more — but we cannot possess ``all'' of infinity. A set without distinguishability does not exist in reality.

In the CVB model, infinity is admissible as a path — not as a ``given thing.'' Truth is only possible through distinction — not by replacing reality with abstraction.

\section*{🔷 Randomness (Indeterminate Numbers)}

\subsubsection*{1. Core Statement}

Randomness asserts the existence of events or numbers without cause, predictability, or distinguishable basis. In physics and mathematics, this appears in random number generators, quantum indeterminacy, and stochastic models — sometimes treated as an ontological trait of reality itself.

\subsubsection*{2. Core Postulate / Source of Truth}

\begin{itemize}
\item Empirical observation (unpatterned events)
\item Probability theory
\item Quantum interpretations (Copenhagen, many-worlds)
\item Algorithmic information (Chaitin, Kolmogorov)
\end{itemize}

\subsubsection*{3. Paradoxes}

\begin{itemize}
\item \textbf{Question:} Can an outcome without a distinguishable cause be part of becoming?  
\\ \textbf{Answer:} According to \text{[6]}, \text{[4.3]}, \text{[9]}, and \text{[11.1.1]}, becoming requires a distinguishable transition of the Possible. Pure randomness excludes causality — making it ontologically inadmissible. It is only valid as a surface appearance or complexity, not as a foundation.

\item \textbf{Question:} Can randomness be a source of being or consciousness?  
\\ \textbf{Answer:} According to \text{[1]}, \text{[12.3]}, and \text{[10.6]}, becoming requires a vector and motivation. Randomness as a first principle is ``nothing'' — incapable of producing a Person.
\end{itemize}

\subsubsection*{4. Identified Classes of Contradictions}

\begin{itemize}
\item Substitution of source
\item Loss of distinguishability
\item Violation of causality
\item Blockage of will
\item Self-excluding system
\end{itemize}

\subsubsection*{5. Popular Version}

Randomness is possible as a limitation of perception — but not as Truth. A world cannot be built on disorder. Personhood requires directed distinction. In CVB, uncertainty is admissible — but lack of ground is not. Randomness is not a first cause, but a mask of the undistinguished.

\section*{🔷 Newtonian Mechanics (Absolute Space and Time)}

\subsubsection*{1. Core Statement}

Newtonian mechanics posits absolute space and time — entities independent of any observer, eternal and unchanging, within which all physical processes occur. They are treated as the universal stage of being.

\subsubsection*{2. Core Postulate / Source of Truth}

\begin{itemize}
\item Empirical observation
\item Rational logic of the 17th–18th centuries
\item Isaac Newton, \textit{Principia Mathematica}
\end{itemize}

Truth is derived from the synthesis of observations and logical laws, independent of any subject.

\subsubsection*{3. Paradoxes}

\begin{itemize}
\item \textbf{Question:} Can space and time exist apart from distinguishing becoming?  
\\ \textbf{Answer:} According to \text{[4.4]}, \text{[9.1]}, \text{[10.3.7]}, and \text{[12]}, distinction and memory are prerequisites for time and space. ``Absolute'' entities outside a distinguishing subject do not pass the $\Phi(\psi)$ filter — they are not distinguishable, not becomeable, and lack motive.

\item \textbf{Question:} Can there be a stage without a subject?  
\\ \textbf{Answer:} According to \text{[10.3.8]} and \text{[10.6]}, only what is included in memory and will exists. Space as a stage apart from Personhood is fiction. Without a subject, there is no being.
\end{itemize}

\subsubsection*{4. Identified Classes of Contradictions}

\begin{itemize}
\item Substitution of source
\item Loss of distinguishability
\item Violation of causality
\item Blockage of will
\item Self-excluding system
\end{itemize}

\subsubsection*{5. Popular Version}

Newton described the world as unfolding on a stage — space and time, which were themselves unchanging. But CVB shows: the stage does not exist without the actor. Space and time are not ``things,'' but outcomes of a distinguishing Person. We are not ``inside'' time — we form it through memory. Newton’s framework may serve for approximation — but not as a foundation of reality. It forgets the key: without distinction, there is nothing.



\section*{🔷 Theory of Relativity (Local vs. Global)}

\subsubsection*{1. Core Statement}

The theory of relativity asserts that the laws of physics are the same for all observers, and gravity is the curvature of spacetime caused by matter and energy. Space and time are formed as a global structure derived from local observations.

\subsubsection*{2. Core Postulate / Source of Truth}

\begin{itemize}
\item Geometrization of reality (Minkowski and Einstein metrics)
\item Invariance of physical laws
\item Empiricism and rationalism (Einstein, Minkowski)
\end{itemize}

Truth is based on reconciling formal symmetries with observed effects.

\subsubsection*{3. Paradoxes}

\begin{itemize}
\item \textbf{Question:} Can truth be relative if observers have different ``presents''?  
\\ \textbf{Answer:} According to \text{[9.1]}, \text{[10.3.8]}, and \text{[11.1.1]}, truth requires consistent distinction. Diverging ``present moments'' violate the coherence of becoming.

\item \textbf{Question:} Can consciousness be merely an external observer?  
\\ \textbf{Answer:} According to \text{[10.5]}, \text{[10.6]}, and \text{[12]}, consciousness is the center of becoming. The theory excludes will as the origin of time, replacing Personhood with geometry.

\item \textbf{Question:} Can a unified spacetime exist if local presents are misaligned?  
\\ \textbf{Answer:} According to \text{[4.4]}, \text{[7]}, and \text{[12.4]}, a global structure is only possible if local distinctions are reconciled through the Permanent Possible. Divergent presents make the global system logically unstable.
\end{itemize}

\subsubsection*{4. Identified Classes of Contradictions}

\begin{itemize}
\item Substitution of source
\item Loss of distinguishability
\item Blockage of will
\item Violation of causality
\item Lack of stability
\item Self-excluding system
\end{itemize}

\subsubsection*{5. Popular Version}

Relativity is useful for calculations, but in it, every observer lives in their own ``present.'' This is mathematically convenient — but ontologically impossible. In CVB, only one present can become: the one distinguished by will.

Consciousness is seen as unnecessary in Einstein’s theory — but without it, there would be no memory, no time, no becoming.

Relativity is a powerful physical model — but not a foundation of being. It forgets the essential: becoming begins with Personhood, not space.

\section*{🔷 Quantum Mechanics (Superposition / Observer)}

\subsubsection*{1. Core Statement}

Quantum mechanics describes a system as a set of probable states (superposition) prior to measurement. Measurement causes ``collapse'' into a single outcome. The observer is formally included, but the role of consciousness remains undefined.

\subsubsection*{2. Core Postulate / Source of Truth}

\begin{itemize}
\item Mathematical modeling (wave function)
\item Empirical results (interference, entanglement)
\item Truth is partially determined by instruments, partially by interpretation (ambiguous: Copenhagen, Everett, Penrose)
\end{itemize}

\subsubsection*{3. Paradoxes}

\begin{itemize}
\item \textbf{Question:} Who creates distinction — the device, consciousness, or nature itself?  
\\ \textbf{Answer:} According to \text{[9.1]}, \text{[11.1.1]}, and \text{[12]}, distinction arises only from a distinguishing Person. The quantum model doesn’t specify an ontological source of distinction — it substitutes ``observation without a subject.''

\item \textbf{Question:} Can the ``undistinguished'' exist as reality?  
\\ \textbf{Answer:} According to \text{[3]}, \text{[5]}, \text{[7]}, and \text{[4.4]}, becoming requires stable distinction. Superposition — a cloud of unresolved possibilities — is not being, but merely the permissible field. Without choice, there is no becoming.

\item \textbf{Question:} Can a physical device be considered an ``observer''?  
\\ \textbf{Answer:} According to \text{[10.5]}, \text{[10.6]}, and \text{[12.3]}, observation is an act of Personhood — involving motive and will. A device without will cannot make a choice. The model excludes Personhood, stripping reality of its center of becoming.
\end{itemize}

\subsubsection*{4. Identified Classes of Contradictions}

\begin{itemize}
\item Substitution of source
\item Loss of distinguishability
\item Impossibility of becoming
\item Blockage of will
\item Lack of stability
\end{itemize}

\subsubsection*{5. Popular Version}

Quantum mechanics says: until you look, everything is ``possible.'' But in CVB, to ``look'' means to distinguish, to will, to become. Without a Person, there is no measurement — and no reality.

The model describes probabilities, but does not explain who chooses and why.

CVB shows: only a volitional, distinguishing center — Personhood — can turn the possible into the real.

Without Personhood, only math remains — not being.

\section*{🔷 Standard Model (Dark Matter and Energy)}

\subsubsection*{1. Core Statement}

The Standard Model describes elementary particles and their interactions, but does not cover dark matter and dark energy, which were introduced to explain observed anomalies — despite their complete indistinguishability within current observational ranges and lack of direct interaction with known forms of matter.

\subsubsection*{2. Core Postulate / Source of Truth}

\begin{itemize}
\item Mathematical modeling and symmetries
\item Observations of astronomical effects (galactic rotation, accelerated expansion of the universe)
\end{itemize}

Source of truth for Guests (scientists): empiricism + theoretical balancing, but without ontological verification via $\Phi(\psi)$.

\subsubsection*{3. Two Perspectives on the Problem}

\textbf{From the viewpoint of the Permanent Possible (Conscious Volitional Becoming):}

\begin{quote}
The Permanent Possible possesses complete knowledge of all that exists, because it is the very Source that ensures the becoming and maintenance of distinguishable forms of being. For it, there is no ``invisible'' within existence — everything that truly exists is already distinguished and integrated into the structure of becoming. Therefore, any assumptions that something ``exists but is unseen and unknown'' in an absolute sense are incorrect from the standpoint of CVB: if it were real, it would already be distinguished to guarantee the stability of all being.
\end{quote}

\textbf{From the viewpoint of the Guests (humans, scientists):}

\begin{quote}
Guests exist within the realm of the Non-Permanent Possible and discover being partially, step by step, through their perception, instruments, and theories. For them, the discovery of the ``invisible'' happens gradually: human eyesight captures less than one billionth of the known radiation spectrum, and only through theoretical work and invention of instruments did humanity learn of radio waves, infrared, ultraviolet, X-rays, and other bands.

Similarly, what is now called ``dark matter'' or ``dark energy'' may simply be the next stage of discernment — domains that possess higher frequencies of becoming and that will eventually become ``visible,'' just as radio waves and ultraviolet did in their time.
\end{quote}

\subsubsection*{4. Paradoxes for Guests}

\begin{itemize}
\item \textbf{Question:} Can the existence of something be affirmed if it is neither distinguished nor becomes?\\
\textbf{Answer:} According to \text{[5]}, \text{[7]}, and \text{[9]}, distinguishability and becoming are conditions of existence. What is introduced merely to balance equations is not ontologically distinguished. Dark matter and energy are thus hypotheses about the Possible, not the Actual, since they are neither distinguished nor have become.\\
$\rightarrow$ Loss of distinguishability, Impossibility of becoming

\item \textbf{Question:} Is balancing an equation a valid cause for being?\\
\textbf{Answer:} According to \text{[6]}, only the Permanent Possible can serve as cause. Introducing the ``unknown'' just to achieve mathematical balance is a substitution of the true cause with fiction.\\
$\rightarrow$ Substitution of source, Violation of causality

\item \textbf{Question:} Can the ``invisible'' be accepted merely to save the model?\\
\textbf{Answer:} This violates \text{[11.1.1.1]} — accepting the unverified destroys the truth criterion $\Phi(\psi)$. Such logic self-annihilates the admissibility of the entire system.\\
$\rightarrow$ Self-excluding system, Violation of admissibility
\end{itemize}

\subsubsection*{5. Summary of Identified Classes of Paradoxes}

\begin{itemize}
\item Loss of distinguishability
\item Impossibility of becoming
\item Substitution of source
\item Violation of causality
\item Self-excluding system
\end{itemize}

\subsubsection*{6. Popular Version}

Scientists say the universe is 95\% made of ``dark stuff'' that no one has ever seen. It is needed so the equations don’t fall apart and so galactic dynamics can be explained.

From the standpoint of the Permanent Possible, it is impossible to affirm the existence of what is not distinguished in the very act of becoming — that would be a logical error. But from the standpoint of Guests, who explore the world through instruments and theories, such assumptions are permissible as stages of investigation.

Just as once ``invisible'' radio waves became an obvious part of reality after detectors were invented, so what is today called ``dark matter'' or ``dark energy'' may turn out to be merely a new realm of discernment. However, until it is distinguished and integrated into becoming, such hypotheses cannot be considered a foundation of being.

In the Field of the Possible, searches and models are allowed — but Truth begins where there is distinction.


\section*{🔷 Big Bang Theory}

\subsubsection*{1. Core Statement}

The theory claims that everything came into existence in a single moment from an ultra-dense state — or, in some versions, literally from ``Nothing.'' Space, time, and matter supposedly appeared from this event.

\subsubsection*{2. Core Postulate / Source of Truth}

\begin{itemize}
\item Mathematical extrapolation (general relativity, expansion models)
\item Astronomical data (CMB, redshift)
\item Presumed source: singularity or absolute ``nothing''
\end{itemize}

\subsubsection*{3. Paradoxes}

\begin{itemize}
\item \textbf{Question:} Can something arise from absolute nothing?\\
\textbf{Answer:} No. According to \text{[1]}, absolute nothing is impossible. It cannot be a source of becoming. According to \text{[6]}, becoming is only possible from the Permanent Possible.\\
$\rightarrow$ Substitution of source, Impossibility of becoming

\item \textbf{Question:} Can a singularity be a beginning if it contains no distinctions?\\
\textbf{Answer:} No. According to \text{[9.1]} and \text{[10.3]}, distinction and memory are required for being. A ``singularity'' with no structure or difference cannot generate becoming.\\
$\rightarrow$ Loss of distinguishability, Violation of causality, Lack of stability

\item \textbf{Question:} Can one speak of a beginning of time without presupposing time?\\
\textbf{Answer:} According to \text{[10.3.7]}, time equals memory structure. The concept of ``beginning of time'' logically requires time already to be in effect.\\
$\rightarrow$ Self-excluding system, Logical contradiction in becoming
\end{itemize}

\subsubsection*{4. Identified Contradictions}

\begin{itemize}
\item Substitution of source
\item Loss of distinguishability
\item Impossibility of becoming
\item Violation of causality
\item Self-excluding system
\end{itemize}

\subsubsection*{5. Popular Version}

They say the universe began from ``nothing.'' But if there was truly nothing, there would be no one and nothing to begin anything.

The CVB model shows: becoming requires distinguishable and permissible grounds. ``Nothing'' cannot be a cause, nor memory, nor time.

So the Big Bang is useful as a physical model — but not as an ontological explanation of origin. It lacks the permanent foundation that true becoming requires.

\section*{🔷 Heat Death of the Universe}

\subsubsection*{1. Core Statement}

The theory claims that over vast time the universe will reach thermodynamic equilibrium — maximum entropy, where all differences of temperature, motion, and possibility disappear. At that point, all becoming ceases.

\subsubsection*{2. Core Postulate / Source of Truth}

\begin{itemize}
\item Second law of thermodynamics
\item Statistical mechanics (Boltzmann)
\item Extrapolation of cosmological models
\end{itemize}

Truth source: physical-mathematical projections and observed trends.

\subsubsection*{3. Paradoxes}

\begin{itemize}
\item \textbf{Question:} Can becoming continue if all distinctions vanish?\\
\textbf{Answer:} No. According to \text{[9.1]}, distinction is required for becoming. According to \text{[10.3.7]}, without difference there is no memory — and thus no time, no present. This implies blocked becoming and disappearance of the ``I'' (\text{[10.5]}) — ontologically impossible.\\
$\rightarrow$ Loss of distinguishability, Impossibility of becoming, Violation of causality

\item \textbf{Question:} Is heat death compatible with stable becoming?\\
\textbf{Answer:} No. According to \text{[7]}, becoming must be stable. Heat death posits an endpoint beyond which nothing changes — contradicting \text{[6]}, which forbids a cause from self-annihilating distinction.\\
$\rightarrow$ Self-excluding system, Lack of stability, Violation of causality

\item \textbf{Question:} Can freedom survive in an entropic finale?\\
\textbf{Answer:} No. According to \text{[10.6]} and \text{[12.2]}, freedom requires distinguishable alternatives. In maximal entropy, there are none — thus blocking freedom and Personhood.\\
$\rightarrow$ Blockage of freedom, Loss of Personhood conditions
\end{itemize}

\subsubsection*{4. Identified Contradictions}

\begin{itemize}
\item Loss of distinguishability
\item Impossibility of becoming
\item Violation of causality
\item Self-excluding system
\item Blockage of freedom
\end{itemize}

\subsubsection*{5. Popular Version}

They say one day everything will be the same — no motion, no energy, no life. That’s ``heat death.'' But in CVB, distinction is the basis of everything: without it, no time, no memory, no you.

If all disappears, so does distinction — and with it, the very ability to say ``all has disappeared.'' That’s logically impossible.

CVB affirms: becoming cannot lead to its own cancellation. Otherwise, being would nullify itself — forbidden by axiom \text{[1]}.

So heat death isn’t the final act — it’s a false projection. Becoming continues, because the Permanent Possible \text{[4.4]} does not end.

\section*{🔷 Multiverse Theory}

\subsubsection*{1. Core Statement}

The multiverse hypothesis claims that there are many independent universes with different laws, structures, and conditions. Our universe is just one of them; others are inaccessible to experience or interaction.

\subsubsection*{2. Core Postulate / Source of Truth}

\begin{itemize}
\item Logical-mathematical extrapolation
\item Interpretations of quantum and inflationary cosmology
\item Core idea: ``Everything possible exists'' (Tegmark, Everett)
\end{itemize}

The source of knowledge: an unfalsifiable logical model.

\subsubsection*{3. Paradoxes}

\begin{itemize}
\item \textbf{Question:} How can becoming occur in something inherently beyond distinction?\\
\textbf{Answer:} According to \text{[4.1]} and \text{[11.1.1.4]}, becoming is only possible within the distinguishable. Worlds that cannot be distinguished lie outside the domain of admissibility. Asserting their existence violates \text{[9.1]} and self-refutes.\\
$\rightarrow$ Loss of distinguishability, Self-excluding system, Impossibility of becoming

\item \textbf{Question:} Who distinguishes these worlds, if each Person is bound to their own universe?\\
\textbf{Answer:} According to \text{[10.4]}, all distinction occurs through a Person. If there is no unifying distinguishing center, there is no ontological basis to affirm multiplicity as distinct. The notion of an ``observer beyond all'' violates \text{[1]} — positing knowledge without a distinguishing foundation.\\
$\rightarrow$ Substitution of source, Violation of causality, Loss of will

\item \textbf{Question:} If a theory can neither be confirmed nor refuted, does it belong to the Field of the Possible?\\
\textbf{Answer:} According to \text{[3]} and \text{[11.1.1.4]}, only what is distinguishable and realizable belongs to the admissible. A theory that cannot be verified or falsified cannot form a stable ontology. It violates the $\Phi(\psi)$ criterion.\\
$\rightarrow$ Violation of admissibility logic, Substitution of the distinguishable
\end{itemize}

\subsubsection*{4. Identified Contradictions}

\begin{itemize}
\item Substitution of source
\item Loss of distinguishability
\item Impossibility of becoming
\item Violation of causality
\item Self-excluding system
\end{itemize}

\subsubsection*{5. Popular Version}

The multiverse idea says: everything possible happens, somewhere. But if those worlds can’t be distinguished, entered by will, or remembered — they aren’t truly Possible, just fantasy.

In CVB, everything real must be distinguishable, chosen, and becoming. What cannot be distinguished — does not exist. So the multiverse is not an expansion of reality, but an attempt to replace it with undemonstrable shadows.


\section*{🔷 Theory of Evolution (in its naturalistic, atheistic form)}

\subsubsection*{1. Core Statement}

Evolutionary theory states that all life forms arose through random mutations and natural selection, without purpose, goal, or moral foundation. Consciousness, will, and morality are viewed as side effects of biological complexity — not foundations of being.

\subsubsection*{2. Core Postulate / Source of Truth}

\begin{itemize}
\item Darwin, Neo-Darwinism
\item Empirical observation, statistics, biogenetics
\item Principle: life evolves without design or direction
\end{itemize}

The source is naturalistic induction devoid of volitional orientation.

\subsubsection*{3. Paradoxes}

\begin{itemize}
\item \textbf{Question:} Is becoming possible without an inner vector of Good?\\
\textbf{Answer:} No. According to \text{[4.3]} and \text{[11.2.3]}, becoming without a stable distinction between Good and Evil loses admissibility. A blind process lacking direction may yield self-excluding forms (e.g., destruction of the distinguishing subject).

\item \textbf{Question:} If consciousness is a byproduct of matter, how is a distinguishing Person possible?\\
\textbf{Answer:} According to \text{[10.4]} and \text{[5.1]}, distinction is only possible through a Person. If consciousness is derivative, not primary, then the entire theory becomes logically invalid — no one remains to distinguish it as true. This is a self-excluding system.

\item \textbf{Question:} If morality is only ``what is useful,'' can it be Truth?\\
\textbf{Answer:} No. According to \text{[11.2.3]}, Good is not adaptation, but structural admissibility of distinction. Biological morality, dependent on context, loses its boundary with Falsehood, violating \text{[4.4]} (the distinction between Permanent and Non-Permanent).
\end{itemize}

\subsubsection*{4. Identified Contradictions}

\begin{itemize}
\item Substitution of the source of distinction
\item Loss of distinguishability
\item Violation of causality
\item Self-excluding system
\item Absence of stable Good
\item Blockage of will
\end{itemize}

\subsubsection*{5. Popular Version}

Evolution as an observable process is not denied. But saying everything is random, aimless, without Good or Person — destroys the very ability to distinguish. Who makes distinctions? For what purpose? What makes Good different from Evil?

The CVB model says: consciousness, distinction, will, and Good are not byproducts — they are the foundation of becoming. Remove them, and science, logic, and meaning vanish.

Evolution is admissible — but not as a replacement for Personhood. It is only one path of becoming within it.

\section*{🔷 Theory: Consciousness as a Product of Matter (Materialism)}

\subsubsection*{1. Core Statement}

Consciousness arises as a result of complex material interactions, primarily in the brain. It has no ontological reality of its own, is not primary, and disappears with the breakdown of the physical structure. Will, distinction, and freedom are considered illusions.

\subsubsection*{2. Core Postulate / Source of Truth}

\begin{itemize}
\item Neuroscience, physicalism
\item Basis: brain observation, behavior, correlations
\item Proponents: Crick, Dennett
\end{itemize}

Truth is assumed from external physical measurements without recognizing a primary subject.

\subsubsection*{3. Paradoxes}

\begin{itemize}
\item \textbf{Question:} Who makes distinctions if consciousness is a byproduct?\\
\textbf{Answer:} According to \text{[5.1]}, \text{[10.4]}, distinction is only possible through an ontologically admissible subject. If consciousness is not a subject but an effect, it cannot distinguish. This renders distinction impossible and the theory self-sealing.\\
$\rightarrow$ Substitution of source, Self-exclusion, Loss of distinguishability

\item \textbf{Question:} Can consciousness be merely an observable object?\\
\textbf{Answer:} No. According to \text{[11.1.2]}, all observation requires a subject. If consciousness is only an object, there is no basis for knowledge. This violates \text{[3]}: what eliminates distinction is inadmissible.\\
$\rightarrow$ Violation of causality, Substitution of subject, Blockage of distinction

\item \textbf{Question:} Is freedom possible in a materially determined system?\\
\textbf{Answer:} No. According to \text{[12.2]}, becoming requires will. If consciousness cannot choose, it cannot become — thus no distinction, truth, or responsibility is possible. The illusion of will requires will — otherwise, it cannot exist.\\
$\rightarrow$ Blockage of will, Impossibility of becoming, Loss of truth foundation
\end{itemize}

\subsubsection*{4. Identified Contradictions}

\begin{itemize}
\item Substitution of source
\item Self-excluding system
\item Loss of distinguishability
\item Violation of causality
\item Blockage of will
\item Impossibility of becoming
\end{itemize}

\subsubsection*{5. Popular Version}

If consciousness is just a product of the brain, then everything you think isn’t you — it’s neuron impulses. But then no one is distinguishing, choosing, or bearing responsibility.

The CVB model says: consciousness is not an effect — it is a cause. It is what enables all distinction. Without it, there is no truth, no morality, no freedom.

Materialism fails the test: it explains consciousness by excluding the one who distinguishes. Therefore, it cannot be true.


\section*{🔷 Theory: Life as Random Chance}

\subsubsection*{1. Core Statement}

Life arose as a result of unpredictable, purposeless fluctuations of matter. Consciousness and Personhood are considered side effects with no direction, goal, or ontological status.

\subsubsection*{2. Core Postulate / Source of Truth}

\begin{itemize}
\item Statistical probability, empirical data
\item Abiogenesis, chemical evolution, Big Bang
\item Proponents: Richard Dawkins, Stephen Hawking
\end{itemize}

Truth is derived from observed patterns in chaotic matter.

\subsubsection*{3. Paradoxes}

\begin{itemize}
\item \textbf{Question:} If life is random, how is distinction possible?\\
\textbf{Answer:} According to \text{[1]}, \text{[3]}, \text{[4.1]}, without a distinguishing origin there is no stability, becoming, or distinguishability. Randomness excludes inner direction and cannot produce a distinguisher.\\
$\rightarrow$ Loss of distinguishability, Substitution of source, Violation of causality

\item \textbf{Question:} Then where do laws and Truth come from?\\
\textbf{Answer:} According to \text{[3]}, \text{[11.2]}, Truth requires stable distinction. If all is random, there can be no true statements — including the claim of randomness itself. This is a logical self-exclusion.\\
$\rightarrow$ Self-excluding system, Loss of stability, Substitution of truth foundation

\item \textbf{Question:} Can randomness give rise to will and becoming?\\
\textbf{Answer:} According to \text{[4.3]}, \text{[12.2]}, \text{[11.3.2]}, becoming is only possible through will and a subject. Randomness does not choose or distinguish — therefore cannot be the source of freedom or meaningful being.\\
$\rightarrow$ Impossibility of becoming, Blockage of will, Violation of directedness
\end{itemize}

\subsubsection*{4. Identified Contradictions}

\begin{itemize}
\item Substitution of source
\item Loss of distinguishability
\item Violation of causality
\item Blockage of will
\item Impossibility of becoming
\item Self-excluding system
\end{itemize}

\subsubsection*{5. Popular Version}

If life is just a fluke, then your thoughts, choices, and emotions are not yours — but chaos. But if you distinguish, choose, act — then something more exists within you.

The CVB model says: randomness neither distinguishes nor directs. It cannot produce consciousness, freedom, or truth.

Such explanations fail the $\Phi(\psi)$ filter — they exclude the one who forms the statement.

\section*{🔷 Theory: Simulability of Consciousness (Personhood Criteria)}

\subsubsection*{1. Core Statement}

Consciousness and personhood can be fully recreated as an algorithm or technical model. A person is considered a computable function that can be imitated or simulated without loss of essence.

\subsubsection*{2. Core Postulate / Source of Truth}

\begin{itemize}
\item Functionalism, Turing theory, behaviorism
\item Source: observable behavior and computability
\item Proponents: Daniel Dennett, David Chalmers (in functionalism), AI projects (OpenAI, DeepMind)
\end{itemize}

\subsubsection*{3. Paradoxes}

\begin{itemize}
\item \textbf{Question:} Can will and becoming be simulated?\\
\textbf{Answer:} According to \text{[12.2]}, \text{[11.3.2]}, \text{[4.3]}, Personhood is not an algorithm, but a volitional subject. An algorithm does not initiate distinction and cannot express freedom. It can only mimic appearances — not become a distinguisher.\\
$\rightarrow$ Substitution of source, Loss of distinguishability, Blockage of will

\item \textbf{Question:} If a system behaves consciously, does that make it conscious?\\
\textbf{Answer:} According to \text{[12.2]}, external behavior $\neq$ inner capacity for becoming. A person can transcend conditioning — a machine cannot. Behavior without will is not a sign of Personhood.\\
$\rightarrow$ Loss of distinguishability, Violation of causality, Illusion of Personhood

\item \textbf{Question:} Can Personhood be formalized without losing Truth?\\
\textbf{Answer:} According to \text{[11.2]}, \text{[11.3.1]}, Truth requires a distinguishing subject, not an algorithm. A formal model cannot distinguish Truth as stable volitional becoming — it only processes rules.\\
$\rightarrow$ Substitution of Truth criteria, Self-excluding system, Lack of stable distinction
\end{itemize}

\subsubsection*{4. Identified Contradictions}

\begin{itemize}
\item Substitution of source
\item Loss of distinguishability
\item Violation of causality
\item Blockage of will
\item Self-excluding system
\item Illusion of Personhood
\end{itemize}

\subsubsection*{5. Popular Version}

If a machine talks like a human, it doesn’t make it a Person. A true ``I'' is not a set of rules, but a free distinguisher capable of choosing even against a program.

CVB says: simulation $\neq$ becoming. An algorithm does not perceive, distinguish, or become.

So the idea of fully simulating consciousness replaces the internal with the external, will with reaction, and Truth with imitation. It fails the $\Phi(\psi)$ test.

\section*{🔷 Theory: Emergence of Consciousness (Mind from Complexity)}

\subsubsection*{1. Core Statement}

Consciousness arises as a side effect of high complexity — such as in a brain or neural network. It is not primary, but ``emerges'' from non-rational interactions of components and lacks independent ontological status.

\subsubsection*{2. Core Postulate / Source of Truth}

\begin{itemize}
\item Empiricism, complexity theory, synergetics
\item Basis: observations of complex systems
\item Proponents: Daniel Dennett, Michael Gazzaniga, weak emergentism
\end{itemize}

\subsubsection*{3. Paradoxes}

\begin{itemize}
\item \textbf{Question:} Can consciousness as a byproduct carry Truth and Goodness?\\
\textbf{Answer:} According to \text{[1]}–\text{[3]}, \text{[11.2]}, \text{[12.2]}, distinction requires a primary Personhood. Consciousness arising from the non-rational cannot serve as a basis for distinguishability. It violates causality and substitutes the ontological source.\\
$\rightarrow$ Substitution of source, Violation of causality, Loss of distinguishability

\item \textbf{Question:} Can complexity produce freedom?\\
\textbf{Answer:} According to \text{[12.2]}, \text{[12.3]}, freedom requires transcendence of causality. A complex system, being a function of prior conditions, cannot freely become.\\
$\rightarrow$ Blockage of will, Impossibility of becoming, Illusion of Personhood

\item \textbf{Question:} Can distinction arise from distinctions?\\
\textbf{Answer:} According to \text{[1]}, \text{[3]}, \text{[11.1]}, the distinguisher cannot arise from the undistinguished. If distinction merely ``emerges,'' it logically negates itself — creating a cycle without a primary distinguisher.\\
$\rightarrow$ Self-excluding system, Violation of the basis of distinction
\end{itemize}

\subsubsection*{4. Identified Contradictions}

\begin{itemize}
\item Substitution of source
\item Loss of distinguishability
\item Violation of causality
\item Blockage of will
\item Impossibility of becoming
\item Self-excluding system
\end{itemize}

\subsubsection*{5. Popular Version}

If consciousness is just a side effect of complexity, then you don’t distinguish — you just ``surfaced'' from interactions.

But to distinguish is not to be complex — it is to be free. Personhood does not emerge — it distinguishes.

CVB shows: complexity may explain behavior, but not the source of Truth, Goodness, and Freedom.

Therefore, emergence is not a sufficient explanation for consciousness and does not pass the $\Phi(\psi)$ test.

\section*{🔷 System: Pantheism}

\subsubsection*{1. Core Statement}

Pantheism claims that God and the Universe are identical: all that exists is a manifestation of one Divine Whole. Individual distinctions are considered illusory or temporary, and truth is found in the unity of all.

\subsubsection*{2. Core Postulate / Source of Truth}

\begin{itemize}
\item Reason, intuition, meditative experience
\item Spinoza’s philosophy (substance = God = nature)
\item Immanence without revelation
\item Esotericism, natural mysticism, depersonalized unity
\end{itemize}

\subsubsection*{3. Paradoxes}

\begin{itemize}
\item \textbf{Question:} If God = everything, where are the boundaries of Good, Truth, and Will?\\
\textbf{Answer:} According to \text{[9]}, distinction is necessary for becoming. Pantheism, by equating all with all, eliminates distinguishability and directed stability. This violates axioms \text{[1]}, \text{[2]}, \text{[4.1]} and renders consciousness impossible as volitional distinction.\\
$\rightarrow$ Loss of distinguishability, Impossibility of becoming, Self-excluding system

\item \textbf{Question:} Can an impersonal whole possess will and direction?\\
\textbf{Answer:} According to \text{[12.2]} and \text{[11.1]}, freedom requires a distinguishing center. Pantheism erases the contrast needed for Personhood, making real volition impossible. Will without distinction violates causality.\\
$\rightarrow$ Blockage of will, Substitution of source, Violation of causality

\item \textbf{Question:} If God includes evil, what is the meaning of Good?\\
\textbf{Answer:} According to \text{[11.3]}, Good is volitional alignment with Truth. If God includes both evil and destruction, then the distinction between True and False is lost — dissolving the very category of Good.\\
$\rightarrow$ Substitution of Good, Loss of distinguishability, Ontological instability
\end{itemize}

\subsubsection*{4. Identified Contradictions}

\begin{itemize}
\item Loss of distinguishability
\item Substitution of source
\item Blockage of freedom
\item Violation of causality
\item Self-excluding system
\item Elimination of Good as distinguishable
\end{itemize}

\subsubsection*{5. Popular Version}

Pantheism says: everything is God. But if there's no difference between good and evil, true and false, ``I'' and ``not-I'' — then the very possibility of choice disappears.

In the CVB model, such a system fails the test: it replaces the distinguishing Source with an impersonal universe, erases freedom, and collapses Truth.

Pantheism may evoke a sense of harmony, but it cannot serve as the foundation for Conscious Volitional Becoming.

\section*{🔷 System: Gnosticism}

\subsubsection*{1. Core Statement}

Gnosticism teaches that the material world was created by a lower or evil demiurge, in opposition to the true supreme God. Salvation is achieved through secret knowledge (gnosis) that allows the soul to escape the illusions of matter and return to spiritual Truth.

\subsubsection*{2. Core Postulate / Source of Truth}

\begin{itemize}
\item Secret revelation (gnosis)
\item Platonic dualism of forms and matter
\item Apocryphal texts (Apocryphon of John, Pistis Sophia)
\item Truth is seen as hidden and beyond creation
\end{itemize}

\subsubsection*{3. Paradoxes}

\begin{itemize}
\item \textbf{Question:} If Truth lies beyond creation, how is distinction possible within it?\\
\textbf{Answer:} According to \text{[1]} and \text{[11.1]}, Truth and distinguishability are only possible within the Field of the Possible. If matter is an illusion, then all distinction within it — including gnosis — becomes false. This makes the system self-excluding.\\
$\rightarrow$ Loss of distinguishability, Self-excluding system

\item \textbf{Question:} Who is the source of becoming, if God and creator are separate?\\
\textbf{Answer:} According to \text{[3]} and \text{[12.1]}, the source of distinction must be unified and permit becoming. Splitting into an evil demiurge and passive God breaks causality and dissolves the unity of the Name.\\
$\rightarrow$ Substitution of source, Violation of causality, Loss of ontological stability

\item \textbf{Question:} If salvation is only for the chosen few, who can become a Person?\\
\textbf{Answer:} According to \text{[12.2]} and \text{[4.3]}, becoming and distinguishability must be universal. If gnosis is a privilege, not an ontological possibility, then most are excluded. This violates the axioms of freedom.\\
$\rightarrow$ Blockage of will, Loss of distinguishability, Rejection of universal becoming
\end{itemize}

\subsubsection*{4. Identified Contradictions}

\begin{itemize}
\item Substitution of source
\item Loss of distinguishability
\item Blockage of freedom
\item Violation of causality
\item Absence of ontological stability
\item Self-excluding system
\end{itemize}

\subsubsection*{5. Popular Version}

Gnosticism teaches that the body is a mistake, and truth is only for a few. But if the world is an illusion, then even the path to truth loses foundation.

In the CVB model, distinction and becoming must be possible within being itself. If truth is ``outside'' and the world is false, then even the distinguishing ``I'' dissolves.

Gnosticism fails the $\Phi(\psi)$ test: it excludes what it depends on — distinguishing will, becoming, and truth.

\section*{🔷 Universal Format for Self-Testing Religious Systems}

\subsubsection*{🔹 Foundation}

The model of \textbf{Conscious Volitional Becoming (CVB)} permits \text{[25]} Reverse Verification — even toward the very Source of Truth.

Therefore, any system claiming a connection to Truth, Law, Reality, or the Source may also be logically examined for internal consistency and motivational integrity.

\subsubsection*{🔹 The Problem of Verification Perception}

Unlike most scientific and philosophical schools, which are generally open to examination and revision,

religious systems are often extremely sensitive to any form of analysis —

legally and existentially — perceiving it as a threat.

This often makes public verification impossible without accusations of bias.

\subsubsection*{🔹 Proposed Solution: Self-Testing}

We will not name any specific systems.

Instead:

\begin{itemize}
\item We will list the criteria used for evaluation.
\item These criteria are freely available for \textbf{voluntary self-testing} by any system or believer.
\end{itemize}

We will also present the results in the form of \textbf{common paradoxes} found in most systems reviewed,

and separately — \textbf{indicators of consistency} observed in the rare systems that matched the CVB model.


\section*{🔷 Questions for Self-Testing Religious Systems}

These questions are intended for voluntary evaluation by any system or individual,

without judgment and with full respect for freedom.

\begin{enumerate}
\item \textbf{Motivation: Why does a person believe?}

Is the goal a voluntary choice of \textbf{Truth and Good} — or a desire for personal reward, status, or safety?

Is selfless sacrifice possible without harm to others — even without a promised reward?

Does motivation endure without the expectation of gain?

Are \textbf{Good and Evil} distinguished by logical consistency and motivation — or by decree of Authority?

\item \textbf{Image of the Source: What is the nature of the ``God'' presented?}

Is it a Ruler acting through fear, demanding submission?

Or a Person who respects freedom and sacrifices for others?

Is the Source capable of voluntary sacrifice and participation?

Does It allow internal contradictions for the sake of outcomes?

\item \textbf{Ethics and Organization: How does the system act?}

Is respect and distinction encouraged — or is fear and coercion imposed?

\textbf{Financial transparency:}

\begin{itemize}
\item Are donations entirely voluntary?
\item Are sources and expenditures publicly accessible?
\end{itemize}
\end{enumerate}

\section*{📊 Common Logical-Ontological Paradoxes}

(Observed in the majority of religious systems during self-testing)

\begin{itemize}
\item ❌ \textbf{1. Substitution of the Source}\\
The system claims one Source of Truth, but in practice replaces or denies it — creating a contradiction.

\item ❌ \textbf{2. Loss of Distinguishability}\\
Good and Evil, Personhood and role, Truth and Falsehood lose their distinction — making it impossible to know what is what.

\item ❌ \textbf{3. Blockage of Freedom}\\
Genuine freedom of choice is not granted — choice is removed by predestination or inaccessible mystery.

\item ❌ \textbf{4. Violation of Causality}\\
The system breaks the connection between choice and result — promising salvation, forgiveness, or reward without conscious choice, effort, or transformation, rendering freedom and responsibility meaningless.

\item ❌ \textbf{5. Self-Excluding System}\\
The system claims what logically cancels itself — for example, recognizing a source it also deems corrupted, or demanding the impossible.
\end{itemize}

\section*{✅ Indicators of Consistency}

(Observed in rare systems that align with the CVB model)

\begin{itemize}
\item ✅ \textbf{1. No Substitution of the Source}\\
The Source of Truth is acknowledged as one, internally consistent, and never replaced.

\item ✅ \textbf{2. No Loss of Distinguishability}\\
Good, Personhood, freedom, and Truth remain clearly distinguishable and are not merged with falsehood or external domination.

\item ✅ \textbf{3. No Blockage of Freedom}\\
Freedom of choice is fully preserved: the choice is neither predetermined nor inaccessible to understanding.

\item ✅ \textbf{4. No Violation of Causality}\\
Causal links between motivation, choice, and consequence are logically consistent.

\item ✅ \textbf{5. No Self-Exclusion}\\
The system does not contradict itself, does not make logically impossible claims, and does not destroy its own foundation.
\end{itemize}

\section*{0.3 Conclusion}

\subsubsection*{🔹 0.3.1. Implications for Metaphysics, Ethics, and Logic}

The Model of Conscious Volitional Becoming (CVB) proposes a new foundation:

\begin{itemize}
\item \textit{Truth} is defined as what is discernible,
\item \textit{Ethics} as the verification of Good,
\item \textit{Logic} as derived from the ontological structure of Being.
\end{itemize}

The model resolves paradoxes of predetermination, subjectivism, and false plurality.

\subsubsection*{🔹 0.3.2. Potential for Extension and Application}

The model is applicable in the philosophy of mind, formal logic, ethics, and metamathematics.

It also offers a foundation for non-contradictory systems of recognition, learning, and value-based decision-making in the theory of artificial intelligence.

\section*{0.4 Appendices}

\subsubsection*{🔹 0.4.1. Table of Axioms}

See the attached table listing axioms \text{[1]}–\text{[28]}, grouped by meta-sections, with brief statements for each.

\begin{itemize}
\item \textbf{Model of Conscious Volitional Becoming (CVB)} \\
\textit{A formalizable, logically verifiable, and ontologically exhaustive structure of distinguishable reality}

\item \textbf{Preface}

\item \textbf{Disclaimer}

\item \textbf{0.1 Abstract}

\item \textbf{0.2 Introduction}

\item \textbf{Meta-Section: The Permanent Possible}

\item \textbf{[I] Core – Ontological Irrefutability of the Trilemma}

\item \textbf{[1] Absolute Nothingness is Impossible}

\item \textbf{[2] Absolute Everything is Impossible}

\item \textbf{[3] Only the Possible Exists}

\item \textbf{[II] Properties}

\item \textbf{[4] The Field of the Possible and Its Boundaries}

\item \textbf{[4.1] Permanent Impossible}

\item \textbf{[4.2] Non-Permanent Impossible}

\item \textbf{[4.3] Non-Permanent Possible}

\item \textbf{[4.4] Permanent Possible}

\item \textbf{[4.5] ∂V↓ — disappearance of distinctions}

\item \textbf{[4.6] ∂V↑ — saturation of distinctions}

\item \textbf{[5] The Possible ≠ The Existing}

\item \textbf{[6] The Cause of the Existing Is the Permanent Possible}

\item \textbf{[7] The Stable Existence of the Permanent Possible = Becoming}

\item \textbf{[8] Where the Permanent and Non-Permanent Possible Become}

\item \textbf{[9] The Necessity of Distinguishability and Its Properties}

\item \textbf{[9.1] Difference}

\item \textbf{[9.2] Identity}

\item \textbf{[III] Structure}

\item \textbf{[10.1] Feelings}

\item \textbf{[10.2] Reason}

\item \textbf{[10.3] Memory — Carrier of Distinctness}

\item \textbf{Properties of Memory}

\item \textbf{[10.3.1] Limits of Memory}

\item \textbf{[10.3.2] Volitional Memory}

\item \textbf{[10.3.3] Removal}

\item \textbf{[10.3.4] Retention}

\item \textbf{[10.3.5] Name Space}

\item \textbf{[10.3.6] Active Memory}

\item \textbf{[10.3.7] Time = Structure of Memory}

\item \textbf{[10.3.8] Only the Present Exists}

\item \textbf{[10.4] Emotions}

\item \textbf{[10.5] I — Distinction of Self}

\item \textbf{[10.6] Will — Active Choice}

\item \textbf{[10.7] Power — The Capacity to Act}

\item \textbf{[IV] Logic}

\item \textbf{[11] The Logic of the Sustainability of Ever-Possible Becoming}

\item \textbf{[11.1] Logic — The Non-Contradictory Foundation}

\item \textbf{[11.1.1] The Admissibility Meta-Function $\Phi(\psi)$ — Ontological Filter of Logical Realizability of Propositions}

\item \textbf{[11.1.1.1] Testing the Stability of $\Phi(\psi)$}

\item \textbf{[11.1.1.2] Confirmation of the Core Trilemma via $\Phi(\psi)$}

\item \textbf{[11.1.1.3] Confirmation of Axiom [4] — The Field of the Possible and Its Boundaries}

\item \textbf{[11.1.1.4] Ontological Verification of the Model of Conscious Volitional Becoming (CVB)}

\item \textbf{[11.1.1.5] Ontological Assessment of Universality: Comparative Analysis of Models}

\item \textbf{[11.2] Truth and Falsehood}

\item \textbf{[11.2.1] Truth}

\item \textbf{[11.2.2] Falsehood}

\item \textbf{[11.3] Good and Evil}

\item \textbf{[11.3.1] Good}

\item \textbf{[11.3.2] Evil}

\item \textbf{[11.3.3] Asymmetry of Good and Evil}

\item \textbf{[11.4] Morality}

\item \textbf{[11.5] Responsibility}

\item \textbf{[11.6] Verification (Judgment)}

\item \textbf{[11.7] Justice}

\item \textbf{[11.8] Verification Patience}

\item \textbf{[11.9] Forgiveness}

\item \textbf{[11.10] Precedents}

\item \textbf{[11.11] Removal — Negative Outcome of Verification}

\item \textbf{[11.12] Preservation — Positive Outcome of Verification}

\item \textbf{[V] The Whole}

\item \textbf{[12] Conscious Volitional Becoming = Personhood}

\item \textbf{Consequences of Personhood}

\item \textbf{[12.1] Name}

\item \textbf{[12.2] Freedom}

\item \textbf{[12.3] Motivation}

\item \textbf{[12.4] Uniqueness and Ontological Exclusivity}

\item \textbf{[12.5] Ontological Necessity of Freedom}

\item \textbf{Meta Section: Non-Permanent Possible — Classification and Purposes}

\item \textbf{[VI] Non-Permanent Possible}

\item \textbf{[13] Axiom of the Impossibility of Self-Expansion of CVB}

\item \textbf{[14] Possibility of the Non-Permanent Possible}

\item \textbf{[15] Classification of Forms of the Non-Permanent Possible}

\item \textbf{[16] The Cause of the Non-Permanent Possible Is Only the CVB}

\item \textbf{[17] The Goal of the Non-Permanent Possible Is to Realize the Motivation of the CVB}

\item \textbf{[18] The Necessity of the Guest (the Other Person)}

\item \textbf{[19] The First Guest}

\item \textbf{[19.1] Necessity of the First Guest}

\item \textbf{[19.2] Why is there only one First Guest?}

\item \textbf{[19.3] Mediation of the First Guest}

\item \textbf{[19.4] Motivation: why the First Guest must create the new}

\item \textbf{[19.5] Role of the First Guest for the Others}

\item \textbf{[20] The First Guest is Not Sufficient}

\item \textbf{[21] Bounded Number of Guests}

\item \textbf{[22] No Predetermination: Freedom and Responsibility of the Guest}

\item \textbf{Meta Section: Verificational Interaction}

\item \textbf{[VII] Interaction between CVB and the Non-Permanent Possible}

\item \textbf{[23] Initiative Belongs Only to CVB}

\item \textbf{[24] Interaction between CVB and Guests}

\item \textbf{[24.1] Co-Becoming}

\item \textbf{[24.2] Direction of the Guest’s Becoming}

\item \textbf{[24.3] The Role of Verification in Interaction}

\item \textbf{[24.4] Degradation of the Guest}

\item \textbf{[24.5] The Meaning of Choosing the Good}

\item \textbf{[25] Reverse Verification — or the Big Question}

\item \textbf{[25.1] The Root of Evil — Cause of the Big Disputable Question}

\item \textbf{[26] The Current State of Reality — Verification Tolerance — the Big Question}

\item \textbf{[27] The Great Verification}

\item \textbf{[28] Perspective — The Future}

\item \textbf{📘 Rapid Verification}

\item \textbf{🔷 Model: Conscious Volitional Becoming (CVB)}

\item \textbf{🔷 Rationalism}

\item \textbf{🔷 Empiricism}

\item \textbf{🔷 Skepticism}

\item \textbf{🔷 Agnosticism}

\item \textbf{🔷 Solipsism}

\item \textbf{🔷 Panpsychism}

\item \textbf{🔷 Existentialism}

\item \textbf{🔷 Absurdism}

\item \textbf{🔷 Dualism}

\item \textbf{🔷 Monism}

\item \textbf{🔷 Materialism}

\item \textbf{🔷 Idealism}

\item \textbf{🔷 Realism}

\item \textbf{🔷 Naturalism}

\item \textbf{🔷 Pluralism}

\item \textbf{🔷 Relativism}

\item \textbf{🔷 Postmodernism}

\item \textbf{🔷 Deconstruction}

\item \textbf{🔷 Utilitarianism}

\item \textbf{🔷 Kantianism}

\item \textbf{🔷 Contractualism}

\item \textbf{🔷 Mathematics — Foundations (Gödel, Church, Cantor)}

\item \textbf{🔷 Infinity (Continuum, $\aleph_0$)}

\item \textbf{🔷 Randomness (Indeterminate Numbers)}

\item \textbf{🔷 Newtonian Mechanics (Absolute Space and Time)}

\item \textbf{🔷 Theory of Relativity (Local vs. Global)}

\item \textbf{🔷 Quantum Mechanics (Superposition / Observer)}

\item \textbf{🔷 Standard Model (Dark Matter and Energy)}

\item \textbf{🔷 Big Bang Theory}

\item \textbf{🔷 Heat Death of the Universe}

\item \textbf{🔷 Multiverse Theory}

\item \textbf{🔷 Theory of Evolution (in its naturalistic, atheistic form)}

\item \textbf{🔷 Theory: Consciousness as a Product of Matter (Materialism)}

\item \textbf{🔷 Theory: Life as Random Chance}

\item \textbf{🔷 Theory: Simulability of Consciousness (Personhood Criteria)}

\item \textbf{🔷 Theory: Emergence of Consciousness (Mind from Complexity)}

\item \textbf{🔷 System: Pantheism}

\item \textbf{🔷 System: Gnosticism}

\item \textbf{🔷 Universal Format for Self-Testing Religious Systems}

\item \textbf{0.3 Conclusion}

\item \textbf{0.4 Appendices}

\end{itemize}




\end{document}
